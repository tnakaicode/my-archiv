\documentclass[paper.tex]{subfiles}

\begin{document}

\section{Mathematical Function Implementation Background}
\label{sec:implementations}

While the mathematics behind approximating transcendental functions
  is well understood,
  numerous choices, like approximation method,
  polynomial order, table size, and range reduction strategy,
  all impact both accuracy and speed.


\begin{figure}
\begin{subfigure}{0.60\textwidth}
\includegraphics[width=\textwidth]{images/poly.pdf}
\caption{Polynomial approximation and error}
\label{fig:poly}
\end{subfigure} \hfill%
\begin{subfigure}{0.315\textwidth}
\includegraphics[width=\textwidth]{images/table.pdf}
\caption{Table approximation}
\label{fig:table-approx}
\end{subfigure}
\caption{
Approximations $P(x)$ of an arbitrary function $f(x)$.
A order-1 polynomial approximation is shown on the left;
  a $4$-entry table based approximation is shown on the right.
In the middle, an illustration of the equioscillation theorem:
  the polynomial approximation is optimal
  error reaches its highest value the maximum
  number of times.
}
\label{fig:approx}
\end{figure}


\begin{figure}
\begin{subfigure}{0.48\textwidth}
\scriptsize
\begin{verbatim}
double ml2_raw_wide_sin_13(double x){
  double x2 = x * x;
  double pa, pa1, pa3, pa5, pa7, pa9, pa11, pa13;
  pa13 =  0x1.52a851954275cp-33;
  pa11 = -0x1.ae00bdd2a86a8p-26 + x2 * pa13;
  pa9  =  0x1.71dce463cf737p-19 + x2 * pa11;
  pa7  = -0x1.a019fce360596p-13 + x2 * pa9;
  pa5  =  0x1.11111109020a6p-7  + x2 * pa7;
  pa3  = -0x1.5555555540916p-3  + x2 * pa5;
  pa1  =  0x1.ffffffffffdc9p-1  + x2 * pa3;
  pa   = x * pa1;
  return pa;
}
\end{verbatim}
\caption{An order-13 polynomial function approximation}
\label{fig:poly-based}
\end{subfigure}%
\hfill%
\begin{subfigure}{0.48\textwidth}
\scriptsize
\begin{verbatim}
double sin_table[255];

void fill_table() {
  for(i=0; i<255; i++){
    theta = (double)(i-127)*M_PI/127.0;
    sin_table[i] = sin(theta);
  }
}

double table_sin(double x){
  int index = (x*127.0/M_PI) + 127;
  return sin_table[index];
}
\end{verbatim}
\caption{A size-255 table-based function approximation}
\label{fig:table-based}
\end{subfigure}
\caption{
  Two possible approximations of \F{sin} for $x \in [-\pi, \pi]$.
  The one on the left uses an order-13 order polynomial
    and achieves an absolute error of \nScarySinError
    and a run time of \nScarySinTime on our reference machine.
  The on on the right
    is the table based implementation used by POV-Ray,
    with an absolute error of $0.02473$
    and a run time of \nPovRayTableImplTime.}
\end{figure}


\subsection{Approximating over an Interval}

The core technique in implementing a mathematical function
  is approximating a function $f$ over a small input range $I$.
One method is to find a polynomial $P(x)$ that approximates $f$.
While techniques like Taylor series and Chebyshev approximations,
  which optimize for the average case, are better known,
  implementations usually optimize for worst-case errors.
The Remez exchange algorithm~\cite{remez},
  which solves for local minima and maxima of $f(x) - P(x)$
  by the Chebyshev equioscillation theorem~\cite{equioscillation},
  is the standard way of deriving coefficients for $P$
  that minimize worst-case error.
\Cref{fig:poly} illustrates this approach.
Remez exchange is implemented in the the Sollya tool~\cite{sollya},
  which additionally uses the LLL algorithm~\cite{lll}
  to tune the coefficients for the floating-point domain.
The end result of this process is an implementation
  similar to the one shown in \Cref{fig:poly-based}.
The number of terms in the polynomial is the key parameter to this process;
  an implementation can approximate a target functions like $\sin$ or $\exp$
  with anywhere from one to dozens of terms,
  offering a spectrum with higher-order approximations
  being more accurate but also slower.
The input interval $I$ is also an important input;
  generally speaking, wider input intervals yield worse approximations.
Even once a polynomial is found, there are more choices to make.
For example, the polynomial
\(
  P(x) = a_0 + a_1 \cdot x + a_2 \cdot x^2 + \dotsb
\)
is usually evaluated via Horner's rule,
\(
  P(x) = a_0 + x\cdot(a_1 + x\cdot(a_2 + x\cdot(\dotsb)))
\),
  but there are other evaluation schemes as well,
  like Estrin’s Method for parallelization with SIMD
  or compensated summation for higher accuracy.

Table-based implementations are an alternative to pure polynomial approximation.
In a table-based implementation,
  the interval $I$ is split into many smaller intervals,
  often hundreds or thousands of them,
  and then some uniform method is used to approximate the function
  on each of those smaller intervals.
A simple table-based implementation like this
  tabulates $f(x)$ for evenly spaced $x$ values,
  like the $\sin$ table in the back of an old-school math textbook;
  such an implementation is shown in \Cref{fig:table-based}.
A more sophisticated approach
  may store polynomial coefficients in the table,
  or use complex interpolation schemes to fill in intermediate values.
Just as with polynomial implementations,
  a table-based scheme has many parameters for the implementor to choose:
  table size, interpolation scheme, subdivision algorithm, and so on;
  and again, these parameters affect both speed and accuracy,
  and may depend on machine-specific parameters like cache size.

\subsection{Range Reduction}

Neither polynomial nor table-based approximation
  works well over a large input interval $I$,
  so the most challenging part of approximating a mathematical function
  is making sure the function can be applied to an arbitrary input.
This is called \textit{range reduction and reconstruction}.
Range reduction and reconstruction bookend the polynomial or table,
  transforming the input to lie within the polynomial or table's input range
  and then adjusting the output to fit the original input.

\begin{figure}
\hfill%
\begin{minipage}[c]{.56\textwidth}
\includegraphics[width=\textwidth]{images/range_reduction.pdf}
\end{minipage}\hfill
\begin{minipage}[c]{0.38\textwidth}
\caption{
\small
Range reduction and reconstruction for \F{sin}.
The input point, somewhere along the input domain, is mapped via range reduction
  into the interval $[0,\pi/2]$ for evaluation, then mapped back via
  reconstruction to the correct output value.
This reduction takes advantage of the cyclic nature of \F{sin}, the symmetry of
  the sections of \F{sin} above and below $y=0$, and the mirror symmetry
  present in each half cycle.
Reconstruction of \F{sin} needs only determine the correct sign of the output.
}
\end{minipage}\hfill
\label{fig:range}
\end{figure}


Consider implementing \F{sin} over a large range.
Since \F{sin} is periodic, $\sin(x) = \sin(x \bmod 2\pi)$;
  and furthermore, $x \bmod 2\pi$ is bounded
  to the range $[0, 2\pi]$.
Thus, \F{sin} can be implemented over a large range
  by using a polynomial or table for the range $[0, 2\pi]$
  and \emph{reducing} other inputs to lie within that range.
We can take advantage of other symmetries
  to restrict the range further.
For example, $\sin(x) = -\sin(x - \pi)$;
  for inputs $x \in [\pi, 2\pi]$ the right hand side of this equation
  calls $\sin$ on inputs $x - \pi \in [0, \pi]$.
So, a polynomial or table fit to $[0, \pi]$ is enough,
  but in this case a \emph{reconstruction} step is necessary
  to take the output on the reduced input
  (in this case, $\sin(x - \pi)$)
  and compute from it the output for the original input
  (in this case, by negating the reduced output).
Both reduction and reconstruction are shown in \cref{fig:range}.
Other identities, like the double-angle formula,
  can also be used for range reduction,
  and the same general principle
  of using function identities to reduce the input range
  can apply to $\log$, $\exp$, or any other functions
  with identities to leverage.

Reducing the interval $I$ that the polynomial or table is fit to
  usually makes the approximation itself
  either more accurate or faster (by using fewer terms).
But range reduction and  reconstruction
  can slow the implementation down
  and add error of their own.
Traditionally, math libraries have used higher-precision arithmetic,
  such as in Cody \& Waite reduction~\cite{cody-waite},
  to implement range reduction with no error.
This maximizes accuracy, especially for very large inputs
  (imagine computing $\F{sin}(10^{10})$),
  but is also very slow because it requires computing $x \bmod 2\pi$
  in very high precision using thousands of bits of $\pi$.
But range reduction can also trade accuracy for speed.
For example, $x \bmod 2\pi$ can be computed
  by just evaluating $x - \lfloor x / 2 \pi \rfloor$ in double precision.
This introduces errors
  (from approximating $\pi$ and from division)
  but is significantly faster than high-precision arithmetic.
There are also intermediate choices with more or less error.
Thus, the choice of range reduction
  again presents a trade-off between accuracy and speed.

\end{document}

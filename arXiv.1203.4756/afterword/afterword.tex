
\chapter{Afterword}
\label{cha:afterword}

An alternative title for this thesis could be ``Prior information in
the phase retrieval problem''. This also can describe the main line of
the work presented here. In fact we discovered and showed how to use
two powerful priors: approximately known Fourier phase, and sparsity
of the sought signal. The material in each chapter represents
an idea and the main results in a succinct form that is suitable for 
publication in a scientific journal. Therefore, some of the
results were not included in this thesis. Part of them is available as
technical reports listed below
\begin{itemize}
\item \shortcite{osherovich10algorithms}
\item \shortcite{osherovich10simultaneous}
\item \shortcite{osherovich09image}
\item \shortcite{osherovich08signal}
\end{itemize}

Other forms of prior knowledge, for example, defocused/blurred version of the
sought signal can be found in~\shortcite{osherovich10numerical}, which
summarizes our work done in  collaboration with KLA Tencor Inc. 

We should also mention a standalone work done
in~\shortcite{osherovich09designing}, where we used Fienup's HIO
algorithm to design an overcomplete dictionary in a way that its atoms
(columns) are maximally uncorrelated. The method produces excellent
results that are significantly better than the results produced by
currently used algorithms.

%%% Local Variables: 
%%% mode: latex
%%% TeX-master: "../thesis"
%%% End: 



\chapter{Introduction}
\label{cha:introduction}

\section{Motivation}
\label{sec:motivation}
Recent development of nanotechnology has resulted in great interest in
imaging techniques suitable for visualization of nano-structures.  One
of the most promising techniques for such high resolution imaging is
Coherent Diffraction Imaging (CDI). In CDI, a highly coherent beam of
X-rays or electrons is incident on a specimen, generating a
diffraction pattern.  Under certain conditions the diffracted
wavefront is approximately equal (within a scale factor) to the
Fourier transform of the specimen.  After being recorded by a CCD
sensor, the diffraction pattern is used to reconstruct the
specimen~\shortcite{sayre52implications,miao99extending,quiney10coherent}.
Effectively, in CDI we replace the objective lens of a typical
microscope with a software algorithm. The advantage in using no lenses
is that the final image is aberration-free and the final resolution is
only diffraction and dose limited, that is, dependent only on the
wavelength, aperture size and exposure time. This process is
illustrated in Figure~\ref{fig:cdi-process}.
\begin{figure}[H]
  \centering
  \subfloat[]{
    \includegraphics[width=0.3\textwidth]{cnt2_cdi_simulation}
  }
  \subfloat[]{
    \includegraphics[width=0.3\textwidth]{cnt2_cdi_fourier}
  }
  \subfloat[]{
    \includegraphics[width=0.3\textwidth]{cnt2_cdi_reconstruction}
  }
  \caption[CDI process]{CDI process: a specimen (a) scatters a wave of
    X-rays or electrons that otherwise diffracts freely; the
    diffraction pattern (b) is used for an algorithmic reconstruction
    (c) of the original specimen. Illustrations taken
    from~\shortcite{wikipedia11coherent}.}
  \label{fig:cdi-process}
\end{figure}
The method has been successfully applied to visualizing a variety of
nano-structures, such as carbon nano-tubes~\shortcite{zuo03atomic},
defects inside nano-crystals~\shortcite{pfeifer06three-dimensional},
proteins, and
more~\shortcite{neutze00potential,chapman06high-resolution,chapman07femtosecond}.
Furthermore, exactly the same problem---the reconstruction of a signal
from the magnitude of its Fourier transform---arises in may other
areas of science. Notable examples include astronomy, crystallography,
and speckle interferometry.

It is important to note that, due to the physical nature of the
sensor, we are limited to recording only the intensity (squared
amplitude) of the diffracted wave, hence its phase is lost. As will
be shown later, this loss of the phase leaves us with highly incomplete
data, which makes the problem of reconstruction hard.



\section{Data acquisition model}
\label{sec:data-acqu-model}
Of course, in the real world, the sought object $x(t)$ and its Fourier
transform (denoted by $\hat{x}(\omega)$) are both continuous functions
of $t$ and $w$, respectively, where $t$ and $w$ are multidimensional
coordinate vectors. Furthermore, the support of $x(t)$ is limited,
which means that $\hat{x}(\omega)$ is spread over all frequencies:
from $-\infty$ to $\infty$.  However, during the data acquisition
process we capture only a finite extent in the Fourier domain and all
further processing is done on digital computers. This naturally leads
to discrete approximations of $x(t)$ and $\hat{x}(\omega)$, that are
well justified in  view of the finite resolution that stems from
the measurements and, in general, from the fact that all optical
systems have resolution limits. Given that $x[n]$ (an adequate
sampling of $x(t)$) contains $N$ points (for simplicity we assume that
$x$ is one-dimensional---generalization for the multi-dimensional case
is straightforward) we assume that $x[n]$ vanishes outside the interval
$[0,N-1]$. Furthermore, if we assume, without loss of generality, that
the physical extent of $x$ is unity we immediately conclude that the
sampling rate in the Fourier domain must be $1/N$ to acquire the
measurements that are related to $x[n]$ via the Discrete Fourier
Transform (DFT). However, a more thorough examination of the problem
yields a higher sampling rate requirement. Recall that we record only
the intensity of the diffraction pattern. This intensity can be
represented as follows:
\begin{equation}
  \label{eq:intro-1}
  I(\omega) =
  |\hat{x}(\omega)|^{2}=\bar{\hat{x}}(\omega)\circ\hat{x}(\omega)\,, 
\end{equation}
where the overbar denotes the complex conjugate and $\circ$ denotes
element-wise multiplication. Hence, the inverse Fourier transform of
the measured intensity $I(\omega)$ results in the auto-correlation
function (denoted by $\star$) of the sought object,
\begin{equation}
  \label{eq:intro-2}
  \mathcal{F}^{-1}[I(\omega)] = x(t)\star x(t) \,. 
\end{equation}
Obviously, the autocorrelation $x(t)\star x(t)$ has support that is
twice as large as the support of $x(t)$ (in each dimension),
therefore, the diffraction pattern intensity must be sampled with the
rate two times higher than $1/N$ to capture all the information about
the auto-correlation function. To this end, we always assume that the
signal $x(t)$ (or $x[n]$) is ``padded'' with zeros so that its
size is doubled (in each dimension). This requirement is an
approximation to the physical constraint on $x(t)$ having finite
support. Without adding it into the reconstruction scheme, the problem
would be severely undetermined with multiple solutions that are
unrelated to the original signal $x$. To illustrate the last claim, one
can imagine the case where $x[n]$ is reconstructed from its Fourier
magnitude without additional constraints. Obviously, $\emph{any}$
choice of the Fourier phase will give rise to a valid solution which,
unfortunately,  has little to do  with the sought signal. 

\section{Reconstruction from incomplete Fourier data}
\label{sec:reconstr-from-incomp}
Before we proceed to the main subject of this work---the
reconstruction of a signal from the magnitude of its Fourier
transform---let us consider a number of toy problems, where we
evaluate the importance of different parts of the Fourier transform.

The Fourier transform is, in general, complex. There are two common
representations of a complex number: one is the sum of its real
($\Re$) and
imaginary ($\Im$) parts
\begin{equation}
  \label{eq:intro-3}
  z = \Re + j\Im \,, 
\end{equation}
and the other is the product of its magnitude ($r$) and the complex
exponent of its phase $e^{j\phi}$, 
\begin{equation}
  \label{eq:intro-4}
  z = re^{j\phi} \,. 
\end{equation}
From these formulas it is not clear whether one part of the
representation is more important
than the other. Below we demonstrate that the real and the
imaginary parts carry about equal amount of information and the loss
of one of them can often be recovered. However, the  phase carries
most of the information, and consequently, its loss is more difficult to
overcome. 

\subsection{Reconstruction from the real part}
\label{sec:reconstr-from-real}
Let us assume that $x[n]$ is a real one-dimensional signal\footnote{A
  generalization to a complex multidimensional  $x$ is
  straightforward.}. Furthermore, we assume that $x$ vanishes outside
the interval $[0,N-1]$, specifically, we assume that
$x[n]=0$ for $n = -1, -2, \ldots, -(N-1)$. Recall that any real signal
can be represented uniquely as a sum of two signals
\begin{equation}
  \label{eq:intro-5}
  x = x_{e}+x_{o} \,, 
\end{equation}
where $x_{e}$ is even and $x_{o}$ is odd\footnote{In the complex case,
  $x_{e}$ is Hermitian, and $x_{o}$ is anti-Hermitian.}
(see Figure~\ref{fig:even-odd}).  It can be easily
shown that
\begin{equation}
  \label{eq:intro-6}
  x_{e} = \frac{x[n] + x[-n]}{2}\,, \quad x_{o}=\frac{x[n] - x[-n]}{2}
  \,. 
\end{equation}
Recall also that the Fourier transform of an even signal is real and
that of an odd signal is purely imaginary. Hence, we conclude that the
real part of $\hat{x}$ is nothing but the Fourier transform of
$x_{e}$. Thus, we can obtain the even part $x_{e}$ by the inverse
Fourier transform of the real part of $\hat{x}$.  Furthermore,
reconstructing $x$ from $x_{e}$ is trivial---we should take the right-hand
side ($n\geq0$) of $x_{e}$ multiplied by two everywhere except the
origin (see Figure~\ref{fig:even-odd}).

\begin{figure}[H]
  \centering
  \subfloat[]{
    \includegraphics[width=0.8\textwidth{}]{signal}
  }\\
  \subfloat[]{
    \includegraphics[width=0.8\textwidth{}]{signalrev}
  }\\
  \subfloat[]{
    \includegraphics[width=0.8\textwidth{}]{signal_even}
  }\\
  \subfloat[]{
    \includegraphics[width=0.8\textwidth{}]{signal_odd}
  }
  \caption[Signal decomposition into even and odd parts]{Signal decomposition into even and odd parts: (a) original
    signal $x[n]$, (b) its reversed version $x[-n]$, (c) the even
    part, and (d) the odd part.}
  \label{fig:even-odd}
\end{figure}

\subsection{Reconstruction from the imaginary part}
\label{sec:reconstr-from-imag}
The reconstruction from the imaginary part is also easy. The method is
very similar to the reconstruction from the real part. The only
difference is that now we obtain the odd part of the sought
signal. But, again, taking the right-hand side of $x_{o}$ and multiplying
it by two we obtain the original signal $x$ everywhere besides the
origin.

For the most general case, where $x$ is complex, it is easy to show
that the missing imaginary part leads to almost perfect
reconstruction---only $\Im(x[0])$ is lost. Similarly, when
reconstructing from the imaginary part of $\hat{x}$---only $\Re(x[n])$
is lost. Therefore, we can conclude that the real and the imaginary
parts of the Fourier transform carry the same amount of information and
losing either one of them can be easily overcome if $x[n]$ is
sufficiently padded with zeros and vanishes for $n<0$.


\subsection{Reconstruction from the phase}
\label{sec:reconstr-from-phase}
Now we switch to the second (polar) representation of complex numbers and
consider first the situation where the magnitude is lost, namely,
reconstruction from the Fourier phase. Several researchers
(see~\shortcite{hayes82reconstruction,hayes80signal,millane96multidimensional,millane90phase,oppenheim81importance})
showed that sufficiently padded signals can be reconstructed from the
Fourier phase. However, the reconstruction is possible only up to
a scale factor, that is, one obtains  $\alpha x$ for some
real positive scalar $\alpha$. Furthermore, is was shown that this is
the only type of non-uniqueness possible in signal reconstruction from
the Fourier phase~\shortcite{hayes80signal}.

It seems that all previous works used a variation of the method
of alternating projections (see
Section~\ref{sec:gerchb-saxt-meth}). However, the problem can easily
be represented as linear programming, for which much more efficient
algorithms exist. Actually, we solve a variation of this problem, where
the Fourier phase is known to lie within  certain interval in
Chapters~\ref{cha:appr-four-phase-first} and
\ref{cha:appr-four-phase-explanation}, and our method is much faster
than the method of alternating projections. 

\subsection{Reconstruction from the magnitude}
\label{sec:reconstr-from-magn}
We finally arrive at the case that is the main subject of this
research---reconstruction from the Fourier magnitude alone. It turns
out that this case of incomplete Fourier information is the most
difficult among the four possibilities considered here. This
phenomenon can be related to the fact that the Fourier phase carries
the majority of the information in a signal. As an informal ``proof''
of the latter claim, look at Figure~\ref{fig:phase-importance}, where
we exchange the Fourier phase between two different images, while
keeping the Fourier magnitude intact. The results show, unmistakably,
an exchange of the images.

It turns out that this problem is very different in the
one-dimensional case and in the multi-dimensional case. The former
suffers from multiplicity of possible solutions (no matter how much
padding we use in $x$), while the latter is usually less prone to
multiple solutions (aside from trivial transformations: lateral
shifts, axis reversal, and constant phase factor), but finding a
solution is very difficult. An explanation for this phenomenon is
given in Chapter~\ref{cha:math-found}.

\begin{figure}[H]
  \centering
  \subfloat[]{
    \includegraphics[height=0.4\textheight{}]{dog}
  }\qquad{}
  \subfloat[]{
    \includegraphics[height=0.4\textheight{}]{cat}
  }\\
  \subfloat[]{
    \includegraphics[height=0.4\textheight{}]{dogMag_catPhase}
  }\qquad{}
  \subfloat[]{
    \includegraphics[height=0.4\textheight{}]{catMag_dogPhase}
  }
  \caption[The importnace of phase in signals]{The importnace of phase in signals---given two images (a) a
    dog, and (b) a cat, we exchange their Fourier phase while keeping
    the magnitude. The results, unmistakably, show the cat instead of
    the dog (c), and the dog instead of the cat (d).}
  \label{fig:phase-importance}
\end{figure}

\section{A short overview}
\label{sec:short-overview}
After this short introduction we are ready to proceed to the main
part of this work. In the next chapter we present the mathematical
foundations of the phase retrieval problem: our main concern is
uniqueness of the reconstruction, moreover, we demonstrate why
the one-dimensional and the multi-dimensional cases behave extremely
differently. In Chapter~\ref{cha:curr-reconstr-meth} we present the
methods that are used today for phase retrieval. These chapters are
based on a compilation of known facts and do not contain our original
research. Our development starts in
Chapter~\ref{cha:found-optim-meth}, where we introduce machinery for
efficient optimization of a real function of complex
argument. Moreover, we find and analyze the eigen-decomposition of the
Hessian of one of the most frequently used objective function in phase
retrieval. Chapters~\ref{cha:appr-four-phase-first} and
\ref{cha:appr-four-phase-explanation} are dedicated to a variant of
the phase retrieval problem where the Fourier phase is not lost
completely, but rather a rough estimate of it is available. We demonstrate
empirically and then prove mathematically that this scenario leads to
a new family of algorithms that are orders of magnitude faster than
the existing methods. The ideas from these chapters are taken into the
new field of the Fourier domain holography in
Chapters~\ref{cha:phase-retr-holography}, and
\ref{cha:design-bound-phase} where we develop a new method of
reconstruction and present guide rules for a reference beam design
that lead to fast and robust reconstruction. A new type of prior
knowledge---image sparsity---is exploited in
Chapter~\ref{cha:bandw-extr-using} where we present a CDI with
resolution several times higher than the maximal theoretical
limit. Finally, in Chapter~\ref{cha:afterword} we present concluding
remarks and provide reference to the works that was done in the course
of this Ph.D. research but have not been included in this thesis. For
example, we used Fienup's Hybrid Input-Output algorithm (see
Section~\ref{sec:fienup-algor-phase}) for creation of Grassmanian
matrices~\shortcite{osherovich09designing}---an interesting work
that was left outside because it is not related to the phase retrieval
problem. 


%%% Local Variables: 
%%% mode: latex
%%% TeX-master: "../thesis"
%%% End: 

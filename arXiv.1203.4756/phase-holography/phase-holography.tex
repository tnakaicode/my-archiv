\chapter{Phase retrieval combined with digital
  holography\footnotemark{}}
\label{cha:phase-retr-holography}

\footnotetext{The material presented in this section is currently in
  preparation for submission to a journal.}

In this chapter we take our algorithm developed in the previous
chapters into a new niche: signal reconstruction from two intensity
measurements made in the Fourier plane. One is the Fourier magnitude
of the sought image, as in classical phase retrieval, and the second
is the intensity pattern resulting from the interference of the
original signal with a known reference beam, as in the Fourier domain
holography. Although either one of these measurements may, in theory,
be sufficient for successful reconstruction of the unknown image, our
method provides significant advantages over such reconstructions. For
example, comparing with reconstruction from the Fourier magnitude
alone by HIO, our method gives a much faster speed and better quality
in case of noisy measurements as we showed earlier. Furthermore,
unlike classical holography methods, our algorithm does not require
any special design of the reference beam. Finally and most
importantly, very good reconstruction quality is obtained even when
the reference beam contains severe errors.


\section{Basic reconstruction algorithm}
\label{sec:basic-reconstr-algor} Let us start with the notation used
throughout the chapter. The unknown two-dimensional signal that we wish
to reconstruct is represented by the complex-valued function
$z(\xi,\eta)=|z(\xi,\eta)|\exp[j\varphi(\xi,\eta)]$. To address the
phase of a complex-valued number we use the angle notation, as before:
$\angle(z(\xi,\eta)) \equiv \varphi(\xi,\eta)$. Our measurements are
done in the Fourier plane $(\xi',\eta')$, hence the transformation
that $z$ undergoes when transforming from the $(\xi,\eta)$ plane to
the $(\xi',\eta')$ plane is simply the unitary Fourier transform
\begin{equation}
  \label{eq:phase-holo-1}
  \hat{z}(\xi',\eta') = \mathcal{F}\{z(\xi,\eta)\}\ .
\end{equation}
% Strictly speaking, the actual transformation is a bit more
% complicated as it includes some constant multipliers and scale
% factors \shortcite{goodman05introduction}.  However, these are
% immaterial for our discussion. 
Hereinafter, we use  the
usual convention that a pair of symbols like $x$ and $\hat{x}$ denotes a
signal in the $(\xi,\eta)$ plane (also known as the object domain) and
its counterpart in the $(\xi',\eta')$ plane (also referred to as the
Fourier domain), respectively. For the sake of brevity, we may
omit the location designator $(\xi,\eta)$  or $(\xi',\eta')$ and use $x$
or $\hat{x}$ when the entire signal is considered.

The main purpose of our work is to develop a robust reconstruction
method that can tolerate severe errors in the reference beam. To
this end we use the reference beam only for \textit{estimating}
the Fourier phase of the sought image. Once a rough phase estimate
is available we can use the method of phase retrieval with
approximately known Fourier phase that was developed in
the previous chapters.

The two measurements available at our disposal are used as follows.
$I_{1}$ provides the Fourier magnitude of the sought image via the
simple relationship between the two:
\begin{equation}
  \label{eq:phase-holo-2}
  I_{1}(\xi',\eta') = |\hat{z}(\xi',\eta')|^{2}\ .
\end{equation}
The second measurement reads
\begin{equation}
  \label{eq:phase-holo-3}
  I_{2}(\xi',\eta') =|\hat{z}(\xi',\eta') + \hat{R}(\xi',\eta')|^{2}\ ,
\end{equation}
where $\hat{R}(\xi, \eta)$ denotes a known reference beam that is
used to obtain the Fourier phase estimate as described below. One
possible schematic setup that provides these measurements is shown
in the next figure.
\begin{figure}[H]
  \centering
  \includegraphics[width=\textwidth{}]{fourierHolography}
  \caption{Schematic representation of the experiment.}
  \label{fig:experiment-schematic}
\end{figure}
Note that $\hat{R}$ is not
necessarily a Fourier transform of some physical signal $R$ in the
object domain. This means that $\hat{R}$ can be formed directly in
the Fourier plane without forming first $R$ and applying then an
optical Fourier transform to obtain $\hat{R}$.  Nevertheless,
there exists the mathematical inverse
\begin{equation}
  \label{eq:phase-holo-4}
  R(\xi,\eta) = \mathcal{F}^{-1}\{\hat{R}(\xi',\eta')\}\ ,
\end{equation}
whose properties, such as extent, magnitude, etc. can be
considered. The only requirement of $\hat{R}$ is that it must not
vanish in the region of our measurements. This is an important
point that provides an advantage for our method over the classical
holography techniques. We shall elaborate more on this in
Section~\ref{sec:relation-holography}.

Let us now describe  how $\hat{z}$'s phase information is extracted
from $I_{2}$, and, more importantly, how it is used in our
reconstruction method. Consider the two signals:
\begin{equation}
  \label{eq:phase-holo-5}
  \hat{z}(\xi',\eta') = |\hat{z}(\xi',\eta')|\exp[j\phi(\xi',\eta')]\ ,
\end{equation}
and
\begin{equation}
  \label{eq:phase-holo-6}
  \hat{R}(\xi',\eta') = |\hat{R}(\xi',\eta')|\exp[j\psi(\xi',\eta')]\ .
\end{equation}
The intensity pattern of their interference can be written as:
\begin{equation}
  \label{eq:phase-holo-7}
\begin{split}
  I_{2}(\xi',\eta')
  = &|\hat{z}(\xi',\eta') + \hat{R}(\xi',\eta')|^{2}\\
  = &|\hat{z}(\xi',\eta')|^{2} + |\hat{R}(\xi',\eta')|^{2}
  + \hat{z}^{*}(\xi',\eta')\hat{R}(\xi',\eta')
  + \hat{z}(\xi',\eta')\hat{R}^{*}(\xi',\eta')\\
  = & |\hat{z}(\xi',\eta')|^{2} + |\hat{R}(\xi',\eta')|^{2}
  +
  2|\hat{z}(\xi',\eta')|\,|\hat{R}(\xi',\eta')|\cos[\phi(\xi',\eta')-\psi(\xi',\eta')] 
  \ .
\end{split}
\end{equation}
From this formula we can easily extract the difference between the
unknown phase $\phi(\xi',\eta')$ and the known phase
$\psi(\xi',\eta')$:
\begin{equation}
  \label{eq:phase-holo-8}
  \cos[\phi(\xi',\eta')-\psi(\xi',\eta')] =
  \frac
  {I_{2}(\xi',\eta') -|\hat{z}(\xi',\eta')|^{2} - |\hat{R}(\xi',\eta')|^{2}}
  {2|\hat{z}(\xi',\eta')|\,|\hat{R}(\xi',\eta')|} \ .
\end{equation}
This gives us:
\begin{equation}
  \label{eq:phase-holo-9}
  \phi(\xi',\eta') = \psi(\xi',\eta') \pm \alpha(\xi',\eta')\ ,
\end{equation}
where
\begin{equation}
  \label{eq:phase-holo-10}
  \alpha(\xi',\eta') = \arccos
  \left[
    \frac
    {I_{2}(\xi',\eta') -|\hat{z}(\xi',\eta')|^{2} - |\hat{R}(\xi',\eta')|^{2}}
    {2|\hat{z}(\xi',\eta')|\,|\hat{R}(\xi',\eta')|}
  \right]\ .
\end{equation}
This expression is well defined, as $|\hat{R}(\xi',\eta')|$ is
assumed to be non-zero everywhere in the region of interest, and
the places where $|\hat{z}(\xi',\eta')|=0$ can be simply excluded
from our consideration as there is nothing to be recovered because
their phase has no influence. We assume that $\pm\alpha$, that is,
the difference between the phases $\phi$ and $\psi$, lies within
the interval $[-\pi,\pi]$, hence, no phase unwrapping is
necessary. The phase $\phi(\xi',\eta')$ can assume either one
(rarely) or two possible values at every location. The two
possible situations are shown in
Figure~\ref{fig:holography-possible-situations}.
\begin{figure}[H]
  \centering
  \subfloat[]{
    \label{fig:projection-fourier}
    \includegraphics[width=0.45\textwidth]{holography2solutions}
  }
  \qquad{}
  \subfloat[]{
    \label{fig:convexrelaxation}
    \includegraphics[width=0.45\textwidth]{holography1solution}
  }
  \caption[Possible scenarios in Fourier domain holography]{Given a reference beam (black) whose magnitude and phase
    are known, and an unknown signal of known magnitude (dotted
    circle radius), one can try to find the phase of the unknown
    signal by measuring the magnitude of the sum (solid circle
    radius). Evidently, in most cases there are two possible
    solutions (a). However, in certain cases, the is only one
    solution (b).}
  \label{fig:holography-possible-situations}
\end{figure}


Hence, if the
intensity $I_{2}(\xi',\eta')$ is sampled at $N$ points there are
generally $2^{N}$ possible solutions $\hat{z}(\xi',\eta')$ and,
consequently, the same number of possible reconstructions
$z(\xi,\eta)$. (Here we consider the worst case scenario where all
sampled values give rise to two solutions.) To guarantee a unique
and meaningful reconstruction we must use additional information
about the sought signal $z(\xi,\eta)$. In the phase retrieval problem,
as well as in holography, 
it is usually assumed that $z(\xi,\eta)$ has limited support, namely,
part of the image is occupied by zeros.  In practice, it is
usually assumed that, in each direction, half or more pixels of
$z(\xi,\eta)$ are zeros. To capture this information in the Fourier
domain one should ``over-sample'' by a factor of two (or more) in
each dimension. Hence, if the known (not necessarily tight)
support area of $z$ is $n\times n$ pixels, then in the Fourier
plane it must be sampled with a sensor of size $2n\times 2n$
pixels. Such ``over-sampling'' usually guarantees unique (up to
trivial transformations: shifts, constant phase factor, and axis
reversal) reconstruction in the case of the classical phase
retrieval problem, where only $|\hat{z}|$ is available
\shortcite{hayes82reconstruction}. It is not known whether this
two-fold oversampling is absolutely necessary in our case where
two measurements are available. However, our experiments indicate
that for a general complex-valued signal it still seems to be
necessary to over-sample by a factor of two. Hence, the
reconstruction problem reads: find $z(\xi,\eta)$ such that $|\hat{z}|$
is known, $\angle(\hat{z})=\psi\pm\alpha$, and $z(\xi_{O},\eta_{O}) =
0$. Here, $\alpha$ is the known phase difference between $\hat{z}$
and $\hat{R}$ as defined by Equations~(\ref{eq:phase-holo-9})
and~(\ref{eq:phase-holo-10}); $(x_{O},y_{O})$ denotes the known off-support
area in the $(\xi,\eta)$ plane.

The problem is combinatorial in nature, and many different methods
can be applied to find a solution. Our method is based on
replacing the equality $\angle(\hat{z})=\psi\pm\alpha$ with the
less strict inequality
\begin{equation}
  \label{eq:phase-holo-11}
  \psi - \alpha \leq \angle(\hat{z}) \leq \psi + \alpha\ .
\end{equation}
By this relaxation we reduce the original problem into the phase
retrieval problem with approximately known phase.  For this
situation we have developed an efficient Quasi-Newton optimization
method based on convex relaxation. Note that the problem we are facing
here:
\begin{equation}
  \label{eq:phase-holo-12}
  \begin{split}
    \min_{x} &\quad \||\mathcal{F}\{x\}| -  |\hat{z}|\|^{2}\\
    \mathrm{subject\ to} &\quad \psi - \alpha \leq \angle(\mathcal{F}\{x\}) \leq
    \psi+\alpha \, , \\
    &\quad x(x_{O}, y_{O}) = 0 \, ,
  \end{split}
\end{equation}
is exactly the same as the one we solved in
Chapters~\ref{cha:appr-four-phase-first},
and~\ref{cha:appr-four-phase-explanation}. Hence, we use the same
convex relaxation as we did before. Likewise, the solution of the
above minimization problem $x$ is not guaranteed to be equal to the
sought signal $z$. Due to the relaxation we performed, the phase of $\hat{x}$ is
allowed to assume the continuum of values in the interval
$[\psi-\alpha, \psi+\alpha]$ instead of the two discrete values
$\psi\pm\alpha$.  However, due to the uniqueness of the phase
retrieval problem, the phase of $x$ may differ from the phase of $z$
only by a constant.  That is, $x(\xi,\eta)=z(\xi,\eta)\exp[jc]$ for
some scalar $c$ (see~\shortcite{hayes82reconstruction} for details). This
does not pose any problems, as only the relative phase distribution inside
the support area of $z(\xi,\eta)$ is usually of interest. Moreover, in
the case where the absolute phase is required, $c$ can be recovered by
adding a post-processing step that solves the one-dimensional
optimization problem:
\begin{equation}
  \label{eq:phase-holo-13}
  \displaystyle\min_{c} \||\hat{x}\exp[jc] + \hat{R}| -
  I_{2}^{1/2}\|^{2}\ .
\end{equation}
Note that we intentionally do not add a penalty term like
$\||\hat{x} + \hat{R}| - I_{2}^{1/2}\|^{2}$ into the main
minimization scheme as defined by Equation~(\ref{eq:phase-holo-12}).  Adding
such a term would introduce a strong connection between $x$ and
the reference beam $\hat{R}$. This connection will inevitably
deteriorate the quality of reconstruction when the reference beam
$\hat{R}$ contains errors.

There is, however, an important dissimilarity between the current
situation and the one we considered in the previous chapter: the phase
uncertainty interval can be as large as $\pi$ radians. Nevertheless,
our experiments indicate that the reconstruction is stable and its
speed is very fast (see Figure~\ref{fig:phase-holo-we-reconstruction-speed}).
Moreover, it can be further accelerated, and our experiments indicate
that more aggressive oversampling (zero padding in the object domain)
results in faster convergence in terms of the number of iterations. In
fact, use of a special reference beam can, in theory, result in a
trivial non-iterative reconstruction in a way similar to
holography. However, such special reference beams may not be easily
realizable in physical systems and the quality of the reconstructed
signal is strongly influenced by the quality of the reference beam. We
discuss this setup in the next section and compare its sensitivity to
possible errors in $\hat{R}$ against our method in
Section~\ref{sec:phase-holo-results}.

\section{Relation to holography}
\label{sec:relation-holography}

Our method was initially developed for the phase retrieval
problem. However, the use of interference patterns creates a
strong connection with holography. Therefore, it may be pertinent
to discuss the advantages that our method provides over the classical
holographic reconstruction. Note that our discussion is limited to
basic holography only, and no attempt is made to cover all possible
setups and techniques that can be used in digital holography. We
nonetheless believe that this novel approach can compete with or
improve upon existing algorithms used in digital holography.

In classical holography one uses a specially designed reference
beam so as to allow easy non-iterative recovery of the sought
image. This has an obvious advantage over iterative methods,
especially when the speed of the reconstruction is of high
importance. However, reliance on the reference beam means that
reconstruction quality may deteriorate badly when the reference
beam contains errors, that is, when it differs from the ``known''
values. To review the non-iterative reconstruction method used in
holography, recall that the recorded intensity $I_{2}$ is the
result of superimposition of $\hat{z}$ and $\hat{R}$ as defined by
Equation~(\ref{eq:phase-holo-7}). In optical Fourier holography, this
intensity is recorded on optical material. The recorded image is
used then as an amplitude modulator for an illuminating beam
$\hat{A}(\xi',\eta')$, which then undergoes a Fourier transform to
form a new signal $B(x',y')$. In a digital computer we may use the
same technique. Moreover, we are free to use either the forward or
the inverse Fourier transform, as it makes no practical difference
(the resulting images will be reversed conjugate copies of each
other). Here we use the inverse Fourier transform:
\begin{equation}
  \label{eq:phase-holo-14}
  \begin{aligned}
    B(\xi,\eta)
    =&\,  \mathcal{F}^{-1}\{\hat{A}(\xi',\eta')\,I_{2}(\xi',\eta')\}\\
    =&\,   \mathcal{F}^{-1}
    \left\{
      \hat{A}
      \left[
        |\hat{z}|^{2} + |\hat{R}|^{2}
        + \hat{z}^{*}\hat{R} + \hat{z}\hat{R}^{*}
      \right]
    \right\}\\
    = &\,  
    A(\xi,\eta) \otimes z(\xi,\eta) \otimes z^{*}(-x,-y) +
    A(\xi,\eta) \otimes R(\xi,\eta) \otimes R^{*}(-x,-y) +\\
    &\,
    A(\xi,\eta) \otimes z^{*}(-x,-y) \otimes R(\xi,\eta) +
    A(\xi,\eta) \otimes z(\xi,\eta) \otimes R^{*}(-x,-y) \, ,
  \end{aligned}
\end{equation}
where $\otimes$ denotes convolution. Note that in this case the
fourth and the third terms are equal to the sought wavefront
$z(\xi,\eta)$ and its Hermitian counterpart $z^{*}(-x,-y)$ convolved
with $A(\xi,\eta)\otimes R(\xi,\eta)$ and $A(\xi,\eta)\otimes R^{*}(-x,-y)$,
respectively.  The best possible choice is $A(\xi,\eta) = \delta(\xi,\eta)$
and $R(\xi,\eta)= \delta(\xi-\xi_{0},\eta-\eta_{0})$, where $\delta(\xi,\eta)$ is the
Dirac delta function.  In this case the obtained wave becomes:
\begin{equation}
  \label{eq:phase-holo-15}
  B(\xi,\eta) =
  z(\xi,\eta) \otimes z^{*}(-\xi,-\eta) + \delta(\xi,\eta) +
  z^{*}(-\xi+\xi_{0}, -\eta+\eta_{0}) + z(\xi-\xi_{0}, \eta-\eta_{0})\ .
\end{equation}
Hence, if $\xi_{0}$ and/or $\eta_{0}$ are large enough, the four terms
in the above sum will not overlap in the $(\xi,\eta)$ plane. Thus, we
can easily obtain the sought signal $z(\xi,\eta)$, albeit shifted by
$(\xi_{0}, \eta_{0})$, provided that the spatial extent of $z(\xi,\eta)$ is
limited by the box $\xi\in[-L_{\xi}, L_{\xi}]$, $\eta\in[-L_{\eta},L_{\eta}]$.
The spatial extent of the autocorrelation $z(\xi,\eta) \otimes
z^{*}(-\xi,-\eta)$ is twice as large, that is, limited by the box
$\xi\in[-2L_{\xi}, 2L_{\xi}]$, $\eta\in[-2L_{\eta},2L_{\eta}]$. Hence, to avoid
overlapping we must have $\xi_{0} > 3L_{\xi}$, or $\eta_{0}>3L_{\eta}$.
Thus, in theory, one can generate an ideal delta function in the
$(\xi,\eta)$ plane located at a sufficient distance from the support area
of $z(\xi,\eta)$. In the Fourier domain, this delta function
corresponds to a plane wave arriving at a certain angle at the
plane of measurements. If such a construction is possible, then a
simple inverse transform of the intensity obtained in the Fourier
plane is sufficient to obtain the sought signal $z(\xi,\eta)$. However,
as mentioned earlier, this approach has some drawbacks. First, it
is impossible to create an ideal delta function. Any physical
realization will necessarily have a finite spatial extent, and
this will result in a ``blurred'' reconstructed image. Note that
the term ``blurring'' describes well the resulting image in the
case where $z(\xi,\eta)$ is real-valued or has constant phase. In the
more general case, where the phase of $z(\xi,\eta)$ varies at
non-negligible speed, the result appears more distorted (see
Figure~\ref{fig:holography-reconstruction-intensity}).
The second
drawback is the sensitivity of this method to errors in $\hat{R}$.
In Section~\ref{sec:phase-holo-results} we demonstrate how the quality of
reconstruction depends on the error in $\hat{R}$ (see
Figures~\ref{fig:objectdomain-error}, 
\ref{fig:objectdomain-error-corrected}, and
\ref{fig:visual-results-phase-error}). Our method, on the other
hand shows very little sensitivity to the reference beam shape.
Moreover, its modification described in the next section allows
the reference beam to contain severe errors without deteriorating
significantly the quality of reconstruction.

\section{Reconstruction method for imprecise reference beam}
\label{sec:reconstr-meth-impr} Here we consider the situation
where the reference beam is not known precisely, that is, we
assume that the phase of $\hat{R}$ contains some unknown error. It
is easy to verify that if the reference beam phase
$\psi(\xi',\eta')$ has error $\epsilon(\xi',\eta')$ then the sought
phase $\phi(\xi',\eta')$ becomes
\begin{equation}
  \label{eq:phase-holo-15}
  \phi(\xi',\eta') = \psi(\xi',\eta') + \epsilon(\xi',\eta') \pm
  \alpha(\xi',\eta')\ ,
\end{equation}
in a manner similar to Equation~(\ref{eq:phase-holo-9}). We do not consider
errors in the magnitude $|\hat{R}|$ for several reasons. Many
aberrations manifest themselves through phase
distortion~\shortcite{goodman05introduction}. Also, the magnitude of
$\hat{R}$ can
be measured. Moreover, looking at the above equation, it is
obvious that any error in $\psi$ can be viewed as an error in
$\alpha$. That is, the situation would be the same if the reference
beam phase $\psi$ were known precisely, while the difference
$\alpha$ would contain some errors. This observation is relevant
because errors in the phase $\alpha$ can arise from many different
sources, including imperfect measurements and errors in the
reference beam magnitude.

The true error $\epsilon(\xi',\eta')$ is, of course, unknown. Hence,
we assume just an upper bound (assumed known) on the absolute
phase error:
\begin{equation}
  \label{eq:phase-holo-17}
  \psi-\epsilon-\alpha
  \leq\angle(\hat{z})\leq\psi+\epsilon+\alpha\ ,
\end{equation}
as in Equation~(\ref{eq:phase-holo-11}). This time, however, the phase
uncertainty interval may be larger than $\pi$ radians which makes our
method inapplicable. On the other hand, limiting the phase uncertainty
interval by $\pi$ radians will prevent us from reconstructing the
precise image, because the true phase may lie outside this interval. A
possible solution is to measure the intensity of the reference beam
and then to reconstruct its phase using the method presented in
Chapter~\ref{cha:appr-four-phase-explanation}, because this problem
itself can be seen as a phase retrieval with approximately known
phase. However, taking another measurement may be undesirable, and
therefore we developed the following reconstruction method:
\begin{description}
\item[Step 1:] Set the phase uncertainty interval as defined by
  Equation~(\ref{eq:phase-holo-11}) (as if there were no errors in the reference
  beam phase).

\item[Step 2:] Solve the resulting minimization problem, obtaining a
solution $x(\xi,\eta)$).

\item[Step 3:] If not converged, set the phase uncertainty interval to
  $[\angle(\hat{x}) - \pi/2, \angle(\hat{x}) + \pi/2]$. Clip it, if
  applicable, to the limits defined by Equation~(\ref{eq:phase-holo-17}) and go to
  Step 2.
\end{description}

In this algorithm we perform a number of outer iterations, each
time updating the phase uncertainty interval. This approach leads
to a successful reconstruction method even in cases where the
reference beam contains severe errors. The results are much better
than those of non-iterative holographic reconstruction (see
Figures~\ref{fig:objectdomain-error},
and~\ref{fig:objectdomain-error-corrected}).  This improvement is
achieved by decoupling the reconstruction problem (which becomes
the pure phase retrieval with approximately known phase) and the
erroneous interferometric measurements.

\section{Experimental results}
\label{sec:phase-holo-results} 
The method was tested on a variety of images with similar
results. Here we present numerical experiments conducted on one
natural image so as to allow easy perception of the reconstruction
quality under various conditions. The image intensity (squared
magnitude) is shown in Figure~\ref{fig:experimenal-image}.
\begin{figure}[H]
  \centering
  \includegraphics{lena_128}
  \caption[Test image]{Original image (intensity).}
  \label{fig:experimenal-image}
\end{figure}


The technical
details of the image are as follows: the size is $128\times 128$
pixels, and pixel values (amplitude) vary from $0.2915$ to $0.9634$ on
a scale of $0$ to $1$.
with mean value of $0.6835$. These parameters will become important
later, when we will consider the reference beam design, and when we will
assess the reconstruction quality. Since the original image is a
photograph, it does not have any phase information. Hence, we
generated three different phase distributions to account for the
assortment of possible real-world problems where our method can be
applied. The first distribution assumes that the image is non-negative
real-valued, that is, the phase is zero everywhere. The second
distribution is designed to mimic a relatively smooth phase. To this
end, the phase is set to be proportional to the image values (scaled
to the interval $[-\pi,\pi]$). Finally, in the third distribution the
phase is chosen at random, uniformly spread over the interval
$[-\pi,\pi]$. This distribution is designed to show the behavior of
our reconstruction method in cases where the true phase varies
rapidly. We also consider three possible reference beams, again, to
demonstrate the robustness of our method. The first reference beam is
an ideal delta-function in the $(\xi,\eta)$ plane, located at the
coordinate $(256, 256)$ so that the holographic condition is
satisfied. With this reference beam exact reconstruction is obtained
as long as the sampling in the Fourier domain is sufficiently dense
($512\times512$ pixels, or more). We do not present the visual results
of reconstruction for this reference beam as both methods produce
images that are indistinguishable from the true image.  The speed of
convergence of our method is shown in
Figure~\ref{fig:phase-holo-we-reconstruction-speed}.  Later, we show also how the
reconstruction quality of both methods is affected by Fourier phase
errors in the reference beam. Before this, we demonstrate the effect
of departure from the ideal delta function: the second reference beam
is a small square of size $3\times3$ pixels, located at the
coordinates $(256:258,256:258)$.  In this setup the reconstruction
quality of the holographic method is degraded, as evident from
Figure~\ref{fig:holography-reconstruction-intensity}.
\begin{figure}[H]
  \centering
  \subfloat[]{
    \label{fig:holography-real-square}
    \includegraphics{holog/lena_real_2011-08-01-18-16-26-hol-square-0}
  }
  \quad{}
  \subfloat[]{
    \label{fig:holography-complexSmooth-square}
    \includegraphics{holog/lena_cmplxSmooth_2011-08-01-22-18-58-hol-square-0}
  }
  \quad{}
  \subfloat[]{
    \label{fig:holography-complexRandom-square}
    \includegraphics{holog/lena_cmplxRand_2011-08-02-06-37-24-hol-square-0}
  }
  \caption[Image reconstructed by the holographic
    technique using a small square as the reference beam]{Image (intensity) reconstructed by the holographic
    technique using a small square as the reference beam: (a) the
    image is real-valued, (b) image phase varies slowly, (c)
    image phase is random (varies rapidly).}
  \label{fig:holography-reconstruction-intensity}
\end{figure}


It is also evident
that faster variations in the object phase result in greater
deterioration in the reconstruction, in agreement with our
expectations. Our method, on the other hand, is insensitive to the
reference beam form. In Figure~\ref{fig:we-reconstruction-intensity} we
demonstrate our reconstruction results for the aforementioned small
square as the reference beam (the first row), and for another
reference beam that was formed in the Fourier plane by combining unit
magnitude with random phase (in the interval $[-\pi, \pi]$). This
beam, of course, is not suitable for holography, as its extent in the
object plane occupies the whole space.
\begin{figure}[H]
  \centering
  \subfloat[]{
    \label{fig:we-real-square}
    \includegraphics{holog/lena_real_2011-08-01-18-16-26-we-square-0}
  }
  \quad{}
  \subfloat[]{
    \label{fig:we-complexSmooth-square}
    \includegraphics{holog/lena_cmplxSmooth_2011-08-01-22-18-58-we-square-0}
  }
  \quad{}
  \subfloat[]{
    \label{fig:we-complexRandom-square}
    \includegraphics{holog/lena_cmplxRand_2011-08-02-06-37-24-we-square-0}
  }\\
   \subfloat[]{
    \label{fig:we-real-random}
    \includegraphics{holog/lena_real_2011-08-01-18-16-26-we-random-0}
  }
  \quad{}
  \subfloat[]{
    \label{fig:we-complexSmooth-random}
    \includegraphics{holog/lena_cmplxSmooth_2011-08-01-22-18-58-we-random-0}
  }
  \quad{}
  \subfloat[]{
    \label{fig:we-complexRandom-random}
    \includegraphics{holog/lena_cmplxRand_2011-08-02-06-37-24-we-random-0}
  }
  \caption[Image reconstructed by our method]{Image reconstructed (intensity) by our method:
    (a), (b), and (c) --- reference beam is a small square and object
    phase is zero (a), smooth (b), random (c).
    (d), (e), and (f) --- reference beam is random and object phase
    is zero (d), smooth (e), random (f).}
  \label{fig:we-reconstruction-intensity}
\end{figure}


Reconstruction is very fast
and, in fact, is almost independent of the sought image and reference
beam type. Figure~\ref{fig:phase-holo-we-reconstruction-speed} demonstrates that
less than 20 iterations are required to solve the minimization problem
as defined by Equation~(\ref{eq:phase-holo-12}).
\begin{figure}[H]
  \centering
  \subfloat[]{
    \label{fig:we-speed-real}
    \includegraphics[height=0.25\textheight]{holog/lena_real_2011-08-01-18-16-26-speed}
  }
  \subfloat[]{
    \label{fig:we-speed-complexSmooth}
    \includegraphics[height=0.25\textheight]{holog/lena_cmplxSmooth_2011-08-01-22-18-58-speed}
  }
  \qquad{}
  \subfloat[]{
    \label{fig:we-speed-complexRandom}
    \includegraphics[height=0.25\textheight]{holog/lena_cmplxRand_2011-08-02-06-37-24-speed}
  }
  \caption[Reconstruction speed of our method]{Reconstruction speed of our method: (a) real valued image,
    (b) image phase is smooth, (c) image phase is random.}
  \label{fig:phase-holo-we-reconstruction-speed}
\end{figure}




In all these examples we assume perfect knowledge of the reference
beam. Next we consider the situation where the actual reference
beam does not match the expected signal in the Fourier plane.
Following our discussion in Section~\ref{sec:reconstr-meth-impr},
we evaluate how the reconstruction quality of the holographic
approach and our method are affected by errors in the reference
beam Fourier phase. We use again the three aforementioned models
of the sought image (real-valued, smooth phase, and random phase)
and the three reference beams (delta function, small square, and
random). From Figure~\ref{fig:we-successrate} it is evident that in
all these cases we were able to solve the minimization problem to
sufficient accuracy as long as the phase error was below 25\%.
That is, our method can tolerate reference
beam Fourier phase errors of up to $\pi/2$ radians. The sharp
discontinuity that happens at this value has a
simple explanation: phase error greater than $\pi/2$ radians can
result in the phase uncertainty interval greater than $2\pi$ radians
(see Equation~(\ref{eq:phase-holo-17})).  Hence, all phase information is
lost.

\begin{figure}[H]
  \centering
  \subfloat[]{
    \label{fig:we-success-real}
    \includegraphics[height=0.25\textheight]{holog/lena_real_2011-08-01-18-16-26-success}
  }
  \subfloat[]{
    \label{fig:we-success-complexSmooth}
    \includegraphics[height=0.25\textheight]{holog/lena_cmplxSmooth_2011-08-01-22-18-58-success}
  }
  \qquad{}
  \subfloat[]{
    \label{fig:we-success-complexRandom}
    \includegraphics[height=0.25\textheight]{holog/lena_cmplxRand_2011-08-02-06-37-24-success}
  }
  \caption[Fourier domain error vs. phase
    error in the reference beam]{Fourier domain ($\||\hat{x}| - |\hat{z}|\|^{2}$) error vs. phase
    error in the reference beam: (a) real valued image,
    (b) image phase is smooth, (c) image phase is random. }
  \label{fig:we-successrate}
\end{figure}

A comparison
with the holographic reconstruction is given in
Figure~\ref{fig:objectdomain-error} where the error norm in the
object domain is depicted. Note that the objective function values
are about $10^{-10}$, hence, one would expect the object domain
error norm to be of order $10^{-5}$ (the difference stems from the
fact that the objective function uses \textit{squared} norm). This
is not so in the case of complex-valued images and the random
reference beam. This effect is due to the relaxation we perform in
the Fourier phase, as discussed in
Section~\ref{sec:basic-reconstr-algor}. It does not change the
relative phase distribution, but all phases can get a constant
addition. This can be corrected by solving the one dimensional
minimization defined by Equation~\eqref{eq:phase-holo-13}.
Figure~\ref{fig:objectdomain-error-corrected} depicts the corrected
values yielded by this process. Visual results comparing the
holographic reconstruction with our method are provided in
Figure~\ref{fig:visual-results-phase-error}. 




\begin{figure}[H]
  \centering
  \subfloat[]{
    \label{fig:we-quality-real}
    \includegraphics[height=0.25\textheight]{holog/lena_real_2011-08-01-18-16-26-quality}
  }
  \subfloat[]{
    \label{fig:we-quality-complexSmooth}
    \includegraphics[height=0.25\textheight]{holog/lena_cmplxSmooth_2011-08-01-22-18-58-quality}
  }
  \qquad{}
  \subfloat[]{
    \label{fig:we-quality-complexRandom}
    \includegraphics[height=0.25\textheight]{holog/lena_cmplxRand_2011-08-02-06-37-24-quality}
  }
  \caption[Object domain error vs. phase error in the
    reference beam]{Object domain error ($\|x-z\|$) vs. phase error in the
    reference beam: (a) real valued image,
    (b) image phase is smooth, (c) image phase is random.}
  \label{fig:objectdomain-error}
\end{figure}


\begin{figure}[H]
  \centering
  \subfloat[]{
    \label{fig:we-qualitycorrected-real}
    \includegraphics[height=0.25\textheight]{holog/lena_real_2011-08-01-18-16-26-qualitycorrected}
  }
  \subfloat[]{
    \label{fig:we-qualitycorrected-complexSmooth}
    \includegraphics[height=0.25\textheight]{holog/lena_cmplxSmooth_2011-08-01-22-18-58-qualitycorrected}
  }
  \qquad{}
  \subfloat[]{
    \label{fig:we-qualitycorrected-complexRandom}
    \includegraphics[height=0.25\textheight]{holog/lena_cmplxRand_2011-08-02-06-37-24-qualitycorrected}
  }
  \caption[Corrected object domain error  vs. phase error in the
    reference beam]{Corrected object domain error ($\|x-z\|$) vs. phase error in the
    reference beam: (a) real valued image,
    (b) image phase is smooth, (c) image phase is random.}
  \label{fig:objectdomain-error-corrected}
\end{figure}


\begin{figure}[H]
  \centering
  \subfloat[]{
    \includegraphics{holog/lena_cmplxRand_2011-08-02-06-37-24-hol-delta-0.01}
  }
  \quad{}
  \subfloat[]{
    \includegraphics{holog/lena_cmplxRand_2011-08-02-06-37-24-hol-delta-0.1}
  }
  \quad{}
  \subfloat[]{
    \includegraphics{holog/lena_cmplxRand_2011-08-02-06-37-24-hol-delta-0.2}
  }\\
   \subfloat[]{
    \includegraphics{holog/lena_cmplxRand_2011-08-02-06-37-24-we-delta-0.01}
  }
  \quad{}
  \subfloat[]{
    \includegraphics{holog/lena_cmplxRand_2011-08-02-06-37-24-we-delta-0.1}
  }
  \quad{}
  \subfloat[]{
    \includegraphics{holog/lena_cmplxRand_2011-08-02-06-37-24-we-delta-0.2}
  }
  \caption[Image reconstructed by the holographic method
    and our method]{Image (intensity) reconstructed by the holographic method
    (upper row) and our method (lower row). Object phase is random,
    and the reference beam is a delta function with Fourier phase
    errors: 
    (a), (b), and (c) --- phase error is 1\%, 10\%, and 20\% respectively
    (d), (e), and (f) --- phase error is 1\%, 10\%, and 20\% respectively.}
  \label{fig:visual-results-phase-error}
\end{figure}



As the results show, our method demonstrates a substantial
advantage over ordinary holographic reconstruction. It is
remarkable that even when the minimization is not particularly
successful (in cases of very large phase errors) our
reconstruction is still closer to the true signal than the
holographic method. This success is due to our approach of
decoupling the phase retrieval from the interferometric
measurements. As mentioned earlier, we deliberately avoid strong
dependence on the reference beam. The interference pattern is only
used to \textit{estimate} the Fourier phase bounds. The results
indicate that this approach is well justified.


\section{Concluding remarks}
\label{sec:phase-holo-conclusions}
In this chapter we presented a new reconstruction method from two
intensities measured in the Fourier plane. One is the magnitude of the
sought signal's Fourier transform, and the other is the intensity
resulting from the superimposition of the original image and an
approximately known reference beam. While the method was originally
developed for the phase retrieval problem, it can be useful in digital
holography, because it poses less stringent requirements on the
reference beam. The method is designed specifically to allow severe
errors in the reference beam without compromising the quality of
reconstruction. Numerical simulations justify our approach, exhibiting
reconstruction that is superior to that of holographic techniques.

It is also important to note that this method can, potentially,  be used in the
classical phase retrieval problem without knowing the Fourier phase
withing the error of $\pi/2$ radians as was done in
Chapters~\ref{cha:appr-four-phase-first},
and~\ref{cha:appr-four-phase-explanation} as it uses some sort of
bootstrapping  technique that allows much rougher phase estimation.  













%%% Local Variables: 
%%% mode: latex
%%% TeX-master: "../thesis"
%%% End: 



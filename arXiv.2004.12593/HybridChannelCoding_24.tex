\documentclass[journal]{IEEEtran}
 
\usepackage{cite}
\usepackage[dvips]{graphicx}
\usepackage[cmex10]{amsmath}
\interdisplaylinepenalty=2500
\usepackage{color}
\usepackage{bm}
\usepackage{theorem} 
\usepackage{amscd}  
\usepackage{amssymb}
\usepackage {subfigmat}  
\usepackage{multirow}      
  
 
%THEOREMS

{\theorembodyfont{\normalfont\it}
\theoremheaderfont{\normalfont\bf}
\newtheorem{thm}{ Theorem}
\newtheorem{dfn}[thm]{ Definition}
\newtheorem{lmm}[thm]{ Lemma}
\newtheorem{exa}[thm]{ Example}
\newtheorem{crl}[thm]{ Corollary}
\newtheorem{asm}[thm]{ Assumption}
\newtheorem{prp}[thm]{ Proposition}
\newtheorem{cjt}[thm]{ Conjecture}}

{\theorembodyfont{\normalfont}
\theoremheaderfont{\normalfont\it}
\newtheorem{prf}{ Proof:}}

{\theorembodyfont{\normalfont}
\theoremheaderfont{\normalfont\it}
\newtheorem{skprf}{ Sketch of the proof.}}

{\theorembodyfont{\normalfont}
\theoremheaderfont{\normalfont\it}
\newtheorem{outprf}{ Outline of the proof.}}

{\theorembodyfont{\normalfont}
\theoremheaderfont{\normalfont\it}
\newtheorem{prftyp}{ Proof of Theorem \ref{thm:typicalsubspace}.}}

{\theorembodyfont{\normalfont}
\theoremheaderfont{\normalfont\it}
\newtheorem{prfconv}{ Proof of Theorem \ref{thm:misoptach}.}}

{\theorembodyfont{\normalfont}
\theoremheaderfont{\normalfont\it}
\newtheorem{prfdirectOSD}{ Proof of \rThm{OSD}.}}



{\theorembodyfont{\normalfont}
\theoremheaderfont{\normalfont\it}
\newtheorem{rmk}{ Remark.}}





%COMMANDS

\renewcommand{\theprf}{}
\renewcommand{\theprftyp}{}
\renewcommand{\theoutprf}{}
\renewcommand{\theprfconv}{}
\renewcommand{\thermk}{}
  

\newcommand{\bra}[1]{\mbox{$\langle#1|$}}
\newcommand{\Bra}[1]{\mbox{$\left\langle#1\right|$}}
\newcommand{\ket}[1]{\mbox{$|#1\rangle$}}
\newcommand{\Ket}[1]{\mbox{$\left|#1\right\rangle$}}
\newcommand{\inpro}[2]{\mbox{$\left\langle#1|#2\right\rangle$}}
\newcommand{\outpro}[2]{\mbox{$\ket{#1}\!\bra{#2}$}}
\newcommand{\Outpro}[2]{\mbox{$\Ket{#1}\!\Bra{#2}$}}
\newcommand{\proj}[1]{\mbox{$\ket{#1}\!\bra{#1}$}}
\newcommand{\Proj}[1]{\mbox{$\Ket{#1}\!\Bra{#1}$}}
\newcommand{\norm}[1]{\|#1\|}
\newcommand{\Norm}[1]{\left\|#1\right\|}
\newcommand{\kakko}[1]{\mbox{$\left(#1\right)$}}
\newcommand{\proof}{\noindent{\bf Proof.}  }
\newcommand{\ave}[1]{\mbox{$\left\langle#1\right\rangle$}}
\newcommand{\eq}[1]{Eq.(\ref{eq:#1})}
\newcommand{\sect}[1]{Sec.\,\ref{sec:#1}}
\newcommand{\ep}[1]{\noindent{\underline{\bf EP{#1}:}}}
\newcommand{\alg}[1]{\begin{align}#1\end{align}}
\newcommand{\rarrow}{\rightarrow}
\newcommand{\larrow}{\leftarrow}
\newcommand{\nn}{\nonumber}
\newcommand{\ca}[1]{{\mathcal #1}}
\newcommand{\mbb}[1]{{\mathbb #1}}
\newcommand{\mfk}[1]{{\mathfrak #1}}
\newcommand{\Tr}[2]{{\rm Tr}_{#1}\left[#2\right]}
\newcommand{\bthm}[1]{\begin{thm}\label{thm:#1}}
\newcommand{\ethm}{\end{thm}}
\newcommand{\rthm}[1]{\ref{thm:#1}}
\newcommand{\rThm}[1]{Theorem \ref{thm:#1}}
\newcommand{\blmm}[1]{\begin{lmm}\label{lmm:#1}}
\newcommand{\elmm}{\end{lmm}}
\newcommand{\rLmm}[1]{Lemma \ref{lmm:#1}}
\newcommand{\rlmm}[1]{\ref{lmm:#1}}
\newcommand{\bdfn}[1]{\begin{dfn}\label{dfn:#1}}
\newcommand{\edfn}{\end{dfn}}
\newcommand{\rdfn}[1]{\ref{dfn:#1}}
\newcommand{\rDfn}[1]{Definition \ref{dfn:#1}}
\newcommand{\basm}[1]{\begin{asm}\label{asm:#1}}
\newcommand{\easm}{\end{asm}}
\newcommand{\bprp}[1]{\begin{prp}\label{prp:#1}}
\newcommand{\eprp}{\end{prp}}
\newcommand{\rprp}[1]{\ref{prp:#1}}
\newcommand{\rPrp}[1]{Proposition \ref{prp:#1}}
\newcommand{\bcrl}[1]{\begin{crl}\label{crl:#1}}
\newcommand{\ecrl}{\end{crl}}
\newcommand{\rcrl}[1]{\ref{crl:#1}}
\newcommand{\rCrl}[1]{Corollary \ref{crl:#1}}
\newcommand{\bcjt}[1]{\begin{cjt}\label{cjt:#1}}
\newcommand{\ecjt}{\end{cjt}}
\newcommand{\rcjt}[1]{\ref{cjt:#1}}
\newcommand{\rCjt}[1]{Conjecture \ref{cjt:#1}}
\newcommand{\bprf}{\begin{prf}}
\newcommand{\eprf}{\end{prf}}
\newcommand{\brmk}{\begin{rmk}}
\newcommand{\ermk}{\end{rmk}}
\newcommand{\laeq}[1]{\label{eq:#1}}
\newcommand{\req}[1]{(\ref{eq:#1})}
\newcommand{\rineq}[1]{Inequality (\ref{eq:#1})}
\newcommand{\QED}{\hfill$\blacksquare$}
\newcommand{\lsec}[1]{\label{sec:#1}}
\newcommand{\rsec}[1]{\ref{sec:#1}}
\newcommand{\rSec}[1]{Section \ref{sec:#1}}
\newcommand{\lapp}[1]{\label{app:#1}}
\newcommand{\rapp}[1]{\ref{app:#1}}
\newcommand{\rApp}[1]{Appendix \ref{app:#1}}
\newcommand{\aR}{a_{\scalebox{0.45}{$R$}}}
\newcommand{\aL}{a_{\scalebox{0.45}{$L$}}}
\newcommand{\az}{a_{\scalebox{0.45}{$0$}}}
\newcommand{\bR}{b_{\scalebox{0.45}{$R$}}}
\newcommand{\bL}{b_{\scalebox{0.45}{$L$}}}
\newcommand{\bz}{b_{\scalebox{0.45}{$0$}}}
\newcommand{\cR}{c_{\scalebox{0.45}{$R$}}}
\newcommand{\cL}{c_{\scalebox{0.45}{$L$}}}
\newcommand{\cz}{c_{\scalebox{0.45}{$0$}}}
\newcommand{\sR}{s_{\scalebox{0.45}{$R$}}}
\newcommand{\sL}{s_{\scalebox{0.45}{$L$}}}
\newcommand{\sz}{s_{\scalebox{0.45}{$0$}}}
\newcommand{\baR}{{\bar a}_{\scalebox{0.45}{$R$}}}
\newcommand{\baL}{{\bar a}_{\scalebox{0.45}{$L$}}}
\newcommand{\baz}{{\bar a}_{\scalebox{0.45}{$0$}}}
\newcommand{\bbR}{{\bar b}_{\scalebox{0.45}{$R$}}}
\newcommand{\bbL}{{\bar b}_{\scalebox{0.45}{$L$}}}
\newcommand{\bbz}{{\bar b}_{\scalebox{0.45}{$0$}}}
\newcommand{\bsR}{{\bar s}_{\scalebox{0.45}{$R$}}}
\newcommand{\bsL}{{\bar s}_{\scalebox{0.45}{$L$}}}
\newcommand{\bsz}{{\bar s}_{\scalebox{0.45}{$0$}}}
\newcommand{\haR}{{\hat a}_{\scalebox{0.45}{$R$}}}
\newcommand{\haL}{{\hat a}_{\scalebox{0.45}{$L$}}}
\newcommand{\haz}{{\hat a}_{\scalebox{0.45}{$0$}}}
\newcommand{\hbR}{{\hat b}_{\scalebox{0.45}{$R$}}}
\newcommand{\hbL}{{\hat b}_{\scalebox{0.45}{$L$}}}
\newcommand{\hbz}{{\hat b}_{\scalebox{0.45}{$0$}}}
\newcommand{\conv}{{\rm conv}}
\newcommand{\red}{\color{red}}
\newcommand{\blue}{\color{blue}}
\newcommand{\white}{\color{white}}
\newcommand{\notations}{\noindent{\it Notations.} }
\newcommand{\bitem}{\begin{itemize}}
\newcommand{\entem}{\end{itemize}}
\newcommand{\benum}{\begin{enumerate}}
\newcommand{\ennum}{\end{enumerate}}
\newcommand{\bb}{\mathbb}
\newcommand{\otm}{\otimes}
\newcommand{\opl}{\oplus}
\newcommand{\vro}{\varrho}
\newcommand{\rFig}[1]{Figure \ref{fig:#1}}
\newcommand{\rfig}[1]{\ref{fig:#1}}
\newcommand{\SH}{\noindent{\red [START HERE]} }
\newcommand{\TBA}{{\red @@@} }
\newcommand{\TBC}{{\red [TO BE CONFIRMED]} }
\newcommand{\argmax}{\mathop{\rm arg~max}\limits}
\newcommand{\argmin}{\mathop{\rm arg~min}\limits}
\newcommand{\prlsection}[1]{{\it{#1}}.---}
%\newcommand{\dim}{{\rm dim}}


 
% correct bad hyphenation here
%\hyphenation{op-tical net-works semi-conduc-tor}

\newcommand{\beq}{\begin{eqnarray}}
\newcommand{\eeq}{\end{eqnarray}}

\newcommand{\green}{\color{green}} %
\newcommand{\magenta}{\color{magenta}} %
\newcommand{\black}{\color{black}} %
\newcommand{\yellow}{\color{yellow}}


\newcommand{\lv}{\left \vert}
\newcommand{\rv}{\right \vert}
\newcommand{\la}{\left \langle}
\newcommand{\ra}{\right \rangle}
\newcommand{\average}[1]{\la #1 \ra} 
\newcommand{\expectation}[1]{\mathbf{E}[#1]}
\newcommand{\Paverage}[1]{\la #1 \ra_{\mathrm{phase}}}
\newcommand{\Raverage}[1]{\la #1 \ra_{\mathrm{rand}}}
\newcommand{\averageInf}[1]{\la #1 \ra_{T;\infty}}
\newcommand{\Rsigma}{\zeta_{\mathrm{rand}}}
\newcommand{\tr}{\mathrm{Tr}}



%\onecolumn

\begin{document}


\title{Randomized Partial Decoupling Unifies\\ One-Shot Quantum Channel Capacities}

\author{Eyuri Wakakuwa, and Yoshifumi Nakata \\ (Report number: YITP-20-61)
 

\thanks{E. Wakakuwa is with the Department of Communication Engineering and Informatics, Graduate School of Informatics and Engineering, The University of Electro-Communications, Tokyo 182-8585, Japan (email: e.wakakuwa@gmail.com).}
\thanks{Yoshifumi Nakata is with Yukawa Institute for Theoretical Physics, Kyoto university, Kitashirakawa Oiwakecho, Sakyo-ku, Kyoto, 606-8502, Japan, and also with JST, PRESTO, 4-1-8 Honcho, Kawaguchi, Saitama, 332-0012, Japan (email: yoshifumi.nakata@yukawa.kyoto-u.ac.jp).}
}


%\setlength{\baselineskip}{5.3mm}

\maketitle


\begin{abstract}
We analyze a task in which classical and quantum messages are simultaneously communicated via a noisy quantum channel, assisted with a limited amount of shared entanglement. 
We derive the direct and converse bounds for the one-shot capacity region.
The bounds are represented in terms of the smooth conditional entropies and the error tolerance, and coincide in the asymptotic limit of infinitely many uses of the channel.
The direct and converse bounds for various communication tasks are obtained as corollaries, both for one-shot and asymptotic scenarios.
The proof is based on the {\it randomized partial decoupling theorem}, which is a generalization of the decoupling theorem.
Thereby we provide a unified decoupling approach to the one-shot quantum channel coding, by fully incorporating classical communication, quantum communication and shared entanglement.
\end{abstract}


\begin{IEEEkeywords}
Quantum Channel Capacity, One-Shot, Decoupling
\end{IEEEkeywords}


\section{Introduction}

One of the major goals of quantum communication theory is to investigate the ultimate capacity of a noisy quantum channel for transmitting classical and quantum information,
particularly in an asymptotic limit of many uses of a discrete memoryless channel (see e.g. \cite{wildetext,nielsentext}).
Classical and quantum capacities of a noisy quantum channel were obtained in \cite{holevo98,schumacher97,bennett1999entanglement} and \cite{lloyd1997capacity,devetak2005private,devetak2004family}, respectively, both for entanglement non-assisted and assisted scenarios.
The capacity of a quantum channel for simultaneously transmitting classical and quantum messages was also obtained in \cite{devetak2005capacity,hsieh2010entanglement}.
Refs.~\cite{devetak2004relating,devetak2005distillation} provided a unified approach to another goal of quantum communication theory, namely, evaluation of the maximum amount of pure entanglement or secrecy that can be extracted from a mixed quantum states. 
This result was subsequently developed into quantum state merging \cite{horo05,horo07} and the fully quantum Slepian-Wolf (FQSW) protocol \cite{ADHW2009}, in addition to state redistribution \cite{devetak2008exact,yard2009optimal}.
Remarkably, various coding theorems including quantum capacity theorems are obtained by reduction from FQSW \cite{ADHW2009}.
These results provided a unified picture for various quantum communication tasks, referred to as the protocol family \cite{ADHW2009,devetak2004family}.




The concept of decoupling plays a central role in the above analyses of quantum protocols.
Decoupling refers to the fact that we may destroy correlation between two quantum systems by applying an operation on one of the two subsystems. 
In various scenarios, the amount of communication required for accomplishing a quantum protocol is equal to the amount of ``randomness'' of the operation that decouples a particular quantum state (see e.g. \cite{horo07,ADHW2009}).
The decoupling approach simplifies many problems of our interest, particularly when combined with the fact that
any purification of a mixed quantum state is convertible to another reversibly.
This fact is known as Uhlmann's theorem \cite{uhlmann1976transition}, and enables us to prove the existence of a decoder for quantum communication without explicitly constructing it.





The decoupling approach to quantum protocols has been generalized to the one-shot scenario.
Ref.~\cite{DBWR2010} proved one of the most general formulations of decoupling, which is referred to as the {\it one-shot decoupling theorem}.
The decoupling theorem provides a necessary condition and a sufficient condition for an operation to decouple a quantum state with high precision, in terms of smooth min- and max- entropies of the operation and the state, respectively.
In the same way as in the asymptotic scenario, various coding theorems in the one-shot scenario are obtained by reduction from the decoupling theorem~\cite{fred10_2}.
Furthermore, due to the fully quantum asymptotic equipartition property \cite{tomamichel2009fully}, the results also lead to a reconstruction of the existing results in an asymptotic scenario.
A generalization and refinement of the decoupling theorem, called {\it catalytic decoupling}, was also proposed in \cite{majenz2017catalytic}.



There is, however, a limitation in the decoupling approach, in that it does not incorporate classical communication tasks. 
Classical communication over noisy quantum channels was addressed in Ref.~\cite{dupuis2014decoupling} using a decoupling-like approach based on the {\it dequantizing theorem}. However, it still does not fully incorporate a general scenario, in which classical and quantum messages are simultaneously transmitted possibly with the assistance of shared entanglement. For addressing this scenario by the existing decoupling methods, it is necessary to apply the decoupling theorem and the dequantizing theorem separately for the two sources (see also Refs.~\cite{salek2019one} for a different approach based on the convex splitting).
In this sense, the unified approach to communication over noisy quantum channel based on decoupling has not been fully completed.



\begin{table*}[t]
\renewcommand{\arraystretch}{1.5}
  \begin{center}
    \begin{tabular}{c|c|c|c|c|c} \hline
	 & \multicolumn{2}{c}{decoupling}                  &  \multicolumn{2}{|c|}{hypothesis testing} & other \\\cline{2-6}
            &   direct &  converse  &  direct &  converse  & direct/converse \\ \hline
    \multirow{2}{*}{$c$}   &    \multirow{2}{*}{Dupuis et al.~\cite{dupuis2014decoupling}} &         \multirow{2}{*}{-}                                  & \multicolumn{2}{c|}{ \begin{tabular}{c} Mosonyi et al.~\cite{mosonyi2009generalized}, Wang et al.~\cite{wang2012one}, \\[-1mm] Datta et al.~\cite{datta2013smooth}  \end{tabular} } &  \multirow{2}{*}{Renes et al.~\cite{renes2011noisy}} \\\cline{4-5}
    	& 	&	&	-	&  Matthews et al.~\cite{matthews2014finite}	&    \\\hline
   $q$ &   \multicolumn{2}{|c|}{Datta et al.~\cite{datta2011apex}, Buscemi et al.~\cite{buscemi2010quantum}}                         & - & -	&- \\\hline
   $c$, $q$ &  -  &                    -             & \multicolumn{2}{|c|}{Salek et al.~\cite{salek2019one}} & - \\ \hline
   \multirow{2}{*}{\begin{tabular}{c}\white{-} \\ \; $c$, $e^*$ \end{tabular} } & \multicolumn{2}{|c|}{}     &  \multicolumn{2}{|c|}{Qi et al.~\cite{qi2018applications}} &	\multirow{3}{*}{-}\\ \cline{4-5}
     & \multicolumn{2}{|c|}{ \begin{tabular}{c} Datta et al.~\cite{datta2012one}   \\  \white{-} \end{tabular}        }     &  \begin{tabular}{c} Datta et al.~\cite{datta2016second}, \\[-1mm] Anshu et al.~\cite{anshu2018building}  \end{tabular}  & \begin{tabular}{c} Matthews et al.~\cite{matthews2014finite}, \\[-1mm] Anshu et al.~\cite{anshu2019near}  \end{tabular} &	\\ \hline
   $q$, $e^*$ & \multicolumn{2}{|c|}{Datta et al.~\cite{datta2012one}         }        & - & -	&	-\\\hline
   $c$, $q$, $e$ & \multicolumn{2}{|c|}{[THIS PAPER]}       & - & -	&-	\\\hline
    $c$, $q$, $e^*$ & \multicolumn{2}{|c|}{[THIS PAPER]}       & - & -	&-	\\\hline
   \multirow{2}{*}{private} &  \multirow{2}{*}{-} & \multirow{2}{*}{-}  & \multicolumn{2}{|c|}{Salek et al. \cite{salek2018one}} 	&	\multirow{2}{*}{ \begin{tabular}{c} Renes et al.~\cite{renes2011noisy},   \\[-1mm]  Radhakrishnan et al.~\cite{radhakrishnan2017one}, \end{tabular}      } \\\cline{4-5}
       & & & Wilde ~\cite{wilde2017position} & -	& \\\hline
    \end{tabular}
  \end{center}
  \caption{Prior works on the one-shot capacities of a quantum channel. 
}
  \label{tb:values}
\end{table*}




In this paper, we propose a unified decoupling approach that allows us a simultaneous treatment of classical communication, quantum communication and shared entanglement.
Namely, we consider a task in which classical and quantum messages are transmitted via a noisy quantum channel with the assistance of a limited amount of shared entanglement. 
This task was analyzed in \cite{hsieh2010entanglement} for an asymptotic scenario, whereas we consider the one-shot scenario based on the decoupling approach. 
The main result is that we derive a one-shot rate region for this task, which is simply represented in terms of the smooth conditional entropy of the channel.
The direct and converse bounds for various communication tasks in one-shot scenario readily follow from the main result.
In the asymptotic limit of infinitely many uses of the channel, the direct and converse bounds coincide and recover the full achievable rate region obtained in \cite{hsieh2010entanglement}.
Thus, we substantially develop the unified decoupling approach for quantum channel coding.




The proof is based on the concept of {\it randomized partial decoupling} \cite{wakakuwa2019one}, which is a generalization of decoupling and may be of independent interest.
Here, 
we consider a scenario in which a bipartite quantum state on system $AR$ is subject to a unitary operation on $A$, followed by the action of a linear completely-positive (CP) map. Unlike the usual setting of decoupling, we assume that
the subsystem is decomposed into a direct-sum form, and 
the unitary is chosen at random from the set of unitaries that are block-diagonal under the decomposition. 
In \cite{wakakuwa2019one}, we proved the {\it randomized partial decoupling theorem}, which shows that the distance between the final state and the averaged one is bounded in terms of smooth conditional entropies of the initial state and the channel.
The existing results of the one-shot decoupling theorem \cite{DBWR2010} and the dequantization theorem \cite{dupuis2014decoupling}
are obtained from this result as corollaries (see Section III D in \cite{wakakuwa2019one} for the details).

Prior works on the one-shot capacity of quantum channels in various scenarios are summarized in Table \ref{tb:values}.
  Most of them are classified into the decoupling-based approaches and approaches based on quantum hypothesis testing. 
  The tasks analyzed therein are communication of classical message ($c$) or quantum message ($q$) or both $(c,q)$, with or without the assistance of shared entanglement that is limited ($e$) or unlimited ($e^*$).
    The result presented in this paper incorporates all these scenarios.
On the other hand, it is not directly applicable to the task of privately communicating classical messages, which was analyzed in some of the literatures.
We, however, expect that our result can also cover the private classical communication,
by adopting the idea that quantum communication and private classical communication over a noisy quantum channel are equivalent.
This idea was originally introduced in \cite{devetak2005private} for the asymptotic scenario, and was applied in \cite{salek2019one} to the one-shot scenario.



This paper is organized as follows. 
In \rSec{prelimi}, we introduce notations and definitions that will be used throughout this paper. 
In \rSec{RPD}, we summarize the statement of the randomized partial decoupling theorem.
In \rSec{mainresults}, we present formulations of the problem and the main results. 
The proofs of the main theorems are provided in Section \rsec{direct} and \rsec{converse}.


\section{Preliminaries}
\lsec{prelimi}

We summarize notations and definitions that will be used throughout this paper. 

\subsection{Notations}
We denote the set of linear operators and that of Hermitian operators on a Hilbert space $\ca{H}$ by $\ca{L}(\ca{H})$ and ${\rm Her}(\ca{H})$, respectively.
For positive semidefinite operators, density operators and sub-normalized density operators, we use the following notations, respectively:
\begin{align}
&
\ca{P}(\ca{H}) = \{\rho \in {\rm Her}(\ca{H}) : \rho \geq 0 \},
\\
&
\ca{S}_=(\ca{H}) = \{\rho \in \ca{P}(\ca{H}) : \tr [\rho]=1 \},
\\
&
\ca{S}_{\leq}(\ca{H}) = \{\rho \in \ca{P}(\ca{H}) : \tr [\rho] \leq 1 \}.
\end{align}
A Hilbert space associated with a quantum system $A$ is denoted by ${\mathcal H}^A$, and its dimension is denoted by $d_A$. A system composed of two subsystems $A$ and $B$ is denoted by $AB$. When $M$  and $N$ are linear operators on ${\mathcal H}^A$ and ${\mathcal H}^B$, respectively, we denote $M\otimes N$ as $M^A\otimes N^B$ for clarity. In the case of pure states, we often abbreviate $|\psi\rangle^A\otimes|\phi\rangle^B$ as $|\psi\rangle^A|\phi\rangle^B$. 
For $\rho^{AB} \in \ca{L}(\ca{H}^{AB})$, $\rho^{A}$ represents ${\rm Tr}_B[\rho^{AB}]$.  
We denote $|\psi\rangle\!\langle\psi|$ simply by $\psi$.
The maximally entangled state between $A$ and $A'$, where $\ca{H}^{A} \cong \ca{H}^{A'}$, is defined by
\alg{
\ket{\Phi}^{AA'}:=\frac{1}{d_A}\sum_{\alpha=1}^{d_A}\ket{\alpha}^A\ket{\alpha}^{A'}
}
with respect to a fixed orthonormal basis $\{\ket{\alpha}\}_{\alpha=1}^{d_A}$.

The identity operator is denoted by $I$. 
We denote $(M^A\otimes I^B)\ket{\psi}^{AB}$ as $M^A\ket{\psi}^{AB}$, and $(M^A\otimes I^B)\rho^{AB}(M^A\otimes I^B)^{\dagger}$ as $M^A\rho^{AB}M^{A\dagger}$. 
%As the identity operator acts trivially, $M^{AB}(\rho^A\otimes I^B)M^{\dagger AB}$ is often abbreviated to be $M^{AB}\rho^AM^{\dagger AB}$. 
When ${\mathcal T}$ is a supermap from $\ca{L}(\ca{H}^{A})$ to $\ca{L}(\ca{H}^{B})$, we denote it by $\ca{T}^{A \rightarrow B}$. When $A = B$, we use $\ca{T}^{A}$ for short.
We also denote $({\mathcal T}^{A \rightarrow B} \otimes{\rm id}^C)(\rho^{AC})$ by ${\mathcal T}^{A \rightarrow B} (\rho^{AC})$.  
When a supermap is given by a conjugation of a unitary $U^A$ or a linear operator $W^{A \rightarrow B}$, we especially denote it by its calligraphic font such as 
$
\ca{U}^{A}(X^A):= (U^{A }) X^A (U^{A })^{\dagger}
$
and
$
\ca{W}^{A \rightarrow B}(X^A):= (W^{A \rightarrow B}) X^A (W^{A \rightarrow B})^{\dagger}
$.
In that case, the adjoint map of $\ca{W}^{A \rightarrow B}$ is defined by $\ca{W}^{\dagger B \rightarrow A}(\cdot):=(W^{A \rightarrow B})^\dagger(\cdot)(W^{A \rightarrow B})$.

For any linear CP map $\ca{T}^{A\rightarrow B}$, there exists a finite dimensional quantum system $E$ and a linear operator $\Gamma_{\ca{T}}^{A\rightarrow BE}$ such that $\ca{T}^{A\rightarrow B}(\cdot)={\rm Tr}_E[\Gamma_{\ca{T}}(\cdot)\Gamma_{\ca{T}}^\dagger]$.
The operator $\Gamma_{\ca{T}}$ is called the Stinespring dilation of $\ca{T}^{A\rightarrow B}$ \cite{stinespring1955positive},
and the linear CP map defined by ${\rm Tr}_B[\Gamma_{\ca{T}}(\cdot)\Gamma_{\ca{T}}^\dagger]$ is called the {\it complementary map} of $\ca{T}^{A\rightarrow B}$.
With a slight abuse of notation, we denote the complementary map by $\ca{T}^{A\rightarrow E}$.



\subsection{Norms and Distances}

For a linear operator $X$, the trace norm is defined as $|\! | X |\! |_1 = \tr[ \sqrt{X^{\dagger}X}]$. 
The trace distance between two unnormalized states $\rho,\rho'\in\ca{P}(\ca{H})$ is defined by $\|\rho-\rho'\|_1$. 
For subnormalized states $\rho,\rho'\in\ca{S}_\leq(\ca{H})$, the generalized fidelity and the purified distance are defined by
\alg{
&
\bar{F}(\rho,\rho')
:=
\|\sqrt{\rho}\sqrt{\rho'}\|_1
+
\sqrt{(1-{\rm Tr}[\rho])(1-{\rm Tr}[\rho'])},
\\
&
P(\rho,\rho')
:=
\sqrt{1-\bar{F}(\rho,\rho')^2},
\laeq{dfnPD}
}
respectively \cite{tomamichel2010duality}.
The trace distance and the purified distance are related as
 \alg{
\frac{1}{2}\|\rho-\varsigma\|_1
\leq
P(\rho,\varsigma)
\leq
\sqrt{2\|\rho-\varsigma\|_1}
\laeq{relTDPD}
}
for any $\rho,\varsigma\in\ca{S}_\leq(\ca{H})$.
The epsilon ball of a subnormalized state $\rho\in\ca{S}_\leq(\ca{H})$ is defined by
\begin{align}
\ca{B}^\epsilon(\rho):=\{\rho'\in\ca{S}_\leq(\ca{H})|\:P(\rho,\rho')\leq\epsilon\}.
\label{eq:epsilon}
\end{align}



\subsection{One-shot entropies}


For any subnormalized state $\rho\in\ca{S}_\leq(\ca{H}^{AB})$ and normalized state $\varsigma\in\ca{S}_=(\ca{H}^{B})$, define
\alg{
&
H_{\rm min}(A|B)_{\rho|\varsigma} := \sup \{ \lambda \in \mathbb{R}| 2^{-\lambda} I^A \otimes \varsigma^B \geq \rho^{AB} \}, \label{eq:condmins} \\
&
H_{\rm max}(A|B)_{\rho|\varsigma} := \log{\|\sqrt{\rho^{AB}}\sqrt{I^A\otm\varsigma^B}\|_1^2}.
\label{eq:condmaxs}
}
The conditional min- and max- entropies (see e.g.~\cite{T16}) are defined by
\begin{align}
H_{\rm min}(A|B)_{\rho}& := \sup_{\varsigma^B \in \ca{S}_=(\ca{H}^B)}H_{\rm min}(A|B)_{\rho|\varsigma},\laeq{condmin} \\
H_{\rm max}(A|B)_{\rho}& := \sup_{\varsigma^B \in \ca{S}_=(\ca{H}^B)}H_{\rm max}(A|B)_{\rho|\varsigma},\laeq{condmax} 
\end{align}
respectively. We also use the smooth conditional min- and max-entropies defined by
\begin{align}
H_{\rm min}^\epsilon(A|B)_{\rho}& := \sup_{\hat{\rho}^{AB} \in \ca{B}^\epsilon(\rho)}H_{\rm min}(A|B)_{\hat\rho},\laeq{condminsm} \\
H_{\rm max}^\epsilon(A|B)_{\rho}& := \inf_{\hat{\rho}^{AB} \in \ca{B}^\epsilon(\rho)}H_{\rm max}(A|B)_{\hat\rho}\laeq{condmaxsm} 
\end{align}
for a smoothing parameter $\epsilon\geq0$.  
The properties of the smooth conditional entropies used in this paper are summarized in \rApp{PSE}.


\subsection{Choi-Jamiolkowski representation}

Let $\ca{T}^{A \rightarrow B}$ be a linear supermap from  $\ca{L}(\ca{H}^A)$ to $\ca{L}(\ca{H}^B)$, and let $\Phi^{AA'}$ be the maximally entangled state between $A$ and $A'$. A linear operator $\mfk{J}(\ca{T}^{A \rightarrow B})\in\ca{L}(\ca{H}^{AB})$ defined by $\mfk{J}(\ca{T}^{A \rightarrow B}) := \ca{T}^{A' \rightarrow B}(\Phi^{AA'})$ is called the {\it Choi-Jamio\l kowski representation of $\ca{T}$} \cite{J1972,C1975}. The representation is an isomorphism. The inverse map is given by, for an operator $X^{AB} \in \ca{L}(\ca{H}^{AB})$,
\alg{
\mfk{J}_{A}^{-1}(X^{AB}) (\varsigma^A) = d_A \tr_A \bigl[ (\varsigma^{A^T} \otimes I^B) X^{AB} \bigr],
\laeq{CJinverse}
}
where $A^T$ denotes the transposition of $A$ with respect to the Schmidt basis of $\Phi^{AA'}$.
When $\ca{T}$ is completely positive, then $\mfk{J}(\ca{T}^{A \rightarrow B})$ is an unnormalized state on $AB$ and is called the {\it Choi-Jamio\l kowski state of $\ca{T}$}.


\subsection{Haar measure}

For a unitary group of finite degree, there exists the unique left- and right- unitarily invariant probability measure, known as the Haar measure. We denote it by ${\sf H}$. More specifically, the Haar measure satisfies the property that, for any unitary $U$ and a set of unitaries $\mathcal{V}$,
\begin{equation}
{\sf H}(U \mathcal{V}) = {\sf H}(\mathcal{V}U) = {\sf H}(\mathcal{V}).
\end{equation}
When a unitary $U$ is chosen uniformly at random with respect to the Haar measure, we denote it by $U \sim {\sf H}$.



\section{Randomized Partial Decoupling} \lsec{RPD}

In this section, we briefly review a task that we call randomized partial decoupling \cite{wakakuwa2019one}. We present the direct and converse bounds for randomized partial decoupling, which is a generalization of the decoupling theorem in the version of \cite{DBWR2010}. For the details and proofs, see the paper by the same authors \cite{wakakuwa2019one}.


Randomized partial decoupling is a task in which a bipartite quantum state $\Psi^{AR}$ is transformed by a unitary operation on $A$ and then is subject to the action of a linear CP map $\ca{T}^{A\rightarrow E}$. 
We assume that the Hilbert space $\ca{H}^A$ is decomposed into a direct-sum form as $\ca{H}^A=\bigoplus_{j=1}^J  \ca{H}_j^A$, where $\ca{H}_j^{A}\:(j=1,\cdots,J)$ has the same dimension $r$. 
Let ${\mathcal H}^{A_c}$ be a $J$-dimensional Hilbert space with a fixed orthonormal basis $\{|j\rangle\}_{j=1}^J$, and ${\mathcal H}^{A_r}$ be an $r$-dimensional Hilbert space. 
The Hilbert space $\ca{H}^A$ is then isomorphic to a tensor product Hilbert space ${\ca H}^{A_c} \otimes{\ca H}^{A_r}$, i.e., $A\cong A_cA_r$.
In terms of this decomposition, any state $\Psi^{AR}$ is written as
\alg{
\Psi^{AR}=\sum_{j,k=1}^J\outpro{j}{k}^{A_c}\otm\Psi_{jk}^{A_rR},
}
where $\Psi_{jk}^{A_rR}:=\Pi_j^A\Psi^{AR}\Pi_k^A$ with $\Pi_j$ being the projection onto $\ca{H}_j^{A}$. In particular, using properly chosen orthonormal bases, a maximally entangled state $\ket{\Phi}^{AA'}$ is given by
\alg{
|\Phi\rangle^{AA'}=\left(\frac{1}{\sqrt{J}}\sum_{j=1}^J\ket{j}^{A_c}\ket{j}^{A_c'}\right)\otm\left(\frac{1}{\sqrt{r}}\sum_{\alpha=1}^r\ket{\alpha}^{A_r}\ket{\alpha}^{A_r'}\right).
\laeq{maxentdfn}
} 
where $A\cong A'$.


Consider a random unitary $U$ on the system $A$ in the form of
\begin{align}
U:=\sum_{j=1}^J\outpro{j}{j}^{A_c}  \otimes U_j^{A_r}.
\laeq{RUrpd}
\end{align}
Here, $U_j$ is independently chosen for each $j$ from the Haar measure ${\sf H}$ on the unitary group on $\ca{H}^{A_r}$.
For any state $\Psi^{AR}$, the averaged state after the action of this random unitary is given by
\begin{align}
\Psi_{\rm av}^{AR}
&:=\mbb{E}_{U_1, \dots, U_J \sim {\sf H}} [ 
U^A ( \Psi^{AR} ) U^{\dagger A}]
\laeq{qui}\\
&
=\sum_{j=1}^Jp_j \proj{j}^{A_c}\otimes \pi^{A_r}\otm\Psi_{j}^{R}. 
\end{align}
Here, $\pi^{A_r}$ is the maximally mixed state on $\ca{H}^{A_r}$, $\{p_j\}_{j=1}^J$ is a probability distribution defined by $p_j:={\rm Tr}[\bra{j}^{A_c}\Psi^{AR}\ket{j}^{A_c}]$, and $\Psi_{j}^{R}$ is a normalized state on $\ca{H}^R$ defined by $\Psi_{j}^{R}:=p_j^{-1}\Psi_{jj}^{R}$.
In the following, we denote $\mbb{E}_{U_1, \dots, U_J \sim {\sf H}}$ simply by $\mbb{E}_{U}$ when there is no ambiguity.
Consider also the permutation group $\mbb{P}$ on $\{ 1,\cdots,J \}$, and  define a unitary $G_s$ for each $s\in\mbb{P}$ by
\alg{
G_s:=\sum_{j=1}^J\outpro{s(j)}{j}^{A_c}  \otimes I^{A_r}.
\label{eq:RPrpd}
}
The permutation $s$ is chosen at random according to the uniform distribution on $\mbb{P}$.
Our concern is how close the final state $\ca{T}^{A \rightarrow E} \circ \ca{G}_s^A  \circ \ca{U}^A ( \Psi^{AR} )$ is, on average over all $U$, to the averaged final state $\ca{T}^{A \rightarrow E} \circ \ca{G}_s^A ( \Psi_{\rm av}^{AR} )$, for typical choices of the permutation $s$ (see Figure \ref{Fig:test} as well). 



\begin{figure}[t]
\begin{center}
\includegraphics[bb={0 0 780 180}, scale=0.3]{figure16.pdf}
\caption{
The procedure used in the randomized partial decoupling is depicted. For a given initial state $\Psi^{AR}$, a random unitary $U$ in the form of \req{RUrpd} and a permutation $G_s$ are first applied to $A$. Then, the system $A$ is mapped to another one $E$ by $\mathcal{T}$. 
}
\label{Fig:test}
\end{center}
\end{figure}


For the simplicity of analysis, we assume that $R\cong R_cR_r$, where $R_c$ is a quantum system with dimension $J$.
We also assume that $\Psi^{AR}$ is decomposed in the form of
\begin{align}
\!\!\!\!\!
\Psi^{AR}=\sum_{k,l=1}^J\outpro{k}{l}^{A_c}\!\otimes\psi_{kl}^{A_rR_r}\!\otimes\outpro{k}{l}^{R_c}\!,\!\laeq{romanof}
\end{align}
where $\psi_{kl}\in\ca{L}(\ca{H}^{A_r}\otimes\ca{H}^{R_r})$ for each $k$ and $l$.
Such a state is called a {\it classically coherent state} \cite{dupuis2014decoupling}.



The following theorem is the direct part of the randomized partial decoupling theorem.
We assume that $\dim{A_r}\geq2$,
although the original version (Theorem 3 in \cite{wakakuwa2019one}) is also applicable to the cases where $\dim{A_r}=1$.


\bthm{SmoothExMarkov}(Corollary of Theorem 3 in \cite{wakakuwa2019one})

Consider a linear CP map $\ca{T}^{A \rightarrow E}$ and
a state $\Psi^{AR}\in\ca{S}_=(\ca{H}^{AR})$ that is decomposed as \req{romanof}.
Let $U$ and $G_s$ be random unitaries defined by (\ref{eq:RUrpd}) and (\ref{eq:RPrpd}), respectively.
Define the partial decoupling error $\Delta_{s,U}(\ca{T},\Psi)$ by
\begin{equation}
\Delta_{s,U}(\ca{T},\Psi)
:=
\left\|
\ca{T}^{A \rightarrow E} \circ \ca{G}_s^A  \bigl( \ca{U}^A ( \Psi^{AR} ) -\Psi_{\rm av}^{AR} ) \bigr)
\right\|_1,
\nn
\end{equation}
where
$
\Psi_{\rm av}^{AR}:=\mbb{E}_{U_1, \dots, U_J \sim {\sf H}} [ \ca{U}^A ( \Psi^{AR} )]
$.
Then, for any $\epsilon,\mu\geq0$,
it holds that
\begin{align}
\mbb{E}_{s,U } [\Delta_{s,U}(\ca{T},\Psi) ]
\leq  
\theta_I+\theta_{I\!I}
+4(\epsilon+\mu+\epsilon\mu).
\laeq{SmExMa}
\end{align}
The terms $\theta_I$ and $\theta_{I\!I}$ are represented by
\alg{
\!
\theta_I=
\begin{cases}
2^{-\frac{1}{2}H_I}
 &\!\!\! (J\geq2)
\\
0 &\!\!\! (J=1)
\end{cases},
\;\;
\theta_{I\!I}=
\begin{cases}
2^{-\frac{1}{2}H_{I\!I}}
 &\!\!\! (d_{A_r}\geq2)
\\
0 &\!\!\! (d_{A_r}=1)
\end{cases},
\!
}
where the exponents $H_I$ and $H_{I\!I}$ are given by
\alg{
&
H_I=
\log{(J-1)}+
H_{\rm min}^\epsilon(A|R)_{\Psi}-H_{\rm max}^\mu(A|C)_{\ca{C}(\tau)},
\\
&
H_{I\!I}=
H_{\rm min}^\epsilon(A|R)_{\ca{C}(\Psi)}-H_{\rm max}^\mu(A_r|CA_c)_{\ca{C}(\tau)},
}
respectively.
Here, $\ca{C}$ is the completely dephasing operation on $A_c$ with respect to the basis $\{|j\rangle\}_{j=1}^J$, and $\tau$ is the Choi-Jamiolkowski state of  the complementary map $\ca{T}^{A\rightarrow C}$ of $\ca{T}^{A\rightarrow E}$, i.e. $\tau = \mfk{J}(\ca{T}^{A\rightarrow E})$. 
\ethm


\begin{figure*}[th]
\begin{center}
\includegraphics[bb={0 0 1141 404}, scale=0.37]{figure7.pdf}
\end{center}
\caption{
One-shot channel coding is depicted.
The normal arrows represent quantum systems, and the dotted arrows represent the classical part of the source state.
}
\label{fig:oneshotchannelcoding}
\end{figure*}


In \cite{wakakuwa2019one}, we also obtained a converse bound for randomized partial decoupling, which is stated by the following theorem.

\bthm{converse}(Theorem 4 in \cite{wakakuwa2019one})
Consider a linear trace-presersing CP map $\ca{T}^{A \rightarrow E}$ and
a state $\Psi^{AR}\in\ca{S}_=(\ca{H}^{AR})$ that is decomposed as \req{romanof}.
Suppose that, for $\delta>0$, there exists a normalized state in the form of
\alg{
\Omega^{ER}:=\sum_{j=1}^Jp_j\varsigma_j^E\otm\Psi_{j}^{R_r}\otm\proj{j}^{R_c},
}
such that
\alg{
\left\|
\ca{T}^{A \rightarrow E} ( \Psi^{AR} ) -\Omega^{ER}
\right\|_1
\leq
\delta.
}
Then, for any $\upsilon\in[0,1/2)$ and $\iota\in(0,1]$, it holds that
\alg{
&
\!\!
H_{\rm min}^{\lambda}(A|R)_\Psi
-H_{\rm min}^{\upsilon}(BR|C)_{\ca{T}\circ\ca{C}(\Psi)}+\log{J}
\geq
\log{\iota},
\laeq{tokiyotomare}
\\
&
\!\!
H_{\rm min}^{\lambda'}(A|R)_{\ca{C}(\Psi)}
-H_{\rm min}^{\upsilon}(BR_r|CR_c)_{\ca{T}\circ\ca{C}(\Psi)}
\nn\\
&
\quad\quad\quad\quad\quad\quad\quad\quad\quad\quad
\geq
\log{\iota}+\log{(1-2\upsilon)}.
\laeq{tokiyougoke}
}
The second terms in the L.H.S.s of \req{tokiyotomare} and \req{tokiyougoke} are for a purification $|\Psi\rangle^{ABR}$ of $\Psi^{AR}$ and the complementary channel $\ca{T}^{A \rightarrow C}$ of $\ca{T}^{A \rightarrow E}$,
with $\ca{C}$ being the completely dephasing channel on $A_c$.
The smoothing parameters $\lambda$ and $\lambda'$ are given by
\alg{
\lambda:=
&
2\sqrt{\iota+4\sqrt{20\upsilon+2\delta}}
+\sqrt{2\sqrt{20\upsilon+2\delta}}
\nn\\
&+2\sqrt{2\delta}
+2\sqrt{20\upsilon+2\delta}
+3\upsilon,
\laeq{dfnsmtlambda}
\\
\lambda':=
&
\upsilon+\sqrt{4\sqrt{\iota+2x}+2\sqrt{x}+(4\sqrt{\iota+8}+24) x
\laeq{dfnsmtlambdaII}
}
}
and $x:=\sqrt{2}\sqrt[4]{24\upsilon+2\delta}$.
\ethm






\section{The One-Shot Capacity Theorems}
\lsec{mainresults}





We consider a scenario in which the sender, Alice, transmits classical and quantum messages simultaneously to the receiver, Bob, through a noisy quantum channel assisted by a limited or unlimited amount of shared entanglement. 
We assume that Bob initially has no side information about the messages. 
We denote by $c$ and $q$ the lengths of classical and quantum messages to be transmitted.
The available resources are a noisy quantum channel $\ca{N}^{A \rightarrow B}$ and a pure entangled state $\Phi_{2^e}^{F_AF_B}$ shared in advance (see Figure \ref{fig:oneshotchannelcoding}).
Our goal is to obtain the conditions for this task to be achievable within error tolerance $\delta$.
A rigorous definition of achievability is as follows:




\bdfn{oneshotcodeproto}
Consider  a quantum channel $\ca{N}^{A\rightarrow B}$. Let $M_q$ and $R_q$ be $2^q$-dimensional quantum systems, and $F_A$ and $F_B$ be $2^e$-dimensional quantum systems.
Let $\Phi_{2^q}^{M_qR_q}$ and $\Phi_{2^e}^{F_AF_B}$ be the maximally entangled state with Schmidt rank $2^q$ and $2^e$, respectively.
A pair of a set of encoding CPTP maps $\{\ca{E}_j^{M_qF_A\rightarrow A}\}_{j=1}^{2^c}$ and a decoding instrument $\{\ca{D}_j^{BF_B\rightarrow M_q}\}_{j=1}^{2^c}$ is called a $(c,q,e,\delta)$ code for the channel $\ca{N}$ if it holds that
\alg{
\frac{1}{2^c}
\sum_{j=1}^{2^c}
\left\|
 \ca{D}_j \!\circ\! \ca{N} \!\circ\! \ca{E}_j (\Phi_{2^q}^{M_qR_q} \!\otm\! \Phi_{2^e}^{F_AF_B}) \! -\! \Phi_{2^q}^{M_qR_q}
\right\|_1
\leq 
\delta.
\laeq{tsuzuiteku2}
}
\edfn

\noindent
The condition \req{tsuzuiteku2} implies that both the classical and quantum parts of the message are transmitted within the total error $\delta$.
The above definition is equivalent to the following definition, which is more convenient in our analysis:


\bdfn{oneshotcode}
Consider the same setting as in \rDfn{oneshotcodeproto}.
Let $M_c$ and $R_c$ be $2^c$-dimensional quantum system with a fixed orthonormal basis $\{\ket{j}\}_{j=1}^{2^c}$.
We denote $M_cM_q$ by $M$ and $R_cR_q$ by $R$ for brevity.
Let $\Phi_{2^c,2^q}'^{MR}$ be a source state defined by
\alg{
\Phi_{2^c,2^q}'^{MR}=\frac{1}{2^c}\sum_{j=1}^{2^c}\proj{j}^{M_c}\otm\proj{\Phi_{2^q}}^{M_qR_q}\otm\proj{j}^{R_c}.
\laeq{sourcestate}
}
A pair of an encoding CPTP map $\ca{E}^{MF_A\rightarrow A}$ and a decoding CPTP map $\ca{D}^{BF_B\rightarrow M}$ is called a $(c,q,e,\delta)$ code for the channel $\ca{N}$ if it holds that
\alg{
\left\|
 \ca{D} \circ \ca{N} \circ \ca{E} (\Phi_{2^c,2^q}'^{MR}\otm\Phi_{2^e}^{F_AF_B})  - \Phi_{2^c,2^q}'^{MR}
\right\|_1
\leq 
\delta \laeq{tsuzuiteku}
}
and
\alg{
&
\ca{E}^{MF_A\rightarrow A}=\ca{E}^{MF_A\rightarrow A}\circ\ca{C}^{M_c},
\laeq{encdeccls}\\
&\ca{D}^{BF_B\rightarrow M}=\ca{C}^{M_c}\circ\ca{D}^{BF_B\rightarrow M},
\laeq{encdecclst}
}
where $\ca{C}$ is the completely dephasing operation on $M_c$ with respect to the basis $\{\ket{j}\}_{j=1}^{2^c}$.
\edfn


\noindent
Note that the correspondence between the encoding and decoding operations in \rDfn{oneshotcodeproto} and \rDfn{oneshotcode} are given by
\alg{
&
\ca{E}_j^{M_qF_A\rightarrow A}(\cdot)
=
\ca{E}^{MF_A\rightarrow A}( \proj{j}^{M_c}\otm(\cdot)^{M_qF_A}),
\\
&
 \ca{D}_j^{BF_B\rightarrow M_q}(\cdot)
 =
 \bra{j}^{M_c}\ca{D}^{BF_B\rightarrow M}(\cdot) \ket{j}^{M_c}
}
and
\alg{
&
\ca{E}^{MF_A\rightarrow A}(\cdot)
=
\sum_{j=1}^{2^c}\ca{E}_j(\bra{j}^{M_c}(\cdot)\ket{j}^{M_c}),
\\
&
 \ca{D}^{BF_B\rightarrow M}(\cdot)
 =
 \sum_{j=1}^{2^c}
 \proj{j}^{M_c}\otm\ca{D}_j^{BF_B\rightarrow M_q}(\cdot).
}


It should be noted that the capacity theorems obtained in terms of the average probability of error, as in \req{tsuzuiteku2} and \req{tsuzuiteku}, are translated into those based on the worst-case error, up to halving of the message length. See, for example, Corollary 1 in \cite{buscemi2010quantum} and Theorem 11 in \cite{datta2012one} for the quantum part and Lemma 1 in \cite{renes2011noisy} for the classical part. The latter is known as the expurgation trick (see e.g.~\cite{wildetext}).

\begin{figure*}[th!]
\begin{center}
\includegraphics[bb={0 40 1224 286}, scale=0.3]{figure18.pdf}
\end{center}
\caption{
The protocol for communication over the channel $\ca{N}$ constructed in terms of randomized partial decoupling is depicted.
The encoding operation $\ca{E}_{\rho,s,U}$ is composed of (i) a linear isometry $P^{MF_A\rightarrow S}$ that embeds the message system $M$ and Alice's share of the entanglement resource $F_A$ to a larger system $S$, (ii) the permutation $s$ and the unitary $U$ that appear in randomized partial decoupling, and (iii) a linear CPTP map $\ca{E}_\rho$ that is obtained from $\rho^{SA}$ by the Choi-Jamiolkowski correspondence. 
The explicit form of the decoder $\ca{D}$ is left open, because we only prove the {\it existence} of a proper decoder $\ca{D}$ in the proof of the direct part.
}
\label{fig:directprotocol}
\end{figure*}

\subsection{Channel Capacity with Limited Entanglement}


First, we consider the situation in which the amount of the resource of shared entanglement is limited.

\bdfn{achievableratelim}
A triplet $(c,q,e)$ is said to be achievable within the error $\delta$ for the channel $\ca{N}^{A\rightarrow B}$ if there exists a $(c,q,e,\delta)$ code for $\ca{N}^{A\rightarrow B}$.
\edfn

\noindent
The direct part is represented by the following theorem.  
The proof is based on the direct part of randomized partial decoupling (\rThm{SmoothExMarkov}), and will be provided in \rSec{direct}.
A protocol that achieves the direct bound is depicted in Figure \ref{fig:directprotocol}.


\bthm{OSDcomp}
Let $S_r$ be a finite dimensional quantum system, and let $S_c$ be a quantum system with a fixed orthonormal basis $\{\ket{j}\}_{j=1}^{d_{S_c}}$ such that $d_{S_c}\geq2$.
We denote $S_cS_r$ by $S$.
Consider a state in the form of
\alg{
\rho^{SA}=\frac{1}{d_{S_c}}\sum_{j=1}^{d_{S_c}}\proj{j}^{S_c}\otm\rho_j^{S_rA}, 
\laeq{yapzet23}
 }
where$\{\rho_j\}_{j=1}^{d_{S_c}}$ is a set of normalized states on $S_rA$,
such that $\rho^{S}$ is the full-rank maximally mixed state on $S$.
For any such $S_c$, $S_r$, $\rho^{SA}$, any $\delta_1,\delta_2>0$ and $\epsilon\geq0$, a triplet $(c,q,e)$ is achievable within the error 
\alg{
\delta=
2\sqrt{\sqrt{\delta_1}+\sqrt{\delta_2}+4\epsilon}
\laeq{thesixthings}
}
 for the channel $\ca{N}^{A\rightarrow B}$
if $d_{S_c}\geq2^c$ and the following three inequalities hold: 
\alg{
q+e
&
\leq
\log{d_{S_r}},
\laeq{righton23}\\
c+q-e
&
\leq
-H_{\rm max}^{\epsilon}(S|B)_{\ca{N}(\rho)}
\nn\\
&
\quad\quad\quad\quad
+\log{(d_{S_c}-1)}
+\log{\delta_1},
\laeq{rightonf23}
\\
q-e
&
\leq
-H_{\rm max}^{\epsilon}(S_r|BS_c)_{\ca{N}(\rho)}
+\log{\delta_2}.
\laeq{rightoff23}
}
The same statement holds in the cases of $(c=0,d_{S_c}=1)$ and $(q=e=0,d_{S_r}=1)$.
In the former case, the condition \req{rightonf23} is removed and $\delta_1$ in \req{thesixthings} is assumed to be zero.
In the latter, the condition \req{rightoff23} is removed and $\delta_2$ in \req{thesixthings} is considered to be zero.
\ethm


\noindent
The converse part is stated by the following theorem, which will be proved in \rSec{converse} based on the converse part for randomized partial decoupling (\rThm{converse}).

\bthm{converseonehyb}
Suppose that a triplet $(c,q,e)$ is achievable within the error $\delta$ for the channel $\ca{N}^{A\rightarrow B}$.
Then, there exist a quantum system $S$ satisfying $d_S\leq2^{c+q+e}$ and a state $\rho^{SA}$ such that the following conditions hold.
First, $S$ is composed of finite dimensional quantum systems $S_c$ and $S_r$, the former of which is equipped with an orthonormal basis $\{\ket{j}\}_{j=1}^{d_{S_c}}$. 
Second, the state $\rho^{SA}$ is in the form of
\alg{
\rho^{SA}=\frac{1}{d_{S_c}}\sum_{j=1}^{d_{S_c}}\proj{j}^{S_c}\otm\rho_j^{S_rA},
\laeq{yapururu}
 } 
 with $\{\rho_j\}_{j=1}^{d_{S_c}}$ being a set of normalized states on $S_rA$,
and $\rho^{S}$ is the full-rank maximally mixed state on $S$.
Third, for any $\iota\in(0,1]$, it holds that
\alg{
q+e
&
\leq
\log{d_{S_r}},
\laeq{convineq1}\\
c+q-e
&
\leq
-H_{\rm max}^{\lambda}(S|B)_{\ca{N}(\rho)}
+\log{d_{S_c}}-
\log{\iota},
\laeq{convineq2}\\
q-e
&
\leq
-H_{\rm max}^{\lambda'}(S_r|BS_c)_{\ca{N}(\rho)}-\log{\iota}.
\laeq{convineq3}
}
The smoothing parameters $\lambda$ and $\lambda'$ are given by
\alg{
\lambda:=
&
2\sqrt{\iota+2x^2}
+x
+2x^2,
\laeq{smlam}\\
\lambda':=
&
\sqrt{4\sqrt{\iota+2x}+2\sqrt{x}+(4\sqrt{\iota+8}+24) x}
\laeq{smlamp}
}
and $x:=2\sqrt[8]{\delta}$.
\ethm



\subsection{Channel Capacity with Unlimited Entanglement}

Second, we consider the situation in which the amount of the resource of shared entanglement is unlimited.
Following \cite{datta2012one}, we assume that the entanglement resource is given in the form of the maximally entangled state.


\bdfn{achievablerateunlim}
A pair $(c,q)$ is said to be achievable within the error $\delta$ for the channel $\ca{N}^{A\rightarrow B}$ with the assistance of entanglement if there exists $e\geq0$ such that a triplet $(c,q,e)$ is achievable within the error $\delta$ for $\ca{N}^{A\rightarrow B}$.
\edfn

\noindent
The direct and converse bounds for the scenario of unlimited entanglement immediately follow from the direct bound and the converse bound for the case of limited entanglement, i.e., from \rThm{OSDcomp} and \rThm{converseonehyb}.


\begin{table*}[th!]
   \begin{center}
   \begin{tabular}{c|c||c|c}
   \hline
  & \begin{tabular}{c} \!\!\!\!\! entanglement \!\!\!\!\! \\ \!\!\!\!\!\! resource \!\!\!\!\!\! \end{tabular}
  %\shortstack{shared \\ entanglement} 
  &   lower bound & upper bound  \\ \hline \hline
     \multirow{2}{*}{  \begin{tabular}{c}
  \shortstack{\\ \\ \!\!\!\!\!\! $\delta$-classical capacity \\ \!\!\!\!\!\!\!\!\!\! of a quantum channel $\cal{N}$ \!\!\!\!\!\!\!\!}  \end{tabular}}  & \shortstack{\\ none} &  \shortstack{\\ $\displaystyle \sup_{S,\rho,\delta'}
\left[
\log{d_{S}}
-H_{\rm max}^{\epsilon(\delta,\delta')}(S|B)_{\ca{C}\otm\ca{N}(\rho)}
+\log{2\delta'}
\right]$} & 
\shortstack{\\  $\displaystyle \sup_{S,\rho} \inf_{\iota}
\left[
\log{d_{S}}
-H_{\rm max}^{\lambda(\delta,\iota)}(S|B)_{\ca{C}\otm\ca{N}(\rho)}
-\log{\iota}
\right]$} \\ \cline{2-4}
  &  \shortstack{\\ unlimited } &  \shortstack{\\  $\displaystyle \sup_{S,\rho,\delta'}
\left[
\log{d_{S}}
-H_{\rm max}^{\epsilon(\delta,\delta')}(S|B)_{\ca{N}(\rho)}
+\log{\delta'}
\right]$} &  \shortstack{\\ $\displaystyle \sup_{S,\rho}
\inf_{\iota}
\left[
\log{d_{S}}
-H_{\rm max}^{\lambda'(\delta,\iota)}(S|B)_{\ca{N}(\rho)}
-\log{\iota}
\right]$} \\ \hline
     \multirow{2}{*}{  \begin{tabular}{c}   \shortstack{\\ \\ \!\!\!\!\!\! $\delta$-quantum capacity \\ \!\!\!\!\!\!\!\!\!\! of a quantum channel $\cal{N}$ \!\!\!\!\!\!\!\!}  \end{tabular}}  & \shortstack{\\ none} &  \shortstack{\\ $\displaystyle \sup_{S,\rho, \delta'}
\left[
-H_{\rm max}^{\epsilon(\delta,\delta')}(S|B)_{\ca{N}(\rho)}
+\log{\delta'}
\right]$} & 
\shortstack{\\  $\displaystyle \sup_{S,\rho}
\inf_{\iota}
\left[
-H_{\rm max}^{\lambda'(\delta,\iota)}(S|B)_{\ca{N}(\rho)}
-\log{\iota}
\right]$} \\ \cline{2-4}
  &  \shortstack{\\ unlimited } &  \shortstack{\\  $\displaystyle \frac{1}{2}
\sup_{S,\rho, \delta'}
\left[
\log{d_{S}}
-H_{\rm max}^{\epsilon(\delta,\delta')}(S|B)_{\ca{N}(\rho)}
+\log{\delta'}
\right]$} &  \shortstack{\\ $\displaystyle \frac{1}{2}
\sup_{S,\rho}
\inf_{\iota}
\left[
\log{d_{S}}
-H_{\rm max}^{\lambda'(\delta,\iota)}(S|B)_{\ca{N}(\rho)}
-\log{\iota}
\right]$}
\\\hline
  \end{tabular}
    \end{center}
\caption{One-shot capacities of a quantum channel for specific cases.}
\label{fig:variouscapacities}
\end{table*}

\bcrl{OSDcompunlim}
For any $\epsilon\in[0,1/2)$ and $\delta'\in(0,1-2\epsilon]$, a pair $(c,q)$ is achievable within the error
\alg{
\delta=
2\sqrt{\sqrt{2\delta'}+\sqrt{\delta'}+4\epsilon}
\laeq{thesixthingsunlim}
}
for the channel $\ca{N}^{A\rightarrow B}$  with the assistance of entanglement, 
if there exist a quantum system $S$ and a state $\rho^{SA}$ 
such that $\rho^{S}$ is the full-rank maximally mixed state on $S$ and the following inequality holds: 
\alg{
c+2q
\leq
\log{d_{S}}-H_{\rm max}^{\epsilon}(S|B)_{\ca{N}(\rho)}
+\log{\delta'}.
\laeq{rightonf23unl}
}
\end{crl}



\bcrl{converseonehybunlim}

Suppose that a pair $(c,q)$ is achievable within the error $\delta$ for the channel $\ca{N}^{A\rightarrow B}$ with the assistance of entanglement.
Then, there exist a quantum system $S$ and a state $\rho^{SA}$ such that $\rho^{S}$ is the full-rank maximally mixed state on $S$,
and for any $\iota\in(0,1]$, it holds that
\alg{
c+2q
\leq
\log{d_{S}}-H_{\rm max}^{\lambda}(S|B)_{\ca{N}(\rho)}
-
\log{\iota}.
}
The smoothing parameter $\lambda$ is given by \req{smlam}.
\end{crl}





\noindent
{\bf Proof of Corollaries:}
\rCrl{converseonehybunlim} immediately follows from Inequalities \req{convineq1} and \req{convineq2} in \rThm{converseonehyb}.
To prove \rCrl{OSDcompunlim} from \rThm{OSDcomp},
suppose that there exists a state $\rho^{SA}$ that satisfies the conditions in \rCrl{OSDcompunlim}.
Let $S_c'$ be a system such that $d_{S_c'}\geq2^c$.
Define $S_r':=S$, $S':=S_c'S_r'$ and
consider a state
\alg{
\rho'^{S'A}
:=
\frac{1}{d_{S_c'}}
\sum_{j=1}^{d_{S_c'}}\proj{j}^{S_c'}\otm\rho^{S_r'A}.
}
Due to the property of the smooth max entropy for product states (\rLmm{Hmaxprod}), we have
\alg{
H_{\rm max}^{\epsilon}(S_r'|BS_c')_{\ca{N}(\rho')}
&
=
H_{\rm max}^{\epsilon}(S|B)_{\ca{N}(\rho)}
\laeq{wantit2}\\
&
\geq
H_{\rm max}^{\epsilon}(S'|B)_{\ca{N}(\rho')}
-
\log{d_{S_c'}}.
\laeq{wantit}
}
It follows from \req{wantit2} and \req{rightonf23unl} that
%\alg{
%c+2q
%\leq
%\log{d_{S_r'}}-H_{\rm max}^{\epsilon}(S_r'|BS_c')_{\ca{N}(\rho)}
%+\log{\delta'},
%}
%which is equivalent to
\alg{
q-\log{d_{S_r'}}
\leq
-H_{\rm max}^{\epsilon}(S_r'|BS_c')_{\ca{N}(\rho')}
+\log{\delta'}-c-q.
}
Thus, there exists $e\in\mbb{R}$ such that
\alg{
\begin{cases}
q+e
&\leq 
\log{d_{S_r'}},
%\laeq{mainichi0}
\\
c+q-e
&\leq
-H_{\rm max}^{\epsilon}(S_r'|BS_c')_{\ca{N}(\rho')}
+\log{\delta'}.
\laeq{mainichi}
\end{cases}
}
We may assume that $e\geq0$, since $q\leq \log{d_{S_r'}}$.
This is because the dimension bound for the smooth max entropy (see \rLmm{tomerare} in \rApp{PSE}) imply
\alg{
2q
\leq
\log{d_{S}}-H_{\rm max}^{\epsilon}(S|B)_{\ca{N}(\rho)}
\leq2\log{d_{S}}
+
\log{\left(\frac{\delta'}{1-2\epsilon}\right)}.
\nn
}
The second inequality in \req{mainichi} further leads to
\alg{
c+q-e
&\leq
\!-\!H_{\rm max}^{\epsilon}(S'|B)_{\ca{N}(\rho')}
\!+\!\log{(d_{S_c'}\!-\!1)}\!+\!\log{2\delta'}
\laeq{mainichi2}
}
due to Inequality \req{wantit} and the relation $d_{S_c'}/(d_{S_c'}-1)\leq2$,
and to \alg{
q-e
\leq
-H_{\rm max}^{\epsilon}(S_r'|BS_c')_{\ca{N}(\rho')}
+\log{\delta'}
\laeq{mainichi3}
}
since $c\geq0$.
Combining \req{mainichi2}, \req{mainichi3} and the first inequality in \req{mainichi} with \rThm{OSDcomp}, we complete the proof.
\QED





\subsection{Special cases}

The one-shot capacity theorems for various scenarios directly follow from \rThm{OSDcomp}, \rThm{converseonehyb}, \rCrl{OSDcompunlim} and \rCrl{converseonehybunlim}.
In Table \ref{fig:variouscapacities}, we summarize the one-shot capacity theorems for the classical and quantum capacities with the assistance of limited/unlimited amount of entanglement.
There, the $\delta$-classical capacity of the channel $\ca{N}$ is defined as the supremum of $c$ such that the triplet $(c,q=0,e=0)$ is achievable within the error $\delta$ for the channel $\ca{N}$.
The $\delta$-quantum capacity and the entanglement assisted capacities are defined along the same line.
The smoothing parameters $\epsilon(\delta,\delta')$, $\lambda(\delta,\iota)$ and $\lambda'(\delta,\iota)$ in Table \ref{fig:variouscapacities} are defined by $\epsilon(\delta,\delta'):=\delta^2/16-\sqrt{\delta'}/4$, \req{smlam} and \req{smlamp}, respectively. The supremum over $\delta'$ and the infimum over $\iota$ are taken in the intervals $\delta'\in(0,\delta^4/16]$ and $\iota\in(0,1]$. The supremum over $\rho$ is take over all states $\rho^{SA}$ such that $\rho^S$ is the full-rank maximally mixed state on $S$. The map $\ca{C}$ is the completely dephasing operation on $S$ with respect to a fixed orthonormal basis.


One can also obtain the one-shot capacity region for simultaneously transmitting classical and quantum messages through the channel without shared entanglement. 
The $\delta$-simultaneous capacity region of the channel $\ca{N}$ is defined as the set of all pairs $(c,q)\in\mbb{R}_\geq^2$ such that the triplet $(c,q,e=0)$ is achievable within the error $\delta$ for the channel $\ca{N}$. 
We denote the achievable rate region by $\Gamma_\delta(\ca{N})$, and assume that $\delta\in(0,2]$. 
For an arbitrary system $S\equiv S_cS_r$ and an arbitrary state $\rho^{SA}$ that is diagonal in $S_c$, let $\Gamma_{\delta,\delta',\epsilon}^{\rm in}(\ca{N},\rho)$ be the set of all pairs $(c,q)\in\mbb{R}_\geq^2$ that satisfy
\alg{
\begin{cases}
c+q
&
\leq
-H_{\rm max}^{\epsilon(\delta,\delta')}(S|B)_{\ca{N}(\rho)}
+\log{(d_{S_c}-1)}
+\log{\delta'},
\\
q
&
\leq
-H_{\rm max}^{\epsilon(\delta,\delta')}(S_r|BS_c)_{\ca{N}(\rho)}
+\log{\delta'(1-2\epsilon(\delta,\delta'))}
\end{cases}
\nn
}
and $\Gamma_{\delta,\iota}^{\rm out}(\ca{N},\rho)$ be the set that satisfy
\alg{
\begin{cases}
c+q
&
\leq
-H_{\rm max}^{\lambda(\delta,\iota)}(S|B)_{\ca{N}(\rho)}
+\log{d_{S_c}}-
\log{\iota},
\\
q
&
\leq
-H_{\rm max}^{\lambda'(\delta,\iota)}(S_r|BS_c)_{\ca{N}(\rho)}-\log{\iota}.
\end{cases}
\nn
}
The smoothing parameters $\epsilon(\delta,\delta')$, $\lambda(\delta,\iota)$ and $\lambda'(\delta,\iota)$ are defined by $\epsilon(\delta,\delta'):=\delta^2/16-\sqrt{\delta'}/4$, \req{smlam} and \req{smlamp}, respectively.
It follows from \rThm{OSDcomp} and \rThm{converseonehyb} that
\alg{
\bigcup_{S,\rho}
\bigcup_{\delta'}
\Gamma_{\delta,\delta'}^{\rm in}(\ca{N},\rho)
\subseteq
\Gamma_\delta(\ca{N})
\subseteq
\bigcup_{S,\rho}
\bigcap_{\iota}
\Gamma_{\delta,\iota}^{\rm out}(\ca{N},\rho).
\laeq{cqregionone}
}
Here, the union over $\delta'$ and the intersection over $\iota$ are taken in the intervals $\delta'\in(0,\delta^4/16]$ and $\iota\in(0,1]$, respectively.
The union over $\rho$ is taken for all states $\rho^{SA}$ such that $\rho^S$ is the full-rank maximally mixed state on $S$.


The rate region \req{cqregionone} coincides the one given by \cite{hsieh2010entanglement} in the asymptotic limit of infinitely many copies and vanishingly small error (see Theorem 5 therein).
The same communication task has also been analyzed in \cite{devetak2005capacity} for the asymptotic scenario and in \cite{salek2019one} for the one-shot scenario. 
There, the rate regions {\it with respect to a fixed $\rho$} is given in the form of $c\leq H_1(\rho,\ca{N})$ and $q\leq H_2(\rho,\ca{N})$, with $H_1$ and $H_2$ being some entropic functions of $\rho$ and $\ca{N}$.
All the rate regions in these literatures shall coincide in the asymptotic limit, by taking the union over all states $\rho$.



\begin{figure*}[t]
\begin{center}
\includegraphics[bb={0 40 1369 582}, scale=0.3]{figure15.pdf}
\end{center}
\caption{
The definition of the state $|\Psi_\rho\rangle$ given by \req{narutame} and its transformation by partial decoupling are depicted.
CJ, SD and UT stand for the Choi-Jamiolkowski correspondense, the Stinespring dilation and Uhlmann's theorem, respectively.
Note that $R\equiv R_cR_q$ and $M\equiv M_cM_q$.
In composing the encoding operation, we use the fact that the actions of $U$, $s$ and $\tilde{\ca{P}}$ on system $\hat{R}$ are replaced by those of $\tilde{\ca{P}}$, $G_s^t$ and $U^t$ in this order, on system $\hat{S}$ before applying $V_\rho$. 
This trick was introduced in \cite{ADHW2009} to analyze the quantum channel capacity in an asymptotic scenario.}
\label{fig:direct}
\end{figure*}



\section{Proof of The Direct Part\\(\rThm{OSDcomp})}
\lsec{direct}

We prove the direct part of the capacity theorem (\rThm{OSDcomp}) based on the direct part of the randomized partial decoupling theorem (\rThm{SmoothExMarkov}).
We follow the idea of Ref.~\cite{ADHW2009}, that the problem of finding a good code for a quantum channel is equivalent to the problem of finding a good way to decouple a bipartite state constructed from the channel.


Fix an arbitrary triplet $(c,q,e)$, $\delta>0$, system $S\equiv S_cS_r$ and a state $\rho^{SA}$ that satisfy the conditions in \rThm{OSDcomp}.
We prove achievability of the triplet $(c,q,e)$ within the error $\delta$ along the following lines:
First, we construct a state $\Psi_\rho$ from the state $\rho^{SA}$, the channel $\ca{N}^{A\rightarrow B}$ and a state obtained by ``purifying'' and ``extending'' the source state and the resource state.
Second, we prove that, if a CP map achieves randomized partial decoupling of the state $\Psi_\rho$ for a particular choice of $s$ and $U$ in high precision, there exist an encoder $\ca{E}_{\rho,s,U}$ and a decoder $\ca{D}$ that accomplish the communication with a small error.
Finally, we evaluate the precision of randomized partial decoupling based on \rThm{SmoothExMarkov}, by which we complete the proof of \rThm{OSDcomp}.



\subsection{Definitions of States and Operations}



We embed $M_c$ and $M_qF_A$ into $S_c$ and $S_r$, respectively.
This is possible because of the conditions $\dim S_c \geq 2^c$ and $\dim S_r  \geq 2^{q+e}$ in \rThm{OSDcomp}.
Similarly, we embed $R_c$ and $R_qF_B$ to larger systems $\hat{R}_c$ and $\hat{R}_r$, respectively, such that $\hat{R}_c\cong S_c$ and $\hat{R}_r\cong S_r$.
The embeddings are represented by linear isometries $P^{M_c\rightarrow S_c}$, $P^{M_qF_A \rightarrow S_r}$, $P^{R_c\rightarrow \hat{R}_c}$ and $P^{R_qF_B \rightarrow \hat{R}_r}$.
For the simplicity of notations, we denote $\hat{R}_c\hat{R}_r$ by $\hat{R}$.
In total, the systems $MF_A$ and $RF_B$ are embedded into systems $S$ and $\hat{R}$ by linear isometries
\alg{
&
P^{MF_A\rightarrow S} :=P^{M_c\rightarrow S_c}\otm P^{M_qF_A\rightarrow S_r},
\laeq{ninen0}
\\
&
P^{RF_B\rightarrow \hat{R}} :=P^{R_c\rightarrow \hat{R}_c}\otm P^{R_qF_B\rightarrow \hat{R}_r},
\laeq{ninen}
 }
respectively.
The adjoint map corresponding to these isometries are given by
\alg{
\ca{P}^{\dagger S\rightarrow MF_A}(\cdot)
&
:=(P^{MF_A\rightarrow S} )^\dagger(\cdot)(P^{MF_A\rightarrow S} ),
\\
\ca{P}^{\dagger \hat{R}\rightarrow  RF_B}(\cdot)
&
:=(P^{RF_B\rightarrow \hat{R}} )^\dagger(\cdot)(P^{RF_B\rightarrow \hat{R}} ).
}

Define a ``purified'' source-resource state by 
\alg{
&
\ket{\Phi_{\rm pur}}^{MF_ARF_B}
:=\ket{\Phi_{2^{c+q}}}^{MR}\ket{\Phi_{2^e}}^{F_AF_B}
\laeq{Phipurdfn}\\
&\quad
=
\frac{1}{\sqrt{2^c}}\sum_{j=1}^{2^c}\ket{j}^{M_c}\ket{j}^{R_c}\ket{\Phi_{2^q}}^{M_qR_q}\ket{\Phi_{2^e}}^{F_AF_B}.
\laeq{pusource}
}
Note that
\alg{
\ca{C}^{M_c}(\Phi_{\rm pur})
=
\Phi_{2^c,2^q}'^{MR}\otm\Phi_{2^e}^{F_AF_B},
\laeq{haveyou}
}
where $\ca{C}^{M_c}$ the completely dephasing operation on $M_c$ with respect to the basis $\{\ket{j}\}_{j=1}^{2^c}$.
We also introduce an ``extended''  one by
\alg{
\ket{\Phi_{\rm ext}}^{S\hat{R}}:=
\frac{1}{\sqrt{d_{S_c}}}\sum_{j=1}^{d_{S_c}}\ket{j}^{S_c}\ket{j}^{\hat{R}_c}\ket{\Phi_{d_{S_r}}}^{S_r\hat{R}_r}.
\laeq{exsource}
}
We properly choose the embedding isometries so that
\alg{
&
\ket{\Phi_{\rm pur}}^{MF_ARF_B}
\nn\\
&\quad
=
\sqrt{\frac{d_S}{2^{c+q+e}}}(P^{MF_A\rightarrow S}\otm P^{RF_B\rightarrow \hat{R}}  )^\dagger\ket{\Phi_{\rm ext}}^{S\hat{R}}
\laeq{extnonext}
}
and that
\alg{
P^{M_c\rightarrow S_c}\ket{j}^{M_c}=\ket{j}^{S_c},
\quad
P^{R_c\rightarrow \hat{R}_c}\ket{j}^{R_c}=\ket{j}^{\hat{R}_c}
\laeq{UY}
}
for any $j=1,\cdots,2^c$.
Using
\alg{
\tilde{P}^{\hat{R}\rightarrow RF_B}
:=
\sqrt{\frac{d_S}{2^{c+q+e}}}\:
(P^{  RF_B\rightarrow \hat{R}})^\dagger,
\laeq{honoto}
}
the purified one and the extended one are simply related as
\alg{
P^{MF_A\rightarrow S}
\ket{\Phi_{\rm pur}}^{MF_ARF_B}
=
\tilde{P}^{\hat{R}\rightarrow RF_B}\ket{\Phi_{\rm ext}}^{S\hat{R}}.
\laeq{extnonextyy}
}


Let $\ca{E}_\rho^{S\rightarrow A}$ and $\ca{E}_{\rho_j}^{S_r\rightarrow A}$ be linear maps defined by the Choi-Jamiolkowski correspondence from $\rho^{SA}$ and $\rho_j^{S_rA}$, respectively. 
That is, $\ca{E}_\rho^{S\rightarrow A}:=\mfk{J}_S^{-1}(\rho^{SA})$ and $\ca{E}_{\rho_j}^{S_r\rightarrow A}:=\mfk{J}_{S_r}^{-1}(\rho_j^{S_rA})$.
Due to the condition that $\rho^S$ is the full-rank maximally mixed state, the two maps are completely positive and trace preserving.
From the decomposition \req{yapzet23}, it follows that
\alg{
\ca{E}_\rho^{S\rightarrow A}(\tau)
=
\sum_{j=1}^{d_{S_c}}\ca{E}_{\rho_j}^{S_r\rightarrow A}(\bra{j}^{S_c}\tau\ket{j}^{S_c}).
\laeq{calErhodec}
}
We denote by $V_{\rho_j}^{S_r\rightarrow AE_0}$ the Stinespring dilation of $\ca{E}_{\rho_j}^{S_r\rightarrow A}$ for each $j$.
Introducing a quantum system $E_c$ with a fixed orthonormal basis $\{\ket{j}\}_{j=1}^{d_{S_c}}$, the Stinespring dilation $V_{\rho}$ of $\ca{E}_\rho^{S\rightarrow A}$ is given by
\alg{
V_\rho^{S\rightarrow AE_0E_c}
=
\sum_{j=1}^{d_{S_c}}\ket{j}^{E_c}\bra{j}^{S_c}\otm V_{\rho_j}^{S_r\rightarrow AE_0}.
\laeq{ichinen}
}
It is straightforward from \req{calErhodec} that
\alg{
\ca{E}_\rho^{S\rightarrow A}\circ\ca{C}^{S_c}
=
\ca{E}_\rho^{S\rightarrow A},
\laeq{haveyoume}
}
with $\ca{C}^{S_c}$ being the completely dephasing operation on $S_c$ with respect to the basis $\{\ket{j}\}_{j=1}^{d_{S_c}}$.



Let $W_{\ca{N}}^{A\rightarrow BE}$ be the Stinespring dilation of $\ca{N}^{A\rightarrow B}$. 
Using the extended source-resource state \req{exsource},
we define the following pure state, where $\bar{E}\equiv E_cEE_0$ (see Figure \ref{fig:direct}):
\alg{
\ket{\Psi_\rho}^{\hat{R}B\bar{E}}
:=
W_{\ca{N}}^{A\rightarrow BE}\circ
V_\rho^{S\rightarrow AE_0E_c}\ket{\Phi_{\rm ext}}^{S\hat{R}}.
\laeq{narutane}
}
Defining the state
\alg{
\ket{\rho_j}^{\hat{R}_rAE_0}:=V_{\rho_j}^{S_r\rightarrow AE_0}\ket{\Phi_{d_{S_r}}}^{S_r\hat{R}_r},
}
it follows from \req{ichinen} that
\alg{
\ket{\Psi_\rho}^{\hat{R}B\bar{E}}
=
\frac{1}{\sqrt{d_{S_c}}}\sum_{j=1}^{d_{S_c}}
\ket{j}^{\hat{R}_c}\ket{j}^{E_c}
W_{\ca{N}}^{A\rightarrow BE}\ket{\rho_j}^{\hat{R}_rAE_0}.
\laeq{narutame}
}
We trace out system $B$ to obtain the state $\Psi_\rho^{\hat{R}\bar{E}}:={\rm Tr}_B[\proj{\Psi_\rho}]$.





\begin{figure*}[t]
\begin{center}
\includegraphics[bb={0 40 1641 661}, scale=0.3]{figure17.pdf}
\end{center}
\caption{
Transformation of the protocol for randomized partial decoupling of the state $\Psi_\rho$ into that for communication over the channel $\ca{N}$ is depicted. We first trace out $\bar{E}$ in Figure \ref{fig:direct} and apply these transformations to obtain Figure~\ref{fig:directprotocol}.
(i) is due to the fact that $\Phi_{\rm ext}$ is the maximally entangled state between $S$ and $\hat{R}$. 
(ii) is from Equality \req{extnonextyy}, and (iii) is because $\ca{E}_{\rho,s,U}^{MF_A\rightarrow A}=\ca{E}_{\rho,s,U}^{MF_A\rightarrow A}\circ\ca{C}^{M_c}$.
Note that $|\Phi_{\rm pur}\rangle^{\bar{M}\bar{R}}=|\Phi_{2^{c+q}}\rangle^{MR}|\Phi_{2^e}\rangle^{F_AF_B}$ as \req{Phipurdfn}.
(iv) follows from \req{haveyou}.
}
\label{fig:directtransform}
\end{figure*}




\subsection{Construction of Encoding and Decoding Operations}


We consider randomized partial decoupling of the ``bipartite'' state $\Psi_\rho^{\hat{R}\bar{E}}$ by a linear CP map ${\rm Tr}_{F_B}\circ\tilde{\ca{P}}^{\hat{R} \rightarrow RF_B}:\hat{R}\rightarrow RF_B$ (see Figure \ref{fig:direct}), where $\tilde{P}^{\hat{R}\rightarrow RF_B}$ is defined by \req{honoto}.
For a unitary $U^{\hat{R}}=\sum_{j=1}^{d_{S_c}}\proj{j}^{\hat{R}_c}\otm U_j^{\hat{R}_r}$ and a permutation $s$ on $\{ 1,\cdots,d_{S_c} \}$, define the {\it partial decoupling error} $\Delta_{s,U}$ by
\alg{
\!
\Delta_{s,U}
\!:=\!
\left\|
{\rm Tr}_{F_B}\!\circ\!\tilde{\ca{P}}^{\hat{R}\rightarrow RF_B} \! \circ \ca{G}_s^{\hat{R}} \! \circ \ca{U}^{\hat{R}} ( \Psi_\rho^{\hat{R}\bar{E}} ) 
-
\tilde{\Psi}_{\rho,s}^{R\bar{E}}
\right\|_1.
\laeq{SmExMaMaMa}
}
Here, we have defined
\alg{
\tilde{\Psi}_{\rho,s}^{R\bar{E}}
:=
{\rm Tr}_{F_B}\circ\tilde{\ca{P}}^{\hat{R}\rightarrow RF_B}  \circ \ca{G}_s^{\hat{R}} ( \Psi_{\rho,{\rm av}}^{\hat{R}\bar{E}} ),
\laeq{tukae}
}
where
\alg{
\Psi_{\rho,{\rm av}}^{\hat{R}\bar{E}}
:=
\mbb{E}_U[\ca{U}^{\hat{R}} ( \Psi_\rho^{\hat{R}\bar{E}} ) ].
\laeq{tukaa}
}
An evaluation of the partial decoupling error $\Delta_{s,U}$ will be given in the next subsection, based on the direct part of the randomized partial decoupling theorem (\rThm{SmoothExMarkov}).
We introduce the operator
\alg{
\tilde{P}_{s,U}^{\hat{R}\rightarrow RF_B}
:=
\tilde{P}^{\hat{R} \rightarrow RF_B}  G_s^{\hat{R}} U^{\hat{R}},
\laeq{dfnpsigU}
}
by which \req{SmExMaMaMa} is simply represented as
\alg{
\Delta_{s,U}
:=
\left\|
{\rm Tr}_{F_B}\!\circ\!\tilde{\ca{P}}_{s,U}^{\hat{R}\rightarrow RF_B} ( \Psi_\rho^{\hat{R}\bar{E}} ) 
-
\tilde{\Psi}_{\rho,s}^{R\bar{E}}
\right\|_1.
\laeq{SmExMaMaMa4}
}


Let $|\tilde{\Psi}_{\rho,s}\rangle^{MR\bar{E}M_0}$ be a purification of $\tilde{\Psi}_{\rho,s}^{R\bar{E}}$ with $MM_0$ being a purifying system.
%An explicit form of the purification will be given later.
Due to Uhlmann's theorem (\cite{uhlmann1976transition}; see also e.g. Chapter 9 in \cite{wildetext}) and \req{SmExMaMaMa4}, there exists a linear isometry $\tilde{V}^{BF_B\rightarrow MM_0}$ such that
\begin{eqnarray}
\left\|
\tilde{\ca{V}}^{BF_B \rightarrow MM_0} \otm
\tilde{\ca{P}}_{s,U}^{\hat{R}\rightarrow RF_B} (\Psi_\rho^{\hat{R}B\bar{E}})
-
\tilde{\Psi}_{\rho,s}^{MR\bar{E}M_0} \right\|_1 
\nn\\
\leq 2\sqrt{\Delta_{s,U}}.
\quad
\laeq{asuwo}
\end{eqnarray}
Note that $\tilde{V}^{BF_B\rightarrow MM_0}$ depends on $\rho$, $s$ and $U$ in general.
Defining
\alg{
\ca{D}^{BF_B\rightarrow M}:=({\rm Tr}_{M_0}\otm\ca{C}^{M_c})\circ\tilde{\ca{V}}^{BF_B\rightarrow MM_0},
} 
and tracing out the unnecessary systems $\bar{E}$ and $M_0$ in \req{asuwo},
 we obtain
\begin{eqnarray}
\left\|
\ca{D}^{BF_B\rightarrow M} \otm
\tilde{\ca{P}}_{s,U}^{\hat{R}\rightarrow RF_B}
(\Psi_\rho^{\hat{R}B})
-
\ca{C}^{M_c}( \tilde{\Psi}_{\rho,s}^{MR}) \right\|_1 
\nn\\
\leq 2\sqrt{\Delta_{s,U}}.
 \laeq{tobeforme}
\end{eqnarray}



To obtain an explicit form of the state $\tilde{\Psi}_{\rho,s}^{MR}$, we use \req{narutame} and \req{tukaa} to have
\alg{
\Psi_{\rho,{\rm av}}^{\hat{R}\bar{E}}
=
\frac{1}{d_{S_c}}\sum_{j=1}^{d_{S_c}}
\proj{j}^{\hat{R}_c}\otm\proj{j}^{E_c}
\otm\pi^{\hat{R}_r}\otm\ca{N}^{A\rightarrow E}(\rho_j^{AE_0}).
\nn
}
Combining this with \req{tukae}, and by using \req{honoto}, we obtain
\alg{
\tilde{\Psi}_{\rho,s}^{R\bar{E}}
&
=
\frac{1}{2^c}\sum_{j=1}^{2^c}
\proj{j}^{R_c}\otm\proj{s^{-1}(j)}^{E_c}
\nn\\
&\quad\quad\quad\quad\quad\quad
\otm\pi_{2^q}^{R_q}\otm\ca{N}^{A\rightarrow E}(\rho_{s^{-1}(j)}^{AE_0})
.
}
A purification of this state is given by
\begin{align}
&
\ket{\tilde{\Psi}_{\rho,s}}^{MR\bar{E}M_0}: = 
\nn\\
&\quad
\frac{1}{\sqrt{2^c}}\sum_{j=1}^{2^c}   \ket{j}^{M_c}\ket{j}^{R_c}\ket{s^{-1}(j)}^{E_c} \ket{\Phi_{2^q}}^{M_qR_q}|\varrho_{s^{-1}(j)}\rangle^{E_0EM_0},
\nn
\end{align}
with $|\varrho_j\rangle^{E_0EM_0}$ being a purification of $\ca{N}^{A\rightarrow E}(\rho_j^{AE_0})$.
Thus, we trace out $\bar{E}M_0$ to obtain $\tilde{\Psi}_{\rho,s}^{MR}=\Phi_{2^c,2^q}'^{MR}$.
Substituting this to \req{tobeforme}, and noting that $\ca{C}^{M_c}(\Phi_{2^c,2^q}'^{MR} ) =\Phi_{2^c,2^q}'^{MR}$, 
we arrive at
\begin{eqnarray}
\left\|
\ca{D}^{BF_B\rightarrow M} \otm
\tilde{\ca{P}}_{s,U}^{\hat{R}\rightarrow RF_B}
(\Psi_\rho^{\hat{R}B})
-
\Phi_{2^c,2^q}'^{MR} \right\|_1
\nn\\
 \leq 2\sqrt{\Delta_{s,U}}.
 \laeq{tobeforme2}
\end{eqnarray}





The first term in the L.H.S. of \req{tobeforme2} is calculated as follows.
Note that the state $\ket{\Phi_{\rm ext}}^{S\hat{R}}$ defined by \req{exsource} is the maximally entangled state on $S\hat{R}$.
Thus, from \req{dfnpsigU}  and \req{extnonextyy}  (see Figure \ref{fig:directtransform}), it holds that 
\alg{
&
\tilde{P}_{s,U}^{\hat{R}\rightarrow RF_B}\ket{\Phi_{\rm ext}}^{S\hat{R}}
\nn\\
&=
(G_s U)^{S^T}\otm\tilde{P}^{\hat{R}\rightarrow RF_B} 
\ket{\Phi_{\rm ext}}^{S\hat{R}}
\nn\\
&=
(G_s U)^{S^T}P^{MF_A\rightarrow S}
\ket{\Phi_{\rm pur}}^{MF_ARF_B}
\nn\\
&=
P_{s,U}^{MF_A\rightarrow S}
\ket{\Phi_{\rm pur}}^{MF_ARF_B},
\laeq{fuha}
}
where we have defined a linear isometry
\alg{
P_{s,U}^{MF_A\rightarrow S}
:=U^{S^T}G_{s^{-1}}^{S}P^{MF_A\rightarrow S}.
\laeq{hikikae}
}
Note that $G_s^{S^T}=G_{s^{-1}}^{S}$.
Combining \req{fuha} with \req{narutane}, we obtain
\alg{
&\!\!
\tilde{P}_{s,U}^{\hat{R}\rightarrow RF_B}\ket{\Psi_\rho}^{\hat{R}B\bar{E}}
\nn\\
&\!\!
=
W_{\ca{N}}^{A\rightarrow BE}\circ
V_\rho^{S\rightarrow AE_0E_c}
\circ P_{s,U}^{MF_A\rightarrow S}
\ket{\Phi_{\rm pur}}^{MF_ARF_B}\!.\!\!
\laeq{stephen}
}
We now construct an encoding operation $\ca{E}_{\rho,s,U}$ by
\alg{
\ca{E}_{\rho,s,U}^{MF_A\rightarrow A}
&
:=
\ca{E}_\rho^{S\rightarrow A} \circ \ca{P}_{s,U}^{MF_A\rightarrow S}
\laeq{dfnencrsU}
\\
&=
{\rm Tr}_{E_0E_c}\circ
\ca{V}_\rho^{S\rightarrow AE_0E_c}
\circ \ca{P}_{s,U}^{MF_A\rightarrow S}.
}
Tracing out $\bar{E} = E E_0 E_c$ in \req{stephen} yields
\alg{
\tilde{\ca{P}}_{s,U}^{\hat{R}\rightarrow RF_B}
(\Psi_\rho^{\hat{R}B})
=
\ca{N}^{A\rightarrow B}\circ\ca{E}_{\rho,s,U}^{MF_A\rightarrow A}(\Phi_{\rm pur}^{MF_ARF_B}).
\laeq{tsuji}
}
Substituting this to \req{tobeforme2}, 
we arrive at
\alg{
&
\left\|
\ca{D}^{BF_B \rightarrow M} \circ
\ca{N}^{A \rightarrow B} \circ \ca{E}_{\rho,s,U}^{MF_A\rightarrow A} (\Phi_{\rm pur}^{MF_ARF_B})
-
\Phi_{2^c,2^q}'^{MR} \right\|_1
\nn\\
&
 \leq 2\sqrt{\Delta_{s,U}}.
 \laeq{tobeforme3}
}



It remains to prove that the encoding operation defined by \req{dfnencrsU} satisfies the condition \req{encdeccls} (see Figure \ref{fig:directtransform}), i.e.,
\alg{
\ca{E}_{\rho,s,U}^{MF_A\rightarrow A}=\ca{E}_{\rho,s,U}^{MF_A\rightarrow A}\circ\ca{C}^{M_c}.
\laeq{clencrsU}
} 
From \req{ninen0}, \req{UY} and \req{hikikae}, we have
\alg{
P_{s,U}^{MF_A\rightarrow S}
\!=\!
\sum_{j=1}^{2^c}\ket{s^{-1}(j)}^{S_c}\bra{j}^{M_c}\!\otm U_{s^{-1}(j)}^{S_r^T}P^{M_qF_A\rightarrow S_r}.
\!
}
Thus, it holds that
\alg{
\ca{P}_{s,U}^{MF_A\rightarrow S}\circ\ca{C}^{M_c}
=
\ca{C}^{S_c}\circ\ca{P}_{s,U}^{MF_A\rightarrow S},
}
with $\ca{C}^{S_c}$ being the completely dephasing operation on $S_c$ with respect to the basis $\{\ket{j}\}_{j=1}^J$.
Combining this with \req{haveyoume}, we obtain
\alg{
\ca{E}_\rho^{S\rightarrow A} \circ \ca{P}_{s,U}^{MF_A\rightarrow S}\circ\ca{C}^{M_c}
=
\ca{E}_\rho^{S\rightarrow A} \circ \ca{P}_{s,U}^{MF_A\rightarrow S},
}
which implies \req{clencrsU}.
Substituting this to \req{tobeforme3}, and by using the relation \req{haveyou}, we finally arrive at
\alg{
&
\left\|
\ca{D}^{BF_B \rightarrow M} \circ
\ca{N}^{A \rightarrow B} \circ \ca{E}_{\rho,s,U}^{MF_A\rightarrow A} (\Phi_{2^c,2^q}'^{MR}\otm\Phi_{2^e}^{F_AF_B})
-
\Phi_{2^c,2^q}'^{MR} \right\|_1
\nn\\
&
 \leq 2\sqrt{\Delta_{s,U}}.
}


The deformations of expressions that we have done in this subsection are depicted in Figure~\ref{fig:directtransform}.
The figure explains the way how the communication protocol depicted in Figure \ref{fig:directprotocol} is obtained from the randomized partial decoupling protocol in Figure \ref{fig:direct}.


\subsection{Evaluation of The Errors}


To evaluate the partial decoupling error $\Delta_{s,U}$ defined by \req{SmExMaMaMa},
we apply the direct part of randomized partial decoupling (\rThm{SmoothExMarkov}) under the following correspondence:
\alg{
A_c, A_r, R_c, R_r
&
\rightarrow
\hat{R}_c, \hat{R}_r, E_c, EE_0
\\
A,R,E,C
&\rightarrow
\hat{R},\bar{E},R,F_B
\\
\Psi^{AR}
&\rightarrow\Psi_\rho^{\hat{R}\bar{E}}
\\
\ca{T}^{A\rightarrow E}
&\rightarrow {\rm Tr}_{F_B}\circ\tilde{\ca{P}}^{\hat{R}\rightarrow RF_B}.
}
We especially choose $\mu=0$.
 It follows that
there exist a unitary $U^{\hat{R}}=\sum_{j=1}^{d_{S_c}}\proj{j}^{R_c}\otm U_j^{R_qF_B}$ and a permutation $s$ such that the partial decoupling error is bounded as
\begin{align}
\Delta_{s,U}
\leq
\begin{cases}
2^{-\frac{1}{2}H_I}
+
2^{-\frac{1}{2}H_{I\!I}}
+4\epsilon
\quad\quad
(d_{S_c},d_{S_r}\geq2)
\\
2^{-\frac{1}{2}H_{I}}
+4\epsilon
\quad\quad
(q=e=0,d_{S_c}\geq2,d_{S_r}=1)
\\
2^{-\frac{1}{2}H_{I\!I}}
+4\epsilon
\quad\quad
(c=0,d_{S_c}=1,d_{S_r}\geq2).
\end{cases}
\laeq{SmExMaMa}
\end{align}
Here, the exponents $H_I$ and $H_{I\!I}$ are given by
\alg{
&
H_I=
\log{(d_{S_c}-1)}+
H_{\rm min}^\epsilon(\hat{R}|\bar{E})_{\Psi_\rho}-H_{\rm max}(\hat{R}|F_B)_{\ca{C}(\tau)},
\laeq{amaimonohoshii}\\
&
H_{I\!I}=
H_{\rm min}^\epsilon(\hat{R}|\bar{E})_{\ca{C}(\Psi_\rho)}-H_{\rm max}(\hat{R}_r|F_B\hat{R}_c)_{\ca{C}(\tau)},
\laeq{nomimonogaii}
}
and $\tau^{\hat{R}F_B}$ is the Choi-Jamiolkowski state of the complementary channel of ${\rm Tr}_{F_B}\circ\tilde{\ca{P}}^{\hat{R}\rightarrow RF_B}$. 
Using \req{pusource}-\req{extnonext}, it holds that
\alg{
\tau^{\hat{R}F_B}
&:={\rm Tr}_{F_B}\circ\tilde{\ca{P}}^{\hat{R}'\rightarrow RF_B}(\Phi_{\rm ext}^{\hat{R}\hat{R}'})
\\
&=\pi_{2^c}^{\hat{R}_c}\otm\pi_{2^q}^{\hat{R}_q}\otm\Phi_{2^e}^{\hat{F}_BF_B}.
\laeq{yotteru}
}
A simple calculation yields
\alg{
&
H_{\rm max}(\hat{R}|F_B)_{\ca{C}(\tau)}
=c+q-e
\\
&
H_{\rm max}(\hat{R}_r|F_B\hat{R}_c)_{\ca{C}(\tau)}
=q-e.
}
Using the duality of the conditional smooth entropies, we have
\alg{
&
H_{\rm min}^\epsilon(\hat{R}|\bar{E})_{\Psi_\rho}
=-H_{\rm max}^\epsilon(\hat{R}|B)_{\Psi_\rho},
\\
&
H_{\rm min}^\epsilon(\hat{R}|\bar{E})_{\ca{C}(\Psi_\rho)}
=-H_{\rm max}^\epsilon(\hat{R}_r|BR_c)_{\Psi_\rho}.
}
Noting that $\hat{R}_c$ and $\hat{R}_r$ are isomorphic to $S_c$ and $S_r$, respectively, it follows from the definition \req{narutame} of $\Psi_\rho$ that
\alg{
\Psi_\rho^{\hat{R}B}
\cong
\ca{N}^{A\rightarrow B}(\rho^{SA}),
} 
which leads to
\alg{
&
H_{\rm max}^\epsilon(\hat{R}|B)_{\Psi_\rho}
=H_{\rm max}^\epsilon(S|B)_{\ca{N}(\rho)},
\\
&
H_{\rm max}^\epsilon(\hat{R}_r|B\hat{R}_c)_{\Psi_\rho}
=H_{\rm max}^\epsilon(S_r|BS_c)_{\ca{N}(\rho)}.
}
Substituting all these equalities to  \req{amaimonohoshii} and \req{nomimonogaii}, 
we obtain
\alg{
&
H_I=
-c-q+e-H_{\rm max}^\epsilon(S|B)_{\ca{N}(\rho)}+\log{(d_{S_c}-1)},
\laeq{amaimonohoshiiyou}
\\
&
H_{I\!I}=
-q+e-H_{\rm max}^\epsilon(S_r|BS_c)_{\ca{N}(\rho)}.
\laeq{nomimonogaiiyou}
}
We now use the conditions \req{rightonf23} and \req{rightoff23} to have
\alg{
H_I\geq -\log{\delta_1},
\quad
H_{I\!I}\geq -\log{\delta_2}.
}
Substituting these inequalities to \req{SmExMaMa}, we finally arrive at 
\alg{
\Delta_{s,U}
\leq
\begin{cases}
\sqrt{\delta_1}+\sqrt{\delta_2}+4\epsilon&\!(d_{S_c},d_{S_r}\geq2)
\\
\sqrt{\delta_1}+4\epsilon&\!(q=e=0,d_{S_c}\geq2,d_{S_r}=1)
\\
\sqrt{\delta_2}+4\epsilon&\!(c=0,d_{S_c}=1,d_{S_r}\geq2)
\end{cases}
}
and complete the proof of \rThm{OSDcomp}.
\QED




\begin{figure*}[t]
\begin{center}
\includegraphics[bb={0 0 1266 579}, scale=0.3]{figure14.pdf}
\end{center}
\caption{
The definition of the state $\Psi_{\ca{E}}$ given by \req{ninshou} and its transformation by partial decoupling are depicted.
SD stands for the Stinespring dilation. Note that $R\equiv R_cR_q$, $M\equiv M_cM_q$ and that $|\Phi_{\rm pur}\rangle^{\bar{M}\bar{R}}=|\Phi_{2^{c+q}}\rangle^{MR}|\Phi_{2^e}\rangle^{F_AF_B}$.
}
\label{fig:converse}
\end{figure*}




\section{Proof of The Converse Part (\rThm{converseonehyb})}
\lsec{converse}

We prove the converse part of the capacity theorem (\rThm{converseonehyb}) based on the converse part of randomized partial decoupling (\rThm{converse}).
The proof proceeds as follows: 
First, we construct a state $\Psi_{\ca{E}}$ from the source state $\Phi_{2^c,2^q}'^{MR}$, the resource state $\Phi_{2^e}$, the channel $\ca{N}$ and an encoding operation $\ca{E}$.
Second, we prove that the small error condition \req{tsuzuiteku} implies that a partial-trace operation achieves randomized partial decoupling of the state $\Psi_{\ca{E}}$.
Applying the converse part for randomized partial decoupling, we obtain a set of inequalities represented in terms of the conditional entropies of the state $\Psi_{\ca{E}}$.
Finally, we evaluate the entropies of the state to complete the proof of \rThm{converseonehyb}.




\subsection{Application of the Converse Bound for Randomized Partial Decoupling}

Suppose that a triplet $(c,q,e)$ is achievable within the error $\delta$ for the channel $\ca{N}$.
By definition, there exists an encoding operation $\ca{E}^{MF_A\rightarrow A}$ and a decoding operation $\ca{D}^{BF_B\rightarrow M}$ that satisfy the conditions 
\alg{
\left\|
 \ca{D} \circ \ca{N} \circ \ca{E} (\Phi_{2^c,2^q}'^{MR}\otm\Phi_{2^e}^{F_AF_B})  - \Phi_{2^c,2^q}'^{MR}
\right\|_1
\leq 
\delta
\laeq{tsuzuitekuu}
}
and 
\alg{
\ca{E}^{MF_A\rightarrow A}=\ca{E}^{MF_A\rightarrow A}\circ\ca{C}^{M_c}.
\laeq{encdecclss}
}
Let $\ca{V}_{\ca{D}}^{BF_B\rightarrow MM_0}$ and $\ca{W}_{\ca{N}}^{A\rightarrow BE}$ be the Stinespring dilations of $\ca{D}$ and $\ca{N}$, respectively. 
Let $E_c$ be a $2^c$-dimensional quantum system with a fixed orthonormal basis $\{\ket{j}\}_{j=1}^{2^c}$.
Due to \req{encdecclss}, a Stinespring dilation $\ca{V}_{\ca{E}}^{MF_A\rightarrow AE_0E_c}$ of $\ca{E}$ is given by
\alg{
V_{\ca{E}}^{MF_A\rightarrow AE_0E_c}
=
\sum_{j=1}^{2^c}\ket{j}^{E_c}\bra{j}^{M_c}\otm V_{\ca{E},j}^{M_qF_A\rightarrow AE_0},
\laeq{sorato}
} 
where $V_{\ca{E},j}$ is a linear isometry for each $j$.
We introduce notations $\bar{R}\equiv RF_B$, $\bar{M}\equiv MF_A$, and define a ``purified'' source-resource state $\Phi_{\rm pur}^{\bar{M}\bar{R}}$ (see Figure \ref{fig:converse}) by
\alg{
\!\!\!
\ket{\Phi_{\rm pur}}^{\bar{M}\bar{R}}\!:=\!\frac{1}{\sqrt{2^c}}\sum_{j=1}^{2^c}\ket{j}^{M_c}\ket{j}^{R_c}\ket{\Phi_{2^q}}^{M_qR_q}|\Phi_{2^e}\rangle^{F_AF_B}\!.\!\!
\laeq{kenshou}
}
Denoting $E_0EE_c$ by $\bar{E}$, we define 
a pure state $\ket{\Psi_{\ca{E}}}$ by
\alg{
\ket{\Psi_{\ca{E}}}^{\bar{R}B\bar{E}}
:=W_{\ca{N}}^{A\rightarrow BE}\circ V_{\ca{E}}^{\bar{M}\rightarrow AE_0E_c}|\Phi_{\rm pur}\rangle^{\bar{M}\bar{R}}.
\laeq{ninshou}
}
Note that $\ket{\Psi_{\ca{E}}}$ is classically coherent in $R_cE_c$, in the sense of \req{romanof}.
A purification of the state $\ca{D}\circ\ca{N} \circ \ca{E} (\Phi_{2^c,2^q}'^{MR}\otm\Phi_{2^e}^{F_AF_B})$ is then given by $\ca{V}_{\ca{D}}^{BF_B\rightarrow MM_0}(\Psi_{\ca{E}}^{\bar{R}B\bar{E}})$, with $M_0\bar{E}$ being a purifying system.
Due to the small error condition \req{tsuzuitekuu} and Uhlmann's theorem (see also \rLmm{cohpurifi} in \rApp{TechLmm}), there exists a purification $|\Omega\rangle^{R\bar{E}MM_0}$ of $\Phi_{2^c,2^q}'^{MR}$ in the form of
\alg{
\!\!\!
|\Omega\rangle
=
\frac{1}{\sqrt{2^c}}\sum_{j=1}^{2^c}\ket{j}^{M_c}\ket{j}^{R_c}\ket{j}^{E_c}\ket{\omega_j}^{E_0M_0E}\ket{\Phi_{2^q}}^{M_qR_q},
\!\!
\laeq{irut}
}
where $\ket{\omega_j}$ are normalized pure states, and satisfies
\alg{
\left\|
\ca{V}_{\ca{D}}^{BF_B\rightarrow MM_0} (\Psi_{\ca{E}}^{\bar{R}B\bar{E}}) 
 - \Omega^{R\bar{E}MM_0}
\right\|_1
\leq 
2\sqrt{\delta}. \laeq{garakuta}
}
Tracing out $MM_0$, it follows that
\alg{
\left\|
{\rm Tr}_{F_B}[\Psi_{\ca{E}}^{\bar{R}\bar{E}}]
 - \Omega^{R\bar{E}}
\right\|_1
\leq 
2\sqrt{\delta}. \laeq{kitai}
}
From \req{irut},
we have
\alg{
\Omega^{R\bar{E}}
=
\frac{1}{2^c}\sum_{j=1}^{2^c}\proj{j}^{R_c}\otm\proj{j}^{E_c}\otm\omega_j^{E_0E}\otm\pi_q^{R_q}.
}
Thus, the condition \req{kitai} implies that the map ${\rm id}^R\otm{\rm Tr}_{F_B}:\bar{R}\rightarrow R$ achieves partial decoupling of the state $\Psi_{\ca{E}}^{\bar{R}\bar{E}}$ (see Figure \ref{fig:converse}).




We apply the converse bound for randomized partial decoupling (\rThm{converse}) under the following correspondence:
\alg{
A_c,A_r,R_c,R_r
&
\rightarrow
R_c,R_qF_B,E_c,E_0E
\\
A,B,R,E,C
&\rightarrow
\bar{R},B,\bar{E},R,F_B
\\
\ket{\Psi_{\ca{E}}}^{ABR}
&\rightarrow \ket{\Psi_{\ca{E}}}^{\bar{R}B\bar{E}}
\\
\ca{T}^{A\rightarrow E}
&\rightarrow {\rm id}^R\otm{\rm Tr}_{F_B}
\\
\delta
&
\rightarrow 
2\sqrt{\delta}
}
We choose $\upsilon=0$. Noting that the complementary channel of ${\rm id}^R\otm{\rm Tr}_{F_B}$ is given by ${\rm Tr}_R\otm{\rm id}^{F_B}$,
we obtain
\alg{
&
\!\!
H_{\rm min}^{\lambda}(\bar{R}|\bar{E})_{\Psi_{\ca{E}}}
-H_{\rm min}(B\bar{E}|F_B)_{\ca{C}(\Psi_{\ca{E}})}+\log{d_{R_c}}
\nn\\
&
\quad\quad\quad\quad\quad\quad\quad\quad\quad\quad
\geq
\log{\iota},
\laeq{tokiyotomatte}
\\
&
\!\!
H_{\rm min}^{\lambda'}(\bar{R}|\bar{E})_{\ca{C}(\Psi_{\ca{E}})}
-H_{\rm min}(BE_0E|F_BE_c)_{\ca{C}(\Psi_{\ca{E}})}
\nn\\
&
\quad\quad\quad\quad\quad\quad\quad\quad\quad\quad
\geq
\log{\iota},
\laeq{tokiyougoite}
}
where $\ca{C}$ is the completely dephasing operation on $E_c$.
Substituting $\upsilon=0$ to \req{dfnsmtlambda} and \req{dfnsmtlambdaII}, the smoothing parameters $\lambda$ and $\lambda'$ are given by \req{smlam} and \req{smlamp}, respectively.



\subsection{Evaluation of Entropies}

By using the duality of the smooth conditional entropies \cite{tomamichel2010duality} (see also Lemma 27 and 29 in \cite{wakakuwa2019one}),
the first terms in \req{tokiyotomatte} and \req{tokiyougoite} are calculated to be
\alg{
&
H_{\rm min}^{\lambda}(\bar{R}|\bar{E})_{\Psi_{\ca{E}}}
=
-H_{\rm max}^{\lambda}(\bar{R}|B)_{\Psi_{\ca{E}}},
\\
&
H_{\rm min}^{\lambda'}(\bar{R}|\bar{E})_{\ca{C}(\Psi_{\ca{E}})}
=
-H_{\rm max}^{\lambda'}(R_qF_B|BR_c)_{\Psi_{\ca{E}}}.
}
To calculate the second term in \req{tokiyotomatte}, note that \req{sorato}, \req{kenshou} and \req{ninshou} imply
\alg{
&
\ca{C}^{E_c}(\Psi_{\ca{E}}^{B\bar{E}F_B})
=
\Psi_{\ca{E}}^{B\bar{E}F_B}
\\
&
\quad\quad
=
\ca{W}_{\ca{N}}^{A\rightarrow BE}\circ\ca{V}_{\ca{E}}^{MF_A\rightarrow AE_0E_c}(\Phi_{2^c,2^q}'^M\otm\Phi_{2^e}^{F_AF_B}).
}
Thus, due to the isometric invariance of the conditional max entropy (Lemma 15 in \cite{tomamichel2010duality}), we have
\alg{
&H_{\rm min}(B\bar{E}|F_B)_{\ca{C}(\Psi_{\ca{E}})}
\nn\\
&
=
H_{\rm min}(MF_A|F_B)_{{\Phi_{2^c,2^q}'}\otm\Phi_{2^e}}
\\
&
=
c+q-e.
}
Similarly, the second term in \req{tokiyougoite} is calculated to be
\alg{
&
H_{\rm min}(BE_0E|F_BE_c)_{\ca{C}(\Psi_{\ca{E}})}
\nn\\
&
=
H_{\rm min}(M_qF_A|F_BM_c)_{{\Phi_{2^c,2^q}'}\otm\Phi_{2^e}}
\\
&
=
q-e.
}
Substituting all these equalities into \req{tokiyotomatte} and \req{tokiyougoite}, we arrive at
\alg{
&
\!\!
-H_{\rm max}^{\lambda}(\bar{R}|B)_{\Psi_{\ca{E}}}
-c-q+e+\log{d_{R_c}}
\geq
\log{\iota},
\\
&
\!\!
-H_{\rm max}^{\lambda'}(R_qF_B|BR_c)_{\ca{C}(\Psi_{\ca{E}})}
-q+e
\geq
\log{\iota}.
}



Finally, we calculate the reduced state of $\Psi_{\ca{E}}$ by using  \req{sorato} and \req{ninshou} to obtain
\alg{
\Psi_{\ca{E}}^{RBF_B}
&=\ca{C}^{R_c}(\Psi_{\ca{E}}^{RBF_B})
\\
&
=
\frac{1}{J}\sum_{j=1}^{J}\proj{j}^{R_c}\otm\ca{N}^{A\rightarrow B}(\rho_j^{R_qF_BC}),
}
where
\alg{
\rho_j^{R_qF_BA}:=
\ca{E}^{MF_A\rightarrow A}(\proj{j}^{M_c}\otm\Phi_{2^q}^{M_qR_q}\otm\Phi_{2^e}^{F_AF_B}).
}
We relabel $R_c$ and $R_qF_B$ by $S_c$ and $S_r$, respectively, in which case we have $d_S\leq2^{c+q+e}$.
Noting that $\rho_j^{R_qF_B}$ is the maximally mixed state, we complete the proof of \rThm{converseonehyb}.
\QED




\section{Application to A Memoryless Channel}
\lsec{iid}


In this section, we consider a scenario in which a memoryless quantum channel is used many times to transmit classical and quantum messages with the assistance of shared entanglement.
We apply the direct and converse bounds for the one-shot scenario (\rThm{OSDcomp} and \rthm{converseonehyb}),
and obtain the achievable rate region for the asymptotic limit of infinitely many uses of the channel. 



\subsection{Definition and Result}

The definition of a code for the channel $\ca{N}$ in the scenario of many uses of the channel follows from \rDfn{oneshotcode} by the correspondence $(c,q,e)\rightarrow(nC,nQ,nE)$ and $\ca{N}\rightarrow\ca{N}^{\otm n}$, where $n$ is the number of the uses of the channel.
The three dimensional achievable rate region for $(C,Q,E)$ is defined as follows:

\bdfn{achievableRT}
A rate triplet $(C,Q,E)$ is achievable if, for any $\delta>0$, there exists $n_0\in\mbb{N}$ such that for any $n\geq n_0$, there exists an $(nC,nQ,nE,\delta)$ code for $(\ca{N}^{A\rightarrow B})^{\otm n}$.
The closure of the set of all achievable rate triplets is called the achievable rate region, and is denoted by $\ca{C}(\ca{N})$.
\edfn

\bdfn{rateSN}
Let $\ca{N}^{A\rightarrow B}$ be a quantum channel.
Consider finite dimensional quantum systems $S_c$ and $S_r$. Let $S$ be $S_cS_r$, and $\{\ket{j}\}_{j=1}^{d_{S_c}}$ be a fixed orthonormal basis in $S_c$.
Consider a state $\rho^{SA}$ in the form of
\alg{
\rho^{SA}
=
\sum_{j=1}^{d_{S_c}}p_j\proj{j}^{S_c}\otm\rho_j^{S_rA},
\laeq{proogo}
}
where $\{p_j,\rho_j\}_{j=1}^{d_{S_c}}$ is an ensemble of quantum states on $S_rA$.
Let $\Theta(\ca{N},\rho)\subset\mbb{R}^3$ be the set of all triplets $(C,Q,E)$ that satisfy
\alg{
Q+E
&\leq
 H(S_r|S_c)_\rho,
\laeq{proogo1}\\
C+Q-E
&
\leq
H(S_c)_\rho-H(S|B)_{\ca{N}(\rho)},
\laeq{proogo2}\\
Q-E
&\leq -H(S_r|BS_c)_{\ca{N}(\rho)},
\laeq{proogo3}\\
C,Q,E&\geq0,
\laeq{proogo4}
}
and define
\alg{
\Theta_{\pi}(\ca{N})
:=
\bigcup_{S_c,S_r,\rho}
\Theta(\ca{N},\rho).
}
Here, the union is taken over all finite dimensional systems $S_c$, $S_r$ and all states $\rho$ that is decomposed as \req{proogo}, 
for which we assume that $\rho^{S}$ is the full-rank maximally mixed state on $S$.
The set $\Theta_{\pi}(\ca{N}^{\otm n})$ is defined for any $n\in\mbb{N}$ along the same line.
The regularized version of $\Theta_{\pi}$ is defined by
\alg{
\Theta_{\pi}^{\infty}(\ca{N})
:=
\bigcup_{n=1}^\infty
\frac{1}{n}\Theta_{\pi}(\ca{N}^{\otm n}).
\laeq{sarusaru}
}
\edfn




\bthm{RRegionIID1}
For any $\ca{N}$, it holds that $\ca{C}(\ca{N})=\overline{\Theta_{\pi}^{\infty}(\ca{N})}$.
Here, the overline represents the closure of the set.
\ethm




\subsection{Proof of The Direct Part}

We prove the direct part of \rThm{RRegionIID1}, i.e.
\alg{
\ca{C}(\ca{N})
\supseteq
\overline{\Theta_{\pi}^{\infty}(\ca{N})}.
\laeq{motorcycle}
}
Since $\ca{C}(\ca{N})$ is a closed set, 
it suffices to prove that, for any $n$, a rate triplet $(C,Q,E)$ is achievable if $(C,Q,E)$ is an inner point of $\frac{1}{n}\Theta_{\pi}(\ca{N}^{\otm n})$.
We only consider the case where $n=1$.
It is straightforward to generalize the proof for $n\geq 2$. 
Fix an arbitrary state $\rho$ that satisfies the condition \req{proogo}, in addition to the condition that $\rho^S$ is the full-rank maximally mixed state on $S$. Fix an arbitrary triplet $(C,Q,E)$ that is an inner point of $\Theta(\ca{N},\rho)$.
Then, there exists $\nu>0$ such that
\alg{
Q+E
&\leq
 H(S_r|S_c)_\rho,
\laeq{proogol1}\\
C+Q-E
&
\leq
H(S_c)_\rho-H(S|B)_{\ca{N}(\rho)}-2\nu,
\laeq{proogol2}\\
Q-E
&\leq -H(S_r|BS_c)_{\ca{N}(\rho)}-2\nu.
\laeq{proogol3}
}
Fix arbitrary $\epsilon,\delta>0$ and choose sufficiently large $m$.
Due to the fully-quantum asymptotic equipartition property (\cite{tomamichel2009fully}: see also \rLmm{fqaep} in Appendix \rapp{PSE}), it holds that
\alg{
&
mH(S|B)_{\ca{N}(\rho)}
\geq
H_{\rm max}^{\epsilon}(S^m|B^m)_{\ca{N}(\rho)^{\otm m}}
-m\nu,
\nn\\
&
mH(S_r|BS_c)_{\ca{N}(\rho)}
\geq
H_{\rm max}^{\epsilon}(S_r^m|B^mS_c^m)_{\ca{N}(\rho)^{\otm m}}
-m\nu.
\nn
}
Combining this with \req{proogol1}-\req{proogol3}, and noting that $H(S_r|S_c)_\rho\leq\log{\dim{S_r}}$ and $H(S_c)_\rho=\log{ d_{S_c}}$, we obtain
\alg{
m(Q+E)
&\leq
m\log{\dim{S_r}},
\laeq{majen0}\\
m(C+Q-E)
&
\leq
m\log{d_{S_c}}
-H_{\rm max}^{\epsilon}(S^m|B^m)_{\ca{N}(\rho)^{\otm m}}
\nn\\
&
\quad\quad\quad\quad\quad\quad\quad\quad\quad
-m\nu,
\laeq{majen1}\\
m(Q-E)
&\leq
 -H_{\rm max}^{\epsilon}(S_r^m|B^mS_c^m)_{\ca{N}(\rho)^{\otm m}}
\nn\\
&
\quad\quad\quad\quad\quad\quad\quad\quad\quad
-m\nu.
\laeq{majen2}
}
We choose $m$ sufficiently large so that we have $-m\nu\leq\log{\delta}$. 
Denoting $mC,mQ,mE$ by $c,q,e$, respectively, it follows that
\alg{
q+e
&
\leq
\log{\dim{S_r^m}},
\laeq{tcard}\\
c+q-e
&
\leq
-H_{\rm max}^{\epsilon}(S^m|B^m)_{\ca{N}^{\otm m}(\rho^{\otm m})}
\nn\\
&\quad\quad\quad\quad\quad\quad\quad
+\log{d_{S_c}^m}
+\log{\delta},
\laeq{madema1}\\
q-e
&
\leq
-H_{\rm max}^{\epsilon}(S_r^m|B^mS_c^m)_{\ca{N}^{\otm m}(\rho^{\otm m})}
+\log{\delta}.
\laeq{madema2}
}


We separately consider the cases $d_{S_c}^m\geq\max\{2^c,2\}$ and $d_{S_c}^m<\max\{2^c,2\}$.
For the former case, we simply proceed with \req{madema1} to obtain
\alg{
c+q-e
&
\leq
-H_{\rm max}^{\epsilon}(S^m|B^m)_{\ca{N}(\rho)^{\otm m}}
\nn\\
&\quad\quad\quad\quad
+\log{(d_{S_c}^m-1)}
+\log{2\delta},
\laeq{madema1-2}
}
where we have used $d_{S_c}^m/(d_{S_c}^m-1)\leq2$.
Combining this with \req{tcard} and \req{madema2}, it follows from \rThm{OSDcomp} that there exists a $(c,q,e,\delta')$ code for the channel $\ca{N}^{\otm m}$, where $\delta':=2\sqrt{\sqrt{2\delta}+\sqrt{\delta}+4\epsilon}$.
For the latter case, we introduce a system $S_c'$ such that $d_{S_c}^md_{S_c'}\geq\max\{2^c,2\}$.
Denoting $S_c'S_c^m$ by $\hat{S}_c$ and $\hat{S}_cS_r^m=S_c'S^m$ by $\hat{S}$, we define the state
\alg{
\hat{\rho}_m^{\hat{S}A^m}
:=
\frac{1}{d_{S_c'}}\sum_{j'=1}^{d_{S_c'}}\proj{j'}^{S_c'}\otm\rho^{SA}.
}
Due to the property of the smooth max entropy for product states (see \rLmm{Hmaxprod} in \rApp{PSE}), we have
\alg{
&
H_{\rm max}^{\epsilon}(S^m|B^m)_{\ca{N}^{\otm m}(\rho^{\otm m})}
\geq
H_{\rm max}^{\epsilon}(\hat{S}|B^m)_{\ca{N}^{\otm m}(\rho^{\otm m})}
\nn\\
&
\quad\quad\quad\quad\quad\quad\quad\quad\quad\quad\quad\quad\quad\quad\quad\quad\quad\quad
-\log{d_{\hat{S}_c}},
\nn\\
&
H_{\rm max}^{\epsilon}(S_r^m|B^mS_c^m)_{\ca{N}^{\otm m}(\rho^{\otm m})}
=
H_{\rm max}^{\epsilon}(S_r^m|B^m\hat{S}_c)_{\ca{N}^{\otm m}(\hat{\rho}_m)}.
\nn
}
Thus, Inequalities \req{madema1} and \req{madema2} yield
\alg{
c+q-e
&
\leq
-H_{\rm max}^{\epsilon}(\hat{S}|B^m)_{\ca{N}^{\otm m}(\hat{\rho}_m)}
\nn\\
&\quad\quad\quad\quad\quad\quad\quad
+\log{(d_{\hat{S}_c}-1)}
+\log{2\delta},
\laeq{madema1-3}\\
q-e
&
\leq
-H_{\rm max}^{\epsilon}(S_r^m|B^m\hat{S}_c)_{\ca{N}^{\otm m}(\hat{\rho}_m)}
+\log{\delta},
\laeq{madema2-3}
}
where we have used $d_{\hat{S}_c}/(d_{\hat{S}_c}-1)\leq2$ in the first line.
These two inequalities together with \req{tcard} imply that there exists a $(c,q,e,\delta')$ code for the channel $\ca{N}^{\otm m}$.
Since $\epsilon$ and $\delta$ can be arbitrarily small in both cases, we completes the proof of the direct part \req{motorcycle}.
\QED

\hfill





\subsection{Proof of The Converse Part}

The converse part of \rThm{RRegionIID1} is given by
\alg{
\ca{C}(\ca{N})
\subseteq
\overline{\Theta_{\pi}^{\infty}(\ca{N})},
}
and is proved as follows.
Suppose that a rate triplet $(C,Q,E)$ is an inner point of $\ca{C}(\ca{N})$.
By definition, for any $\delta>0$ and sufficiently large $n$, there exists a $(nC,nQ,nE,\delta)$ code for the channel $\ca{N}^{\otm n}$.
Due to \rThm{converseonehyb}, there exists a quantum system $S\equiv S_cS_r$ and a state in the form of
\alg{
\rho_n^{SA^n}=\frac{1}{d_{S_c}}\sum_{j=1}^{d_{S_c}}\proj{j}^{S_c}\otm\rho_j^{S_rA^n},
\laeq{yapuyapu}
 }
 satisfying $d_{S}\leq2^{n(C+Q+E)}$ and $\rho^{S}=\pi^{S}$,
such that for any $\iota\in(0,1]$, it holds that
\alg{
n(Q+E)
&
\leq
\log{\dim{S_r}}
,
\\
n(C+Q-E)
&
\leq
\log{d_{S_c}}
-H_{\rm max}^{\lambda}(S|B^n)
-
\log{\iota},
\\
n(Q-E)
&
\leq
-H_{\rm max}^{\lambda'}(S_r|B^nS_c)-\log{\iota}.
}
The entropies are for the state $\ca{N}^{\otm n}(\rho_n)$, and the smoothing parameters $\lambda$ and $\lambda'$ are given by \req{smlam} and \req{smlamp}.
Note that we have $\rho^{S}=\ca{N}(\rho)^{S}$ since $\ca{N}$ acts only on $A$.
From the condition $\rho^{S}=\pi^{S}$, we have $\log{\dim{S_r}}=H(S_r|S_c)$.
Using the relation between the smooth max entropy and the von Neumann entropy (\rLmm{contHmax} in \rApp{PSE}), we also have
\alg{
&
H_{\rm max}^{\lambda}(S|B^n)
\geq
H(S|B^n)
-
\eta(\lambda)\log{d_{S}},
\\
&
H_{\rm max}^{\lambda'}(S_r|B^nS_c)
\geq
H(S_r|B^nS_c)
-
\eta(\lambda')\log{d_{S}},
}
where $\eta$ is a function that satisfies $\lim_{x\rightarrow0}\eta(x)=0$.
Combining these inequalities, we arrive at
\alg{
Q+E
&
\leq
\frac{1}{n}H(S_r|S_c)
,
\nn\\
C+Q-E
&
\leq
\frac{1}{n}(\log{d_{S_c}}-H(S|B^n)
+
\eta(\lambda)\log{d_{S}}
-
\log{\iota}),
\nn
\\
Q-E
&
\leq
\frac{1}{n}(
-H(S_r|B^nS_c)
+
\eta(\lambda')\log{d_{S}}-\log{\iota}).
\nn
}
Thus, by taking the limit of $n\rightarrow\infty$ and $\iota,\delta\rightarrow0$, we obtain $(C,Q,E)\in{\Theta_{\pi}^{\infty}(\ca{N})}$ and completes the proof. 
\QED


\subsection{Alternative Expression for Achievable Rate Region}

In \cite{hsieh2010entanglement}, the authors analyzed the coding problem described in \rSec{iid} and obtained a complete characterization of the achievable rate region. 
We prove in \rApp{AEiid} that the achievable rate region given by \rThm{RRegionIID1} coincides the one obtained by \cite{hsieh2010entanglement}.




\subsection{On The Non-Optimality of Time-Sharing Strategies}

Ref.~\cite{devetak2005capacity} addressed the task of simultaneously transmitting classical and quantum messages through a noisy quantum channel without the assistance of shared entanglement, i.e. the case where $E=0$ in \rDfn{achievableRT}.
One strategy for this task is to use $n\lambda$ copies of the channel to transmit classical messages and the remaining $n(1-\lambda)$ copies for quantum messages ($0\leq\lambda\leq1$), which is often referred to as the time sharing strategy.
It achieves the rate pair
\alg{
(C,Q)=(\lambda C(\ca{N}),(1-\lambda)Q(\ca{N})),
\laeq{timesharing}
}
with $C(\ca{N})$ and $Q(\ca{N})$ being the classical and quantum capacities of the channel.
Ref.~\cite{devetak2005capacity} proved that the time sharing strategy is not optimal in general: For a certain channel $\ca{N}$, there exists an achievable rate pair $(C,Q)$ that cannot be represented as \req{timesharing}.
On the contrary, the time sharing strategy is always optimal when the {\it unlimited} amount of shared entanglement is available as a resource.
This is because $t>0$ qubits of quantum communication is converted to $2t$ bits of classical communication and vice versa, in terms of superdense coding and quantum teleportation.

In the case where a {\it limited} amount of shared entanglement is available, it is not clear whether or not the time sharing strategy is optimal. 
This problem may be of independent interest, particularly because it would be closely related to the non-additivity of classical capacity of a quantum channel under the limited amount of entanglement assistance \cite{zhu2017superadditivity}.
We leave this problem as an open question.



\section{Conclusion}
\lsec{cncl}

In this paper, we have analyzed the task of simultaneously transmitting classical and quantum messages through a noisy quantum channel, assisted by a limited amount of shared entanglement.
We have considered a single-shot scenario, and derived the direct and converse bounds for the achievable triplets of classical communication, quantum communication and entanglement consumption.  
We have also applied the one-shot result to the asymptotic scenario of infinitely many uses of the identical channel.
We have obtained the full characterization of the achievable rate region, which coincides with the prior result based on the resource inequality calculus \cite{hsieh2010entanglement}. 
Numerical calculations of the achievable rate region for specific qubit channels are given in \cite{NMcapacity}.
In \cite{hybridQSR}, we also analyze quantum state redistribution for a classical-quantum hybrid source in the one-shot scenario, in terms of randomized partial decoupling.
To investigate relations between these two tasks is left as a future work.


\section*{Acknowledgement}

The authors thank Hayata Yamasaki and Min-Hsiu Hsieh for useful discussions.
E. W. is supported by JSPS KAKENHI Grant No.~18J01329. Y. N. is supported by JST, PRESTO Grant No.~JPMJPR1865, Japan.



\appendices


\section{Technical Lemmas}
\lapp{TechLmm}

We introduce some technical lemmas that are used in the main text.


\blmm{cqerroreq}
Consider two states $\rho$ and $s$ in the form of 
\alg{
&
\rho=\frac{1}{K}\sum_{k=1}^K\proj{k}^X\otm\proj{k}^Y\otm\rho_k^A,
\\
&
\sigma=\frac{1}{K}\sum_{k=1}^K\proj{k}^X\!\otm\!\left(\sum_{k'=1}^Kp(k'|k)\proj{k'}^Y\!\otm\!\sigma_{kk'}^A\right)\!,\!
}
where $\{\ket{k}\}_k$ is an orthonormal basis of $\ca{H}^X$ and $\ca{H}^Y$, $\{p(k'|k)\}_{k'=1}^K$ is a conditional probability distribution, and $\rho_k$ and $\sigma_{kk'}$ are normalized states on $A$ for each $k$ and $k'$.
Suppose that we have
\alg{
&
\frac{1}{K}\sum_{k=1}^K(1-p(k|k))\leq\frac{\delta}{3},
\\
&
\frac{1}{K}\sum_{k=1}^K
\left\|\rho_k-\sum_{k'=1}^Kp(k'|k)\sigma_{kk'}\right\|_1
\leq
\frac{\delta}{3}
}
for $\delta>0$.
Then, it holds that
\alg{
\left\|\rho-\sigma\right\|_1\leq\delta.
}
\elmm

\bprf
Using the properties of the trace distance (see e.g.~ Section 9.1 in \cite{wildetext}), we have
\alg{
&
\left\|\rho-\sigma\right\|_1
\nn\\
&
=
\frac{1}{K}\sum_{k=1}^K
\left\|\proj{k}^Y\otm\rho_k^A-\sum_{k'=1}^Kp(k'|k)\proj{k'}^Y\!\otm\!\sigma_{kk'}^A\right\|_1
\nn\\
&
=
\frac{1}{K}\sum_{k=1}^K
\left\|\rho_k^A-p(k|k)\sigma_{kk}^A\right\|_1
\nn\\
&\quad\quad\quad\quad\quad\quad
+
\frac{1}{K}\sum_{k=1}^K\sum_{k'\neq k}p(k'|k)\left\|\sigma_{kk'}^A\right\|_1
\\
&
\leq
\frac{1}{K}\sum_{k=1}^K
\left\|\rho_k^A-\sum_{k'=1}^Kp(k'|k)\sigma_{kk'}^A\right\|_1
\nn\\
&\quad\quad\quad\quad\quad\quad
+
\frac{2}{K}\sum_{k=1}^K\sum_{k'\neq k}p(k'|k)\left\|\sigma_{kk'}^A\right\|_1
\laeq{acchii}\\
&
=
\frac{1}{K}\sum_{k=1}^K
\left\|\rho_k^A-\sum_{k'=1}^Kp(k'|k)\sigma_{kk'}^A\right\|_1
+
\frac{2}{K}\sum_{k=1}^K(1-p(k|k))
\nn\\
&
\leq
\delta,
}
where Inequality \req{acchii} follows due to the triangle inequality.
\QED
\eprf




\blmm{cohpurifi}
Consider a state $\rho$ on $XA$ and a pure state $\ket{\phi}$ on $XYAB$ that take the forms of
\alg{
&
\rho
=
\sum_kp_k\proj{k}^X\otm\varrho_k^A,
\\
&
\ket{\phi}
=
\sum_k\sqrt{p_k}\ket{k}^X\ket{k}^Y\ket{\varphi_k}^{AB},
}
where $\{p_k\}_k$ is a probability distribution, $\varrho_k$ is a state on $A$ and $\ket{\varphi_k}$ is a pure state on $AB$ for each $k$.
Suppose that 
\alg{
\left\|\phi^{XA}-\rho^{XA}\right\|_1\leq\delta.
}
Then, there exists a purification $\ket{\Psi}^{XYAB}$ of $\rho$ that takes the form of 
\alg{
\ket{\Psi}
=
\sum_k\sqrt{p_k}\ket{k}^X\ket{k}^Y\ket{\psi_k}^{AB},
}
and satisfies
$\|\proj{\Psi}-\proj{\phi}\|_1\leq2\sqrt{\delta}$.
\elmm

\bprf
Any purification $\ket{\Psi'}$ of $\rho$ is represented as
\alg{
\ket{\Psi'}
=
\sum_k\sqrt{p_k}\ket{k}^X\ket{\psi_k'}^{YAB},
}
where $\ket{\psi_k'}$ is a purification of $\varrho_k$ for each $k$.
It follows that
\alg{
|\inpro{\Psi'}{\phi}|
&
=
\left|\sum_kp_k\bra{\psi_k'}^{YAB}\ket{k}^Y\ket{\psi_k}^{AB}\right|
\\
&
\leq
\sum_kp_k\left|\bra{\psi_k'}^{YAB}\ket{k}^Y\ket{\psi_k}^{AB}\right|
\\
&
\leq
\sum_kp_k\max_{\psi_k''}|\inpro{\psi_k''}{\psi_k}|
\\
&
=
\sum_kp_k|\inpro{\psi_k^*}{\psi_k}|,
\laeq{riigon}
}
where we have defined
\alg{
\ket{\psi_k^*}
:=
\argmax_{\ket{\psi_k''}}|\inpro{\psi_k''}{\psi_k}|.
} 
The maximization in the fourth line is taken over all purifications $\ket{\psi_k''}^{AB}$ of $\varrho_k^A$.
We consider a state
\alg{
\ket{\Psi}
=
\sum_k\sqrt{p_k}\ket{k}^X\ket{k}^Y\ket{\psi_k^*}^{AB}.
}
Due to \req{riigon}, it holds that
\alg{
F(\rho^{XA},\phi^{XA})
=
\max_{|\Psi'\rangle}|\inpro{\Psi'}{\phi}|
=
|\inpro{\Psi}{\phi}|
=
F(\ket{\Psi},\ket{\phi}),
\nn
}
where $F$ is the fidelity defined by 
$
F(\sigma,\tau):=\|\sqrt{\sigma}\sqrt{\tau}\|_1
$.
The first equality follows from Uhlmann's theorem \cite{uhlmann1976transition}
By using the relation between the trace distance and the fidelity (see e.g.~Section 9.2.3 in \cite{nielsentext}), we obtain
\alg{
&
1-F(\phi^{XA},\rho^{XA})
\leq
\left\|\phi^{XA}-\rho^{XA}\right\|_1
\leq\delta,
\\
&
\left\|\proj{\Psi}-\proj{\phi}\right\|_1
\leq
2\sqrt{1-F(\ket{\Psi},\ket{\phi})}.
}
Combining these all together, we complete the proof.
\QED
\eprf




\section{Properties of Entropies}
\lapp{PSE}

In this section, we summarize properties of quantum entropies that are used in the proofs of the main results.
Note that the set of positive semidefinite operators, normalized states and subnormalized states are defined by
\begin{align}
&
\ca{P}(\ca{H}) = \{\rho \in {\rm Her}(\ca{H}) : \rho \geq 0 \},
\\
&
\ca{S}_=(\ca{H}) = \{\rho \in \ca{P}(\ca{H}) : \tr [\rho]=1 \},
\\
&
\ca{S}_{\leq}(\ca{H}) = \{\rho \in \ca{P}(\ca{H}) : \tr [\rho] \leq 1 \}.
\end{align}


\begin{lmm}
\label{lmm:duality}
(Definition 14, Equality (6) and Lemma 16 in \cite{tomamichel2010duality})
For any subnormalized pure state $|\psi\rangle$ on system $ABC$, and for any $\epsilon>0$, 
$H_{\rm max}^\epsilon(A|B)_\psi=
-
H_{\rm min}^\epsilon(A|C)_\psi$.
\end{lmm}

\blmm{DPIsmoothmax}
(Corollary of Theorem 18 in  \cite{tomamichel2010duality})
For any state $\rho^{AB}\in\ca{S}_=(\ca{H}^{AB})$, any CPTP map $\ca{E}^{A\rightarrow B}$ and any $\epsilon\geq0$, it holds that
\alg{
H_{\rm max}^{\epsilon}(A|B)_\rho
\leq
H_{\rm max}^{\epsilon}(A|C)_{\ca{E}(\rho)}.
}
\elmm

\begin{lmm}\label{lmm:SE2}
(Corollary of Lemma 20 in \cite{tomamichel2010duality})
For any $\rho^{AB}\in\ca{S}_\leq(\ca{H}^{AB})$, it holds that
\alg{
-\log{d_A}
\leq
H_{\rm max}(A|B)_\rho
-
\log{{\rm Tr}[\rho^{AB}]}
\leq
\log{d_A}.
}
\elmm

\begin{lmm}\label{lmm:SE3}
[Lemma A.2 in \cite{DBWR2010}]
For any $\rho^{AB}\in\ca{S}_=(\ca{H}^{AB})$, $\sigma^{CD}\in\ca{S}_=(\ca{H}^{CD})$ and any $\epsilon,\epsilon'\geq0$, it holds that
\alg{
\!\!
H_{\rm min}^{\epsilon+\epsilon'}(AC|BD)_{\rho\otm\sigma}
\geq
H_{\rm min}^{\epsilon}(A|B)_{\rho}
\!+\!
H_{\rm min}^{\epsilon'}(C|D)_{\sigma}.
\!\!\!
}
\elmm



\blmm{fqaep}
(Theorem 1 in \cite{tomamichel2009fully})
For any $\rho\in\ca{S}_=(\ca{H}^{AB})$ and $0<\epsilon<1$, it holds that
\alg{
\lim_{n\rightarrow\infty}\frac{1}{n}H_{\rm max}^\epsilon(A^n|B^n)_{\rho^{\otm n}}
=
H(A|B)_\rho.
}
\elmm

\blmm{maxvNub}
(Corollary of Lemma 2 in \cite{tomamichel2009fully})
For any $\rho\in\ca{S}_=(\ca{H}^{AB})$, it holds that
\alg{
H_{\rm max}(A|B)_\rho\geq H(A|B)_\rho.
}
\elmm


\blmm{Hmaxprod}
For any $\rho^{AB}\in\ca{S}_=(\ca{H}^{AB})$ and $\xi^C\in\ca{S}_=(\ca{H}^{C})$, it holds that
\alg{
H_{\rm max}^{\epsilon}(A|BC)_{\rho\otm\xi}
&
=
H_{\rm max}^{\epsilon}(A|B)_{\rho},
\laeq{Hmaxprodeq}
\\
H_{\rm max}^{\epsilon}(AC|B)_{\rho\otm\xi}
&
\leq
H_{\rm max}^{\epsilon}(A|B)_{\rho}
+
\log{d_C}.
\laeq{Hmaxprodineq}
}
\elmm

\bprf
To prove Equality \req{Hmaxprodeq}, define an operation $\ca{E}_\xi^{B\rightarrow BC}$ by $\ca{E}_\xi(\tau^B)=\tau^B\otm\xi^C$.
Due to the monotonicity of the smooth max entropy (\rLmm{DPIsmoothmax}) under $\ca{E}_\xi^{B\rightarrow BC}$ and the partial trace operation, it holds that
\alg{
&
H_{\rm max}^{\epsilon}(A|B)_{\rho}
\geq
H_{\rm max}^{\epsilon}(A|BC)_{\ca{E}_\xi(\rho)}
\nn\\
&\quad
=
H_{\rm max}^{\epsilon}(A|BC)_{\rho\otm\xi}
\geq
H_{\rm max}^{\epsilon}(A|B)_{\rho},
}
which implies \req{Hmaxprodeq}.
To prove \req{Hmaxprodineq}, note that \rLmm{SE3} and the duality relation (\rLmm{duality}) imply, for any $\eta\in\ca{S}_=(\ca{H}^D)$,
\alg{
&
H_{\rm max}^{\epsilon+\epsilon'}(AC|BD)_{\rho\otm\xi\otm\eta}
\nn\\
&
\quad
\leq
H_{\rm max}^{\epsilon}(A|B)_{\rho}
+
H_{\rm max}^{\epsilon'}(C|D)_{\xi\otm\eta}.
}
We particularly choose $\epsilon'=0$.
Due to Inequality \req{Hmaxprodeq}, the L.H.S. is equal to $H_{\rm max}^{\epsilon}(AC|B)_{\rho\otm\xi}$.
From \rLmm{SE2}, the second term in the R.H.S. is bounded as
$
H_{\rm max}(C|D)_{\xi\otm\eta}
\leq
\log{d_C},
$
which completes the proof.
\QED
\eprf


\blmm{tomerare}
For any $\rho^{AB}\in\ca{S}_=(\ca{H}^{AB})$ and $\epsilon\geq0$, it holds that
\alg{
-\log{d_A}
\leq
H_{\rm max}^\epsilon(A|B)_\rho
-
\log{(1-2\epsilon)}.
}
\elmm

\bprf
Let $\hat{\rho}^{AB}\in\ca{B}^\epsilon(\rho)$ be such that $H_{\rm max}^\epsilon(A|B)_\rho=H_{\rm max}(A|B)_{\hat\rho}$.
Due to \rLmm{SE2}, it holds that
\alg{
-\log{d_A}
\leq
H_{\rm max}(A|B)_{\hat{\rho}}
-
\log{{\rm Tr}[\hat{\rho}^{AB}]}.
\laeq{katachino}
}
Using the triangle inequality for the trace distance, we have
\alg{
{\rm Tr}[\hat{\rho}^{AB}]
=\|\hat{\rho}^{AB}\|_1
\geq
\|\rho^{AB}\|_1-\|\rho^{AB}-\hat{\rho}^{AB}\|_1
\geq
1-2\epsilon,
\nn
}
where the last inequality follows from the relation between the trace distance and the purified distance \req{relTDPD}.
Substituting this to \req{katachino}, we complete the proof.
\QED
\eprf


\blmm{contHmax}
For any $0\leq\epsilon<1$ and any state $\rho\in\ca{S}_=(\ca{H}^{AB})$, it holds that
\alg{
H_{\rm max}^\epsilon(A|B)_\rho
\geq
H(A|B)_\rho
-
\eta(\epsilon)\log{d_A},
\laeq{contHmax}
}
where $\eta$ is a function that satisfies $\lim_{x\rightarrow0}\eta(x)=0$ and is independent of the dimensions of the systems.
\elmm

\bprf
Let $\hat{\rho}\in\ca{B}^\epsilon(\rho)$ be a subnormalized state such that $H_{\rm max}^\epsilon(A|B)_\rho=H_{\rm max}^\epsilon(A|B)_{\hat{\rho}}$.
From \rLmm{maxvNub}, it holds that
\alg{
H_{\rm max}(A|B)_{\hat{\rho}/{\rm Tr}[\hat{\rho}]}\geq H(A|B)_{\hat{\rho}/{\rm Tr}[\hat{\rho}]}.
}
Thus, Inequality \req{contHmax} follows due to the Alicki-Fannes inequality (\cite{alicki04}, see also Inequality (89) in \cite{wakakuwa2017coding}).
\QED
\eprf


\blmm{entpinbo}
Let $\{\Pi_m\}_{m=1}^M$ be a complete set of orthogonal projectors on a finite dimensional Hilbert space $\ca{H}^A$, and let $X$ be a quantum system with a fixed orthonormal basis $\{\ket{m}\}_{m=1}^M$.
Consider a map $\ca{E}:A\rightarrow XA$ defined by
\alg{
\ca{E}(\cdot):=\sum_{m=1}^M\proj{m}^X\otm\Pi_m(\cdot)\Pi_m^A.
}
For any state $\rho$ on system $AB$, it holds that
\alg{
&
H(A|B)_\rho
\leq
H(XA|B)_{\ca{E}(\rho)},
\\
&
H(A|BX)_{\ca{E}(\rho)}
\leq
H(A|B)_\rho
+\log{M}.
}
\elmm



\bprf
The first inequality follows from the isometric invariance of the conditional quantum entropy and its monotonicity under the completely dephasing operation (see e.g. Corollary 11.9.4 in \cite{wildetext}).
Note that the map $\ca{E}$ is represented as $\ca{E}=\ca{C}^X\circ\ca{V}^{A\rightarrow XA}$, where $V$ is a linear isometry defined by $V:=\sum_{m=1}^M\ket{m}^X\otm\Pi_m^A$ and $\ca{C}$ is the completely dephasing operation on $X$ with respect to the basis $\{\ket{m}\}_{m=1}^M$.
To prove the second inequality, let $X'$ be a $M$-dimensional Hilbert space with a fixed orthonormal basis $\{\ket{m}\}_{m=1}^M$.
Define a linear isometry $V:\ca{H}^A\rightarrow \ca{H}^X\otm\ca{H}^{X'}\otm\ca{H}^A$ by
\alg{
V:=\sum_{m=1}^M\ket{m}^{X}\otm\ket{m}^{X'}\otm\Pi_m^A.
}
A Stinespring dilation of the map $\ca{E}$ is given by $\ca{E}={\rm Tr}_{X'}\circ\ca{V}$.
It holds that
\alg{
&
H(A|B)_\rho
\\
&
=
H(XX'A|B)_{\ca{V}(\rho)}
\\
&
\geq
H(X|B)_{\ca{V}(\rho)}+
H(A|BX)_{\ca{V}(\rho)}
\nn\\
&\quad\quad\quad\quad+
H(X'|ABX)_{\ca{V}(\rho)}
\\
&
\geq
H(A|BX)_{\ca{V}(\rho)}-\log{\dim{X}}
\\
&
=
H(A|BX)_{\ca{E}(\rho)}-\log{M},
}
where the first line follows from the isometric invariance of the conditional quantum entropy, the second line from the chain rule, the third line due to $H(X|B)_{\ca{V}(\rho)}\geq0$ and $H(X'|ABX)_{\ca{V}(\rho)}\geq-\log{\dim{E_0}}$, and the last line from  $\ca{V}(\rho)^{AB}=\ca{E}(\rho)$.
\QED
\eprf


\blmm{fano}
Let $\{p_x,\rho_x\}_{x\in\ca{X}}$ be an ensemble of states on system $A$, and suppose that there exists a POVM $\{M_x\}_{x\in\ca{X}}$ such that
\alg{
\sum_{x\in\ca{X}}p_x{\rm Tr}[M_x\rho_x]
\geq
1-\epsilon.
\laeq{caution}
}
Then, for the state
\alg{
\rho^{XA}:=\sum_{x\in\ca{X}}p_x\proj{x}^X\otm\rho_x^A,
}
it holds that
\alg{
H(X|A)_\rho\leq\eta(\epsilon)\log{|\ca{X}|},
}
where $\eta$ is a function that satisfies $\lim_{\epsilon\rightarrow0}\eta(\epsilon)=0$ and is independent of the dimensions of the systems.
\elmm

\bprf
Let $\hat{X}$ be a quantum system with a fixed orthonormal basis $\{\ket{x}\}_{x\in\ca{X}}$, and define a CPTP map $\ca{M}:A\rightarrow \hat{X}$ by $\ca{M}(\cdot):={\rm Tr}[M_x(\cdot)]\proj{x}^{\hat{X}}$.
It follows that
\alg{
\tilde{\rho}^{X\hat{X}}
&:={\rm id}^X\otm\ca{M}(\rho^{XA})
\\
&=
\sum_{x,x'\in\ca{X}}p_x\proj{x}^X\otm p_{x'|x}\proj{x'}^{\hat{X}},
}
where $\{p_{x'|x}\}_{x'\in\ca{X}}$ is a conditional probability distribution defined by $p_{x'|x}={\rm Tr}[M_{x'}\rho_x]$.
Thus, due to the monotonicity of the conditional quantum entropy, we have
\alg{
H(X|A)_\rho
\leq
H(X|\hat{X})_{\tilde{\rho}}.
\laeq{kaikai}
}
With a slight abuse of notation, let $(X,\hat{X})$ be a pair of random variables that takes values in $\ca{X}\times\ca{X}$ according to a joint probability distribution $P\equiv\{p(x,\hat{x})\}$, where $p(x,\hat{x}):=p_xp_{x'|x}$.
Due to the condition \req{caution}, it holds that
\alg{
\!
P(X\neq\hat{X})
=
\sum_{\substack{x,x'\in\ca{X}\\x\neq x'}}p(x,\hat{x})
=
\sum_{x\in\ca{X}}p_x(1-p_x)
\leq
\epsilon.
\!
}
Thus, due to Fano's inequality (see e.g.~Theorem 2.10.1 in \cite{cover05}), it follows that
\alg{
H(X|\hat{X})_{\tilde{\rho}}
=
H(X|\hat{X})_{P}
\leq
\eta(\epsilon)\log{|\ca{X}|}
}
Combining this with \req{kaikai}, we complete the proof.
\QED
\eprf



\section{Method of Types and Type Subspaces}
\lapp{typesub}



In this section, we briefly review the definitions and properties of types and type subspaces.
For the details, see e.g. Section 13.7 and 14.3 in \cite{wildetext}.
The properties of type subspaces presented in this section will be used in \rApp{AEiid} to prove the equivalence between the achievable rate region for the asymptotic limit given by \rThm{RRegionIID1} in the main text and the one obtained in Ref.~\cite{hsieh2010entanglement}


Let $\ca{X}$ be a finite alphabet and $n\in\mbb{N}$.
A probability distribution $\{t(x)\}_{x\in\ca{X}}$ is called a {\it type of length $n$} if $nt(x)\in\mbb{N}$ for all $x\in\ca{X}$.
Let $x^n\equiv x_1\cdots x_n$ be a sequence of variables of length $n$ such that $x_i\in\ca{X}$ for each $i$. 
The sequence $x^n$ is said to be of type $t$ if
\alg{
\frac{1}{n}N(x|x^n)=t(x)
}
for all $x\in\ca{X}$, where  $N(x|x^n)$ is the number of the symbol $x\in\ca{X}$ that appears in the sequence $x^n$.
The {\it type class} corresponding to the type $t$ of length $n$, which we denote by $T_t^n$, is the set of all sequences whose type is $t$.
Let $\mathfrak{T}(\ca{X},n)$ be the set of all types of length $n$ on the alphabet $\ca{X}$.
It holds that
\alg{
T_t^n\cap T_{t'}^n=\emptyset\;(t\neq t'),
\quad
\ca{X}^n=\bigcup_{t\in\mathfrak{T}(\ca{X},n)}T_t^n.
\laeq{kaii}
}
The size of $\mathfrak{T}(\ca{X},n)$ is bounded above by a polynomial function of $n$ as
\alg{
|\mathfrak{T}(\ca{X},n)|
\leq (n+1)^{|\ca{X}|}.
\laeq{shinju}
}
By definition, any two sequences in the same type class are transformed with each other by permuting the elements. That is, for any $t\in\mathfrak{T}(\ca{X},n)$ and $x^n,x'^n\in T_t^n$, there exists a permutation $s$ such that $x_i'=x_{s(i)}$ for any $1\leq i\leq n$.

Let $\ca{H}$ be a Hilbert space with a fixed orthonormal basis $\{\ket{x}\}_{x\in\ca{X}}$.
For any $n\in\mbb{N}$, the {\it type subspace} corresponding to the type $t$ of length $n$ is defined by 
\alg{
\ca{H}_t^n:={\rm span}\{\ket{x^n} :  x^n\in T_t^n\}
\subseteq\ca{H}^{\otm n},
}
and the {\it type projector} is defined by
\alg{
\Pi_t^n:=\sum_{ x^n\in T_t^n}\proj{x^n}.
}
It follows from \req{kaii} that
\alg{
\Pi_t^n\Pi_{t'}^n=0\;(t\neq t'),
\quad
I=\sum_{t\in\mathfrak{T}(\ca{X},n)}\Pi_t^n.
}
Consider a state $\rho\in\ca{S}_=(\ca{H})$ and suppose that the eigen decomposition of $\rho$ is given by $\rho=\sum_{x\in\ca{X}}p_x\proj{x}$. 
By definition, it holds that
\alg{
\Pi_t^n\rho^{\otm n}\Pi_t^n
=
\sum_{ x^n\in T_t^n}p_{x^n}\proj{x^n}
=
q_t\Pi_t^n,
}
where
\alg{
q_t
:=
\prod_{x\in\ca{X}}p_x^{nt(x)}.
}
The projectors $\{\Pi_t^n\}_{t\in\mathfrak{T}(\ca{X},n)}$ are called the {\it type projectors corresponding to the state $\rho^{\otm n}$.}


Consider an ensemble $\{p_j,\rho_j\}_{j=1}^J$, where $J\in\mbb{N}$, $\rho_j\in\ca{S}_=(\ca{H})$, and fix arbitrary $n\in\mbb{N}$.
For any type $t$ of length $n$ over $[J]:=\{1,\cdots,J\}$, define a sequence of length $n$ by
\begin{eqnarray}
{\bm j}_t
:=
\underbrace{11\cdots1}_{nt(1)}
\underbrace{22\cdots2}_{nt(2)}
\cdots 
\underbrace{JJ\cdots J}_{nt(J)}.
\laeq{dfnbfjt}
\end{eqnarray}
For each $j\in[J]$, 
let $\{\Pi_{\nu_j}\}_{\nu_j\in\mathfrak{T}(\ca{X},nt(j))}$ be the set of type projectors corresponding to the state ${\rho_j}^{\otm nt(j)}$. 
We define a projector on ${\mathcal H}^{\otimes n}$ by
\alg{
\Pi_{\vec{\nu}|{\bm j}_t}
:=
\Pi_{\nu_1}\otm\cdots\otm\Pi_{\nu_J}
}
for each $\vec{\nu}:=\nu_1\cdots\nu_J\in\bigotimes_{j=1}^J\mathfrak{T}(\ca{X},nt_j)$.
Using \req{shinju}, the number of conditional type projectors is bounded above by
\alg{
\prod_{j=1}^J
|\mathfrak{T}(\ca{X},nt(j))|
\leq
\prod_{j=1}^J
(nt(j)+1)^{|\ca{X}|}
\leq
(n+1)^{2{\rm dim}\ca{H}}.
}



As mentioned above, for any $t\in\mathfrak{T}(J,n)$ and ${\bm j}\in T_t^n$, there exists a permutation $s$ such that $j_i=j_{t,s(i)}$ for each $1\leq i\leq n$,
where $j_{t,s(i)}$ is the $s(i)$-th element of the sequence ${\bm j}_t$ defined by \req{dfnbfjt}.
Let $P_s$ be a unitary that acts on $\ca{H}^{\otm n}$ as
\alg{
P_s(\ket{\varphi_1}\otm\cdots\otm\ket{\varphi_n})
=
\ket{\varphi_{s(1)}}\otm\cdots\otm\ket{\varphi_{s(n)}}.
}
We define the set of {\it conditional type projectors} $\{\Pi_{\vec{\nu}|{\bm j}}\}_{\vec{\nu}}$ on ${\mathcal H}^{\otimes n}$ by
\alg{
\Pi_{\vec{\nu}|{\bm j}}
=
P_s\Pi_{\vec{\nu}|{\bm j}_t}P_s^\dagger.
}
%Thus, we may construct the set of type projectors $\{\Pi_{\vec{\nu}|{\bm j}}\}_{\vec{\nu}}$ for ${\bm j}$ from that for ${\bm j}_t$ simply by permuting the systems if ${\bm j}\in T_t^n$. 
By definition, it holds that
\alg{
{\rm Tr}[\Pi_{\vec{\nu}|{\bm j}}\rho_{{\bm j}}]
=
{\rm Tr}[P_s\Pi_{\vec{\nu}|{\bm j}}P_s^\dagger P_s\rho_{{\bm j}}P_s^\dagger]
=
{\rm Tr}[\Pi_{\vec{\nu}|{\bm j}_t}\rho_{{\bm j}_t}].
}
Note that the numbers of conditional type projectors for the sequences ${\bm j}$ and ${\bm j}'$ are the same if the two sequences belong to the same type class.



\section{Alternative Expression for Achievable Rate Region}
\lapp{AEiid}

In \cite{hsieh2010entanglement}, the authors analyzed the coding problem described in \rSec{iid} and obtained a complete characterization of the achievable rate region. 
In this section, we provide an alternative proof for their results based on \rThm{RRegionIID1}.


\bdfn{rateSN2}
Let $\ca{N}^{A\rightarrow B}$ be a quantum channel. 
Consider finite dimensional quantum systems $S_c$ and $S_r$, the former of which is equipped with a fixed orthonormal basis $\{\ket{j}\}_{j=1}^{d_{S_c}}$.
We denote $S_cS_r$ by $S$.
Consider a state $\rho^{SA}$ in the form of
\alg{
\rho^{SA}
=
\sum_{j=1}^{d_{S_c}}p_j\proj{j}^{S_c}\otm\rho_j^{S_rA},
\laeq{proogostar}
}
where $\{p_j,\rho_j\}_{j=1}^{d_{S_c}}$ is an ensemble of quantum states on $S_rA$.
Let $\Lambda(\ca{N},\rho)\in\mbb{R}^3$ be the set of all triplets $(C,Q,E)$ that satisfy
\alg{
C+2Q
&\leq I(S:B)_{\ca{N}(\rho)},
\laeq{proogo21}\\
C+Q-E
&
\leq
H(S_c)_\rho-H(S|B)_{\ca{N}(\rho)},
\laeq{proogo22}\\
Q-E
&\leq -H(S_r|BS_c)_{\ca{N}(\rho)},
\laeq{proogo23}\\
C,Q,E&\geq0,
}
and define
\alg{
\Lambda_p(\ca{N})
:=
\bigcup_{S_c,S_r,\rho}
\Lambda(\ca{N},\rho).
}
Here, the union is taken over all $S_c$, $S_r$ and $\rho$ that is decomposed as \req{proogostar}, 
for which we assume that $\rho_j$ is a pure state on $S_rA$ for each $j$.
The set $\Lambda_p(\ca{N}^{\otm n})$ is defined for any $n\in\mbb{N}$ along the same line.
The regularized version of $\Lambda_p$ is defined by
\alg{
\Lambda_p^{\infty}(\ca{N})
:=
\bigcup_{n=1}^\infty
\frac{1}{n}\Lambda_p(\ca{N}^{\otm n}).
\laeq{sarusarusara}
}
\edfn

\bthm{RRegionIID2}
(Theorem 1 in \cite{hsieh2010entanglement})
For any quantum channel $\ca{N}^{A\rightarrow B}$, it holds that $\ca{C}(\ca{N})=\overline{\Lambda_p^{\infty}(\ca{N})}$. 
Here, the overline represents the closure of the set.
\ethm

In the following, we provide an alternative proof for \rThm{RRegionIID2} based on \rThm{RRegionIID1}, by showing that 
\alg{
\overline{\Theta_{\pi}^{\infty}(\ca{N})}
=
\overline{\Lambda_p^{\infty}(\ca{N})}.
\laeq{SfeqR}
}
To this end, 
we define
\alg{
\Theta(\ca{N})
:=
\bigcup_{S_c,S_r,\rho}
\Theta(\ca{N},\rho),
}
where $\Theta(\ca{N},\rho)\in\mbb{R}^3$ is given by \rDfn{rateSN}. 
Here, the union is taken over all $S$ and $\rho$ that is in the form of \req{proogo}.
We do not require the condition that $\rho^S$ is the full-rank maximally mixed state on $S$.
We also define
\alg{
\Lambda(\ca{N})
:=
\bigcup_{S_c,S_r,\rho}
\Lambda(\ca{N},\rho),
}
where $\Lambda(\ca{N},\rho)$ is defined in \rDfn{rateSN2}.
The union is taken over all $S$ and $\rho$ that is decomposed as \req{proogostar}, 
but we do not require that $\rho_j^{S_rA}$ are pure states.
Note that the only difference between $\Theta(\ca{N},\rho)$ and $\Lambda(\ca{N},\rho)$ is in one of the inequality conditions for $(C,Q,E)$. That is, we require $Q+E\leq H(S_r|S_c)_\rho$ for $\Theta(\ca{N},\rho)$ and $C+2Q\leq I(S:B)_{\ca{N}(\rho)}$ for $\Lambda(\ca{N},\rho)$ (see Inequalities \req{proogo1} and \req{proogo21}).
The two sets are regularized into
\alg{
&
\Theta^{\infty}(\ca{N})
:=
\bigcup_{n=1}^\infty
\frac{1}{n}\Theta(\ca{N}^{\otm n}),
\\
&
\Lambda^{\infty}(\ca{N})
:=
\bigcup_{n=1}^\infty
\frac{1}{n}\Lambda(\ca{N}^{\otm n}).
\laeq{sarusaru2}
}
In the following subsections,
we prove the following lemma that implies \req{SfeqR}:
\bprp{SfeqR2}
For any quantum channel $\ca{N}$, it holds that
\alg{
\overline{\Theta_{\pi}^{\infty}(\ca{N})}
=
\overline{\Theta^{\infty}(\ca{N})}
=
\overline{\Lambda^{\infty}(\ca{N})}
=\overline{\Lambda_p^{\infty}(\ca{N})}.
\laeq{SfeqR2}
}
\eprp


For the simplicity of notations, we denote $d_{S_c}$ by $J$.



\subsection{Proof of $\overline{\Theta_{\pi}^{\infty}(\ca{N})}=\overline{\Theta^{\infty}(\ca{N})}$}


By definition, it is straightforward to verify that $\overline{\Theta_{\pi}^{\infty}(\ca{N})}\subseteq\overline{\Theta^{\infty}(\ca{N})}$.
Thus, 
it suffices to prove the converse relation $\overline{\Theta_{\pi}^{\infty}(\ca{N})}\supseteq\overline{\Theta^{\infty}(\ca{N})}$.
We prove this by showing that
$\overline{\Theta_{\pi}^{\infty}(\ca{N})}\supseteq\frac{1}{n}\Theta(\ca{N}^{\otm n},\rho)$ for any $n$ and any state $\rho$
in the form of
\alg{
\rho^{SA^n}
=
\sum_{j=1}^Jp_j\proj{j}^{S_c}\otm\rho_j^{S_rA^n},
}
where we do not require that $\rho^{S}=\pi^{S}$.
We only consider the case where $n=1$. It is straightforward to generalize the proof for $n\geq2$.
 



\subsubsection{Construction of States}

Fix an arbitrary $\epsilon>0$ and choose sufficiently large $m$.
Let $\mathfrak{T}(J,m)$ be the set of all types of length $m$ over $[J]:=\{1,\cdots,J\}$, and $T_t^n\subset [J]^{\times m}$ be the type class corresponding to a type $t\in\mathfrak{T}(J,m)$ (see \rApp{typesub} for the definitions and properties of types and type subspaces). 
For any $t\in\mathfrak{T}(J,m)$ and $\vec{j}\in T_t^n$, 
let $\{\Pi_{\varsigma|\vec{j}}\}_{\varsigma=1}^{\theta(\vec{j})}$
be the set of conditional type projectors on $(\ca{H}^{S_r})^{\otm m}$ with respect to $\rho_{\vec{j}}^{S_r^m}$.
Here, $\theta(\vec{j})$ is the number of the conditional type subspaces.
Note that $\theta(\vec{j})=\theta(\vec{j}')$ if $\vec{j}$ and $\vec{j}'$ belong to the same type class.
Thus, we will denote $\theta(\vec{j})$ also as $\theta(t)$ if $\vec{j}\in T_t^n$. 
Define probability distributions $\{p_t\}_{t\in\mathfrak{T}(J,m)}$ and $\{p_{\varsigma|\vec{j}}\}_{\varsigma=1}^{\theta(\vec{j})}$ for each $\vec{j}\in J^{\times m}$ by
\alg{
p_t:=\sum_{\vec{j}\in T_t^n}p_{\vec{j}},
\quad
p_{\varsigma|\vec{j}}
:=
{\rm Tr}[\Pi_{\varsigma|\vec{j}}\rho_{\vec{j}}^{S_r^m}].
} 
Due to the properties of the conditional type projectors, it holds that $p_{\varsigma|\vec{j}}=p_{\varsigma|\vec{j}'}$ for any $\vec{j}$ and $\vec{j}'$ in the same type set $t$, which we denote by $p_{\varsigma|t}$. 
We define a probability distribution $\{p_{\varsigma,t}\}_{\varsigma\in[\theta(t)],t\in\mathfrak{T}(J,m)}$ by 
\alg{
p_{\varsigma,t}=p_{t}\cdot p_{\varsigma|t}.
\laeq{rrt}
}



Let $Y$ and $Y'$ be quantum systems with dimensions $|\mathfrak{T}(J,m)|$ and $\theta_*:=\max_{t\in\mathfrak{T}(J,m)}|\theta(t)|$, respectively.
Consider states
\alg{
\rho_{\varsigma,\vec{j}}^{S_r^mA^m}
&
:=
p_{\varsigma|\vec{j}}^{-1}\Pi_{\varsigma|\vec{j}}^{S_r^m}\rho_{\vec{j}}^{S_r^mA^m}\Pi_{\varsigma|\vec{j}}^{S_r^m},
\laeq{cofeemilk}\\
\rho_{\varsigma,t}^{S^mA^m}
&
:=
\frac{1}{| T_t^n|}
\sum_{\vec{j}\in T_t^n}\proj{\vec{j}}^{S_c^m}
\otm
\rho_{\varsigma,\vec{j}}^{S_r^mA^m}
\laeq{110}
}
and define
\alg{
&
\rho_m^{YY'S^mA^m}
:=
\sum_{t\in\mathfrak{T}(J,m)}
p_t
\proj{t}^Y
\otm
\frac{1}{| T_t^n|}
\sum_{\vec{j}\in T_t^n}\proj{\vec{j}}^{S_c^m}
\nn\\
&
\quad\quad\quad\quad\quad\quad\quad
\otm
\sum_{\varsigma=1}^{\theta(\vec{j})}p_{\varsigma|\vec{j}}
\proj{\varsigma}^{Y'}\otm
\rho_{\varsigma,\vec{j}}^{S_r^mA^m}.
\laeq{111}
}
By the definition of the type subspaces, it is straightforward to verify that $\rho_{\varsigma,t}^{S^m}$ has a flat distribution on its support, that is,
\alg{
\rho_{\varsigma,t}^{S^m}
=
\frac{1}{| T_t^n|}
\sum_{\vec{j}\in T_t^n}\proj{\vec{j}}^{S_c^m}
\otm
\frac{\Pi_{\varsigma,\vec{j}}^{S_r^m}}{{\rm Tr}[\Pi_{\varsigma,\vec{j}}]}.
\laeq{flDs}
}
By using \req{rrt} and \req{110}, the state \req{111} is rewritten into
\alg{
\rho_m^{YY'S^mA^m}
=
\sum_{t\in\mathfrak{T}(J,m)}\sum_{\varsigma=1}^{\theta(t)}
p_{\varsigma,t}\proj{\varsigma,t}^{YY'}
\otm
\rho_{\varsigma,t}^{S^mA^m}.
\nn
}
For the later convenience, we introduce a map $\ca{E}_{\vec{j}}:S_r^m\rightarrow Y'S_r^m$ by 
\alg{
\ca{E}_{\vec{j}}(\cdot)=
\sum_{\varsigma=1}^{\theta(\vec{j})}
\proj{\varsigma}^{Y'}
\otm
\Pi_{\varsigma|\vec{j}}^{S_r^m}(\cdot)\Pi_{\varsigma|\vec{j}}^{S_r^m}
}
for each $\vec{j}$, which leads to
\alg{
\sum_{\varsigma=1}^{\theta(\vec{j})}p_{\varsigma|\vec{j}}
\proj{\varsigma}^{Y'}\otm
\rho_{\varsigma,\vec{j}}^{S_r^mA^m}
=
\ca{E}_{\vec{j}}(\rho_{\vec{j}}^{S_r^mA^m}).
\laeq{yasa}
}
A useful relation which follows from the properties of the conditional type projectors is that for any $\vec{j}$ and $\vec{j}'$ in the same type set, there exits a permutation $s$ such that
\alg{
\rho_{\varsigma,\vec{j}'}^{S_r^m}
=
\ca{P}_s^{S_r^m}
(\rho_{\varsigma,\vec{j}}^{S_r^m})
\laeq{koori}
}
and
\alg{
&
(\ca{N}^{A\rightarrow B})^{\otm m}(\rho_{\varsigma,\vec{j}'}^{S_r^mA^m})
\nn\\
&
\quad\quad
=
\ca{P}_s^{S_r^m}
\otm
\ca{P}_s^{B^m}
\circ(\ca{N}^{A\rightarrow B})^{\otm m}
(\rho_{\varsigma,\vec{j}}^{S_r^mA^m}).
\laeq{mizu}
}
These properties will be used in the following subsections to calculate the entropies of the states.


\subsubsection{Calculation of Entropies}

Let us calculate entropies and mutual informations of the state defined above.
We use the definition of the state $\rho_m$ given by \req{111} and the fact that $(\rho^{S_cA})^{\otm m}=\rho_m^{S_c^mA^m}$.
Due to the properties of the quantum mutual information, we have
\alg{
&
mI(S_c:B)_{\ca{N}(\rho)}
\\
&
=
I(S_c^m:B^m)_{\ca{N}^{\otm m}(\rho^{\otm m})}
\\
&=
I(S_c^m:B^m)_{\ca{N}^{\otm m}(\rho_m)}
\\
&\leq
I(YY'S_c^m:B^m)_{\ca{N}^{\otm m}(\rho_m)}
\\
&
=
I(YY':B^m)_{\ca{N}^{\otm m}(\rho_m)}
\nn\\
&\quad\quad\quad\quad
+
I(S_c^m:B^m|YY')_{\ca{N}^{\otm m}(\rho_m)}
\\
&\leq
I(S_c^m:B^m|YY')_{\ca{N}^{\otm m}(\rho_m)}
+
H(YY')
\\
&
\leq
\sum_{t\in\mathfrak{T}(J,m)}
\sum_{\varsigma=1}^{\theta(t)}
p_{\varsigma,t}I(S_c^m:B^m)_{\ca{N}^{\otm m}(\rho_{\varsigma,t})}
\nn\\
&\quad\quad\quad\quad\quad\quad\quad\quad\quad\quad\quad\quad
+\log{\theta_*|\mathfrak{T}(J,m)|}.
\laeq{noritake1}
}
We also have
\alg{
&
mH(S_r|BS_c)_{\ca{N}(\rho)}
\\
&
=
H(S_r^m|B^mS_c^m)_{\ca{N}^{\otm m}(\rho^{\otm m})}
\\
&
=
\sum_{\vec{j}\in[J]^{\times m}}p_{\vec{j}}
H(S_r^m|B^m)_{\ca{N}^{\otm m}(\rho_{\vec{j}})}
\\
&
\geq
\sum_{\vec{j}\in[J]^{\times m}}p_{\vec{j}}
[
H(S_r^m|B^mY')_{\ca{N}^{\otm m}\otm\ca{E}_j(\rho_{\vec{j}})}-\log{\theta(\vec{j})}
]
\laeq{norimaki2}\\
&
\geq
\sum_{\vec{j}\in[J]^{\times m}}p_{\vec{j}}
H(S_r^m|B^mY')_{\ca{N}^{\otm m}\otm\ca{E}_j(\rho_{\vec{j}})}
-\log{\theta_*},
\laeq{street2}
}
where \req{norimaki2} follows from \rLmm{entpinbo} in \rApp{PSE}.
Using \req{yasa} and \req{110},
the first term in \req{street2} is calculated as
\alg{
&
\sum_{\vec{j}\in[J]^{\times m}}p_{\vec{j}}
H(S_r^m|B^mY')_{\ca{N}^{\otm m}\otm\ca{E}_j(\rho_{\vec{j}})}
\\
&
=
\sum_{\vec{j}\in[J]^{\times m}}p_{\vec{j}}
\sum_{\varsigma=1}^{\theta(\vec{j})}p_{\varsigma|\vec{j}}
H(S_r^m|B^m)_{\ca{N}^{\otm m}(\rho_{\varsigma,\vec{j}})}
\\
&
=
\sum_{t\in\mathfrak{T}(J,m)}
\sum_{\vec{j}\in T_t^n}
\frac{p_t}{| T_t^n|}
\sum_{\varsigma=1}^{\theta(t)}p_{\varsigma|t}
H(S_r^m|B^m)_{\ca{N}^{\otm m}(\rho_{\varsigma,\vec{j}})}
\\
&
=
\sum_{t\in\mathfrak{T}(J,m)}
\sum_{\varsigma=1}^{\theta(t)}p_{\varsigma,t}
\!\cdot\!
\frac{1}{| T_t^n|}
\sum_{\vec{j}\in T_t^n}
H(S_r^m|B^m)_{\ca{N}^{\otm m}(\rho_{\varsigma,\vec{j}})}
\\
&
=
\sum_{t\in\mathfrak{T}(J,m)}
\sum_{\varsigma=1}^{\theta(t)}p_{\varsigma,t}
\cdot
H(S_r^m|B^mS_c^m)_{\ca{N}^{\otm m}(\rho_{\varsigma,t})}.
\laeq{kagami}
}
Here, the last line follows from the fact that the entropies of the state $\ca{N}^{\otm m}(\rho_{\varsigma,\vec{j}})$ depends only on $\varsigma$ and the type of $\vec{j}$, because of the local unitary equivalence \req{mizu}.
Similarly, we have
\alg{
&
mH(S_r|S_c)_\rho
=
H(S_r^m|S_c^m)_{\rho^{\otm m}}
=
\sum_{\vec{j}\in[J]^{\times m}}p_{\vec{j}}H(S_r^m)_{\rho_{\vec{j}}}
\nn\\
&
\leq
\sum_{\vec{j}\in[J]^{\times m}}p_{\vec{j}}
H(S_r^mY')_{\ca{E}_{\vec{j}}(\rho_{\vec{j}})}
\laeq{norimaki}\\
&
=
\sum_{\vec{j}\in[J]^{\times m}}p_{\vec{j}}
[H(Y')_{\ca{E}_{\vec{j}}(\rho_{\vec{j}})}
+
H(S_r^m|Y')_{\ca{E}_{\vec{j}}(\rho_{\vec{j}})}
]
\\
&
\leq
\sum_{\vec{j}\in[J]^{\times m}}p_{\vec{j}}
[\log{\theta(\vec{j})}
+
H(S_r^m|Y')_{\ca{E}_{\vec{j}}(\rho_{\vec{j}})}
]
\\
&
\leq
\sum_{\vec{j}\in[J]^{\times m}}p_{\vec{j}}
H(S_r^m|Y')_{\ca{E}_{\vec{j}}(\rho_{\vec{j}})}
+
\log{\theta_*}
\\
&
=
\sum_{t\in\mathfrak{T}(J,m)}
\sum_{\varsigma=1}^{\theta(t)}p_{\varsigma,t}
\cdot
H(S_r^m|S_c^m)_{\rho_{\varsigma,t}}
+
\log{\theta_*}.
\laeq{kagami2}
}
The second line follows from \rLmm{entpinbo} in \rApp{PSE}, and the last line from the similar argument as in \req{kagami}, for which we use the local unitary equivalence \req{koori}.
The cardinalities of the type sets $\mathfrak{T}(J,m)$ and $\theta_*$ are bounded from above by
\alg{
|\mathfrak{T}(J,m)|
\leq
(m+1)^J,
\quad
\theta_*
\leq
(m+1)^{2d_A}.
\laeq{punisher}
}

Consider an arbitrary inner point $(C,Q,E)$ of $\Theta(\ca{N},\rho)$ and choose sufficiently large $m$.
Combining Inequalities \req{noritake1}, \req{street2}, \req{kagami}, \req{kagami2} and \req{punisher} with the conditions \req{proogo1}-\req{proogo3},
 it follows that
\alg{
Q+E
&
\leq
\frac{1}{m}\sum_{\varsigma,t}p_{\varsigma,t}H(S_r^m|S_c^m)_{\rho_{\varsigma,t}},
\\
C+Q-E
&\leq
\frac{1}{m}\sum_{\varsigma,t}p_{\varsigma,t}
[H(S_c^m)_{\rho_{\varsigma,t}}
\nn\\
&\quad\quad\quad\quad\quad\quad
-
H(S^m|B^m)_{\ca{N}^{\otm m}(\rho_{\varsigma,t})}
],
\laeq{kore}\\
Q-E
&\leq
-\frac{1}{m}\sum_{\varsigma,t}p_{\varsigma,t}H(S_r^m|B^mS_c^m)_{\rho_{\varsigma,t}}.
}
Here, we have used the fact that
the chain rule of quantum enrtopies implies 
\alg{
&
I(S_c:B)_{\ca{N}(\rho)}-H(S_r|BS_c)_{\ca{N}(\rho)}
\nn\\
&\quad
=
H(S_c)_{\rho}-H(S|B)_{\ca{N}(\rho)}
}
and
\alg{
&
I(S_c^m:B^m)_{\ca{N}^{\otm m}(\rho_{\varsigma,t})}
-
H(S_r^m|B^mS_c^m)_{\ca{N}^{\otm m}(\rho_{\varsigma,t})}
\nn\\
&=
H(S_c^m)_{\rho_{\varsigma,t}}-H(S^m|B^m)_{\ca{N}^{\otm m}(\rho_{\varsigma,t})}.
}
Thus, we arrive at
\alg{
(C,Q,E)
&
\in
\sum_{\varsigma,t}p_{\varsigma,t}\cdot\frac{1}{m}\Theta_{\pi}(\ca{N}^{\otm m},\rho_{\varsigma,t})
\\
&
\subset
{\rm conv}\frac{1}{m}\Theta_{\pi}(\ca{N}^{\otm m})
\\
&
\subset
{\rm conv}\Theta_{\pi}^{\infty}(\ca{N})
\subseteq
\overline{\Theta_{\pi}^{\infty}(\ca{N})},
}
where the last line follows from the convexity of $\overline{\Theta_{\pi}^{\infty}(\ca{N})}$ (see the next subsection).
This completes the proof of $\frac{1}{n}\Theta(\ca{N}^{\otm n},\rho)\subseteq\overline{\Theta_{\pi}^{\infty}(\ca{N})}$ for $n=1$.
The proofs for $n\geq2$ are obtained along the same line.
\QED




\begin{figure*}[t]
\centerline{
\subfigure[]{\includegraphics[bb={0 30 557 486},scale=0.28]{figure1.pdf}}
\quad
\subfigure[]{\includegraphics[bb={0 32 557 486},scale=0.28]{figure2.pdf}}
\quad
\subfigure[]{\includegraphics[bb={0 0 591 456},scale=0.28]{figure3.pdf}}
}
\caption{
The two dimensional regions of $(Q,E)$ satisfying inequalities \req{proogo21}-\req{proogo23} in the case of $-H(S_r|BS_c)_{\ca{N}(\rho)}>0$ are depicted.
The figures (a), (b) and (c) are for $0\leq C\leq I(S_c:B)$, $I(S_c:B)\leq C\leq I(S_c:B)-H(S_r|BS_c)$ and $I(S_c:B)-H(S_r|BS_c)\leq C\leq I(S:B)$, respectively.
Lines $\ell_1$, $\ell_2$ and $\ell_3$ represent the boundaries represented by inequalities \req{proogo21}-\req{proogo23}. 
The points of intersection of the three lines and the axes are denoted by $u_1$, $u_2$, $u_3$, $t_2$, $t_2'$ and $t_3$. 
At $C=0$, both $u_1$ and $u_2$ are on the right than $u_3$.
Thus $u_3$ and $t_3$ are vertices of the region that yield $P_1^+$ and $P_2$, respectively.
The two points $u_1$ and $u_2$ approaches to the origin as $C$ increases.
At $C=I(S_c:B)$, the point $u_2$ coincides $u_3$, in which case the points $u_2=u_3$ and $t_2=t_3$ are the vertices $P_3^+$ and $P_4$.
In $C=H(S_c)-H(S|B)$, the point $u_2$ meets the origin, and yields $P_5$.
Finally, the point $u_1$ coincides the origin at $C=I(S:B)$,  in which case the point of $u_2'=t_2$ yields $P_6$.
}
\label{fig:rateregion1}
\end{figure*}


\begin{figure}[h]
\centerline{
\subfigure[]{\includegraphics[bb={0 -10 422 623},scale=0.28]{figure4.pdf}}
\quad
\subfigure[]{\includegraphics[bb={0 -10 380 623},scale=0.28]{figure5.pdf}}
}
\caption{
The two dimensional regions of $(Q,E)$ satisfying inequalities \req{proogo21}-\req{proogo23} in the case of $-H(S_r|BS_c)_{\ca{N}(\rho)}<0$ are depicted.
The figure (a) is for $0\leq C\leq I(S_c\!:\!B)$, and (b) is for $I(S_c\!:\!B)\leq C\leq I(S\!:\!B)$.
Lines $\ell_1$, $\ell_2$ and $\ell_3$ represent the boundaries represented by inequalities \req{proogo21}-\req{proogo23}, respectively, and the crossing points are denoted by $u_1$,  $u_2'$, $u_3'$, $t_2$ and $t_3$.
At $C=0$, the point $u_2'$ is below $u_3'$,
in which case $u_3'$ and $t_3$ are vertices of the region that yield $P_1^-$ and $P_2$, respectively.
The point $u_2'$ approaches $u_3'$ as $C$ increases, and coincides it at $C=I(S_c:B)$.
In this case, the points $u_2'=u_3'$ and $t_2=t_3$ correspond to the vertices $P_3^-$ and $P_4$.
At $C=I(S:B)$,  the point $u_1$ coincides the origin, where the point $u_2'=t_2$ yields $P_6$.
}
\label{fig:rateregion2}
\end{figure}



\subsubsection{Proof of ${\rm conv}\Theta_{\pi}^{\infty}(\ca{N})\subseteq\overline{\Theta_{\pi}^{\infty}(\ca{N})}$}

We prove that the convex hull of $\Theta_{\pi}^{\infty}(\ca{N})$ is a subset of $\overline{\Theta_{\pi}^{\infty}(\ca{N})}$.
Fix arbitrary $\lambda_1,\lambda_2>0$ such that $\lambda_1+\lambda_2=1$, and suppose that $(C^{(i)},Q^{(i)},E^{(i)})\in\Theta_{\pi}^{\infty}(\ca{N})$ for $i=1,2$.
By definition, for any sufficiently large $n$, there exist quantum systems $S^{(i)}\equiv S_c^{(i)}S_r^{(i)}$ and a quantum state $\rho_i$ on $S^{(i)}C^{n\lambda_i}$ such that $\rho_i^{S^{(i)}}$ is the maximally mixed state, and it holds that
\alg{
n_i(Q^{(i)}\!+\!E^{(i)})
&\leq
 H(S_r^{(i)}|S_c^{(i)})_{\rho_i},
\laeq{hashigo}\\
n_i(C^{(i)}\!+\!Q^{(i)}\!-\!E^{(i)})
&
\leq
H(S_c^{(i)})_{\rho_i}
\nn\\
&\quad
-H(S^{(i)}|B^{n\lambda_i})_{\ca{N}^{\otm n_i}(\rho_i)},
\\
n_i(Q^{(i)}\!-\!E^{(i)})
&\leq
 -H(S_r^{(i)}|B^{n\lambda_i}S_c^{(i)})_{\ca{N}^{\otm n_i}(\rho_i)}
\laeq{hashigo2}
}
for $i=1,2$, where $n_1:=\lfloor n\lambda_1\rfloor$ and $n_2:=\lceil n\lambda_i\rceil$.
Define $S_c\equiv S_c^{(1)}S_c^{(2)}$, $S_r \equiv  S_r^{(1)}S_r^{(2)}$, $S\equiv S_cS_r$ and consider a quantum state $\bar{\rho}$ on $SA^n$ defined by
\alg{
\bar{\rho}^{\:SA^n}:=
\rho_1^{\:S^{(1)}\!A^{n_1}}
\otm
\rho_2^{\:S^{(2)}\!A^{n_2}}.
}
It is straightforward to verify that the state is diagonal on $S_c$ with respect to a fixed basis, and that $\bar{\rho}^{S}$ is the full-rank maximally mixed state on $S$. 
Due to the additivity of the conditional quantum entropy, we have
\alg{
H(S_r|S_c)_{\bar{\rho}}
=
H(S_r^{(1)}|S_c^{(1)})_{\rho_1}
+
H(S_r^{(2)}|S_c^{(2)})_{\rho_2}
}
and so forth. 
Define 
\alg{
(C_n,Q_n,E_n)\equiv\sum_{i=1,2}n_i(C^{(i)},Q^{(i)},E^{(i)}).
}
It follows from \req{hashigo}-\req{hashigo2} that
\alg{
Q_n+E_n
&\leq
 H(S_r|S_c)_{\bar{\rho}}
\\
C_n+Q_n-E_n
&
\leq
H(S_c)_{\bar{\rho}}-H(S|B^n)_{\ca{N}^{\otm n}(\bar{\rho})}
\\
Q_n-E_n
&\leq
 -H(S_r|B^nS_c)_{\ca{N}^{\otm n}(\bar{\rho})},
}
which implies $(C_n,Q_n,E_n)\in\Theta_{\pi}(\ca{N}^{\otm n})$.
Furthermore, it is straightforward to verify that
\alg{
\lim_{n\rightarrow\infty}\frac{1}{n}(C_n,Q_n,E_n)
&
=
(\bar{C},\bar{Q},\bar{E})
\\
&
:=
\sum_{i=1,2}\lambda_i(C^{(i)},Q^{(i)},E^{(i)}).
}
This implies $(\bar{C},\bar{Q},\bar{E})\in\overline{\Theta_{\pi}^\infty(\ca{N})}$, and completes the proof.
\QED


\subsection{Proof of $\overline{\Theta^{\infty}(\ca{N})}=\overline{\Lambda^{\infty}(\ca{N})}$}


Suppose that a triplet $(C,Q,E)$ belongs to $\Theta(\ca{N},\rho)$, which is defined by Inequalities \req{proogo1}-\req{proogo4}. Noting that $\rho^{S}=\ca{N}(\rho)^{S}$, Inequalities \req{proogo1} and \req{proogo2} implies \req{proogo21}.
Thus, we have $\Theta(\ca{N},\rho)\subseteq\Lambda(\ca{N},\rho)$, which leads to $\overline{\Theta^{\infty}(\ca{N})}\subseteq\overline{\Lambda^{\infty}(\ca{N})}$.

To prove the converse relation $\overline{\Theta^{\infty}(\ca{N})}\supseteq\overline{\Lambda^{\infty}(\ca{N})}$, we show that $\Theta^{\infty}(\ca{N})\supseteq\Lambda(\ca{N},\rho)$ for any state $\rho$. 
Note that $\Lambda(\ca{N},\rho)$ is a convex polytope such that $(C,Q,E+\Delta E)\in\Lambda(\ca{N},\rho)$ for any $(C,Q,E)\in\Lambda(\ca{N},\rho)$ and any $\Delta E>0$ (see \rDfn{rateSN2}).
Thus, it suffices to prove that (i) all vertices of $\Lambda(\ca{N},\rho)$ belongs to $\Theta^{\infty}(\ca{N})$, and that (ii) if $(C,Q,E)\in\Theta^{\infty}(\ca{N})$, then $(C,Q,E+\Delta E)\in\Theta^{\infty}(\ca{N})$ for any $\Delta E>0$.



\subsubsection{Vertices of $\Lambda(\ca{N},\rho)$}


Consider the following points in $\mbb{R}^3$,
where all entropies and mutual informations are for the state $\ca{N}(\rho)$:
\alg{
&P_0:=(0,\;0,\;0)
\nn\\
&
P_1^+:=(0,\;-H(S_r|BS_c),\;0)
\nn\\
&
P_1^-:=(0,\;0,\;H(S_r|BS_c))
\nn\\
&
P_2:=\left(0,\;\frac{1}{2}I(S:B),
\frac{1}{2}I(S\!:\!B)+H(S_r|BS_c)\right)
\nn\\
&
P_3^+:=\left(I(S_c:B),\;-H(S_r|BS_c),\;0\right)
\nn\\
&
P_3^-:=\left(I(S_c:B),\;0,\;H(S_r|BS_c)\right)
\nn\\
&
P_4:=\left(I(S_c\!:\!B),\frac{1}{2}I(S_r\!:\!D|S_c),
\frac{1}{2}I(S_r\!:\!D|S_c)\!+\!H(S_r|BS_c)\right)
\nn\\
&
P_5:=\left(H(S_c)-H(S|B),\;0,\;0\right)
\nn\\
&
P_6:=\left(I(S\!:\!B),\;0,\;H(S_r|S_c)\right)
\nn
}
The vertices of $\Lambda(\ca{N},\rho)$ are $P_0,P_1^+,P_2,P_3^+,P_4,P_5,P_6$ in the case of $-H(S_r|BS_c)_{\ca{N}(\rho)}>0$ and $P_1^-,P_2,P_3^-,P_4,P_6$ when $-H(S_r|BS_c)_{\ca{N}(\rho)}<0$ (Figures \ref{fig:rateregion1} and \ref{fig:rateregion2}: see also Section VI in \cite{hsieh2010entanglement}).
Note that, by the chain rule of the mutual information, it holds that
\alg{
&
H(S_c)_\rho
-
H(S|B)_{\ca{N}(\rho)}
\nn\\
&
=
I(S_c:B)_{\ca{N}(\rho)}-H(S_r|BS_c)_{\ca{N}(\rho)}
\\
&
=
I(S:B)_{\ca{N}(\rho)}-H(S_r|S_c)_{\ca{N}(\rho)}.
}
By a simple calculation, it is straightforward to verify that all of the above points except $P_2$ belong to $\Theta(\ca{N},\rho)$, and consequently to $\Theta^{\infty}(\ca{N})$. 


 

\subsubsection{Proof of $P_2\in\Theta^{\infty}(\ca{N})$}


Consider the point $P_2$ represented by the coordinate $(C_2,Q_2,E_2)$, where
\alg{
&
C_2=0,
\quad
Q_2=\frac{1}{2}I(S:B)_{\ca{N}(\rho)},
\\
&
E_2=
\frac{1}{2}I(S:B)_{\ca{N}(\rho)}+H(S_r|S_cB)_{\ca{N}(\rho)}.
}
A simple calculation yields
\alg{
Q_2+E_2
&=H(S_r|S_c)_{\ca{N}(\rho)}+I(S_c:B)_{\ca{N}(\rho)},
\laeq{trop1}\\
C_2+Q_2-E_2
&=
-H(S_r|S_cB)_{\ca{N}(\rho)},
\laeq{trop2}\\
Q_2-E_2
&=
-H(S_r|S_cB)_{\ca{N}(\rho)}.
\laeq{trop3}
}
Fix arbitrary $\epsilon,\delta>0$ and choose sufficiently large $n$. 
Due to the data compression theorem for classical information source with quantum side information (Theorem 1 in \cite{devetak2003classical}),
 there exist a countable set $\ca{Y}_{n,\delta}$ satisfying
 \alg{
|\ca{Y}_{n,\delta}|\leq 2^{n(H(S_c|B)_{\ca{N}(\rho)}+\delta)},
}
a function $f:[J]^{\times n}\rightarrow \ca{Y}_{n,\delta}$ and 
 for each $y\in\ca{Y}_{n,\delta}$, there exists a measurement $\{M_{\vec{j}}^y\}_{\vec{j}\in[J]^{\times n}}$ on $B^n$  that satisfies
\alg{
\sum_{\vec{j}\in[J]^{\times n}}
p_{\vec{j}}
{\rm Tr}\left[
M_{\vec{j}}^y(\ca{N}^{A\rightarrow B})^{\otm n}(\rho_{\vec{j}}^{A^n})
\right]
\geq
1-\epsilon.
\laeq{rect}
}
We introduce a $|\ca{Y}_{n,\delta}|$-dimensional quantum system $Y$ and define a state
\alg{
\rho_n^{YS^nA^n}
:=
\sum_{\vec{j}\in [J]^{\times n}}p_{\vec{j}}
\proj{f(\vec{j})}^Y
\otm
\proj{\vec{j}}^{S_c^n}
\otm
\rho_{\vec{j}}^{S_r^nA^n}.
\nn
}
We denote the system $S_c^nS_r^n$ by $\hat{S}_r$.
It is straightforward to verify that
\alg{
\rho_n^{YS^nA^n}
=
(\rho^{SA})^{\otm n}.
\laeq{ican}
}



Using the properties of quantum entropies and \req{ican}, we have
\alg{
&
H(\hat{S}_r|Y)_{\rho_n}
\nn\\
&
=
H(S_c^nS_r^n|Y)_{\rho_n}
\\
&
=
H(S_c^nY)_{\rho_n}
-
H(Y)_{\rho_n}
+
H(S_r^n|S_c^nY)_{\rho_n}
\\
&=
H(S_c^n)_{\rho_n}
-
H(Y)_{\rho_n}
+
H(S_r^n|S_c^n)_{\rho_n}
\\
&\geq
H(S_c^n)_{\rho_n}
-
|\ca{Y}_{n,\delta}|
+
H(S_r^n|S_c^n)_{\rho_n}
\\
&=
nH(S_c)_{\rho}
-
|\ca{Y}_{n,\delta}|
+
nH(S_r|S_c)_{\rho}
\\
&
\geq
nI(S_c:B)_{\ca{N}(\rho)}
+
nH(S_r|S_c)_{\rho}-n\delta,
}
where $\eta$ is a function that satisfies $\lim_{\epsilon\rightarrow0}\eta(\epsilon)=0$ and is independent of the dimensions of the systems.
From \req{rect}, \req{ican} and \rLmm{fano} in \rApp{PSE}, we also have
\alg{
&
H(\hat{S}_r|YB^n)_{\ca{N}^{\otm n}(\rho_n)}
\nn\\
&=
H(S_c^nS_r^n|YB^n)_{\ca{N}^{\otm n}(\rho_n)}
\\
&
=
H(S_c^n|YB^n)_{\ca{N}^{\otm n}(\rho_n)}
+
H(S_r^n|S_c^nYB^n)_{\ca{N}^{\otm n}(\rho_n)}
\\
&
=
H(S_c^n|YB^n)_{\ca{N}^{\otm n}(\rho_n)}
+
H(S_r^n|S_c^nB^n)_{\ca{N}^{\otm n}(\rho_n)}
\\
&
=
H(S_c^n|YB^n)_{\ca{N}^{\otm n}(\rho_n)}
+
nH(S_r|S_cB)_{\ca{N}(\rho)}
\\
&
\leq
nH(S_r|S_cB)_{\ca{N}(\rho)}
+
2n\eta(\epsilon)\log{d_{S_c}}.
}
In addition, a simple calculation using the chain rule yields
\alg{
&
H(Y)_{\rho_n}
-
H(Y\hat{S}_r|B^n)_{\ca{N}^{\otm n}(\rho_n)}
\nn\\
&=
-H(\hat{S}_r|YB^n)_{\ca{N}^{\otm n}(\rho_n)}
+
I(Y:B^n)_{\ca{N}^{\otm n}(\rho_n)}
\\
&\geq
-H(\hat{S}_r|YB^n)_{\ca{N}^{\otm n}(\rho_n)}.
}
Combining these relations with \req{trop1}-\req{trop3}, we arrive at
\alg{
Q_2+E_2
&\leq
\frac{1}{n}H(\hat{S}_r|Y)_{\ca{N}^{\otm n}(\rho_n)}
+
\delta,
\\
C_2+Q_2-E_2
&\leq
\frac{1}{n}[H(Y)_{\rho_n}
-H(Y\hat{S}_r|B^n)_{\ca{N}^{\otm n}(\rho_n)}]
\nn\\
&
\quad\quad\quad\quad\quad\quad\quad\quad\quad
+
2\eta(\epsilon)\log{d_{S_c}},
\\
Q_2-E_2
&\leq
-\frac{1}{n}H(\hat{S}_r|YB^n)_{\ca{N}^{\otm n}(\rho_n)}
\nn\\
&
\quad\quad\quad\quad\quad\quad\quad\quad
+
2\eta(\epsilon)\log{d_{S_c}}.
}
Since these relations hold for any small $\epsilon,\delta>0$ and sufficiently large $n$,
we obtain  $P_2\in\Theta^{\infty}(\ca{N})$.
\QED



\subsubsection{Proof of $(C,Q,E+\Delta E)\in\Theta^{\infty}(\ca{N})$ for any $\Delta E>0$ and $(C,Q,E)\in\Theta^{\infty}(\ca{N})$}


We complete the proof of $\Theta^{\infty}(\ca{N})\supseteq\Lambda(\ca{N},\rho)$ by showing that, 
if $(C,Q,E)\in\Theta^{\infty}(\ca{N})$, then $(C,Q,E+\Delta E)\in\Theta^{\infty}(\ca{N})$ for any $\Delta E>0$.
It suffices to prove that for any $n\in\mbb{N}$ and $\Delta E>0$, it holds that $(C,Q,E+\Delta E)\in\Theta^{\infty}(\ca{N})$ if $(C,Q,E)\in\frac{1}{n}\Theta(\ca{N}^{\otm n})$.
We only consider the case where $n=1$.
It is straightforward to generalize the proof for $n\geq2$.

Consider a triplet $(C,Q,E)\in\Theta(\ca{N})$, and fix arbitrary $\Delta E>0$ and $m\in\mbb{N}$.
By definition, there exist finite dimensional quantum systems $S_c$, $S_r$ and a state in the form of
\alg{
\rho^{SA}=\sum_{j=1}^Jp_j\proj{j}^{S_c}\otm\rho_j^{S_rA},
}
such that
\alg{
Q+E
&\leq H(S_r|S_c)_\rho,
\laeq{prooga1}\\
C+Q-E
&
\leq
H(S_c)_\rho-H(S|B)_{\ca{N}(\rho)},
\laeq{prooga2}\\
Q-E
&\leq -H(S_r|BS_c)_{\ca{N}(\rho)}.
\laeq{prooga3}
}
Define $\Delta E_m:=\lfloor m\Delta E\rfloor$, and
let $S_r'$ be a quantum system with dimension $2^{\Delta E_m}$.
Consider a state
\alg{
\rho_m^{S^mS_r'A^m}=(\rho^{SA})^{\otm m}\otm\pi^{S_r'},
}
where $\pi$ is the full-rank maximally mixed state on $S_r'$.
Relabelling $S_c^m$ by $\hat{S}_c$, $S_r^mS_r'$ by $\hat{S}_r$ and $\hat{S}_c\hat{S}_r$ by $\hat{S}$,
the above state is represented as
\alg{
\rho_m^{\hat{S}A^m}
:=
\sum_{\vec{j}\in[J]^{\times m}}p_{\vec{j}}\proj{\vec{j}}^{\hat{S}_c}\otm\hat{\rho}_{\vec{j}}^{\hat{S}_rA^m},
}
where
\alg{
\hat{\rho}_{\vec{j}}^{\hat{S}_rA^m}
:=
\rho_{\vec{j}}^{S_r^mA^m}\otm\pi^{S_r'}
}
and
\alg{
p_{\vec{j}}
:=
p_{j_1}\cdots p_{j_m},
\quad
\rho_{\vec{j}}
:=
\rho_{j_1}\otm\cdots\otm\rho_{j_m}
}
for $\vec{j}=j_1\cdots j_m$.
Noting that
\alg{
H(\hat{S}_r|\hat{S}_c)_{\rho_m}
=
mH(S_r|S_c)_\rho
+
\Delta E_m
}
 and so forth, it follows from \req{prooga1}-\req{prooga3} that
\alg{
m(Q+E)+\Delta E_m
&\leq
 H(\hat{S}_r|\hat{S}_c)_{\rho_m},
\\
m(C+Q-E)-\Delta E_m
&
\leq
H(\hat{S}_c)_\rho
\nn\\
&\quad\quad
-H(\hat{S}|B^m)_{\ca{N}^{\otm m}(\rho_m)},
\\
m(Q-E)-\Delta E_m
&\leq
 -H(\hat{S}_r|B^m\hat{S}_c)_{\ca{N}^{\otm m}(\rho_m)}.
}
This implies $(C,Q,E+\Delta E_m/m)\in\Theta^{\infty}(\ca{N})$.
Noting that $\lim_{m\rightarrow\infty}(\Delta E_m/m)=\Delta E$, this implies $(C,Q,E+\Delta E)\in\overline{\Theta^{\infty}(\ca{N})}$ and completes the proof.
\QED


\subsection{Proof of $\overline{\Lambda^{\infty}(\ca{N})}=\overline{\Lambda_p^{\infty}(\ca{N})}$}

It is straightforward to verify that $\overline{\Lambda^{\infty}(\ca{N})}\supseteq\overline{\Lambda_p^{\infty}(\ca{N})}$. Thus, we prove $\overline{\Lambda^{\infty}(\ca{N})}\subseteq\overline{\Lambda_p^{\infty}(\ca{N})}$ by showing that $\Lambda(\ca{N}^{\otm n})\subseteq\Lambda_p(\ca{N}^{\otm n})$ for any $n$. We only consider the case $n=1$. It is straightforward to generalize the proof for $n\geq2$.

Fix an arbitrary state $\rho$ in the form of \req{proogostar}, and suppose that $(C,Q,E)\in\Lambda(\ca{N},\rho)$. For each $j$, let $\{q_{k|j},\ket{\phi_{j,k}}\}_k$ be an ensemble of pure states on $S_rA$ such that $\rho_j=\sum_kq_{k|j}\proj{\phi_{j,k}}$. 
We denote $p_jq_{k|j}$ by $p_{jk}$.
Let $Y$ be a finite dimensional quantum system with a fixed orthonormal basis $\{\ket{k}\}_k$, 
and define a state $\tilde{\rho}^{YSA}$ by
\alg{
\tilde{\rho}^{YSA}
:=
\sum_{j=1}^J\sum_kp_{jk}\proj{k}^Y\otm\proj{j}^{S_c}\otm\proj{\phi_{j,k}}^{S_rA}.
\nn
}
It is straightforward to verify that $\tilde{\rho}^{SA}=\rho^{SA}$.
Denoting $YS_c$ by $\hat{S}_c$ and $\hat{S}_cS_r$ by $\hat{S}$, the data processing inequality yields
\alg{
I(S:B)_{\ca{N}(\rho)}
&
\leq
I(\hat{S}:B)_{\ca{N}(\tilde{\rho})},
\\
-H(S_r|BS_c)_{\ca{N}(\rho)}
&
\leq
-H(S_r|B\hat{S}_c)_{\ca{N}(\tilde{\rho})},
}
in addition to
\alg{
&
H(S_c)_\rho-H(S|B)_{\ca{N}(\rho)}
\nn\\
&=
I(S_c:B)_{\ca{N}(\rho)}-H(S_r|BS_c)_{\ca{N}(\rho)}
\\
&\leq
I(\hat{S}_c:B)_{\ca{N}(\tilde{\rho})}-H(S_r|B\hat{S}_c)_{\ca{N}(\tilde{\rho})}
\\
&
=
H(\hat{S}_c)_{\tilde{\rho}}-H(\hat{S}|B)_{\ca{N}(\tilde{\rho})}.
}
Combining these inequalities with \req{proogo21}-\req{proogo23},
we have $(C,Q,E)\in\Lambda(\ca{N},\tilde{\rho})$, which implies $\Lambda(\ca{N},\rho)\subseteq\Lambda(\ca{N},\tilde{\rho})$.
By taking the union over all $\rho$, we arrive at $\Lambda(\ca{N})\subseteq\Lambda_p(\ca{N})$ and complete the proof.
\QED



\bibliographystyle{IEEEtran}
\bibliography{bibbib.bib}



%\begin{IEEEbiographynophoto}{Eyuri Wakakuwa}
%received his doctorate from The University of Tokyo
%in 2015, and was a postdoctoral fellow at the University of Electro-Communications, Japan, until 2017. He is currently an assistant professor at The University of Tokyo. His research focuses on quantum information theory and the foundation of quantum mechanics.
%\end{IEEEbiographynophoto}
%
%\begin{IEEEbiographynophoto}{Yoshifumi Nakata}
%@@@
%\end{IEEEbiographynophoto}



\end{document}



\chapter{
%$\boldsymbol{\S}$\textbf{24}.\quad  THE WEIL-DELIGNE GROUP}
APPENDIX A: TOPICS IN TOPOLOGY}
\setlength\parindent{2em}
\setcounter{theoremn}{0}



\[
\text{NEIGHBORHOODS}
\]

\[
\text{COMPACTNESS}
\]

\[
\text{CONNECTEDNESS}
\]

\[
\text{TOPOLOGICAL GROUPS}
\]
\newpage
%%----------------------------------------------------------------------------------------------01

\ \indent 

\[
\text{NEIGHBORHOODS}
\]

\begin{x}{\small\bf DEFINITION} \ %01
If \mX is a topological space and if $x \in X$, then a 
\un{neighborhood}
\index{neighborhood} 
of $x$ is a set \mU which contains an open set \mV containing $x$, the collection $\sU_x$ of all neighborhoods of $x$ 
being the 
\un{neighborhood system}
\index{neighborhood system} 
at $x$.
\end{x}

\vspace{0.1cm}

Therefore \mU is a neighborhood of $x$ iff $x \in \itr U$.

\vspace{0.2cm}


\begin{x}{\small\bf PROPERTIES} \ %02
of $\sU_x$

\vspace{0.1cm}

\un{N-a} \quad If $U \in \sU_x$, then $x \in U$.

\un{N-b} \quad If $U_1, U_2 \in \sU_x$, then $U_1 \cap U_2 \in \sU_x$.

\un{N-c} \quad If $U \in \sU_x$, then there is a $U_0 \in \sU_x$ such that $U \in \sU_{x_0}$ for each $x_0 \in U_0$.

\un{N-d} \quad If $U \in \sU_x$ and $U \subset V$, then $V \in \sU_x$.

\end{x}

\vspace{0.1cm}

\begin{x}{\small\bf FACT} \ %03
A subset $G \subset X$ is open iff \mG contains a neighborhood of each of its points.
\end{x}

\vspace{0.1cm}

\begin{x}{\small\bf SCHOLIUM} \ %04
If in a set \mX a nonempty collection $\sU_x$ of subsets of \mX is assigned to each $x \in X$ 
so as to satisfy N-a through N-d and if a subset $G \subset X$ is deemed ``open'' provided 
$\forall \ x \in G$, there is a $U \in \sU_x$ such that $U \subset G$, then the result is a topology on \mX 
in which the neighborhood system at each $x \in X$ is $\sU_x$.
\end{x}

\vspace{0.1cm}

\begin{x}{\small\bf DEFINITION} \ %05
If \mX is a topological space and if $x \in X$, then a 
\un{neighborhood basis}
\index{neighborhood basis} 
at $x$ is a subcollection $\sB_x$ of $\sU_x$ such that $U \in \sU_x$ contains some $V \in \sB_x$.
\end{x}

\vspace{0.1cm}

\begin{x}{\small\bf EXAMPLE} \ %06
Take $X = \R^2$ with the usual topology $-$then the set of all squares with sides parallel to the axes and centered at $x$ 
is a neighborhood basis at $x$.
\end{x}

\vspace{0.1cm}

%%----------------------------------------------------------------------------------------------02
\begin{x}{\small\bf PROPERTIES} \ %07
of $\sB_x$

\vspace{0.1cm}

\un{NB-a} \quad If $V \in \sB_x$, then $x \in V$.

\un{NB-b} \quad If $V_1, V_2 \in \sB_x$, then there is a $V_3 \in \sB_x$ such that $V_3 \subset V_1 \cap V_2$.

\un{NB-c} \quad If $V \in \sB_x$, then there is a $V_0 \in \sB_x$ such that if $x_0 \in V_0$, then there is a 
$W \in \sB_{x_0}$ such that $W \subset V$.
\end{x}

\vspace{0.1cm}

\begin{x}{\small\bf FACT} \ %08
A subset $G \subset X$ is open iff \mG contains a basic neighborhood of each of its points.
\end{x}

\vspace{0.1cm}

\begin{x}{\small\bf SCHOLIUM} \ %09
If in a set \mX a nonempty collection $\sB_x$ of subsets of \mX is assigned to each $x \in X$ so as to satisfy 
NB-a through NB-c and if a subset $G \subset X$ is deemed ``open'' provided $\forall \ x \in G$, there is a 
$V \in \sB_x$ such that $V \subset G$, then the result is a topology on \mX in which a neighborhood  basis 
at each $x \in X$ is $\sB_x$.

\vspace{0.1cm}

[Put
\[
\sU_x \hsx = \hsx \{U \subset X: V \subset U \ (\exists V \in \sB_x)\}.
\]
Then $\sU_x$ satisfies N-a through N-d above.]

\end{x}

\vspace{0.1cm}

\begin{x}{\small\bf EXAMPLE} \ %10
Take $X = \R$ and given $x$, let $\sB_x$ be the $[x,y[$ $(y > x)$ $-$then $\sB_x$ satisfies NB-a through NB-c above, 
from which a topology on the line, the underlying topological space being the 
\un{Sorgenfrey line}.
\index{Sorgenfrey line}.
\end{x}

\vspace{0.1cm}
%%----------------------------------------------------------------------------------------------03
\begin{x}{\small\bf DEFINITION} \ %11
Let \mX be a topological space $-$then a 
\un{basis}
\index{basis} 
for \mX (i.e., for the underlying topology \ldots) is a collection $\sB$ of open sets such that for any open set 
$G \subset X$ and for any point $x \in G$, there is a set $B \in \sB$ such that $x \in B \subset G$.
\end{x}

\vspace{0.1cm}

\begin{x}{\small\bf FACT} \ %12
If $\sB$ is a collection of open sets, then $\sB$ is a basis for \mX iff $\forall \ x \in X$, the collection
\[
\sB_x \hsx = \hsx \{B \in \sB : x \in B \}
\]
is a neighborhood basis at $x$.
\end{x}

\vspace{0.1cm}

\begin{x}{\small\bf FACT} \ %13
If \mX is a set and if $\sB$ is a collection of subsets of \mX, then $\sB$ is a basis for a topology on \mX iff 
\[
X \hsx = \hsx \bigcup\limits_{B \in \sB} B
\]
and given $B_1, B_2 \in \sB$ and $x \in B_1 \cap B_2$, there exists $B_3 \in \sB$ such that $x \in B_3 \subset B_1 \cap B_2$. 
\end{x}

\vspace{0.1cm}


%%%%%%%%%%%%%%%%%%%%%%%%%%%%%%%%%%%%%%
%%%%%%%%%%%%%%%%%%%%%%%%%%%%%%%%%%%%%%
%%%%%%%%%%%%%%%%%%%%%%%%%%%%%%%%%%%%%%

\newpage
\setcounter{theoremn}{0}

%%----------------------------------------------------------------------------------------------01

\ \indent 

\[
\text{COMPACTNESS}
\]

\begin{x}{\small\bf DEFINITION} \ %01
A topological space \mX is 
\un{compact}
\index{compact} 
if every open cover of \mX has a finite subcover.
\end{x}

\vspace{0.1cm}


\begin{x}{\small\bf EXAMPLE} \ %02
The Cantor set is compact.
\end{x}

\vspace{0.1cm}

\begin{x}{\small\bf FACT} \ %03
The continous image of a compact space is compact.
\end{x}

\vspace{0.1cm}

\begin{x}{\small\bf FACT} \ %04
A one-to-one continuous function from a compact space \mX onto a Hausdorff space \mY is a homeomorphism.
\end{x}

\vspace{0.1cm}

\begin{x}{\small\bf DEFINITION} \ %05
A topological space \mX is 
\un{locally compact} 
\index{locally compact} 
if each point in \mX has a neighborhood basis consisting of compact sets.
\end{x}

\vspace{0.1cm}

\begin{x}{\small\bf FACT} \ %06
A Hausdorff space \mX is locally compact iff each point in \mX has a compact neighborhood.
\end{x}

\vspace{0.1cm}

\begin{x}{\small\bf APPLICATION} \ %07
Every compact Hausdorff space \mX is locally compact.
\end{x}

\vspace{0.1cm}

\begin{x}{\small\bf EXAMPLE} \ %08
The Cantor set is a locally compact Hausdorff space.
\end{x}

\vspace{0.1cm}

\begin{x}{\small\bf EXAMPLE} \ %09
$\R$ is a locally compact Hausdorff space.
\end{x}

\vspace{0.1cm} 

\begin{x}{\small\bf EXAMPLE} \ %10
$\Q$ is a Hausdorff space but it is not locally compact ($\Q$ is first category while a locally compact Hausdorff space is second category).
\end{x}

\vspace{0.1cm}

\begin{x}{\small\bf EXAMPLE} \ %11
The Sorgenfrey line is Hausdorff but not locally compact.
\end{x}

\vspace{0.1cm}

\begin{x}{\small\bf FACT} \ %12
Suppose that $X_i$ $(i \in I)$ is a nonempty topological space $-$then the product 
$\prod\limits_{i \in I} X_i$ is locally compact iff each $X_i$ is locally compact and all but a finite number of the $X_i$ are compact.
\end{x}

%%%%%%%%%%%%%%%%%%%%%%%%%%%%%%%%%%%%%%
%%%%%%%%%%%%%%%%%%%%%%%%%%%%%%%%%%%%%%
%%%%%%%%%%%%%%%%%%%%%%%%%%%%%%%%%%%%%%

\newpage
\setcounter{theoremn}{0}


%%----------------------------------------------------------------------------------------------01

\ \indent 

\[
\text{CONNECTEDNESS}
\]

\begin{x}{\small\bf DEFINITION} \ %01
A topological space \mX is 
\un{connected} 
\index{connected} 
if it is not the union of two nonempty disjoint open sets.
\end{x}

\vspace{0.1cm}


\begin{x}{\small\bf EXAMPLE} \ %02
$\Q$ is not connected (write
\[
\Q \hsx = \hsx \{x: x > \sqrt{2}\} \hsx \cap \hsx \Q \cup \{x: x < \sqrt{2}\} \hsx \cap \hsx \Q). 
\]
\end{x}

\vspace{0.1cm}

\begin{x}{\small\bf EXAMPLE} \ %03
$\R$ is connected and the only connected subsets of $\R$ having more than one point are the intervals 
(open, closed, or half-open, half-closed).
\end{x}

\vspace{0.1cm}

\begin{x}{\small\bf FACT} \ %04
A topological space \mX is connected iff the only subsets of \mX that are both open and closed are $\emptyset$ and \mX.
\end{x}

\vspace{0.1cm}

\begin{x}{\small\bf FACT} \ %05
The continuous image of a connected space is connected.
\end{x}

\vspace{0.1cm}

\begin{x}{\small\bf DEFINITION} \ %06
Let \mX be a topological space and let $x \in X$ $-$then the 
\un{component}
\index{component} 
$C(x)$ of $x$ is the union of all connected subsets of $X$ containing $x$.
\end{x}

\vspace{0.1cm}

\begin{x}{\small\bf FACT} \ %07
$C(x)$ is a closed subset of \mX.
\end{x}

\vspace{0.1cm}

\begin{x}{\small\bf FACT} \ %08
$C(x)$ is a maximal connected subset of \mX.
\end{x}

\vspace{0.1cm}

If $x \neq y$ in \mX, then either $C(x) = C(y)$ or $C(x) \cap C(y) = \emptyset$ (otherwise, 
$C(x) \cup C(y)$ would be a connected set containing $x$ and $y$ and larger than $C(x)$ or $C(y)$, which is 
impossible).  Therefore the set of distinct components of \mX forms a partition of \mX.

\vspace{0.2cm}

\begin{x}{\small\bf EXAMPLE} \ %09
Take $X = \Q$ $-$then $\forall \ x \in \Q$, $C(x) = \{x\}$ (under the inclusion $\Q \ra \R$, a connected subset of $\Q$ 
is sent to a connected subset of $\R$).
\end{x}
%%----------------------------------------------------------------------------------------------02

\begin{x}{\small\bf DEFINITION} \ %10
A topological space \mX is 
\un{totally disconnected}
\index{totally disconnected} 
if the components of \mX are singletons, i.e., $\forall \ x \in X$, $C(x) = \{x\}$.



\vspace{0.1cm}


\begin{x}{\small\bf FACT} \ %11
A topological space \mX is totally disconnected iff the only nonempty connected subsets of \mX are the one-point 
sets (hence \mX is $T_1$).

\vspace{0.1cm}

[Note: \ In every topological space \mX, the empty set and the one-point sets are connected and in a totally disconnected 
topological space, these are the only connected subsets.]
\end{x}
\end{x}

\vspace{0.1cm}

\begin{x}{\small\bf REMARK} \ %12
Let \mE be the equivalence relation defined by writing $x \sim y$ if $x$ and $y$ lie in the same component.  
Equip the set $X/E$ with the identification topology determined by the projection 
$p:X \ra X/E$ $-$then $X/E$ is totally disconnected.
\end{x}

\vspace{0.1cm}

\begin{x}{\small\bf EXAMPLE} \ %13
The Cantor set is totally disconnected.
\end{x}

\vspace{0.1cm}



\begin{x}{\small\bf EXAMPLE} \ %14
$\Q$ is totally disconnected. 
\end{x}

\vspace{0.1cm}


\begin{x}{\small\bf EXAMPLE} \ %15
The Sorgenfrey line is totally disconnected.
\end{x}

\vspace{0.1cm}


\begin{x}{\small\bf FACT} \ %16
Every product of totally disconnected topological spaces is totally disconnected.
\end{x}

\vspace{0.1cm}


\begin{x}{\small\bf FACT} \ %17
Every subspace of a totally disconnected topological space is totally disconnected.
\end{x}

\vspace{0.1cm}


\begin{x}{\small\bf REMARK} \ %18
The continuous image of a totally disconnected space need not be totally disconnected.  
To appreciate the point, recall that evey compact metric space is the continuous image of the Cantor set.
\end{x}



%%----------------------------------------------------------------------------------------------03


\begin{x}{\small\bf DEFINITION} \ %19
A topological space \mX is 
\un{0-dimensional}
\index{0-dimensional} 
if each point of \mX has a neighborhood basis consisting of open-closed sets.
\end{x}

\vspace{0.1cm}

\begin{x}{\small\bf FACT} \ %20
A 0-dimensional $T_1$-space is totally disconnected.
\end{x}

\vspace{0.1cm}

\begin{x}{\small\bf EXAMPLE} \ %21
The Cantor set is 0-dimensional.
\end{x}

\vspace{0.1cm}

\begin{x}{\small\bf EXAMPLE} \ %22
$\Q$ is 0-dimensional.
\end{x}

\vspace{0.1cm}

\begin{x}{\small\bf EXAMPLE} \ %23
The Sorgenfrey line is 0-dimensional.
\end{x}

\vspace{0.1cm}

\begin{x}{\small\bf REMARK} \ %24
As can be shown by example, a totally disconnected metric space need not be 0-dimensional.
\end{x}

\vspace{0.1cm}

\begin{x}{\small\bf FACT} \ %25
A locally compact Hausdorff space is 0-dimensional iff it is totally disconnected.

\vspace{0.1cm}

[Note: \ In such a space, each point has a neighborhood basis consisting of open-compact sets.]
\end{x}

\vspace{0.1cm}

A discrete space is 0-dimensional, hence is totally disconnected, hence a product of discrete spaces is totally disconnected, but an infinite product of nontrivial discrete spaces is never discrete.

\vspace{0.2cm}

\begin{x}{\small\bf DEFINITION} \ %26
The 
\un{Cantor space}
\index{Cantor space} 
is the countable product of the two-point discrete space.
\end{x}

\vspace{0.1cm}
\begin{x}{\small\bf FACT} \ %27
The Cantor set is homeomorphic to the Cantor space.
\end{x}

\vspace{0.1cm}





%%%%%%%%%%%%%%%%%%%%%%%%%%%%%%%%%%%%%%
%%%%%%%%%%%%%%%%%%%%%%%%%%%%%%%%%%%%%%
%%%%%%%%%%%%%%%%%%%%%%%%%%%%%%%%%%%%%%

\newpage
\setcounter{theoremn}{0}

%%----------------------------------------------------------------------------------------------01

\ \indent 

\[
\text{TOPOLOGICAL GROUPS}
\]
\begin{x}{\small\bf DEFINITION} \ %01
A 
\un{locally compact (compact)} group 
\index{locally compact (compact) group} 
is a topological group \mG that is both locally compact (compact) and Hausdorff.
\end{x}

\vspace{0.1cm}


\begin{x}{\small\bf FACT} \ %02
If \mG is a locally compact group and if \mH is a closed subgroup, then $G/H$ is a locally compact Hausdorff space.
\end{x}

\vspace{0.1cm}

\begin{x}{\small\bf FACT} \ %03
If \mG is a locally compact group and if \mH is a closed normal subgroup, then $G/H$ is a locally compact group.
\end{x}

\vspace{0.1cm}

\begin{x}{\small\bf FACT} \ %04
If \mG is a locally compact group and if \mH is a locally compact subgroup, then \mH is closed in \mG.
\end{x}

\vspace{0.1cm}

\begin{x}{\small\bf FACT} \ %05
If \mG is a locally compact 0-dimensional group and if \mH is a closed subgroup of \mG, then $G/H$ is 0-dimensional.
\end{x}

\vspace{0.1cm}

\begin{x}{\small\bf FACT} \ %06
If \mG is a totally disconnected locally compact group, then $\{e\}$ has a neighborhood basis consisting of open-compact subgroups.
\end{x}

\vspace{0.1cm}

\begin{x}{\small\bf FACT} \ %07
If \mG is a totally disconnected compact group,  then $\{e\}$ has a neighborhood basis consisting of open-compact normal subgroups. 
\end{x}

\vspace{0.1cm}

\begin{x}{\small\bf FACT} \ %08
If \mG is a locally compact group, then a subgroup \mH is open iff the quotient $G/H$ is discrete.
\end{x}

\vspace{0.1cm}

\begin{x}{\small\bf FACT} \ %09
If \mG is a compact group, then a subgroup \mH is open iff the quotient $G/H$ is finite.
\end{x}

\vspace{0.1cm}

\begin{x}{\small\bf FACT} \ %10
If \mG is a locally compact group, then every open subgroup of \mG is closed and every finite index closed subgroup of \mG is open.
\end{x}



















\chapter{
$\boldsymbol{\S}$\textbf{3}.\quad  COMPLETIONS}
\setlength\parindent{2em}
\setcounter{theoremn}{0}
%%----------------------------------------------------------------------------------------------01

\ \indent 

Let $\acdot$  be a an absolute value on a field $\F$ which satisfies the triangle inequality $-$then per $\acdot$, $\F$ might or might not be complete. 
%dmc
(Recall, a metric space is \un{complete} iff every Cauchy sequence converges.)

\vspace{0.1cm}

\begin{x}{\small\bf EXAMPLE} \ %1
Take $\F = \R$ or $\Q$ and let $\acdot = \acdot_\infty -$then $\R$ is complete but $\Q$ is not.
\end{x}

\vspace{0.1cm}

\begin{x}{\small\bf EXAMPLE} \ %2
Take $\F =  \Q$ and let $\acdot = \acdot_p -$then $\Q$ is not complete.

\vspace{0.05cm}

[To illustrate this, choose $p = 5$ and starting with $x_1 = 2$, define inductively a sequence $\{x_n\}$ of integers subject to
\[\left\{
\begin{array}{l l}
{x_n}^2 + 1 \equiv 0	 \quad \text{ $\mod 5^n$ } \\
{x_{n+1}} \equiv x_n	 \quad \ \  \text{ $\mod 5^n$ }
\end{array}
.
\right.\]
Then
\[
\abs{x_m - x_n}_5  \ \le \ 5^{-n}		\quad (m > n),
\]
so $\{x_n\}$ is a Cauchy sequence and, to get a contradiction, assume that it has a limit x in $\Q$, thus
\[
\begin{aligned}
\abs{x_n^2 + 1}_5 \le \  5^{-n} 
&\implies \abs{x^2 + 1}_5 = 0 \\
&\implies x^2 + 1 = 0 \dots .] 
\end{aligned}
\]
\end{x}


\vspace{0.1cm}

\begin{x}{\small\bf DEFINITION} \ %3
If an absolute value is not non-archimedean, then it is said to be \un{archimedean}.
\end{x}

\vspace{0.1cm}
%%----------------------------------------------------------------------------------------------02

\begin{x}{\small\bf FACT} \ %4
Suppose that $\F$ is a field which is complete with respect to an archimedean absolute value $\acdot$ 
$-$then $\F$ is isomorphic to either $\R$ or $\C$ and $\acdot$ is equivalent to $\acdot_\infty.$
\end{x}


\vspace{0.1cm}

\begin{x}{\small\bf RAPPEL} \ %5
Every metric space X has a completion 
$\compl{X}$
.  
Moreover, there is an isometry $\phi:X \ra \compl{X}$ such that $\phi$(X) is dense in $\compl{X}$ and $\compl{X}$ is unique up to isometric isomorphism.  
%dmc
(Recall, an isometry is a distance preserving mapping.  An isometry is injective, indeed, is a homeomorphism onto its image.)
\end{x}

\vspace{0.1cm}

\begin{x}{\small\bf CONSTRUCTION} \ %6
The standard model for $\compl{X}$ is the set of all Cauchy sequences in $X$ modulo the equivalence relation 
$\sim$, 
where
\[
\{x_n\} \sim \{y_n\} \Leftrightarrow d(x_n,y_n) \ra 0,
\]
the map $\phi:X \ra \compl{X}$ being the rule that sends $x \in$ X to the equivalence class of the constant sequence $x_n = x$.

[Note: The metric on $\compl{X}$ is specified by
\[
\ov{d}(\{x_n\},\{y_n\}) = \lim_{n \ra \infty} d(x_n,y_n).]
\]

Take $X = \F$ and 
\[
d(x,y) = \abs{x - y}.
\]
Then the claim is that $\compl{\F}$ is a field.  
E.g.: Let us deal with addition.  
Given $\ov{x}, \ov{y} \in \compl{\F}$, how does one define $\ov{x} + \ov{y}$ ?  
To this end, choose sequences
$
\begin{cases}
x_n  \\
y_n 
\end{cases}
$
in $\F$ such that
$
\begin{cases}
x_n \ra \ov{x} \\
y_n \ra \ov{y}
\end{cases}
$
$-$then
%%----------------------------------------------------------------------------------------------03
\[
\begin{aligned}
d(x_n + y_n,x_m + y_m) \ 
&= \  \abs{x_n + y_n - x_m - y_m}\\
&= \ \abs{(x_n - x_m ) + (y_n - y_m)}\\
&\leq \ \abs{x_n - x_m } + \abs{y_n - y_m}.
\end{aligned}
\]
Therefore  $\{x_n + y_n\}$ is a Cauchy sequence in $\F$, hence converges in $\compl{\F}$ to an element $\ov{z}$.  If
$
\begin{cases}
{x_n}^\prime  \\
{y_n}^\prime
\end{cases}
$
are sequences in $\F$ converging to 
$
\begin{cases}
\ov{x}  \\
\ov{y}
\end{cases}
$
as well, then $\{x_n^\prime + y_n^\prime\}$ converges in $\compl{\F}$ to an element $\ov{z}^\prime$.  
And
\[
\ov{z} = \ov{z}^\prime.
\]
Proof: \ Choose n $\in \N$ such that\\

$
\indent\indent\indent\indent
\begin{cases}
\abs{\ov{z} - (x_n + y_n)} \ \ < \ds\frac{\epsilon}{3}  \\
\abs{\ov{z}^\prime - (x_n^\prime  + y_n^\prime)} < \ds\frac{\epsilon}{3} 
\end{cases}
$
\\
and
\[
\abs{(x_n + y_n) - (x_n^\prime + y_n^\prime)}  \ 
\le \  \abs{x_n - x_n^\prime} + \abs{y_n - y_n^\prime} < \frac{\epsilon}{3}.
\]
Then
\[
\begin{aligned}
\abs{\ov{z} - \ov{z}^\prime} 
&\le \ \abs{\ov{z} - (x_n + y_n)} + \abs{\ov{z}^\prime - (x_n + y_n)}\\
&\le \ \abs{\ov{z} - (x_n + y_n)} + \abs{\ov{z}^\prime - 
(x_n^\prime + y_n^\prime)} + \abs{(x_n^\prime + y_n^\prime) - (x_n + y_n)} < \epsilon\\
&\implies \ov{z}= \ov{z}^\prime.
\end{aligned}
\]
Therefore addition in $\F$  extends to $\compl{\F}$.  
The same holds for multiplication and 
%%----------------------------------------------------------------------------------------------04
inversion.  
Bottom line: $\compl{\F}$ is a field.  
Furthermore, the prescription 
\[
\abs{\ov{x}} = \ov{d}(x,0)  	\quad (\ov{x} \in \compl{\F})
\]
is an absolute value on $\compl{\F}$ whose underlying topology is the metric topology.  
It thus follows that $\compl{\F}$ is a topological field (cf. $\S$2, $\#5$).
\end{x}

\vspace{0.1cm}

\begin{x}{\small\bf EXAMPLE} \ %7
Take $\F = \Q$, $\acdot = \acdot_p -$then the completion $\compl{\F} = \compl{\Q}$ 
is denoted by $\Q_p$, the field of \un{ $p$-adic numbers}.
\end{x}

\vspace{0.1cm}

\begin{x}{\small\bf LEMMA} \ %8
If $\acdot$ is non-archimedean per $\F$, then $\acdot$ is non-archimedean per $\compl{\F}$.

\vspace{0.1cm}

PROOF \   
Given
$
\begin{cases}
\ov{x}\\
\ov{y}
\end{cases}
\in \compl{\F},
$
choose
$
\begin{cases}
x_n\\
y_n
\end{cases}
\in \F 
$
such that
$
\begin{cases}
x_n \ra \ov{x}_n  \\
y_n \ra \ov{y}_n
\end{cases}
\text{in } \compl{\F}:
$
\allowdisplaybreaks
\[
\begin{aligned}
\abs{\ov{x} - \ov{y}} 
&\le \ \abs{\ov{x} - x_n + x_n - y_n + y_n - \ov{y}}\\
&\le \ \abs{\ov{x} - x_n} + \abs{x_n - y_n} + \abs{y_n - \ov{y}}.\\
& \qquad\quad \downarrow \qquad\qquad\qquad\qquad\quad  \downarrow\\
& \qquad\quad \  0 \qquad\qquad\qquad\qquad\quad  0
\end{aligned}
\]
And
\[
\begin{aligned}
\abs{x_n - y_n}  \ 
&\le \  \sup(\abs{x_n},\abs{y_n})\\
&= \  \frac{1}{2} (\abs{x_n} + \abs{y_n}) + \abs{x_n - y_n})\\
&\ra \ \frac{1}{2} (\abs{\ov{x}} + \abs{\ov{y}}) + \abs{\ov{x} - \ov{y}})\\
&= \  \sup(\abs{\ov{x}},\abs{\ov{y}}).
\end{aligned}
\]
\end{x}

\vspace{0.1cm}
%%----------------------------------------------------------------------------------------------05

\begin{x}{\small\bf LEMMA} \ 
If $\acdot$ is non-archimedean per $\acdot$, then
\[
\{\abs{\ov{x}}: \ov{x} \in \compl{\F}\} = \{\abs{x}: x \in \F\}.
\]

\vspace{0.1cm}

PROOF \    Take $\abs{\ov{x}} \in \compl{\F}:\ov{x} \ne 0$.  
Choose x $\in \F:\abs{\ov{x} - x} \ < \ \abs{\ov{x}}.$ 
Claim: $\abs{\ov{x}} = \abs{x}$.
\\
Thus, consider the other possibilities.
\[
\begin{aligned}
\text{\textbullet} \abs{x} < \abs{\ov{x}}:\\
&\abs{\ov{x} - x} = \abs{\ov{x} + (-x)} = \abs{\ov{x}}	\quad (\text{c.f.}  \ \S1, \ \# 18) < \abs{\ov{x}} \dots \ .\\
\text{\textbullet} \abs{\ov{x}} < \abs{x}:\\
&\abs{\ov{x} - x} = \abs{-x + \ov{x}} = \abs{-x}	 \quad (\text{c.f.} \   \S1, \ \# 18) = \abs{x} < \abs{\ov{x}} \dots \ .
\end{aligned}
\]
\end{x}
\vspace{0.1cm}

\begin{x}{\small\bf EXAMPLE} \ %10
The image of $\Q_p$ under $\acdot_p$ is the same as the image of $\Q$ under $\acdot_p$, namely
\[
\{p^k:k \in\Z\} \cup \{0\}.
\]
\end{x}
\vspace{0.1cm}

Let $\K$ be a field, $\LL/\K$ a finite field extension.\\

\begin{x}{\small\bf EXTENSION PRINCIPLE} \ %11
Let $\acdot_\K$ be a complete absolute value on $\K$ $-$then there is one and only one extension 
$\acdot_\LL$ of $\acdot_\K$  to $\LL$ and it is given by
\[
\abs{x}_\LL = \abs{N_{\LL/\K}(x)}_\K^{1/n},
\]
where n = $[\LL:\K]$.  In addition, $\LL$ is complete with respect to $\acdot_\LL$. 

\vspace{0.1cm}

[Note: $\acdot_\LL$ is non-archimedean if $\acdot_\K$ is non-archimedean.$]$
\end{x}
\vspace{0.1cm}

\begin{x}{\small\bf SCHOLIUM} \ %12
There is a unique extension of $\acdot_\K$ to the algebraic closure 
$\K^{c\ell}$ 
of $\K$.\\
%%----------------------------------------------------------------------------------------------06
\indent [Note: It is not true in general that $\K^{c\ell}$  is complete.$]$\\

Suppose further that $\LL/\K$ is a Galois extension.  
Given $\sigma \in$ $\Gal(\LL/\K)$, define 
$\acdot_\sigma$ by 
$\abs{x}_\sigma = \abs{\sigma x}_\LL$
$-$then
%\[
%\acdot_\sigma |_\K = \acdot_\K,
%\]
\[
\restr{\acdot_\sigma}{\K} = \acdot_\K,
\]
so by uniqueness, $\acdot_\sigma$ = $\acdot_L$. 
But
\[
N_{\LL/\K}(x) = \prod_{\sigma\in \Gal(\LL/\K)} \sigma x
\]
\indent\indent$\implies$\\
\[
\begin{aligned}
\abs{N_{\LL/\K}(x)}_\K \ 
&= \ \abs{N_{\LL/\K}(x)}_\LL \\
&= \  \bigl |\prod_{\sigma\in \Gal(\LL/\K)} \sigma x \bigr |_\LL\\
&= \  \prod_{\sigma\in \Gal(\LL/\K)} \abs{\sigma x}_\LL\\
&= \  \prod_{\sigma\in \Gal(\LL/\K)} \abs{x}_\LL\\
&= \  \abs{x}_\LL^{\#(\Gal(\LL/\K))}\\
&= \  \abs{x}_\LL^{[\LL:\K]}\\
&= \  \abs{x}_\LL^n.
\end{aligned}
\]
\end{x}
\vspace{0.1cm}
%%----------------------------------------------------------------------------------------------App

\[
\textbf{APPENDIX}
\]
\setcounter{theoremn}{0}

\begin{x}{\small\bf APPROXIMATION PRINCIPLE} \ 
Let $\acdot_1, \dots , \acdot_N$ be pairwise inequivalent non-trivial absolute values on $\F$ .  
Fix elements $a_1, \dotsb , a_N$ in $\F$ $-$then $\forall$  $\epsilon > 0$, $\exists$ $a_\epsilon \in \F$:
\[
\abs{a_\epsilon - a_k}_k \  < \  \epsilon			\quad (k = 1, \dotsb, N). 
\]
%%----------------------------------------------------------------------------------------------07
\indent Let $\compl{\F}_1, \dotsb , \compl{\F}_N$ be the associated completions and let
\[
\Delta:\F \ra \prod_{k=1}^N \compl{\F}_k
\]
be the diagonal map $-$then the image $\Delta \F$ is dense (i.e., its closure is the whole of $\prod_{k=1}^N \compl{\F}_k$).

[Fix $ \epsilon > 0$ and elements $\ov{a}_1, \dots , \ov{a}_N$ in $\compl{\F}_1, \dots , \compl{\F}_N$ respectively 
$-$then there exist elements $a_k \in \F$:
\[
\abs{a_k - \ov{a}_k}_k < \epsilon			\quad (k = 1, \dots, N). 
\]
Choose $a_\epsilon \in \F$:
\[
\abs{a_\epsilon  - \ov{a}_k} < \epsilon	\quad (k = 1, \dots, N).
\]
Then
\[
\begin{aligned}
\abs{a_\epsilon  - \ov{a}_k}_k 
&= \ \abs{(a_\epsilon - a_k) + (a_k - \ov{a}_k)}_k\\
&\le \ \abs{a_\epsilon  - a_k}	 + \abs{a_k  - \ov{a}_k}_k\\
&< \ 2\epsilon.]
\end{aligned}
\]
\end{x}

\begin{x}{\small\bf \un{N.B.}} \ 
The product $\prod\limits_{k=1}^N \compl{\F}_k$ carries the product topology and the prescription
\[
\begin{aligned}
d((\ov{a}_1, \dots , \ov{a}_N), (\ov{b}_1, \dots , \ov{b}_N)) 
&= \ \sup_{1 \le k \le N} d_k(\ov{a}_k, \ov{b}_k) \\
&= \  \sup_{1 \le k \le N} \abs{\ov{a}_k - \ov{b}_k}_k
\end{aligned}
\]
%%----------------------------------------------------------------------------------------------08
metrizes the product topology.  Therefore 
\[
\begin{aligned}
d((a_\epsilon, \dots , a_\epsilon), (\ov{a}_1, \dots , \ov{a}_N)) \
&= \  \sup_{1 \le k \le N} d_k(a_\epsilon, \ov{a}_k)\\
&= \ \sup_{1 \le k \le N} \abs{a_\epsilon - \ov{a}_k}_k\\
&< \  2\epsilon.
\end{aligned}
\]
\end{x}

%%%%%%%%%%%%%%%%%%%%%%%%%%%%%%%%%%%%%%
%%%%%%%%%%%%%%%%%%%%%%%%%%%%%%%%%%%%%%
%%%%%%%%%%%%%%%%%%%%%%%%%%%%%%%%%%%%%%






















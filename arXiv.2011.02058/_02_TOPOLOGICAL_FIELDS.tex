\chapter{
$\boldsymbol{\S}$\textbf{2}.\quad  TOPOLOGICAL FIELDS}
\setlength\parindent{2em}
\setcounter{theoremn}{0}

%%----------------------------------------------------------------------------------------------01

\ \indent Let $\abs{\hspace{0.05cm}\cdot \hspace{0.05cm}}$ be an absolute value on a field $\F$. Given a $\in \F, r > 0$, put
\[
N_r(a) = \{b: \abs{b - a} < r\}.
\]


\begin{x}{\small\bf LEMMA} \ %1
There is a topology on $\F$ in which a basis for the neighborhoods of a are the $N_r(a)$.\\

PROOF \  The nontrivial point is to show that given  $V \in$ $\sB_a$
%dmc
($\sB_a$ = the set of open balls centered at $a$)\xspace
, 
there is a $V_0 \in \sB_a$ such that if $a_0 \in V_0$, then there is a $W \in \sB_{a_0}$ such that $W \subset V$.  
So let V = $N_r(a)$, $V_0 = N_{r/2M}(a)$, $W = N_{r/2M}(a_0)$ $(a_0 \in V_0) -$ then $W \subset V:$
\[
\begin{aligned}
b \in W \implies \abs{b - a} \ 
&= \ \abs{(b - a_0) + (a_0 - a)}\\
&\le \ M \sup(\abs{b - a_0}, \abs{a_0 - a}) \\
&\le \ M \sup(r/2M, r/2M)\\
&= \  M(r/2M) \\
&= \ r/2 \\
&< \  r.
\end{aligned}
\]
\end{x}



\begin{x}{\small\bf EXAMPLE} \ %2
The topology induced by $\acdot$ is the discrete topology iff  $\acdot$ is the trivial absolute value.
\end{x}


\begin{x}{\small\bf FACT} \ %3
Absolute values $\acdot_1$, and $\acdot_2$ are equivalent iff they give rise to the same topology.
\end{x}
%\vspace{0.1cm}

\begin{x}{\small\bf LEMMA} \ %4
The topology induced by $\acdot$  is metrizable.\\

PROOF \  This is because $\acdot$ is equivalent to an absolute value satisfying the 
%%----------------------------------------------------------------------------------------------02
triangle inequality (cf. $\S1$, \ \#14), the underlying metric being 
\[
d(a,b) \ = \  \abs{a - b}.
\]
\end{x}
%\vspace{0.1cm}

\begin{x}{\small\bf THEOREM} \ %5
A field with a topology defined by an absolute value is a 
\underline{topological} \underline{field} 
\index{topological field}
i.e., the operations sum, product, and inversion are continuous.\\
\end{x}
\vspace{0.1cm}

Assume now that $\acdot$  is non-archimedean, hence that the ultrametric inequality 
\[
\abs{a - b} \ \le \  \sup (\abs{a},\abs{b})
\]
is in force.\\

\begin{x}{\small\bf LEMMA} \ %6
$N_r(a)$ is closed (open is automatic).\\

\indent PROOF \   Let $p$ be a limit point of $N_r(a)$ $-$then $\forall$  $t > 0$,
\[
(N_t(p) - \{p\}) \cap N_r(a) \ne \emptyset
\]
Take $t = \ds\frac{r}{2}$ and choose $b \in N_r(a):$
\[
d(p,b) < \frac{r}{2}	\quad \text{$(p \ne b)$}.
\]
Then
\[
\begin{aligned}
d(a,p) 
&\le  \ \sup(d(a,b), d(b,p)) \\
&< \  r
\end{aligned}
\]
$\implies$
\[
p \in N_r(a).
\]
Therefore, $N_r(a)$ contains all its limit points, hence is closed.\\
\end{x}
%\vspace{0.1cm}

\begin{x}{\small\bf LEMMA} \ %7
If $a^{\prime} \in N_r(a)$, then $N_r(a^{\prime} ) = N_r(a)$.\\

PROOF \  E.g:
\[
b \in N_r(a) \implies \abs{b - a} < r
\]
%%----------------------------------------------------------------------------------------------03
$\implies$
\[
\begin{aligned}
\abs{b - a^{\prime}} 
&= \  \abs{(b - a) + (a - a^{\prime})}\\
&\le \  \sup(\abs{b - a},\abs{a - a^{\prime}})\\
&< r %\implies N_r(a) \subset N_r(a^{\prime}).
\end{aligned}
\]
$\indent\indent\implies$ 
\[
N_r(a) \subset N_r(a^{\prime}).
\]
\end{x}
%\vspace{0.1cm}

\begin{x}{\small\bf REMARK} \ %8
Put
\[
B_r(a) = \{b:\abs{b - a} \le r\}.
\]
Then a priori, $B_r(a)$ is closed.  But $B_r(a)$ is also open and if $a^{\prime} \in B_r(a)$, then  $B_r(a^{\prime}) = B_r(a)$.
\end{x}
\vspace{0.1cm}

\begin{x}{\small\bf LEMMA} \ %9
If
\[
a_1 + a_2 + \dotsb + a_n = 0, 
\]
then $\exists$  $i \ne j$ such that
\[
\abs{a_i} = \abs{a_j} = \sup\abs{a_k}.
\]

PROOF \
Without loss of generality write $a_1 = \sup\limits_{1 \leq k \leq n}\abs{a_k}$.  Then
%$\implies$
\allowdisplaybreaks
\begin{align*}
\abs{a_1} 	\ 
&=\  \abs{0 - a_1}\\	
&=\  \abs{a_1 + a_2 + \dotsb + a_n - a_1}\\
&=\  \abs{a_2 + \dotsb + a_n}\\	
&\leq\ \sup\limits_{2 \leq k \leq n}\abs{a_k}\\
&=\ \abs{a_j} \qquad (\exists \ j: 2 \leq j \leq n)\\
&\leq\ \sup\limits_{1\leq k \leq n}\abs{a_k}\\
&= \abs{a_1}.
\end{align*}

\end{x}



















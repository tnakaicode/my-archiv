\chapter{
$\boldsymbol{\S}$\textbf{7}.\quad  HARMONIC ANALYSIS}
\setlength\parindent{2em}
\setcounter{theoremn}{0}
%%----------------------------------------------------------------------------------------------01

\ \indent 
Let $G$ be a locally compact abelian group.

\vspace{0.25cm}

\begin{x}{\small\bf DEFINITION} \ %1
A 
\un{character}
\index{character} 
of $G$ is a continuous homomorphism $\chi:G \ra \C^\times$.
\end{x}

\vspace{0.1cm}

\begin{x}{\small\bf NOTATION} \ %2
Write $\widetilde{G}$ for the group whose elements are the characters of $G$.
\end{x}

\vspace{0.1cm}

\begin{x}{\small\bf DEFINITION} \ %3
A 
\un{unitary character}
\index{unitary character} 
of $G$ is a continuous homomorphism $\chi: G \ra \T$.
\end{x}

\vspace{0.1cm}

\begin{x}{\small\bf NOTATION} \ %4
Write $\widehat{G}$ for the group whose elements are the unitary characters of $G$.
\end{x}

\vspace{0.1cm}


\begin{x}{\small\bf LEMMA} \ %5
There is a decomposition
\[
\widetilde{G} \approx \widetilde{G}_+ \times \widehat{G},
\]
where $\widetilde{G}_+$ is the group of positive characters of $G$.

\vspace{0.1cm}

PROOF \  The only positive unitary character is trivial, so $\widetilde{G}_+ \cap \widehat{G} = \{1\}$.  On the other hand, if $\chi$ is a character, then $\abs{\chi}$ is a positive character, $\chi / \abs{\chi}$ is a unitary character, and $\chi = \abs{\chi} (\ds\frac{\chi}{\abs{\chi}}$).
\end{x}

\vspace{0.1cm}

\begin{x}{\small\bf LEMMA} \ %6
Every bounded character of $G$ is a unitary character.

\vspace{0.1cm}

PROOF \  The only compact subgroup of $\R_{>0}$ is the trivial subgroup $\{1\}$.
\end{x}


\vspace{0.1cm}

\begin{x}{\small\bf APPLICATION} \ %7
If $G$ is compact, then every character of $G$ is unitary.
\end{x}


\begin{x}{\small\bf EXAMPLE} \ %8
Take $G = \Z$ $-$then $\widetilde{G} \approx \C^\times$, the isomorphism being given by the map $\chi \ra \chi (1)$.
\end{x}

\vspace{0.1cm}
%%----------------------------------------------------------------------------------------------02


\begin{x}{\small\bf EXAMPLE} \ %9
Take $G = \R$ $-$then $\widetilde{G} \approx \R \times \R $ and every character has the form 
$\chi (x) = e^{zx}$  $(z \in \C)$.
\end{x}

\vspace{0.1cm}


\begin{x}{\small\bf EXAMPLE} \ %10
Take $G = \C$ $-$then $\widetilde{G} \approx \C \times \C $ and every character has the form 
$\chi (x) = \exp(z_1 \Re(x) + z_2 \Im(x))$  \ $(z_1, z_2 \in \C)$.
\end{x}
\vspace{0.1cm}

\begin{x}{\small\bf EXAMPLE} \ %11
Take $G = \R^\times$ $-$then $\widetilde{G} \approx \Z/ 2 \Z \times \C$, and every character has the form 
$\chi (x) = (\sgn x)^\sigma \abs{x}^s$ $(\sigma \in \{0,1\}, \ s \in \C)$.
\end{x}
\vspace{0.1cm}

\begin{x}{\small\bf EXAMPLE} \ %12
Take $G = \C^\times$ $-$then $\widetilde{G} \approx \Z \times \C$, 
and every character has the form $ \chi (x) = \exp(\sqrt{-1} $ $n \arg x) \abs{x}^s$ $(n \in \Z, s \in \C)$.
\end{x}

\vspace{0.1cm}

\begin{x}{\small\bf DEFINITION} \ %13
The 
\un{dual group}
\index{dual group} 
of $G$ is $\widehat{G}$.
\end{x}

\vspace{0.1cm}

\begin{x}{\small\bf RAPPEL} \ %14
Let $X$, $Y$ be topological spaces and let $F$ be a subspace of $C(X,Y)$.  
Given a compact set $K \subset X$ and an open subset $V \subset Y$, 
let $W(K,V)$ be the set of all $f \in F$ such that $f(K) \subset V$ 
$-$then the collection $\{W(K,V)\}$ is a subbasis for the 
\un{compact open topology}
\index{compact open topology} 
on $F$.

\vspace{0.1cm}
$[$Note: The family of finite intersections of sets of the form $W(K,V)$ is then a basis for the compact open topology: 
Each member has the form $\bigcap\limits_{i= 1}^n W(K_i,V_i)$, where the $K_i \subset X$ 
are compact and the $V_i \subset Y$ are open.]
\end{x}

\vspace{0.1cm}

Equip $\widehat{G}$ with the compact open topology.
\vspace{0.1cm}
\begin{x}{\small\bf FACT} \ %15
The compact open topology on $\widehat{G}$ coincides with the topology of uniform convergence on compact subsets of $G$.
\end{x}
%%----------------------------------------------------------------------------------------------03

\begin{x}{\small\bf LEMMA} \ %16
$\widehat{G}$ is a locally compact abelian group.
\end{x}


\begin{x}{\small\bf REMARK} \ %17
$\widetilde{G}$  is also a locally compact abelian group and the decomposition
\[
\widetilde{G} \approx \widetilde{G}_+ \times \widehat{G}
\]
is topological.
\end{x}
\vspace{0.1cm}

\begin{x}{\small\bf EXAMPLE} \ %18
Take $G = \R$ and given a real number $t$, let $\chi_t(x) = e^{\sqrt{-1}\ tx}$ $-$then $\chi_t$ 
is a unitary character of $G$ and for any $\chi \in \widehat{G}$, there is a unique $t \in \R$ such that $\chi = \chi_t$, 
hence $G$ can be identified with $\widehat{G}$.
\end{x}

\vspace{0.1cm}

\begin{x}{\small\bf EXAMPLE} \ %19
Take $G = \R^2$ and given a point $(t_1,t_2)$, let $\chi_{(t_1,t_2)}(x_1,x_2) = e^{\sqrt{-1}(t_1x_1 + t_2x_2)}$ 
$-$then $\chi_{(t_1,t_2)}$ is a unitary character of $G$ and for any $\chi \in \widehat{G}$, 
there is a unique $(t_1,t_2) \in \R^2$ such that $\chi = \chi_{(t_1,t_2)}$, 
hence $G$ can be identified with $\widehat{G}$.
\end{x}

\vspace{0.1cm}

\begin{x}{\small\bf EXAMPLE} \ %20
Take $G = \Z/ n\Z$ and given an integer $m = 0, 1, \dotsb, n-1$, let 
$
\chi_m(k) = \exp\bigl(2\pi \sqrt{-1} \ \ds\frac{km}{n}\bigr)
$
$-$then $\chi_0, \chi_1, \dotsb, \chi_{n-1}$ are characters of $G$, hence $G$ can be identified with $\widehat{G}$.
\end{x}

\vspace{0.1cm}

\begin{x}{\small\bf LEMMA} \ %21
If $G$ is compact, then $\widehat{G}$ is discrete.
\end{x}

\vspace{0.1cm}

\begin{x}{\small\bf EXAMPLE} \ %22
Take $G = \T$ and given $n \in$ $\Z$, let 
$\chi_n\bigl(e^{\sqrt{-1}\ \theta}\bigr ) = e^{\sqrt{-1} \ n\theta}$ 
$-$then $\chi_n$ is a unitary character of $G$ and all such have this form, so $\T$ $\approx \Z$.
\end{x}

\vspace{0.1cm}
%%----------------------------------------------------------------------------------------------04

\begin{x}{\small\bf LEMMA} \ %23
If $G$ is discrete, then $\widehat{G}$ is compact.
\end{x}

\vspace{0.1cm}

\begin{x}{\small\bf EXAMPLE} \ %24
Take $G = \Z$ and given $e^{\sqrt{-1}\ \theta}  \in \T$, let $\chi_\theta(n) = e^{\sqrt{-1} \ \theta n}$ 
$-$then $\chi_\theta$ is unitary character of $G$ and all such have this form, so $\widehat{\Z} \approx \T$.
\end{x}

\vspace{0.1cm}

\begin{x}{\small\bf LEMMA} \ %25
If $G_1$, $G_2$ are locally compact abelian groups, then 
$\reallywidehat{G_1 \times G_2}$
%$\widehat{G_1 \times G_2}$ 
is topologically isomorphic to $\widehat{G_1} \times \widehat{G_2}$.
\end{x}

\vspace{0.1cm}

\begin{x}{\small\bf EXAMPLE} \ %26
Take $G = \R^\times$ $-$then 
\[
G \approx \Z / 2 \Z \times \R_{>0}^\times \ \approx \ \Z / 2 \Z \times \R, 
\]
thus $\widehat{G}$ is topologically isomorphic to $\Z / 2 \Z \times \R:$
\[
(u,t) \ra \chi_{(u,t)}			\quad (u \in \Z / 2 \Z, t \in  \R),
\]
where
\[
\chi_{(u,t)}(x) = \left (\ds\frac{x}{\abs{x}}\right)^u \abs{x}^{\sqrt{-1}\  t}.
\]
\end{x}

\vspace{0.2cm}

\begin{x}{\small\bf EXAMPLE} \ %27
Take $G = \C^\times$ $-$then 
\[
G \approx \T \times \R_{>0}^\times \ \approx \T \ \times \R, 
\]
thus $\widehat{G}$ is topologically isomorphic to $\Z \times \R:$
\[
(n,t) \ra \chi_{n,t}			\quad (n \in \Z , t \in  \R),
\]
where
\[
\chi_{(n,t)}(z) = \left(\ds\frac{z}{\abs{z}}\right)^n \abs{z}^{\sqrt{-1} \  t}.
\]
\end{x}

\vspace{0.5cm}

Denote by $\ev_G$ the canonical arrow 
$G \ra \reallywidehat{\reallywidehat{G}}:$
\[
\ev_G(x) (\chi) = \chi (x).
\]

\vspace{0.1cm}
%%----------------------------------------------------------------------------------------------05

\begin{x}{\small\bf REMARK} \ %28
If $G$, $H$ are locally compact abelian groups and if $\phi : G \ra H$ is a continuous homomorphism, then there is a commutative diagram
\[
\begin{tikzcd}[sep=large]
G \ar{d}[swap]{\phi} \ar{rr}{\text{ev}_G}	
&&\reallywidehat{\reallywidehat{G}} \ar{d}{\reallywidehat{\reallywidehat{\phi}}}\\
H \ar{rr}{\text{ev}_H} 		
&&\reallywidehat{\reallywidehat{H}}
\end{tikzcd}
\quad .
\]
\end{x}

\vspace{0.1cm}


\index{Pontryagin duality}
\begin{x}{\small\bf PONTRYAGIN DUALITY} \ %29
$\ev_G$ is an isomorphism of groups and a homeomorphism of topological spaces.
\end{x}

\vspace{0.1cm}


\begin{x}{\small\bf SCHOLIUM} \ %30
Every compact abelian group is the dual of a discrete abelian group and every discrete abelian group is the dual of a compact abelian group.
\end{x}

\vspace{0.1cm}

\begin{x}{\small\bf REMARK} \ %31
Every finite abelian group $G$ is isomorphic to its dual $\widehat{G}:$ $G \approx \widehat{G}$  (but the isomorphism is not "functorial").
\end{x}

\vspace{0.1cm}

Let $H$ be a closed subgroup of $G$.

\vspace{0.2cm}

\index{$H^\perp$}
\begin{x}{\small\bf NOTATION} \ %32
Put
\[
H^\perp = \{\chi \in \widehat{G}: \restr{\chi}{H} = 1\}.
\]
\end{x}
\vspace{0.1cm}

\begin{x}{\small\bf LEMMA} \ %33
$H^\perp$ is a closed subgroup of $\widehat{G}$ and H = $H^{\perp \perp}$.
\end{x}

\vspace{0.1cm}

Let $\pi_H:G \ra G/H$ be the projection and define
\[\left\{
\begin{array}{l l}
\Phi : \widehat{G/H} \ra H^\perp\\
\Psi : \widehat{G}/H^\perp \ra \widehat{H}
\end{array}
\right.\]
%%----------------------------------------------------------------------------------------------06
by
\[\left\{
\begin{array}{l l}
\Phi(\chi) = \chi \circ \pi_H \\
\Psi(\chi H^\perp) = \restr{\chi}{H}.
\end{array}
\right.\]

\vspace{0.1cm}

\begin{x}{\small\bf LEMMA} \ %34
$\Phi$ and $\Psi$ are isomorphisms of topological groups.
\end{x}

\vspace{0.1cm}

\begin{x}{\small\bf APPLICATION} \ %35
Every unitary character of $H$ extends to a unitary character of $G$.
\end{x}

\vspace{0.1cm}

\begin{x}{\small\bf EXAMPLE} \ %36
Let $G$ be a finite abelian group and let $H$ be subgroup of $G$ $-$then $G$ contains a subgroup isomorphic to $G/H$.

[In fact, 
\[
G/H  \approx \  \widehat{G/H}\  \approx \  H^\perp \ \subset \  \widehat{G} \ \approx \ G.]
\]
\end{x}

\vspace{0.1cm}

\begin{x}{\small\bf REMARK} \ %37
Denote by $\textbf{LCA}$ the category whose objects are the locally compact abelian groups and whose morphisms are the continuous homomorphisms $-$then
\[
\widehat{ } : \mathbf{LCA} \ra \mathbf{LCA}
\]
is a contravariant functor.  This said, consider the short exact sequence
\[
\begin{tikzcd}%[sep=small]
1 	\ar{r}	
&H	\ar{r}	
&G 	\ar{rr}{\pi_H}	
&&{G/H}	\ar{r}  
&{1}
\end{tikzcd}
\]
and apply  $ $   $\widehat{ }:$
\[
\begin{tikzcd}%[sep=small]
1 	\ar{r}	
&{\widehat{G/H} \  \approx \ H^\perp} 	\ar{r}	
&\widehat{G} 	\ar{r}	
&{\widehat{H} \ \approx \  \widehat{G}/H^\perp}	\ar{r}  
&{1}
\end{tikzcd}
,
\]
which is also a short exact sequence.
\end{x}
%%----------------------------------------------------------------------------------------------07

\vspace{0.1cm}

Given f $\in L^1(G)$, its 
\un{Fourier transform}
\index{Fourier transform} 
is the function 
\[
\widehat{f} : \widehat{G} \ra \C
\]
defined by the rule
\[
\widehat{f}(\chi) = \int_G f(x) \chi(x) d \mu_G(x).
\]

\vspace{0.1cm}

\begin{x}{\small\bf EXAMPLE} \ %38
Take $G = \R$ $-$then $\widehat{\R} \approx \R$ and
\[
\widehat{f}(\chi_t) \equiv \widehat{f}(t) = \int_{-\infty}^\infty  f(x) e ^{\sqrt{-1}\  tx} dx.
\]
\end{x}

\vspace{0.1cm}

\begin{x}{\small\bf EXAMPLE} \ %39
Take $G = \R^2$ $-$then $\widehat{\R}^2 \approx \R^2$ and
\[
\widehat{f}(\chi_{(t_1,t_2)}) \equiv \widehat{f}(t_1 , t_2) \ 
= \ \int_{-\infty}^\infty  \int_{-\infty}^\infty  f(x_1, x_2) e ^{\sqrt{-1} \ (t_1 x_1 + t_2 x_2)} dx_1 dx_2.
\]
\end{x}

\vspace{0.1cm}

\begin{x}{\small\bf EXAMPLE} \ %40
Take $G = \T$ $-$then $\widehat{\T} \approx \Z$ and
\[
\widehat{f}(\chi_n) \equiv \widehat{f}(n) 
=  \int_{0}^{2\pi}  f(\theta) e^{\sqrt{-1} \ n \theta} d\theta
\]
\end{x}

\vspace{0.1cm}

\begin{x}{\small\bf EXAMPLE} \ %41
Take $G = \Z$ $-$then $\widehat{Z} \approx \T$ and
\[
\widehat{f}(\chi_\theta) \equiv \widehat{f}(\theta) =  \sum_{n = -\infty}^\infty f(n) e^{\sqrt{-1} \ n \theta}.
\]
\end{x}

\vspace{0.1cm}

\begin{x}{\small\bf EXAMPLE} \ %42
Take $G = \Z/n\Z$ $-$then $\widehat{\Z/n\Z} \approx \Z/n\Z$ and
\[
\widehat{f}(\chi_m) \equiv \widehat{f}(m) =  \sum_{k = 0}^{n - 1}f(k) \exp(2 \pi \sqrt{-1} \ \frac{km}{n}). 
\]
\end{x}

\vspace{0.1cm}

\begin{x}{\small\bf LEMMA} \ %43
$\widehat{f} : \widehat{G} \ra \C$ is a continuous function on $\widehat{G}$ that vanishes at infinity and
%%----------------------------------------------------------------------------------------------08
\[
\| \widehat{f} \|_\infty \le \| f  \|_1.
\]
\end{x}

\vspace{0.1cm}

\index{$\mathbf{INV}$}
\begin{x}{\small\bf NOTATION} \ %44
$\mathbf{INV}(G)$ is the set of continuous functions $f \in L^1(G)$ with the property that $\widehat{f} \in L^1(\widehat{G})$.
\end{x}

\vspace{0.1cm}

\index{Fourier inversion}
\begin{x}{\small\bf FOURIER INVERSION} \ %45
Given a Haar measure $\mu_G$ on $G$, 
there exists a unique Haar measure $\mu_{\widehat{G}}$ on $\widehat{G}$ such that $\forall f \  \in \mathbf{INV}(G)$,
\[
f(x) = \int_{\widehat{G}} \widehat{f}(\chi) \ov{\chi (x)} d\mu_{\widehat{G}} (\chi).
\]
\end{x}

\vspace{0.1cm}

If $G$ is compact, then it is customary to normalize $\mu_G$ by the requirement $\int_G 1 d\mu_G = 1$.

\vspace{0.1cm}


\begin{x}{\small\bf LEMMA} \ %46
\[ 
\int_G \chi(x) d\mu_G(x) \ = \  
\begin{cases}
\ 1  \quad \text{if} \  \chi = 0\\
\ 0 \quad \text{if} \ \chi \ne 0
\end{cases}
.
\]

\vspace{0.1cm}

PROOF \  The case $\chi = 0$ is clear.  On the other hand, if $\chi \ne 0$, then there exists $x_0 : \chi (x_0) \ne 1$, hence
%\[
\begin{align*}
\int_G \chi(x) d\mu_G(x) \ 
&= \ \int_G \chi (x - x_0 + x_0) d\mu_G (x)\\
&= \ \chi (x_0) \int_G \chi (x - x_0) d\mu_G (x)\\
&= \ \chi (x_0) \int_G \chi (x) d\mu_G (x)
\end{align*}
%\]
$\implies$
\[
\int_G \chi (x) d\mu_G (x) = 0 .
\]
\end{x}

\vspace{0.1cm}
%%----------------------------------------------------------------------------------------------09

Assuming still that $G$ is compact ($\implies \widehat{G}$ is discrete), take $f \equiv 1:$
\[
\widehat{f}(0) = 1, \   \widehat{f}(\chi) = 0 	 \quad (\chi \ne 0). 
\]
I.e.$:$	$\widehat{f}$ is the characteristic function of \{0\}, hence is integrable, thus $f \in \mathbf{INV}(G)$.  
Accordingly, if $\mu_{\widehat{G}}$ is the Haar measure on $\widehat{G}$ per Fourier inversion, then
\begin{align*}
1 \ 
&= \ f(0) \\
&= \ \int_{\widehat{G}} \widehat{f}(\chi) d\mu_{\widehat{G}} (\chi) \\
&= \ \mu_{\widehat{G}} (\{0\}), 
\end{align*}
so $\forall$ $\chi \in \widehat{G}$,
\[
\mu_{\widehat{G}} (\{\chi\}) = 1.
\]

\vspace{0.1cm}

\begin{x}{\small\bf EXAMPLE} \ %47
Let $G = \T$ $-$then $d\mu_G = \ds\frac{d\theta}{2\pi}$, so for $f \in \mathbf{INV}(G)$,
\[
f(\theta) = \sum_{n = -\infty}^\infty \widehat{f}(n) e^{-\sqrt{-1} \ n \theta},
\]
where
\[
\widehat{f}(n) = \int_0^{2\pi} f(\theta) e^{\sqrt{-1} \  n \theta} \frac{d\theta}{2\pi}.
\]

\vspace{0.1cm}

If $G$ is discrete, then it is customary to normalize $\mu_G$ by stipulating that singletons are assigned measure 1.
\end{x}

\vspace{0.1cm}

\begin{x}{\small\bf REMARK} \ %48
There is a conflict if $G$ is both compact and discrete, i.e., if $G$ if finite.
\end{x}

\vspace{0.1cm}

\allowdisplaybreaks
Assuming still that $G$ is discrete ( $\implies \widehat{G}$ is compact), take $f(0) = 1, f(x) = 0$ $(x \ne 0):$
%%----------------------------------------------------------------------------------------------10
\begin{align*}
\widehat{f}(\chi)	 \ 
&= \  \int_G f(x) \chi(x) d\mu_G(x)\\	
&= \ f(0) \chi(0) \mu_G(\{0\})\\		
&= \ 1.
\end{align*}
I.e.$:$ $\widehat{f}$ is the constant function 1, hence is integrable, thus $f \in \mathbf{INV}(G)$.  Accordingly, if $\mu_{\widehat{G}}$ is the Haar measure on $\widehat{G}$ per Fourier inversion, then
\begin{align*}
\mu_{\widehat{G}}(\widehat{G}) 	 \ 
&= \ \int_{\widehat{G}} 1 d \mu_{\widehat{G}}(\chi)  \\		 						
&= \  \int_{\widehat{G}} \widehat{f}(\chi) d \mu_{\widehat{G}}(\chi)  \\
&= \ \int_{\widehat{G}} \widehat{f}(\chi) \chi(0)d \mu_{\widehat{G}}(\chi)  \\	
&= \ f(0)\\
&=	\ 1.
\end{align*}


\begin{x}{\small\bf EXAMPLE} \ %49
Take $G = \Z / n\Z$ and let $\mu_G$ be the counting measure $($thus here $\mu_G(G) = n)$ $-$then $\mu_{\widehat{G}}$ is the counting measure divided by $n$ and for $f \in \mathbf{INV}(G)$, 
\[
f(k) = \frac{1}{n} \sum_{m = 0} ^{n - 1} \widehat{f}(m) \exp(-2\pi\sqrt{-1} \ \frac{km}{n}),
\]
where
\[
\widehat{f}(m) = \sum_{k = 0} ^{n - 1} f(k) \exp(2\pi\sqrt{-1} \ \frac{km}{n}).
\]
\end{x}

\begin{x}{\small\bf EXAMPLE} \ %50
Take $G = \R$ and let $\mu_G = \alpha dx$ $(\alpha > 0)$, hence $\mu_{\widehat{G}} = \beta dt $ $(\beta > 0)$ 
%%----------------------------------------------------------------------------------------------11
and we claim that
\[
1 = 2\alpha \beta \pi.
\]
To establish this, recall first that the formalism is
\[
\begin{cases}
\widehat{f}(t) 	&= \ds\int_{-\infty}^\infty f(x) e^{\sqrt{-1}\  tx} \alpha dx \\	
\\
f(x) 	&= \ds\int_{-\infty}^\infty \widehat{f}(t) e^{-\sqrt{-1} \ tx} \beta dx 
\end{cases}
.
\]
Let $f(x) = e^{-\abs{x}}$ $-$then
\[
\frac{2\alpha}{1 + t^2} = \int_{-\infty}^\infty e^{-\abs{x}} e^{\sqrt{-1}\  tx} \alpha dx,
\]
so $f \in \mathbf{INV}(G)$, thus
\[
\begin{aligned}
e^{-\abs{x}} \ 
&=\  \int_{-\infty}^\infty \frac{2\alpha}{1 + t^2} e^{-\sqrt{-1} \ tx} \beta dt \\
&= \  2\alpha \beta \int_{-\infty}^\infty \frac{e^{-\sqrt{-1}\  tx }}{1 + t^2}dt.
\end{aligned}
\]
Now put $x = 0:$
\[
1 = 2\alpha \beta \int_{-\infty}^\infty \frac{dt}{1 + t^2}dt = 2\alpha \beta  \pi,
\]
as claimed.  One choice is to take
\[
\alpha = \beta = \frac{1}{\sqrt{2\pi}},
\]
the upshot being that the Haar measure of $[0,1]$ is not 1 but rather $\ds\frac{1}{\sqrt{2\pi}}$.
\end{x}

\vspace{0.1cm}

\begin{x}{\small\bf NOTATION} \ %51
Given $f \in L^1(\R)$, let
\[
\sF_\R f(t) = \int_{-\infty}^\infty f(x) e^{2\pi \sqrt{-1}\   tx} dx.
\]
%%----------------------------------------------------------------------------------------------12
Therefore
\begin{align*}
\sF_\R f(t) \ 
&=\  \sqrt{2\pi}\   \frac{1}{\sqrt{2\pi}} \int_{-\infty}^\infty f(x) e^{2\pi \sqrt{-1}\   tx} dx \\
&=\   \sqrt{2\pi}\   \widehat{f}(2 \pi t).
\end{align*}
\end{x}

\vspace{0.1cm}

\begin{x}{\small\bf STANDARDIZATION} \ %52
$(G = \R)$ Let $f \in \mathbf{INV}(\R)$, $-$then
\[
\sF_\R \sF_\R f(x) = f(-x).
\]
%\vspace{0.1cm}

[In fact,
\begin{align*}
\sF_\R \sF_\R f(x)	 \ 
&=\  \int_{-\infty}^\infty \sF_\R f(t) e^{2 \pi \sqrt{-1}\  tx} dx \\	
&=\  \ \int_{-\infty}^\infty \sqrt{2\pi} \widehat{f}(2 \pi t) e^{2 \pi \sqrt{-1}\  tx} dx \\	
&=\  \ \sqrt{2\pi}  \int_{-\infty}^\infty  \widehat{f}(u) e^{\sqrt{-1}\  ux} \frac{du}{2\pi} \\	
&=\  \ \frac{1}{\sqrt{2\pi}}  \int_{-\infty}^\infty  \widehat{f}(t) e^{\sqrt{-1}\  tx} dt \\	
&=\  \ f(-x).]
\end{align*}

\vspace{0.1cm}

Fourier inversion in the plane takes the form
\[
\begin{cases}
\widehat{f}(t_1, t_2)  \
= \ \ds\frac{1}{2\pi} \int_{-\infty}^\infty \int_{-\infty}^\infty f(x_1, x_2) e^{\sqrt{-1}\  (t_1x_1 + t_2x_2)} dx_1 dx_2\\
\\
f(x_1, x_2)  \ 
= \ \ds\frac{1}{2\pi} \int_{-\infty}^\infty \int_{-\infty}^\infty \widehat{f}(t_1, t_2) e^{-\sqrt{-1}\  (t_1x_1 + t_2x_2)} dt_1 dt_2
\end{cases}
.
\]
One may then introduce
\begin{align*}
\sF_{\R^2}f(t_1,t_2) \ 
&= \ \int_{-\infty}^\infty \int_{-\infty}^\infty f(x_1, x_2) e^{2\pi\sqrt{-1}\  (t_1x_1 + t_2x_2)} dx_1 dx_2 \\
&= \ 2\pi \widehat{f} (2\pi t_1, 2\pi t_2)
\end{align*}
%%----------------------------------------------------------------------------------------------13
and proceeding as above we find that
\[
\sF_{\R^2} \sF_{\R^2} f(x_1, x_2) = f(-x_1, -x_2).
\]

Now identify $\R^2$ with $\C$ and recall that $\trs_{\C /\R} (z) = z + \bar{z}$.  
Write
\[
\begin{cases}
w = a + \sqrt{-1} \ b\\
z = x + \sqrt{-1}  \ y
\end{cases}
.
\]

Then
\[
wz + \ov{wz} = 2\Re(wz) = 2(ax -by). 
\]
Therefore
\[
\frac{1}{2\pi} \int_{-\infty}^\infty \int_{-\infty}^\infty f(x,y) e^{2 \sqrt{-1}\ (ax-by)}dx dy = \widehat{f}(2a, -2b).
\]

\vspace{0.1cm}

[Note: \  Let $\chi_w(z) = \exp(\sqrt{-1}(wz + \ov{wz}))$ 
$-$then $\chi_w$ is a unitary character of $\C$ and for any $\chi \in \widehat{\C}$, 
there is a unique $w \in \C$ such that $\chi = \chi_w$, hence $\widehat{\C} = \C$.]\\
\end{x}

\begin{x}{\small\bf NOTATION} \ %53
Given $f \in L^1(\R^2)$, let
\begin{align*}
\sF_\C f(w) \ 
&= \  \sF_\C f(a,b)\\
&= \  2 \sF_{\R^2} f(2a,-2b)\\
&= \  4 \pi \widehat{f} (4 \pi a, -4\pi b)\\
&= \  \int_{-\infty}^\infty  \int_{-\infty}^\infty \ f(x,y)e^{4 \pi \sqrt{-1}\  (ax - by)} 2dxdy\\
\end{align*}
\end{x}

%%----------------------------------------------------------------------------------------------14
\begin{x}{\small\bf STANDARDIZATION} \ %54
$(G = \C)$ Let $f \in \mathbf{INV}(\C)$, $-$then
\[
\sF_\C \sF_\C f(x,y) = f(-x, -y).
\]

[In fact,
\begin{align*}
\sF_\C \sF_\C f(x,y) \ 
&= \  \int_{-\infty}^\infty \int_{-\infty}^\infty \sF_\C f(a,b) e^{4 \pi \sqrt{-1} \ (ax - by)} \ 2dadb \\	
&= \  \int_{-\infty}^\infty \int_{-\infty}^\infty 4\pi \widehat{f}(4\pi a, -4\pi b) e^{4\pi\sqrt{-1}\  (ax - by)}\  2dadb \\	
&= \  \frac{4\pi}{(4\pi)^2} \int_{-\infty}^\infty \int_{-\infty}^\infty  \widehat{f}(u,-v) e^{\sqrt{-1}\  (ux - vy)}\  2dudv\\	
&= \  \frac{1}{2\pi}  \int_{-\infty}^\infty \int_{-\infty}^\infty  \widehat{f}(u,-v) e^{\sqrt{-1}\  (ux - vy)}\  dudv\\	
&= \  \frac{1}{2\pi}  \int_{-\infty}^\infty \int_{-\infty}^\infty  \widehat{f}(u,-v) e^{-\sqrt{-1}\  (-ux + vy)}\  dudv\\
&= \  \frac{1}{2\pi}  \int_{-\infty}^\infty \int_{-\infty}^\infty  \widehat{f}(u,v) e^{-\sqrt{-1}\  (-ux - vy)}\  dudv\\
&= \  f(-x, -y).]
\end{align*}
\end{x}

\vspace{0.1cm}


\index{Plancherel theorem}
\begin{x}{\small\bf PLANCHEREL THEOREM} \ %55
The Fourier transform restricted to $L^1(G) \cap L^2(G)$ is an isometry 
(with respect to $L^2$ norms$)$ onto a dense linear subspace of $L^2(\widehat{G})$, 
hence can be extended uniquely to an isometric isomorphism $L^2(G) \ra L^2(\widehat{G})$.\\
\end{x}

\vspace{0.1cm}

\index{Parseval theorem}
\begin{x}{\small\bf PARSEVAL FORMULA} \ %56
$\forall$ $f, g \in  L^2(G)$,
\[
\int_G f(x) \ov{g(x)} d_G(x) = \int_{\widehat{G}} \widehat{f}(\chi) \ov{\widehat{g}(\chi)} d_{\widehat{G}} (\chi).
\]
\end{x}

\vspace{0.1cm}


\begin{x}{\small\bf \un{N.B.}} \ %57
In both of these results, the Haar measure on $\widehat{G}$ is per Fourier inversion.
\end{x}
%%%%%%%%%%%%%%%%%%%%%%%%%%%%%%%%%%%%%%
%%%%%%%%%%%%%%%%%%%%%%%%%%%%%%%%%%%%%%
%%%%%%%%%%%%%%%%%%%%%%%%%%%%%%%%%%%%%%






















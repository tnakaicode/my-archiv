\chapter{
$\boldsymbol{\S}$\textbf{5}.\quad  LOCAL FIELDS}
\setlength\parindent{2em}
\setcounter{theoremn}{0}
%%----------------------------------------------------------------------------------------------01
%\ \indent 


\ \indent Let $\K$ be a field of characteristic 0 equipped with a non-archimedean absolute value $\acdot$.

\vspace{0.1cm}

\begin{x}{\small\bf NOTATION} \ 
Let
\index{$R$}
\index{$R^\times$}
\[
\begin{cases}
R = \{a \in \K: \abs{a} \le 1\}\\
R^\times = \{a \in \K: \abs{a} = 1\}
\end{cases}
.
\]
\end{x}

\begin{x}{\small\bf LEMMA} \ 
$R$ is a commutative ring with unit and $R^\times$ is its multiplicative group of invertible elements.
\end{x}

\vspace{0.1cm}

\begin{x}{\small\bf NOTATION} \ 
Let
\index{$P$}
\[
P = \{a \in \K: \abs{a} < 1\}.
\]
\end{x}

\vspace{0.1cm}

\begin{x}{\small\bf LEMMA} \ %4
$P$ is a maximal ideal.
\end{x}

\vspace{0.1cm}

Therefore the quotient $R/P$ is a field, the 
\un{residue field}
\index{residue field} of $\K$.

\vspace{0.2cm}

\begin{x}{\small\bf THEOREM} \ %5
$\K$ is locally compact iff the following conditions are satisfied.

\vspace{0.1cm}

\indent 1. \ $\K$ is a complete metric space.\\
\indent 2. \  $R/P$ is a finite field.\\
\indent 3. \  $\abs{\K^\times}$ is a nontrivial discrete subgroup of $\R_{>0}$.
\end{x}

\vspace{0.1cm}

\begin{x}{\small\bf DEFINITION} \ %6
A 
\un{local field}
\index{local field} is a locally compact field of characteristic 0.
\end{x}

\vspace{0.1cm}

\begin{x}{\small\bf EXAMPLE} \ %7
$\R$ and $\C$ are local fields.
\end{x}

\vspace{0.1cm}

\begin{x}{\small\bf EXAMPLE} \ %8
$\Q_{p}$ is a local field.
\end{x}

\vspace{0.1cm}
%%----------------------------------------------------------------------------------------------02

Assume that $\K$ is a non-archimedean local field.

\vspace{0.1cm}

\begin{x}{\small\bf LEMMA} \ %9
$R$ is compact.
\end{x}

\vspace{0.1cm}

\begin{x}{\small\bf LEMMA} \ %10
$P$ is principal, say $P = \pi$R, and 
\[
\abs{\K^\times}  = \abs{\pi}^{\Z}, \quad \text{where } 0 < \abs{\pi} < 1.
\]
\end{x}

\vspace{0.1cm}

[Note: Such a $\pi$ is said to be a \un{prime element}\index{prime element}.]

\vspace{0.1cm}

\begin{x}{\small\bf REMARK} \ %11
A nontrivial discrete subgroup $\Gamma$ of $\R_{>0}$ is free on one generator $ 0 < \gamma < 1:$
\[
\Gamma = \{\gamma^n:n \in \Z\}.
\]
This said, choose $\pi$ with the largest absolute value $<$ 1, thus $\pi \in P \subset  R \Rightarrow \pi R \subset P.$
In the other direction, 
\[
a \in P \Rightarrow \abs{a} \le \abs{\pi} \Rightarrow \frac{a}{\pi} \in R.
\]
And
\[
a = \pi \cdot \frac{a}{\pi}  \Rightarrow a \in \pi R.
\]
\end{x}

\vspace{0.1cm}

\begin{x}{\small\bf FACT} \ %12
A locally compact topological vector space over a local field is necessarily finite dimensional.
\end{x}

\vspace{0.1cm}

\begin{x}{\small\bf THEOREM} \ %13
$\K$ is a finite extension of $\Q_{p}$ for some $p$.

\vspace{0.1cm}

PROOF \ 
First, $\K\supset \Q$ (since char $\K = 0$).  
Second, the restriction of $\acdot$ to $\Q$ is equivalent to $\acdot_{p}$ $(\exists \ p$)
(cf. \S1, \#20), 
hence the closure of $\Q$ in $\K$ "is" $\Q_{p}$ $($since $\K$ is complete$)$. Third, $\K$ is finite dimensional over $\Q_{p}$ $($since $\K$ is locally compact).
\end{x}
%%----------------------------------------------------------------------------------------------03
\indent There is also a converse.
\begin{x}{\small\bf THEOREM} \ %14
Let $\K$ be a finite extension of $\Q_{p}$ $-$then $\K$ is a local field.

\vspace{0.1cm}

\indent PROOF \ 
In view of $\#5$, it suffices to equip $\K$ with a non-archimedean absolute value subject to the conditions 1, 2, 3. 
But, by the extension principle (cf. $\S$3, $\#$11), $\acdot_{p}$ extends uniquely to $\K$. 
This extension is non-archimedean and points 1, 3 are manifest. 
As for point 2, it suffices to observe that the canonical arrow
\[
\Z_{p}/p\Z_{p} \ra R/P
\]
is injective and
\[
[R/P:\F_p] \le [\K:\Q_{p}] < \infty.
\] 
\indent [Details: To begin with,
\[
\Q_{p} \cap P = p\Z_{p},
\]
thus the inclusion $\Z_{p} \ra$ R induces an injection
\[
\Z_{p}/p\Z_{p} \ra R/P.
\] 
Put now $n = [\K:\Q_{p}]$ and let $A_{1}, ..., A_{n+1} \in R$ 
$-$then the claim is that the residue classes $\ov{A}_{1}, ..., \ov{A}_{n+1}  \in$ 
$R/P$ are linearly dependent over $\Z_{p}/p\Z_{p}$.  
In any event, 
there are elements $x_{1}, ..., x_{n+1} \in \Q_{p}$ such that 
\[
\sum_{i=1}^{n+1} x_iA_i = 0, 
\]
matters being arranged in such a way that 
\[
\max\abs{x_i}_{p} = 1.
\]
Therefore the $ x_{i} \in \Z_{p}$ and not every residue class $\ov{x}_{i} \in \Z_{p}/p\Z_{p}$ is zero. But then
%%----------------------------------------------------------------------------------------------04
\[
\sum_{i=1}^{n+1} \ov{x}_i \ov{A}_i = 0
\]
is a nontrivial dependence relation.]\\
\end{x}

\vspace{0.1cm}

 
\begin{x}{\small\bf SCHOLIUM} \ %15
A non-archimedean field of characteristic zero is a local field iff it is a finite extension of $\Q_{p}$ $(\exists$ $p)$.
\end{x}

\vspace{0.1cm}

Let $\K/\Q_{p}$ be a finite extension of degree $n$ $-$then the 
\un{canonical absolute value}
\index{canonical absolute value}  
on $\K$ is given by
\[
\abs{a}_{p} = \abs{N_{\K/\Q_{p}}(a)}_{p}^{1/n}.
\]

\vspace{0.1cm}

[Note: The 
\un{normalized absolute value}
\index{normalized absolute value}  
on $\K$ is given by
\[
\abs{a}_\K = \abs{a}_{p}^{n}.
\]
Its intrinsic significance will emerge in due course but for now observe that $\acdot_{\K}$ is equivalent to $\acdot_{p}$ and is non-archimedean (cf. $\S 1, \  \#23$).]


\vspace{0.1cm}

\begin{x}{\small\bf LEMMA} \ %16
The range of $\restr{\acdot_{p}}{\K^\times}$  is $\abs{\pi}_{p}^\Z.$
\end{x}

\vspace{0.1cm}

\begin{x}{\small\bf DEFINITION} \ %17
The 
\un{ramification index}
\index{ramification index} 
 of $\K$ over $\Q_{p}$ is the positive integer
\[
e = [\abs{\K^\times}_{p}:\abs{Q^\times_p}_{p}].
\]
I.e.,
\[
e = [\abs{\pi}_{p}^\Z:\abs{p}_{p}^\Z].
\]
Therefore
\[
\abs{\pi}_{p}^{e} = \abs{p}_{p} \qquad (= \frac{1}{p}).
\]

%%----------------------------------------------------------------------------------------------05
[Consider $\Z$ and e$\Z$  $-$then the generator 1 of $\Z$  is related to the generator e of e$\Z$ by the triviality 
$1 + \cdots + 1 = e\cdot 1 = e$.]
\end{x}

\vspace{0.1cm}

\begin{x}{\small\bf \un{N.B.} } \ %18
If $\pi^\prime$ has the property that $\abs{\pi^\prime}_p^{e}$ = ${\abs{p}}_{p}$ then $\pi^\prime$ is a prime element.
\\
\indent [Using obvious notation, write $\pi^\prime$ = ${\pi^{v(\pi)}}u$, thus
\[
\begin{aligned}
\abs{p}_{p} \ 
&= \  \abs{\pi^\prime}_p^{e}\\ 
&= \  {(\abs{\pi}^{v(\pi)}_p)}^e \\
&= \ {(\abs{\pi}^{e}_p)}^{v(\pi)} \\
&= \ {\abs{p}_{p}}^{v(\pi)},
\end{aligned}
\]
thus v($\pi$) = 1.]\\
\end{x}

\begin{x}{\small\bf NOTATION} \ 
\[
q \equiv \card R/P = (\card \F_p)^f = p^f,
\]
so
\[
f = [R/P:\F_p],
\]
the 
\un{residual index}
\index{residual index}  
of $\K$ over $\Q_{p}$.\\
\end{x}

\vspace{0.1cm}

\begin{x}{\small\bf THEOREM} \ %20
Let $\K/\Q_{p}$ be a finite extension of degree $n$ $-$then
\[
n = [\K:\Q_{p}] = ef.
\]
\end{x}



\begin{x}{\small\bf APPLICATION} \ %21
\[
\begin{aligned}
{\abs{\pi}}_\K \ 
= \ \abs{\pi}_p^n \\
= \ \abs{p}_{p}^{n/e} \\
= \ \left(\frac{1}{p}\right)^{n/e} \\
= \ \left({\frac{1}{p}}\right)^f \\
= \ \frac{1}{p^f} \\
= \ \frac{1}{q}.
\end{aligned}
\]
\end{x}

\vspace{0.1cm}

%%----------------------------------------------------------------------------------------------06

View $p$ as an element of $\K$:

\vspace{0.1cm}

\qquad \textbullet \quad $\abs{p}_{p} =  \abs{N_{\K/\Q_{p}}(p)}_{p}^{1/n}= \abs{p^n}_{p}^{1/n} = \abs{p}_{p}$.

\qquad \textbullet \quad $\abs{p}_{\K} 
= \abs{N_{\K/\Q_{p}}(p)}_{p} 
= \abs{p^n}_{p} 
= \ds\frac{1}{p^n} 
= \ds\frac{1}{p^{ef}} 
= {\left(\ds\frac{1}{p^f}\right)}^e 
= q^{-e}.$

\vspace{0.3cm}

\begin{x}{\small\bf DEFINITION} \  %22
A finite extension $\K/\Q_{p}$ is

\vspace{0.1cm}

\qquad \textbullet\quad \un{unramified}\index{unramified} if $e = 1$

\qquad \textbullet\quad \un{ramified}\index{ramified} if $f = 1$.

\end{x}

\vspace{0.1cm}

Take the case $\K = \Q_{p}$ $-$then $e = 1$, hence $\K$ is unramified, and $f = 1$, hence $\K$ is ramified.

\vspace{0.1cm}

\begin{x}{\small\bf LEMMA} \ %23
If $\K/\Q_{p}$ is is unramified, then $p$ is a prime element.
\end{x}

\vspace{0.1cm}

\begin{x}{\small\bf THEOREM} \ %24
$\forall$ $n = 1, 2, \ldots$,  there is up to isomorphism one unramified extension $\K/\Q_{p}$ of degree $n$.
\end{x}

\vspace{0.1cm}

Let $\K/\Q_{p}$ be a finite extension.

\vspace{0.1cm}

\begin{x}{\small\bf LEMMA} \ %25
The group $M^\times$ of roots of unity of order prime to $p$ in $\K$ is cyclic of order 
\[
p^f - 1 \quad \text{$(= q-1)$}.
\]
\end{x}

\vspace{0.1cm}

\begin{x}{\small\bf LEMMA} \ %26
The set M = $M^\times$ $\cup$ \{0\} is  a set of coset representatives for $R/P$. Therefore (cf. $\S4$, $\#$43)
\[
\K^\times \approx \Z \times R^\times \approx \Z \times \Z/(q-1)\Z \times1 + P.
\]
\end{x}
%%----------------------------------------------------------------------------------------------07

\vspace{0.1cm}

\begin{x}{\small\bf NOTATION} \ %27
Let
\index{$\K_{ur}$}
\[
\K_{ur} = \Q_p(M^\times).
\]
\end{x}


\begin{x}{\small\bf LEMMA} \ %28
$\K_{ur}$ is the maximal unramified extension of $\Q_p$ in $\K$ and
\[
[\K_{ur}:\Q_p] = f.
\]
\end{x}

\begin{x}{\small\bf REMARK} \ %29
The maximal unramified extension 
$(\Q_p^{c\ell})_{ur} \subset \Q_p^{c\ell}$ 
is the field extension generated by all roots of unity of order prime to $p$.
\end{x}

\vspace{0.1cm}

\begin{x}{\small\bf QUADRATIC EXTENSIONS} \ \ %30
\index{Quadratic extensions}
(cf. $\S 4$, $\#56$)  \ Suppose that $p \ne$ 2, 
let $\tau \in \Q_p^\times - (\Q_p^\times)^2$, and form the quadratic extension
\[
\Q_p(\tau) = \{x + y\sqrt{\tau} : x, y \in \Q_p\}.
\]
Then the canonical absolute value on $\Q_p(\sqrt{\tau})$ is given by
\[
\begin{aligned}
\abs{x + y\sqrt{\tau}}_p \ 
&= \ \abs{N_{\Q_p(\sqrt{\tau})/\Q_p}   (x + y\sqrt{\tau}}_p^{1/2} \\
&= \ \abs{x^2 - \tau y^2}_p^{1/2}.
\end{aligned}
\]
\end{x}

\vspace{0.1cm}

\begin{x}{\small\bf CLASSIFICATION} \ %31
Consider the three possibilities
\[
\Q_p(\sqrt{p}), \ \Q_p(\sqrt{\tau}), \ \Q_p(\sqrt{p \tau}),
\]
thus here $ef = 2$.
\begin{itemize}
\item $\Q_p(\sqrt{p})$ is ramified or still, $e = 2$.

\vspace{0.1cm}

[Note that
\[
\abs{\sqrt{p}}_p^2 = \abs{0^2 - (p)1^2}_p = \abs{p}_p = \frac{1}{p}.]
\]
%%----------------------------------------------------------------------------------------------08
\item $\Q_p(\sqrt{p \zeta})$ is ramified or still, $e = 2$.\\

\vspace{0.1cm}

[Note that
\[
\abs{\sqrt{p \zeta}}^2 
= \abs{0^2 - (p \zeta)1^2}_p 
= \abs{p \zeta}_p 
= \abs{p}_p \cdot \abs{\zeta}_p 
= \abs{p}_p = \frac{1}{p}.]
\]

If $e = 1$, then in either case, the value group would be $p^{\Z}$, an impossibility since 
$\ds\frac{1}{\sqrt{p}} \notin p^{\Z}$, so $e = 2$.
\item $\Q_p(\sqrt{\zeta})$ is unramified or still, $e = 1$.
\end{itemize}

\vspace{0.1cm}

[There is up to isomorphism one unramified extension $\K$ of $\Q_p$ of degree 2 (cf. $\# 24$)].

\vspace{0.1cm}

[Instead of quoting theory, one can also proceed directly, it being simplest to work instead with $\Q_p(\sqrt{a})$, 
where $1 < a < p$ is an integer that is not a square mod $p$ (cf. $\S 4$, $\#57$) 
$-$then the residue field of $\Q_p(\sqrt{a})$ is $\F_p(\sqrt{a})$, hence $f = 2$, hence $e = 1$ (since $n = 2$).]
\end{x}

\vspace{0.1cm}

The preceding developments are absolute, i.e., based at $\Q_p$.  
It is also possible to relativize the theory.  
Thus let $\LL/\K$, $\K/\Q_p$ be finite extensions.  
Append subscripts to the various quantities involved:
\[
\begin{cases}
R_\K \supset P_{\K}, \ R_\K/P_{\K}, \ e_{\K}, \ f_{\K}, \ M_\K^\times \\
R_\LL \supset P_{\LL}, \ R_\LL/P_{\LL}, \ e_{\LL}, \ f_{\LL}, \ M_\LL^\times
\end{cases}
.
\]
Introduce
\[
\begin{cases}
e(\LL/\K) = [\abs{\LL^\times}:\abs{\K^\times}] \\
f(\LL/\K) = [R_\LL/P_\LL:R_\K/P_\K]
\end{cases}
.
\]
\vspace{0.2cm}
%%----------------------------------------------------------------------------------------------09

\begin{x}{\small\bf LEMMA} \ %32
\[
[\LL:\K] = e(\LL/\K)f(\LL/\K).
\]

\indent PROOF \   We have
\[
\begin{cases}
[\LL:\Q_p ] = e_\LL f_\LL \\
[\K:\Q_p] = e_\K f_\K
\end{cases}
\qquad (\text{ cf. $\# 20$}).
\]
Therefore
\[
[\LL:\K] = \frac{[\LL:\Q_p] }{[\K:\Q_p]} = \frac{e_\LL f_\LL}{e_\K f_\K} = e(\LL/\K) f(\LL/\K).
\]
\end{x}

\vspace{0.1cm}

\begin{x}{\small\bf THEOREM} \ %33
Let $\LL/\K$, $\K/\Q_p$ be finite extensions $-$then there exists a unique maximal intermediate extension 
$\K \subset \K_{ur} \subset \LL$ that is unramified over $\K$.

\vspace{0.1cm}

[In fact, 
\[
\K_{ur} = \K(M_\LL^\times) \subset \LL.]
\]

[Note: The extension $\LL/\K_{ur}$ is ramified.]
\end{x}

%%%%%%%%%%%%%%%%%%%%%%%%%%%%%%%%%%%%%%
%%%%%%%%%%%%%%%%%%%%%%%%%%%%%%%%%%%%%%
%%%%%%%%%%%%%%%%%%%%%%%%%%%%%%%%%%%%%%






















\chapter{
$\boldsymbol{\S}$\textbf{15}.\quad  GLOBAL ANALYSIS}
\setlength\parindent{2em}
\setcounter{theoremn}{0}
%%----------------------------------------------------------------------------------------------01

\ \indent 
By definition,
\[
\A \ =\  \A_{\fin} \times \R.
\]
Therefore
\[
\widehat{\A} \ \thickapprox \  \widehat{\A}_{\fin} \times \widehat{\R}.
\]
And
\[
\A_{\fin} \ =\  \prod_p  \  (\Q_p: \Z_p)
\]
\qquad\qquad $\implies$
\[
\widehat{\A}_{\fin} \ \thickapprox \   \prod_p \  (\widehat{\Q}_p: \Z_p^\perp)	\qquad (\text{cf.} \  \S13, \  \#15).
\]
Put
\[
\chi_\Q \ =\  \prod_{p \le \infty} \chi_{p},
\]
where
\[
\chi_\infty \ =\  \exp( -2\pi \sqrt{-1}  \ x)  \qquad (x \in \R)	\qquad (\text{cf.}  \ \S8, \  \#27).
\]
Then
\[
\chi_\Q \in \widehat{\A}.
\]
Given $t \in \A$, define $\chi_{\Q, t} \in \widehat{\A}$ by the rule
\[
\chi_{\Q, t} (x) = \chi_\Q (tx).
\]
Then the arrow
\[
\Xi_\Q : \A \ra \widehat{\A}
\]
that sends $t$ to $\chi_{\Q, t}$ is an isomorphism of topological groups (cf.  \S8, \  \#24).
%%----------------------------------------------------------------------------------------------02


Recall now that $\forall$ $q \in \Q$,
\[
\chi_\Q(q) = 1	\qquad (\text{cf.} \  \S8, \  \#28).
\]
Accordingly, $\chi_\Q$ passes to the quotient and defines a unitary character of the adele class group $\A/\Q$.  
So, $\forall$ $q \in \Q$, $\chi_{\Q, q}$ is constant on the cosets of $\A/\Q$, 
thus it too determines an element of $\widehat{\A/\Q}$.

Equip $\Q$ with the discrete topology.
\vspace{0.1cm}

\begin{x}{\small\bf THEOREM} \ %01
The induced map
\begin{align*}
%\Xi_\Q|_\Q: \Q  	
\restr{\Xi_\Q}{\Q}: \Q  	
&\ra \widehat{\A/\Q}\\
q 		
&\mapsto \chi_{\Q, q}
\end{align*}
is an isomorphism of topological groups.

\vspace{0.1cm}

PROOF \  
Form $\Q^\perp \subset \widehat{\A}$, the closed subgroup of $\widehat{\A}$ consisting of those $\chi$ 
that are trivial on $\Q$ $-$then $\Q \subset \Q^\perp$ and $\widehat{\A/\Q} \thickapprox \Q^\perp.$  
But $\A/\Q$ is compact, thus its unitary dual $\widehat{\A/\Q}$ is discrete, thus $\Q^\perp$ is discrete.  
The quotient $\Q^\perp/\Q \subset \A/\Q$ $(\A \thickapprox \widehat{\A})$ is therefore discrete and closed, 
hence discrete and compact, hence finite.  
But $\Q^\perp/\Q$ is a $\Q$-vector space, so $\Q^\perp/\Q = \{0\}$ or still, $\Q^\perp = \Q$, 
which implies that $\Q \thickapprox \widehat{\A/\Q}$.
\end{x}

\vspace{0.1cm}


\begin{x}{\small\bf \un{N.B.}} \ %02
There are two points of detail that have been tacitly invoked in the foregoing derivation.


\qquad \textbullet \quad $\Q^\perp/\Q$ in the quotient topology is discrete.  
Reason$:$  Let $S$ be an arbitrary nonempty subset of $\Q^\perp/\Q$, say 
$S = \{x\Q: x \in U\}$, \mU a subset of $\Q^\perp$ $-$then $U$ is automatically open $(\Q^\perp$ being discrete$)$, 
thus by the very definition of the quotient 
%%----------------------------------------------------------------------------------------------03
topology, $S$ is an open subset of $\Q^\perp/\Q$.

\qquad \textbullet \quad The quotient $\Q^\perp/\Q$ is closed in $\A/\Q$.  
Reason: $\Q^\perp$ is a closed subgroup of $\A$ containing $\Q$, so the following generality is applicable:  
If $G$ is a topological group, if $H$ is a subgroup of $G$, if $F$ is a closed subgroup of $G$ containing $H$, 
then $\pi(F)$ is closed in $G/H$ $(\pi : G \ra G/H$ the projection$)$.
\end{x}

\vspace{0.1cm}


\begin{x}{\small\bf SCHOLIUM} \ %03
\[
\Q \thickapprox \widehat{\A/\Q} \implies \widehat{\Q} \thickapprox \widehat{\widehat{\A/\Q}} \thickapprox \A/\Q.
\]

\vspace{0.1cm}

[Note: \  Bear in mind that $\Q$ carries the discrete topology.]
\end{x}

\vspace{0.1cm}

\begin{x}{\small\bf DISCUSSION} \ %04
Explicated, if $\chi \in \widehat{\Q}$, then there exists a $t \in \A$ such that $\chi = \chi_{\Q, t}$ and $\chi_{\Q, t_1} = \chi_{\Q, t_2}$ iff $ t_1 - t_2 \in \Q$.
\end{x}

\vspace{0.1cm}

\begin{x}{\small\bf DEFINITION} \ %05
The 
\underline{Bruhat space}
\index{Bruhat space} 
$\sB(\A_{\fin})$
\index{$\sB(\A_{\fin})$} 
consists of all finite linear combinations of functions of the form
\[
f = \prod_p f_{p},
\]
where $\forall$ $p$, $f_p \in \sB(\Q_p)$ and $f_p = \chi_{\Z_p}$ for all but a finite number of $p$.
\end{x}

\vspace{0.1cm}

\begin{x}{\small\bf DEFINITION} \ %06
The 
\underline{Bruhat-Schwartz space}
\index{Bruhat-Schwartz space} 
$\sB_\infty(\A)$
\index{$\sB_\infty(\A)$}  
consists of all finite linear combinations of functions of the form
\[
f = \prod_p f_{p} \times f_\infty,
\]
where 
\[
\prod_p f_p = \sB(\A_{\fin}) \text{ and } f_\infty \in \sS(\R).
\]
\end{x}

\vspace{0.1cm}
%%----------------------------------------------------------------------------------------------04

Given an $f \in \sB_\infty(\A)$, its Fourier transform is the function:
\begin{align*}
\widehat{f}:\A 	
&\ra \C\\	
t 		
&\mapsto \int_\A f(x) \chi_{\Q,t}(x) d\mu_\A(x) = \int_\A f(x) \chi_\Q(tx) d\mu_\A(x).	
\end{align*}

\vspace{0.1cm}

\begin{x}{\small\bf LEMMA} \ %07
If
\[
f = \prod_p f_p \times f_\infty
\]
is a Bruhat-Schwartz function, then
\[
\widehat{f} = \prod_p \widehat{f}_p \times \widehat{f}_\infty.
\]
\end{x}
\vspace{0.1cm}

\begin{x}{\small\bf REMARK} \ %08
$\widehat{f}_p$ is computed per \S10, \  \#11 but $\widehat{f}_\infty$ is computed per
\[
\chi_\infty(x) = \exp(-2\pi\sqrt{-1} \ x),
\]
meaning that the sign convention here is the opposite of that laid down in \S10 (a harmless deviation).
\end{x}

\vspace{0.1cm}


\begin{x}{\small\bf APPLICATION} \ %09
\[
f \in \sB_\infty(\A) \implies \widehat{f} \in \sB_\infty(\A) \qquad (\text{cf.} \  \S10, \  \#16).
\]
\end{x}

\vspace{0.1cm}


\begin{x}{\small\bf \un{N.B.}} \ %10
It is clear that
\[
\sB_\infty(\A) \subset \bINV(\A)
\]
and $\forall$ $f \in \sB_\infty(\A)$,
\[
\widehat{\widehat{f}\hspace{0.1cm}} = f(-x) \quad \text{$( x \in \A)$}.
\]
\end{x}

\vspace{0.11cm}
%%----------------------------------------------------------------------------------------------05


\begin{x}{\small\bf LEMMA} \ %11
Given $f \in \sB_\infty(\A)$, the series
\[
\sum_{r \in \Q} f(x+r),  \qquad \sum_{q \in \Q} \widehat{f}(x+q)
\]
are absolutely and uniformly convergent on compact subsets of $\A$.
\end{x}

\vspace{0.1cm}

\begin{x}{\small\bf POISSON SUMMATION FORMULA} \ %12
Given $f \in \sB_\infty(\A)$,
\[
\sum_{r \in \Q} f(r) \ = \  \sum_{q \in \Q} \widehat{f}(q).
\]

The proof is not difficult but there are some measure theoretic issue to be dealt with first.

On general grounds, 
\[
\int_\A \ =\  \int_{\A/\Q} \ \sum_\Q	\qquad (\text{cf.} \ \S6,  \ \#11).
\]
Here the integral $\ds\int_\A$ is with respect to the Haar measure $\mu_\A$ on $\A$ (cf. \S14, \  \#31).  
Taking $\mu_\Q$ to be counting measure, this choice of data fixes the Haar measure $\mu_{\A/\Q}$ on $\A/\Q$.

\vspace{0.1cm}

[Note: \  The restriction of $\mu_\A$ to the fundamental domain
\[
D = \prod_p \Z_p \times [0,1[
\]
for $\A/\Q$ ( cf. \S14, \  \#10 ) determines $\mu_{\A/\Q}$ and
\[
1 \ =\  \mu_\A(D) \ =\  \mu_{\A/\Q} (\A/\Q).]
\]
If $\phi:\Q \ra \C$, then $\widehat{\phi}:\widehat{\Q} \ra \C$, i.e. $\widehat{\phi}:
\A/\Q \ra \C$ or still,
\[
\widehat{\phi}(\chi) \ = \ \sum_{r \in \Q} \phi(r) \chi(r).
\]
%%----------------------------------------------------------------------------------------------06


Specialize and suppose that $\phi$ is the characteristic function of $\{0\}$, so $\forall$ $\chi$,
\[
\widehat{\phi}(\chi) \ =\  \chi(0) \ =\  1.
\]
Therefore $\widehat{\phi}$ is the constant function 1 on $\A/\Q$.  
Pass now to $\widehat{\widehat{\phi}\hspace{0.059cm}}$, thus 
$\widehat{\widehat{\phi}\hspace{0.05cm}}: \widehat{\A/\Q} \ra \C$ or still,
\begin{align*}
\widehat{\widehat{\phi}\hspace{0.05cm}}:(\chi_{\Q,q}) \ 
&=\  \int_{\A/\Q} \widehat{\phi}(x) \chi_{\Q,q} (x) d\mu_{\A/\Q}(x) \\
&=\  \int_{\A/\Q} \chi_{\Q,q} (x) d\mu_{\A/\Q}(x)
\end{align*}
which is 1 if $q = 0$ and is 0 otherwise (cf. \S7, \  \#46 $(\A/\Q$ is compact)), 
hence $\widehat{\widehat{\phi}\hspace{0.05cm}} = \phi$.  
But $\phi(r) = \phi(-r)$, thereby leading to the conclusion that the Haar measure 
$\mu_{\A/\Q}$ on $\A/\Q$ is the one singled out by Fourier inversion ( cf. \S7, \  \#45).

\vspace{0.2cm}

Summary: Per Fourier inversion,

\qquad \textbullet \quad $\mu_\Q$ is paired with $\mu_{\A/\Q}$. 

\qquad \textbullet \quad $\mu_{\A/\Q}$ is paired with $\mu_\Q$.
\vspace{0.2cm}

Given $f \in \sB_\infty(\A)$, put
\[
F(x) = \sum_{r \in \Q} f(x+r).
\]
Then $F$ lives on $\A/\Q$, so $\widehat{F}$ lives on $\widehat{\A/\Q} \thickapprox \Q:$
\begin{align*}
\widehat{F}(q) \ 
&=\  \int_{\A/\Q} F(x) \chi_{\Q, q}(x) d\mu_{\A/\Q}(x) \\
&=\  \int_{\A/\Q} F(x) \chi_\Q(qx) d\mu_{\A/\Q}(x).
\end{align*}
On the other hand,
%%----------------------------------------------------------------------------------------------07
\begin{align*}
\widehat{f}(q)  	
&= \int_\A f(x) \chi_{\Q, q}(x) d\mu_\A(x) \\
&= \int_\A f(x) \chi_\Q (qx) d\mu_\A(x) \\
&= \int_{\A/\Q} \bigl(\sum_{r \in \Q} f(x+r)  \chi_\Q (q(x+r))\bigr) d\mu_{\A/\Q} (x) \\
&= \int_{\A/\Q} \bigl(\sum_{r \in \Q} f(x+r)  \chi_\Q (qx+qr)\bigr) d\mu_{\A/\Q} (x) \\
&= \int_{\A/\Q} \bigl(\sum_{r \in \Q} f(x+r)  \chi_\Q (qx) \chi_\Q (qr)  \bigr) d\mu_{\A/\Q} (x) \\
&= \int_{\A/\Q} \bigl(\sum_{r \in \Q} f(x+r)\bigr)  \chi_\Q (qx)   d\mu_{\A/\Q} (x) \\
&= \int_{\A/\Q} F(x)  \chi_\Q (qx)   d\mu_{\A/\Q} (x) \\
&= \widehat{F}(q).
\end{align*}

To finish the proof, per Fourier inversion, write
\[
F(x) \ =\  \sum_{q \in \Q} \widehat{F}(q) \overline{\chi_\Q(qx)}
\]
and then put $x = 0$:
\[
F(0) \ =\  \sum_{r \in \Q} f(r) \ =\  \sum_{q \in \Q} \widehat{F}(q) \ =\  \sum_{q \in \Q} \widehat{f}(q).
\]
\end{x}

\vspace{0.1cm}


\begin{x}{\small\bf THEOREM} \ %13
Let $x \in \I$ $-$then $\forall$ $f \in \sB_\infty(\A)$,
\[
\sum_{r \in \Q} f(rx) \ =\  \frac{1}{\abs{x}_\A} \  \sum_{q \in \Q} \  \widehat{f}(qx^{-1}).
\]

\vspace{0.1cm}

PROOF \  
Work with $f_x \in \sB_\infty(\A)$ $(f_x(y) = f(xy)):$
\[
\sum_{r \in \Q} f_x(r) \ =\  \sum_{q \in \Q} \widehat{f}_x(q).
\]
%%----------------------------------------------------------------------------------------------08
But
\begin{align*}
\widehat{f}_x(q)  \ 	
&=\  \int_\A f_x(y) \chi_{\Q, q}(y) d\mu_\A(y) \\
&=\  \int_\A f_x(y) \chi_{\Q}(qy) d\mu_\A(y) \\
&=\  \int_\A f(xy) \chi_{\Q}(qxx^{-1}y) d\mu_\A(y) \\
&=\  \frac{1}{\abs{x}_\A} \int_\A f(y) \chi_{\Q}(qx^{-1}y) d\mu_\A(y) \\
&=\  \frac{1}{\abs{x}_\A} \widehat{f}(qx^{-1}).
\end{align*}\\
\end{x}
%%%%%%%%%%%%%%%%%%%%%%%%%%%%%%%%%%%%%%
%%%%%%%%%%%%%%%%%%%%%%%%%%%%%%%%%%%%%%
%%%%%%%%%%%%%%%%%%%%%%%%%%%%%%%%%%%%%%






















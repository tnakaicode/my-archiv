\chapter{
$\boldsymbol{\S}$\textbf{9}.\quad  MULTIPLICATIVE p-ADIC CHARACTER THEORY}
\setlength\parindent{2em}
\setcounter{theoremn}{0}
%%----------------------------------------------------------------------------------------------01

\ \indent 
Recall that
\[
\Q_p^\times \ \thickapprox \  \Z \times \Z_p^\times,
\]
the abstract reflection of the fact that for ever $x \in \Q_p^\times$, there is a unique $v(x) \in \Z$ and a unique $u(x) \in \Z_p^\times$ such that $x = p^{v(x)}u(x)$.  
Therefore
\[
\widehat{(\Q_p^\times)} \ 
\thickapprox \ \widehat{\Z} \times \widehat{(\Z_p^\times)}  \ 
\thickapprox \ \T \times \widehat{(\Z_p^\times)}.
\]

\vspace{0.1cm}

\begin{x}{\small\bf \un{N.B.}} \ %1
A character of $\Q_p$ is necessarily unitary (cf. $ \S8, \  \# 4$) but this is definitely not the case for $\Q_p^\times$ $($cf. infra$)$.
\end{x}

\vspace{0.1cm}

\begin{x}{\small\bf DEFINITION} \ %2
A character $\chi: \Q_p^\times \ra \C^\times$ is \un{unramified} if it is trivial on $\Z_p^\times$.
\end{x}

\vspace{0.1cm}

\begin{x}{\small\bf EXAMPLE} \ %3
Given any complex number $s$, the arrow $x \ra \abs{x}_p^s$ is an unramified character of $\Q_p^\times$.
\end{x}

\vspace{0.1cm}

\begin{x}{\small\bf LEMMA} \ %4
If $\chi: \Q_p^\times \ra \C^\times$ is an unramified character, then there exists a complex number $s$ such that 
$\chi = \acdot_p^s$.

\vspace{0.1cm}

PROOF \ Such a $\chi$ factors through the projection $\Q_p^\times \ra p^\Z$ defined by $x \ra \abs{x}_p$, 
hence gives rise to a character $\widetilde{\chi}: p^\Z \ra \C^\times$ 
which is completely determined by its value on $p$, say $\widetilde{\chi}(p) = p^s$ for the complex number
\[
s = \frac{\log \widetilde{\chi}(p)}{\log p},
\]
%%----------------------------------------------------------------------------------------------02
itself determined up to an integral multiple of
\[
\frac{2\pi \sqrt{-1}}{\log p}.
\]
Therefore
\begin{align*}
\chi(x) \ 		
&=\ \widetilde{\chi}(\abs{x}_p)\\	
&=\ \widetilde{\chi}(p^{-v(x)})\\		
&=\  (\widetilde{\chi}(p))^{-v(x)}\\	
&=\  (p^s)^{-v(x)}\\
&=\  (p^{-v(x)})^s\\
&=\  \abs{x}_p^s.	
\end{align*}

[Note: \  For the record, 
\begin{align*}
\abs{x}_p^{2 \pi \sqrt{-1}/ \log p}	\ 	
&=\  (p^{-v(x)})^{2 \pi \sqrt{-1}/ \log p}\\	
&=\  (e^{-v(x) \log p})^{2 \pi \sqrt{-1} / \log p}\\		
&=\  e^{-v(x) 2 \pi \sqrt{-1}}\\	
&=\  1.]\\	
\end{align*}
\end{x}

\vspace{0.1cm}

Suppose that $\chi: \Q_p^\times \ra \C^\times$ is a character $-$then $\chi$ can be written as
\[
\chi(x) = \abs{x}_p^s \un{\chi}(u(x)),
\]
where $s \in \C$ and $\un{\chi} \equiv \restr{\chi}{\Z_p^\times} \in \widehat{(\Z_p^\times)}$, 
thus $\chi$ is unitary iff $s$ is pure imaginary.
\vspace{0.2cm}


\begin{x}{\small\bf LEMMA} \ %5
If $\un{\chi} \in \widehat{(\Z_p^\times)}$ is nontrivial, then there is an $n \in \N$ such that $\un{\chi} \equiv 1$ on $U_{p,n}$ but $\chi \not\equiv 1$ on $U_{p,n-1}$ (cf. $\S 8, \  \# 5$).
\end{x}

\vspace{0.1cm}
%%----------------------------------------------------------------------------------------------03

Assume again that $\chi: \Q_p^\times \ra \C^\times$ is a character.

\vspace{0.2cm}

\begin{x}{\small\bf DEFINITION} \ %6
$\chi$ is 
\un{ramified of degree $n \ge 1$}
\index{ramified of degree $n \ge 1$} 
if 
$\restr{\un{\chi}}{U_{p,n}} \equiv 1$ 
and 
$\restr{\un{\chi}}{U_{p,n-1}} \not\equiv 1$.
\end{x}

\vspace{0.1cm}

\begin{x}{\small\bf DEFINITION} \ %7
The 
\un{conductor}
\index{conductor} 
$\con\chi$ of $\chi$ is $\Z_p^\times$ if $\chi$ is unramified and $U_{p,n}$ if 
$\chi$ is ramified of degree $n$.
\end{x}

\vspace{0.1cm}

\begin{x}{\small\bf RAPPEL} \ %8
If $G$ is a finite abelian group, then the number of unitary characters of $G$ is card $G$.
\end{x}

\vspace{0.1cm}

\begin{x}{\small\bf LEMMA} \ %9
\[
[\Z_p^\times:U_{p,1}] = p-1  \qquad (\text{cf. } \S 4, \  \#40)
\]
and 
\[
[U_{p,1}:U_{p,n}] = p^{n-1} .
\]
\end{x}

\vspace{0.1cm}


If $\chi$ is ramified of degree $n$, then $\un{\chi}$ can be viewed as a unitary character of $\Z_p^\times / U_{p,n}$.   
But the quotient $\Z_p^\times / U_{p,n}$ is a finite abelian group, thus has 
\[
\card \ \Z_p^\times / U_{p,n} = [\Z_p^\times:U_{p,n}] 
\]
unitary characters.  
And
\begin{align*}
[\Z_p^\times:U_{p,n}] \ 
&=\  [\Z_p^\times:U_{p,1}] \cdot [U_{p,1}:U_{p,n}]  \\
&=\  (p-1) p^{n-1},
\end{align*}
this being the number of unitary characters of $\Z_p^\times$ of degree $\le n$.  
Therefore the 
%%----------------------------------------------------------------------------------------------04
group $\Z_p^\times$ has p-2 unitary characters of degree 1 and for $n \ge 2$, the group $\Z_p^\times$ has
\[
(p-1) p^{n-1} - (p-1) p^{n-2} = p^{n-2}(p-1)^2
\]
unitary characters of degree $n$.

\vspace{0.2cm}

\begin{x}{\small\bf LEMMA} \ %10
Let $\chi \in \widehat{\Q_p^\times}$ $-$then
\[
\chi(x) = \abs{x}_p^{\sqrt{-1} \  t} \un{\chi}(u(x)),
\]
where $t$ is real and 
\[
-(\pi / \log p) < t \le \pi / \log p.
\]
\end{x}

\newpage


%%%%%%%%%%%%%%%%%%%
%%%%%

\[
\textbf{APPENDIX}
\]
\setcounter{theoremn}{0}



Suppose that $p \ne 2$, let $\tau \in \Q_p^\times - (\Q_p^\times)^2$, and form the quadratic extension
\[
\Q_p(\tau) = \{x+y \sqrt{\tau} : x,y \in \Q_p\}.
\]

\begin{x}{\small\bf NOTATION} \ %1
Let $\Q_{p,\tau}$ be the set of points of the form $x^2 - \tau y^2$ $(x \ne 0$, $y \ne 0)$.
\end{x}

\vspace{0.1cm}

\begin{x}{\small\bf LEMMA} \ %2
$\Q_{p,\tau}$ is a subgroup of $\Q_p^\times$ containing $(\Q_p^\times)^2$.
\end{x}

\vspace{0.1cm}

\begin{x}{\small\bf LEMMA} \ %3
\[
[\Q_p^\times:\Q_{p,\tau}] = 2 \text{ and } [\Q_{p,\tau}:(\Q_p^\times)^2] = 2.
\]

[Note:
\[
[\Q_p^\times:(\Q_p^\times)^2] = 4		\qquad (\text{cf. } \S 4, \ \# 53).]
\]
\end{x}

\vspace{0.1cm}

\begin{x}{\small\bf DEFINITION} \ %4
Given $x \in \Q_p^\times$, let
\[
\sgn_\tau(x) \ = \ 
\begin{cases}
\  \ 1	\quad \text{if }x \in \Q_{p,\tau}\\
-1 	\quad \text{if } x \notin \Q_{p,\tau}
\end{cases}
.
\]
%%\sgn_\tau(x)=\left\{
%%\begin{array}{l l}
%%\  \ 1	&\quad \text{if $x \in \Q_{p,\tau}$}\\
%%-1 	&\quad \text{if $x \notin \Q_{p,\tau}$}\\
%%\end{array}
%\right.\]
\end{x}

\vspace{0.1cm}

\begin{x}{\small\bf LEMMA} \ %5
$\sgn_\tau$ is a unitary character of $\widehat{\Q}_p$.
\end{x}
%%%%%%%%%%%%%%%%%%%%%%%%%%%%%%%%%%%%%%
%%%%%%%%%%%%%%%%%%%%%%%%%%%%%%%%%%%%%%
%%%%%%%%%%%%%%%%%%%%%%%%%%%%%%%%%%%%%%






















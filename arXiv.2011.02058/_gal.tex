\chapter{
TOPICS IN GALOIS THEORY}
\setlength\parindent{2em}
\setcounter{theoremn}{0}
%%----------------------------------------------------------------------------------------------01

\[
\text{GALOIS CORRESPONDENCES}
\]

\[
\text{FINITE GALOIS THEORY}
\]

\[
\text{INFINITE GALOIS THEORY}
\]

\[
\text{$\K^\sep$ AND $\K^\ab$}
\]
\newpage
%%----------------------------------------------------------------------------------------------01

\ \indent 
\[
\text{GALOIS CORRESPONDENCES}
\]

Given a field $\F$, $\Aut(\F)$ stands for its associated group of field automorphisms.


\begin{x}{\small\bf EXAMPLE} \ %01
Take $\F = \Q$ $-$then $\Aut(\Q)$ is trivial.
\end{x}

\vspace{0.1cm}


\begin{x}{\small\bf EXAMPLE} \ %02
Take $\F = \R$ $-$then $\Aut(\R)$ is trivial.

\vspace{0.1cm}

[Let $\phi \in \Aut(\R)$ $-$then $\restr{\phi}{\Q} = \id_\Q$.  Next: 
\begin{align*}
x < y \implies \phi(y) - \phi(x) \ 
&=\ \phi(y - x) \\
&=\ \phi((\sqrt{y - x}\hsx)^2) \\
&=\ \phi(\sqrt{y - x}\hsx)^2 \\
&> \ 0.
\end{align*}
If now $\phi \neq \id_\R$, choose $x$ such that $\phi(x) \neq x$ $-$then there are two possibilities.

\vspace{0.1cm}

\qquad\qquad \textbullet \quad $x < \phi(x)$: \ Choose $q \in \Q$: $x < q < \phi(x)$, so 
$\phi(x) < \phi(q) = q < \phi(x)$.   Contradiction.

\vspace{0.1cm}

\qquad\qquad \textbullet \quad $\phi(x) < x$: \ Choose $q \in \Q$: $\phi(x) < q < x$, so
$\phi(x) < q= \phi(q) < \phi(x)$.  Contradiction.
\end{x}

\vspace{0.1cm}

\begin{x}{\small\bf EXAMPLE} \ %03
Take $\F = \C$ $-$then $\Aut(\C)$ is infinite.

\vspace{0.1cm}

[Any automorphism $\phi:\C \ra \C$ will fix $\Q$ and any continuous automorphism $\phi:\C \ra \C$ will fix its closure $\R$, 
there being two such, viz. the identity and the complex conjugation, all others being discontinuous.]

\vspace{0.1cm}

[Note: \ As an illustration, consider the automorphism
\[
a +  b \hsx \sqrt{2} \ra a - b \hsx \sqrt{2} \qquad (a, b \in \Q)
\]
of the field $\Q(\sqrt{2})$ $-$then it can be extended to an automorphism of $\C$ via the following procedure.

\vspace{0.1cm}

\qquad\qquad 1. \ Extend to $\K \equiv \Q(\sqrt{2})^\cl \subset \C$.

%%----------------------------------------------------------------------------------------------02

\vspace{0.1cm}

\qquad\qquad 2. \ Choose a transcendence basis \mS for $\C/\K$ and extend to $\K(S)$. 

\vspace{0.1cm}

\qquad\qquad 3. \ Extend from $\K(S)$ to $\C$.]

\end{x}

\vspace{0.1cm}

\begin{x}{\small\bf DEFINITION} \ %04
Let \mG be a group of automorphisms of $\F$ $-$then the subfield 
\[
\Inv(G) \hsx = \hsx \{x: \sigma x = x\} \qquad (\sigma \in G)
\]
is called the 
\un{invariant field}
\index{invariant field} 
associated with \mG.
\end{x}

\vspace{0.1cm}

\begin{x}{\small\bf DEFINITION} \ %05
Given a subfield $\E \subset \F$, the group consisting of all automorphisms of $\F$ leaving every element of $\E$ invariant is 
denoted by $\Gal(\F/\E)$, the 
\un{Galois group}
\index{Galois group} 
of $\F$ over $\E$.
\end{x}

\vspace{0.1cm}

\begin{x}{\small\bf EXAMPLE} \ %06
Take $\E = \R$, $\F = \C$ $-$then $\Gal(\C/\R) = \{\id_\C,\sigma\}$, where $\sigma$ is the complex conjugation.
\end{x}

\vspace{0.1cm}

\begin{x}{\small\bf EXAMPLE} \ %07
Take $\E = \Q$, $\F = \Q((2)^{1/3})$ $-$then $\Gal(\Q((2)^{1/3}) / \Q)$ is trivial.
\end{x}

\vspace{0.1cm}

\begin{x}{\small\bf EXAMPLE}  \ %08
Take $\E = \Q$, $\F = \Q(\omega_n)$ ($\omega_n$ a primitive $n^\nth$ root of unity in $\C$) $-$then 
\[
\Gal(\Q(\omega_n) / \Q) \approx (\Z/n\Z)^\times.
\]
%$\Gal(\Q(\omega_n) / \Q) \approx (\Z/n\Z)^\times$.
\end{x}

\vspace{0.1cm}

\begin{x}{\small\bf FACT} \ %09
We have
\[
G \subset \Gal(\F/\Inv(G)).
\]
\end{x}


\begin{x}{\small\bf FACT} \ %10
We have
\[
\E \subset \Inv(\Gal(\F/\E)).
\]
\end{x}

\vspace{0.1cm}

\begin{x}{\small\bf FACT} \ %11
\[
G \hsx \subset\hsx \Gal(\F/\E) \hsx\Leftrightarrow\hsx \E \hsx\subset\hsx \Inv(G).
\]
\end{x}

\vspace{0.1cm}

%%----------------------------------------------------------------------------------------------03

\begin{x}{\small\bf FACT} \ %12
\\

\qquad\qquad \textbullet \quad $G_1 \subset G_2 \subset \Aut(\F) \implies \Inv(G_1) \supset \Inv(G_2)$.

\vspace{0.1cm}

\qquad\qquad \textbullet \quad $\E_1 \subset \E_2 \subset \F \implies \Gal(\F/\E_2) \subset \Gal(\F/\E_1)$.

\end{x}

\vspace{0.1cm}

\begin{x}{\small\bf DEFINITION} \ %13
Let $\F$ be a field.

\qquad\qquad \textbullet \quad A 
\un{Galois group}
\index{Galois group}
on $\F$ is a group \mG of automorphisms of $\F$ such that 
\[
G \hsx = \hsx \Gal(\F/\Inv(G)).
\]

\vspace{0.1cm}

\qquad\qquad \textbullet \quad An 
\un{invariant field}
\index{invariant field}
in $\F$ is a subfield $\E$ of $\F$ such that 
\[
\E \hsx = \hsx \Inv(\Gal(\F/\E)).
\]

\vspace{0.1cm}
\end{x}

\vspace{0.1cm}



\begin{x}{\small\bf EXAMPLE} \ %14
$\Aut(\F)$ is a Galois group on $\F$.

\vspace{0.1cm}

[For 
\begin{align*}
\Aut(\F) \ 
&\subset \Gal(\F/\Inv(\Aut(\F)))\\
&=\ \Aut(\F).]
\end{align*}
\end{x}

\vspace{0.1cm}


\begin{x}{\small\bf EXAMPLE} \ %15
$\{\id_\F\}$ is a Galois group on $\F$

\vspace{0.1cm}

[For 
\begin{align*}
\{\id_\F\} \ 
&\subset \Gal(\F/\Inv(\{\id_\F\}))\\
&=\ \Gal(\F/\F) \\
&=\ \{\id_\F\}.]
\end{align*}
\end{x}

\vspace{0.1cm}


\begin{x}{\small\bf EXAMPLE} \ %16
$\F$ is an invariant field on $\F$.
\end{x}

\vspace{0.1cm}


\begin{x}{\small\bf REMARK} \ %17
Recall that a field is 
\un{prime}
\index{prime} 
if it possesses no proper subfields, these being the fields 
isomorphic to $\Q$ (characteristic 0) or 
isomorphic to $\Z/p\Z$ (characteristic p).  
A prime field admits no automorphism other than the identity.
\end{x}

\vspace{0.1cm}
%%----------------------------------------------------------------------------------------------04

\begin{x}{\small\bf ABSOLUTE GALOIS CORRESPONDENCE} \ %18
Let $\F$ be a field.

\qquad \textbullet \quad If $\E$ is a subfield of $\F$, then $\Gal(\F/\E)$ is a Galois group on $\F$.

\vspace{0.1cm}

\qquad \textbullet \quad If \mG is a group of automorphisms of $\F$, then $\Inv(G)$ is an invariant field in $\F$.

\vspace{0.1cm}

And:  \ The arrow $\E \ra \Gal(\F/\E)$ from the set of all invariant fields in $\F$ to the set of all Galois groups 
on $\F$ and the arrow $G \ra \Inv(G)$ from the set of all Galois groups on $\F$ to the set of all invariant fields in $\F$ 
are mutually inverse inclusion reversing bijections. 
\end{x}

\vspace{0.1cm}


\begin{x}{\small\bf RELATIVE GALOIS CORRESPONDENCE} \ %19
Let $\K$ be a field and let $\LL$ be a field extension of $\K$.


\qquad \textbullet \quad If $\K \subset \E \subset \LL$, then $\Gal(\LL/\E)$ is a Galois group on $\LL$ contained in 
$\Gal(\LL/\K)$.

\vspace{0.1cm}

\qquad \textbullet \quad If \mG is a subgroup of $\Gal(\LL/\K)$, then $\Inv(G)$ is an invariant field in $\LL$ containing 
$\K$.

\vspace{0.1cm}

And: \ The arrow $\E \ra \Gal(\LL/\E)$ from the set of all invariant fields in $\LL$ containing $\K$ to the set of all Galois 
groups on $\LL$ contained in $\Gal(\LL/\K)$ and the arrow $G \ra \Inv(G)$ from the set of all Galois groups on $\LL$ 
contained in $\Gal(\LL/\K)$ to the set of all invariant fields in $\LL$ containing $\K$ are mutually inverse inclusion reversing bijections. 


\end{x}

%%%%%%%%%%%%%%%%%%%%%%%%%%%%%%%%%%%%%%
%%%%%%%%%%%%%%%%%%%%%%%%%%%%%%%%%%%%%%
%%%%%%%%%%%%%%%%%%%%%%%%%%%%%%%%%%%%%%


\setcounter{theoremn}{0}

\newpage
%%----------------------------------------------------------------------------------------------01

\ \indent 

\[
\text{FINITE GALOIS THEORY}
\]

\begin{x}{\small\bf DEFINITION} \ %01
A field extension $\LL/\K$ is 
\un{Galois over $\K$}
\index{Galois over $\K$} 
(or is a 
\un{Galois extension of $\K$})
\index{Galois extension of $\K$} 
if $\LL$ is algebraic over $\K$ and $\K$ is an invariant field on $\L$ or still, 
\[
\K \hsx = \hsx \Inv(\Gal(\LL/\K)).
\]
\end{x}

\vspace{0.1cm}


\begin{x}{\small\bf FACT} \ %02
If $\LL/\K$ is a finite Galois extension and if $\LL \supset \E \supset \K$ is an intermediate field, then $\LL$ is Galois over $\E$.
\end{x}

\vspace{0.1cm}

\begin{x}{\small\bf FACT} \ %03
If $\LL/\K$ is a finite Galois extension and if $\LL \supset \E \supset \K$ is an intermediate field, then $\E$ is Galois over $\K$ 
iff $\Gal(\LL/\E)$ is a normal subgroup of $\Gal(\LL/\K)$. 

\vspace{0.1cm}

[Note: \ Under the assumption that $\E$ is Galois over $\K$, there is an arrow of restriction
\[
\Gal(\LL/\K) \ra \Gal(\E/\K).
\]
It is surjective with kernel $\Gal(\LL/\E)$, from which an exact sequence of groups:
\[
1 \ra \Gal(\LL/\E) \ra \Gal(\LL/\K)  \ra \Gal(\E/\K) \ra 1.]
\]
\end{x}

\vspace{0.1cm}



\begin{x}{\small\bf RECOGNITION  PRINCIPLE} \ %04
If $\LL/\K$ is a finite extension, then $\LL$ is Galois over $\K$ iff 
\[
\card \Gal(\LL/\K) \hsx = \hsx [\LL:\K].
\]

\vspace{0.1cm}

[Note: \ If $\LL/\K$ is a finite extension, then a priori
\[
\card \Gal(\LL/\K) \hsx \leq \hsx [\LL:\K],
\]
the inequality being strict in general.  Matters break down if it is a question of infinite extensions.  
E.g.: \ If $\Q^\cl$ is an algebraic closure of $\Q$, then 
\[
[\Q^\cl:\Q] \hsx = \hsx \aleph_0
\]
%%----------------------------------------------------------------------------------------------02
while
\[
\card \Gal(\Q^\cl / \Q) \hsx = \hsx 2^{\aleph_0}.]
\]
\end{x}

\vspace{0.1cm}

\begin{x}{\small\bf EXAMPLE} \ %05
Let $\F$ be a field of characteristic 0 and let $a \in \F^\times - (\F^\times)^2$.

Form the quadratic extension $\F(\sqrt{a})$ $-$then $[\F(\sqrt{a}):\F] = 2$, while 
$\Gal(\F(\sqrt{a}) / \F) = \{\id,\sigma\}$ $(\sigma(\sqrt{a}) = - \sqrt{a})$.  
Therefore $\F(\sqrt{a})$ is a Galois extension of $\F$.
\end{x}

\vspace{0.1cm}

\begin{x}{\small\bf EXAMPLE} \ %06
Take $\K = \Q$, $\LL = \Q((2)^{1/3})$ $-$then $[\Q((2)^{1/3}):\Q] = 3$ but 
$\Gal(\Q((2)^{1/3}) / \Q)$ is trivial.  Therefore $\Q((2)^{1/3})$ is not a Galois extension of $\Q$.
\end{x}

\vspace{0.1cm}

\begin{x}{\small\bf EXAMPLE} \ %07
Take $\K = \Q$, $\LL = \Q((2)^{1/3}, \omega)$, where 
\[
\omega \hsx = \hsx \exp(2 \pi \sqrt{-1} / 3).
\]
Then
\[
[\Q((2)^{1/3}, \omega):\Q] \hsx = \hsx  
[\Q((2)^{1/3}, \omega):\Q((2)^{1/3})]  \cdot [\Q((2)^{1/3}):\Q] \hsx = \hsx
2 \cdot 3 \hsx = \hsx 6.
\]
On the other hand, the six functions 
\begin{align*}
&\vsy(2)^{1/3} \ra (2)^{1/3}, \quad \omega \ra \omega\\
&\vsy(2)^{1/3} \ra \omega(2)^{1/3}, \quad \omega \ra \omega\\
&\vsy(2)^{1/3} \ra (2)^{1/3}, \quad \omega \ra \omega^2\\
&\vsy(2)^{1/3} \ra \omega(2)^{1/3}, \quad \omega \ra \omega^2\\
&\vsy(2)^{1/3} \ra \omega^2 (2)^{1/3}, \quad \omega \ra \omega\\
&\vsy(2)^{1/3} \ra \omega^2 (2)^{1/3}, \quad \omega \ra \omega^2
\end{align*}
extend to distinct automorphisms of $\Q((2)^{1/3}, \omega) /\Q$.
%%----------------------------------------------------------------------------------------------03
Therefore $\Q((2)^{1/3}, \omega)$ is a Galois extension of $\Q$.
\end{x}

\vspace{0.1cm}

\begin{x}{\small\bf FUNDAMENTAL THEOREM OF FINITE GALOIS THEORY}  \ %08
Suppose that $\LL$ is a finite Galois extension of $\K$.

\vspace{0.1cm}

\qquad \textbullet \quad If $\LL \supset \E \supset \K$, then
\[
[\Gal(\LL/\K):\Gal(\LL/\E)] \hsx = \hsx  [\E:\K].
\]
\vspace{0.1cm}

\qquad \textbullet \quad If $G \subset \Gal(\LL/\K)$, then 
\[
[\Inv(G):\K] \hsx = \hsx [\Gal(\LL/\K):G].
\]

\vspace{0.1cm}

And: \ The arow $\E \ra \Gal(\LL/\E)$ from the set of all intermediate fields between $\K$ and $\LL$ 
to the set of all subgroups of $\Gal(\LL/\K)$ and the arrow $G \ra \Inv(G)$ from the 
set of all subgroups of $\Gal(\LL/\K)$ to the set of all intermediate fields between $\K$ and $\LL$ 
are mutually inverse inclusion reversing bijections.
\end{x}

\vspace{0.1cm}

\begin{x}{\small\bf REMARK} \ %09
Given a finite Galois extension $\LL/\K$, the problem of determining all intermediate fields 
$\LL \supset \E \supset \K$ amounts to finding all subgroups of $\Gal(\LL/\K)$, a finite problem.

\vspace{0.1cm}

[Note: \ The fact that there are but finitely many intermediate fields cannot be established by a vector space argument alone.]

\end{x}

\vspace{0.1cm}

\begin{x}{\small\bf EXAMPLE} \ %10
The field $\Q((2)^{1/3}, \omega)$ is Galois over $\Q$ and its Galois group is a group of order 6, there being two 
possibilities, viz. the cyclic group $\Z/6\Z$ and the symmetric group $S_3$.  
Since $\Q((2)^{1/3})$ is not Galois over $\Q$, the group 
\[
\Gal(\Q((2)^{1/3}, \omega) / \Q((2)^{1/3}))
\]
is not a normal subgroup of $\Gal(\Q((2)^{1/3}, \omega) / \Q$).  %XXX
But every subgroup of an abelian group
%%----------------------------------------------------------------------------------------------04
is normal, so the conclusion is that 
\[
G \hsx \equiv \hsx \Gal(\Q((2)^{1/3}, \omega) / \Q) \hsx \approx \hsx S_3.  %XXX
\]
Proceeding, there are $\Q$-automorphisms $\sigma, \tau$ of  $\Q((2)^{1/3}, \omega)$ defined by the specification
\[
\begin{cases}
\ \sigma : (2)^{1/3} \ra \omega(2)^{1/3}, \quad \omega \ra \omega\\
\ \tau : (2)^{1/3} \ra (2)^{1/3}, \quad \omega \ra \omega^2
\end{cases}
.
\]
Then $\sigma$ has order 3, $\tau$ has order 2, and $\sigma \tau \neq \tau \sigma$.  
The subgroups of \mG are 
\[
\langle \id \rangle, \quad 
\langle \sigma \rangle, \quad
\langle \tau \rangle, \quad
\langle \sigma \tau \rangle, \quad
\langle \sigma^2 \tau \rangle, \quad
G
\]
and the corresponding intermediate fields are
\[
\Q((2)^{1/3}, \omega), \quad
\Q(\omega), \quad
\Q((2)^{1/3}), \quad
\Q(\omega^2 (2)^{1/3}), \quad
\Q(\omega(2)^{1/3}), \quad
\Q.
\]
\end{x}

\vspace{0.1cm}

\begin{x}{\small\bf FACT} \ %11
Let $\K$ be a finite Galois extension of $\F$ and let $\LL$ be an arbitrary finite extension of $\F$ $-$then 
$\K \vee \LL \supset \LL$ is a Galois extension and 
\[
\Gal(\K \vee \LL / \LL) \hsx \approx \hsx \Gal(\K / \K \cap \LL).
\]
In addition, 
\[
[\K \vee \LL:\LL] \hsx = \hsx [\K:\K \cap \LL].
\]

\vspace{0.1cm}

[Note: \ Tacitly, $\K$ and $\LL$ lie inside some common field $\M$, hence $\K \vee \LL$ is the subfield of $\M$ 
generated by $\K$ and $\LL$.  This said, the arrow 
\[
\Gal(\K \vee \LL /\LL) \ra \Gal(\K / \K \cap \LL)
\]
sends $\sigma$ to its restriction $\restr{\sigma}{\K}$.]
\end{x}

\vspace{0.1cm}



\begin{x}{\small\bf FACT} \ %12
Suppose that $\LL$ is a finite Galois extension of $\K$ $-$then

\vspace{0.3cm}

\qquad \textbullet \quad $\tN_{\LL / \K} (x) \hsx = \hsx \ds\prod\limits_{\sigma \in \Gal(\LL / \K)} \sigma x$

\vspace{0.1cm}
%%----------------------------------------------------------------------------------------------05

\qquad \textbullet \quad $\tT_{\LL / \K} (x) \hsx = \hsx \ds\sum\limits_{\sigma \in \Gal(\LL / \K)} \sigma x$.
\end{x}

\vspace{0.1cm}

\begin{x}{\small\bf NORMAL BASIS THEOREM} \ %13
If $\LL / \K$ is finite Galois, then $\exists \ x \in \LL$ such that $\{\sigma x: \sigma \in \Gal(\LL / \K)\}$ is a basis for $\LL / \K$.
\end{x}



%%%%%%%%%%%%%%%%%%%%%%%%%%%%%%%%%%%%%%
%%%%%%%%%%%%%%%%%%%%%%%%%%%%%%%%%%%%%%
%%%%%%%%%%%%%%%%%%%%%%%%%%%%%%%%%%%%%%


\setcounter{theoremn}{0}

\newpage
%%----------------------------------------------------------------------------------------------01

\ \indent 

\[
\text{INFINITE GALOIS THEORY}
\]

\begin{x}{\small\bf FACT} \ %01
If $\K$ is a field and if $\LL$ is an infinite Galois extension of $\K$, then 
\[
\card \Gal(\LL / \K) \hsx \geq \hsx 2^{\aleph_0}.
\]
\end{x}

\vspace{0.1cm}


\begin{x}{\small\bf APPLICATION} \ %02
The Galois group of an infinite Galois extension cannot be cyclic.
\end{x}

\vspace{0.1cm}

\begin{x}{\small\bf FACT} \ %03
If $\F$ is a field and if $G \subset \Aut(\F)$ is a finite group of automorphisms of $\F$, then \mG is a Galois group on $\F$: 
The a priori containment
\[
G \hsx\subset\hsx \Gal(\F / \Inv(G))
\]
is an equality: 
\[
G \hsx = \hsx \Gal(\F / \Inv(G)).
\]
\end{x}

\vspace{0.1cm}

\begin{x}{\small\bf REMARK} \ %04
In general, an infinite group of automorphisms of a field need not be a Galois group.
\end{x}

\vspace{0.1cm}

Given a field $\F$ and an element $a \in \F$, let $D_a$ denote the discrete topological space having $\F$ as its set of points 
$-$then the elements of the product
\[
\prod\limits_{a \in \F} D_a
\]
are just the maps $\F^\F$ from $\F$ to $\F$.

When equipped with the product topology, $\F^\F$  is Hausdorff and totally disconnected 
(but not discrete if $\card \F \geq \aleph_0$).  
Since $\Aut(\F)$ is contained in $\F^\F$, it can be endowed with the relativized product topology, the so-called 
\un{finite topology}.
\index{finite topology}
\vspace{0.2cm}

%%----------------------------------------------------------------------------------------------02

\begin{x}{\small\bf \un{N.B.}} \ %05
Given $\phi \in \Aut(\F)$ and a finite subset \mA of $\F$, let $\Omega_\phi(A)$ be the set of all automorphisms of $\F$ 
that agree with $\phi$ on \mA $-$then $\Omega_\phi(A)$ is open and the collection $\{\Omega_\phi(A)\}$ is a neighborhood basis at $\phi$.
\end{x}

\vspace{0.1cm}

\begin{x}{\small\bf FACT} \ %06
In the finite topology, $\Aut(\F)$ is a topological group (as well as being Hausdorff and totally disconnected).
\end{x}

\vspace{0.1cm}

In what follows, if $\Gamma \subset \Aut(\F)$ is a group of automorphisms of $\F$, it will be understood that $\Gamma$ 
carries the relativized finite topology.

\vspace{0.2cm}

\begin{x}{\small\bf FACT} \ %07
Suppose that $\Gamma \subset \Aut(\F)$ is compact $-$then $\Gamma$ is a Galois group on $\F$.
\end{x}

\vspace{0.1cm}

\begin{x}{\small\bf REMARK} \ %08
A group of automorphisms of $\F$ is compact iff it is closed in $\Aut(\F)$ and has finite orbits.
\end{x}

\vspace{0.1cm}

\begin{x}{\small\bf FACT} \ %09
If $\K$ is a field and if $\LL$ is an extension of $\K$, then 
\[
\Gal(\LL / \K) \subset \Aut(\LL)
\]
is closed.
\end{x}


\begin{x}{\small\bf FACT} \ %10
If $\K$ is a field and if $\LL$ is an algebraic extension of $\K$, then 
\[
\Gal(\LL / \K) \subset \Aut(\LL)
\]
is compact.

\vspace{0.1cm}

[Note: \ If $\LL$ is finite over $\K$ (hence algebraic), then $\Gal(\LL / \K)$ is discrete.]
\end{x}

\vspace{0.1cm}

\begin{x}{\small\bf REMARK} \ %11
The compactness of the Galois group does not characterize algebraic extensions (there exist transcendental extensions with a finite Galois group).

%%----------------------------------------------------------------------------------------------03

\vspace{0.1cm}

[Note: \ If $\K$ is an infinite field and if $\K(\xi)$ is a simple transcendental extension of $\K$, then $\Gal(\K(\xi) / \K)$ is not compact.]
\end{x}

\vspace{0.1cm}


\begin{x}{\small\bf FUNDAMENTAL THEOREM OF INFINITE GALOIS THEORY} \ %12

Suppose that $\LL$ is an infinite Galois extension of $\K$ (hence algebraic, hence $\Gal(\LL / \K)$ compact).

\vspace{0.1cm}

\qquad \textbullet \quad If $\LL \supset \E \supset \K$, then $\Gal(\LL / \E)$ is a closed subgroup of $\Gal(\LL/\K)$ 
(thus is a compact subgroup of $\Gal(\LL/\K)$).  

\vspace{0.1cm}

\qquad \textbullet \quad If \mG is a closed subgroup of $\Gal(\LL/\K)$ (thus is a compact subgroup of $\Gal(\LL/\K))$, then 
$\Inv(G)$ is an intermediate field between $\K$ and $\LL$.

\vspace{0.1cm}

And: \ The arrow $\E \ra \Gal(\LL/\E)$ from the set of all intermediate fields between $\K$ and $\LL$  to the set of all closed  
subgroups of $\Gal(\LL/\K)$ and the arrow $G \ra \Inv(G)$ from the set of all closed  
subgroups of $\Gal(\LL/\K)$ to the set of all intermediate fields between $\K$ and $\LL$
are mutually inverse inclusion reversing bijections. 
\end{x}

\vspace{0.1cm}

\begin{x}{\small\bf REMARK} \ %13
Since $\LL / \K$ is an infinite Galois extension, $\Gal(\LL / \K)$ always contains a subgroup that is not closed.

\vspace{0.1cm}

[Any infinite group has a countably infinite subgroup (consider the subgroup generated by a countably infinite subset).  
On the other hand, an infinite compact totally disconnected Hausdorff group has cardinality at least that of the continuum 
(it has a quotient which is homeomorphic to the Cantor set).]
\end{x}

\vspace{0.1cm}



\begin{x}{\small\bf FACT} \ %14
$\E/\K$ is finite iff $\Gal(\LL / \E)$ is open.
\end{x}

\vspace{0.1cm}


\begin{x}{\small\bf FACT} \ %15
$\E/\K$ is Galois iff $\Gal(\LL / \E)$ is normal.

\vspace{0.1cm}

[Note: \ Canonically, 
\[
\Gal(\E / \K) \hsx \approx \hsx \Gal(\LL / \K) / \Gal(\LL  / \E),
\]
this being a topological identification if $\Gal(\LL / \K)  / \Gal(\LL / \E)$ is given the quotient topology.]
\end{x}

\vspace{0.1cm}

%%----------------------------------------------------------------------------------------------04
\begin{x}{\small\bf \un{N.B.}} \ %16
$\LL$ is Galois over $\E$.
\end{x}

\vspace{0.1cm}


\begin{x}{\small\bf NOTATION} \ %17

\vspace{0.2cm}

\qquad \textbullet \quad
$\ds\bigvee\limits_{i \in I} \E_i$ is the subfield generated by the union $\ds\bigcup\limits_{i \in I} \E_i$.

\vspace{0.1cm}


\qquad \textbullet \quad
$\ds\bigvee\limits_{i \in I} G_i$ is the subgroup generated by the union $\ds\bigcup\limits_{i \in I} G_i$.

\vspace{0.1cm}

\end{x}

\vspace{0.1cm}


\begin{x}{\small\bf FACT} \ %18
Let $\LL$ be an infinite Galois extension of $\K$.

\vspace{0.2cm}

\qquad \textbullet \quad If $\E_i$ $(i \in I)$ is a nonempty family of intermediate fields between $\K$ and $\LL$, then 
\[
\Gal \biggl(\hsx \LL / \bigcap\limits_{i \in I} \E_i \biggr) \  = \  
\ov{\bigvee\limits_{i \in I} \Gal(\LL / \E_i)}.
\]

\vspace{0.1cm}

\qquad \textbullet \quad If $G_i$ $(i \in I)$ is a nonempty family of closed subgroups of $\Gal(\LL / \K)$, then 
\[
\Inv \biggl(\hsx\bigcap\limits_{i \in I} G_i \biggr) \  = \  
\bigvee\limits_{i \in I} \Inv(G_i).
\]

\vspace{0.1cm}


\end{x}

\vspace{0.1cm}

\begin{x}{\small\bf EXAMPLE} \ %19
Take $\K = \Q$, $\LL =\Q(\sqrt{2},\sqrt{3},\sqrt{5}, \ldots)$ (incorporate all primes) $-$then $\LL$ is Galois 
(and infinite) over $\K$ (being the union of 
$\Q$, 
$\Q(\sqrt{2})$, 
$\Q(\sqrt{2},\sqrt{3})$, 
$\Q(\sqrt{2},\sqrt{3},\sqrt{5})$ 
and so on).  Here $\Gal(\LL / \K)$ is a countably infinite direct product of copies of $\Z / 2\Z$.  
Accordingly, every $\K$-automorphism of $\LL$ differet from $\id_\LL$ is an element of order 2.
\end{x}

\vspace{0.1cm}

\begin{x}{\small\bf EXAMPLE} \ %20
Take $\K = \Q$, $\LL = A(\C / \Q)$ $-$then $\LL$ is Galois (and infinite) over $\K$.
\end{x}



%%%%%%%%%%%%%%%%%%%%%%%%%%%%%%%%%%%%%%
%%%%%%%%%%%%%%%%%%%%%%%%%%%%%%%%%%%%%%
%%%%%%%%%%%%%%%%%%%%%%%%%%%%%%%%%%%%%%


\setcounter{theoremn}{0}

\newpage
%%----------------------------------------------------------------------------------------------01

\ \indent 

\[
\text{$\K^\sep$ AND $\K^\ab$}
\]

Let $\K$ be a field, $\LL/\K$ a field extension.

\vspace{0.3cm}

\begin{x}{\small\bf DEFINITION} \ %01
An element of $\LL$ is 
\un{separable}
\index{separable (element of a field)} 
if it is algebraic over $\K$ and is a simple zero of its minimal polynomial.
\end{x}

\vspace{0.1cm}


\begin{x}{\small\bf NOTATION} \ %02
$S(\LL/\K)$ is the set of all elements of $\LL$ that are separable over $\K$.

\vspace{0.1cm}

[Note: \ Therefore
\[
S(\LL/\K) \subset A(\LL/\K)
\]
and 
\[
S(\LL/\K) \hsx = \hsx A(\LL/\K)
\]
if the characteristic of $\K$ is zero.]
\end{x}

\vspace{0.1cm}

\begin{x}{\small\bf DEFINITION} \ %03
$S(\LL/\K)$ is the 
\un{separable closure}
\index{separable closure} 
of $\K$ in $\LL$.
\end{x}

\vspace{0.1cm}

\begin{x}{\small\bf FACT} \ %04
$S(\LL/\K)$ is a field.
\end{x}

\vspace{0.1cm}

\begin{x}{\small\bf FACT} \ %05
If $\LL \supset \E \supset \K$ and $\E$ is a separable extension of $\K$, then $\E \subset S(\LL/\K)$.
\end{x}

\vspace{0.1cm}

\begin{x}{\small\bf NOTATION} \ %06
$\K^\cl$ is the algebraic closure of $\K$.
\end{x}

\vspace{0.1cm}

\begin{x}{\small\bf \un{N.B.}} \ %07
If $\K$ is not perfect, then $\K^\cl$ is not Galois over $\K$.
\end{x}

\vspace{0.1cm}

\begin{x}{\small\bf NOTATION}  \ %08
$\K^\sep$ is the separable closure of $\K$ in $\K^\cl$:
\[
\K^\sep \hsx = \hsx S(\K^\cl/\K).
\]
\end{x}

\vspace{0.1cm}

\begin{x}{\small\bf FACT} \ %09
$\K^\sep$ is the maximal separable extension of $\K$.
\end{x}
%%----------------------------------------------------------------------------------------------02

\begin{x}{\small\bf FACT} \ %10
$\K^\sep$ is a Galois extension of $\K$.
\end{x}

\vspace{0.1cm}

\begin{x}{\small\bf DEFINITION} \ %11

\[
\Gal(\K^\sep/\K)
\]
is the 
\un{absolute Galois group}
\index{absolute Galois group} 
of $\K$.
\end{x}

\vspace{0.1cm}

\begin{x}{\small\bf FACT} \ %12
If \hsx $\LL/\K$ is Galois, then $\Gal(\LL/\K)$ is a homomorphic image of $\Gal(\K^\sep/\K)$.

\vspace{0.1cm}

[This is because $\Gal(\LL/\K)$ can be identified with the quotient
\[
\Gal(\K^\sep/\K) / \Gal(\K^\sep/\LL).]
\]
\end{x}

\vspace{0.1cm}

\begin{x}{\small\bf EXAMPLE} \ %13
Take $\K = \F_p$ $-$then $\Gal(\F_p^\sep /\F_p)$ can be identified with \ 
$\lim\limits_{\substack{\lla\\ n \in \N}} \Z/n\Z$ 
(the set of all (equivalence classes) of sequences $\{a_n\} = \{a_1, a_2, \ldots \}$ 
of natural numbers such that 
\[
a_n \hsx \equiv \hsx a_m \hsx \text{(mod $m$)}
\]
whenever $m|n$).

\vspace{0.1cm}

[Bear in mind that $\forall \ n \in \N$, there is a Galois extension $\K_n/\F_p$ with $[\K_n:\F_p] = n$ and 
$\Gal(\K_n/\F_p) \approx \Z/n\Z$.]

\vspace{0.1cm}

[Note: \ Let $\phi:\F_p^\sep \ra \F_p^\sep$ be the Frobenius automorphism: $\phi(x) = x^p$.  
Let $G = \langle \phi \rangle$ $-$then 
\[
\Inv(G)\hsx = \hsx \F_p, \quad  \Inv(\Gal(\F_p^\sep /\F_p)) \hsx = \hsx \F_p,
\]
yet 
\[
G \hsx \neq \hsx \Gal(\F_p^\sep /\F_p).]
\]
\end{x}

\vspace{0.1cm}

%%----------------------------------------------------------------------------------------------03

\begin{x}{\small\bf NOTATION} \ %14
$\Gal^*(\K^\sep/\K)$ 
\index{$\Gal^*(\K^\sep/\K)$ } 
is the commutator subgroup of $\Gal(\K^\sep/\K)$ .
\end{x}

\vspace{0.1cm}


\begin{x}{\small\bf FACT} \ %15
\[
\Inv(\Gal^*(\K^\sep/\K)) \hsx = \hsx \Inv(\ov{\Gal^*(\K^\sep/\K})).
\]

\vspace{0.1cm}

[Put
\[
\Gamma \hsx = \hsx \Gal^*(\K^\sep/\K).
\]
Then
\[
\Gamma \hsx\subset\hsx \ov{\Gamma} \implies \Inv(\ov{\Gamma}) \subset \Inv(\Gamma).
\]
To go the other way, let $x \in \Inv(\Gamma)$, $\ov{\gamma} \in \ov{\Gamma}$ and claim: 
$\ov{\gamma} x = x$ (hence $x \in \Inv(\ov{\Gamma})$).  
If $\ov{\gamma} \in \Gamma$, we are through; otherwise, 
$\ov{\gamma}$ is an accumulation point of $\Gamma$, thus since 
$\Omega_{\ov{\gamma}} (\{x\})$ is a neighborhood of $\ov{\gamma}$, it must contain a 
$\gamma \in \Gamma$ $(\gamma \neq \ov{\gamma}$).  
But
\[
\gamma \in \Gamma \cap \Omega_{\ov{\gamma}} (\{x\}) \implies 
\gamma \in  \Omega_{\ov{\gamma}} (\{x\}) \implies 
\gamma x \hsx = \hsx \ov{\gamma} x.
\]
Meanwhile, 
\[
\gamma \in \Gamma  \ \& \ x \in \Inv(\Gamma) \implies \gamma x = x.
\]
Therefore $\ov{\gamma} x = x$.]
\end{x}

\vspace{0.1cm}


\begin{x}{\small\bf \un{N.B.}} \ %16
\[
\ov{\Gal^*(\K^\sep/\K})
\]
is a closed normal subgroup of $\Gal(\K^\sep/\K)$.
\end{x}

\vspace{0.1cm}


\begin{x}{\small\bf DEFINITION} \ %17

\[
\Inv(\Gal^*(\K^\sep/\K))
\]
is called the 
\un{maximal abelian extension}
\index{maximal abelian extension} 
of $\K$, denote it by $\K^\ab$.
\index{$\K^\ab$}
\end{x}

\vspace{0.1cm}
%%----------------------------------------------------------------------------------------------04

\begin{x}{\small\bf FACT} \ %18
$\K^\ab$ is a Galois extension of $\K$ and $\Gal(\K^\ab/\K)$ is an abelian group.

\vspace{0.1cm}

[Since
\[
\ov{\Gal^*(\K^\sep/\K)}
\]
is a closed normal subgroup of $\Gal(\K^\sep/\K)$, it follows that
\begin{align*}
\K^\ab \ 
&=\ \Inv ( \Gal^*(\K^\sep/\K) )\\
&=\ \Inv(\ov{\Gal^*(\K^\sep/\K)})
\end{align*}
is a Galois extension of  $\K$ and
\begin{align*}
\Gal(\K^\ab/\K) \ 
&\approx\ \Gal(\K^\sep/\K) / \Gal(\K^\sep/\K^\ab)\\
&=\ \Gal(\K^\sep/\K) / \ov{\Gal^*(\K^\sep/\K)}
\end{align*}
But the group on the RHS is isomorphic to 
\[
\Gal(\K^\sep/\K) / \Gal^*(\K^\sep/\K) / \ov{\Gal^*(\K^\sep/\K)} / \Gal^*(\K^\sep/\K),
\]
thus is a homomorphic image of the abelian group
\[
\Gal(\K^\sep/\K) / \Gal^*(\K^\sep/\K).]
\]
\end{x}

\vspace{0.1cm}


\begin{x}{\small\bf DEFINITION} \ %19
A Galois extesnsion $\LL/\K$ is said to be 
\un{abelian}
\index{abelian (field extension)}
if $\Gal(\LL/\K)$ is abelian.
\end{x}

\vspace{0.1cm}

\begin{x}{\small\bf FACT} \ %20
The field $\K^\ab$ has no extensions that are abelian Galois extensions of $\K$.

\vspace{0.1cm}

[Let $\LL/\K^\ab$ be an abelian Galois extensions of $\K$:
\[
\LL \hsx = \hsx \Inv(\Gal(\K^\sep/\K)) \supset \K^\ab \hsx = \hsx \Inv(\Gal^*(\K^\sep/\K))
\]
%%----------------------------------------------------------------------------------------------05
\qquad\qquad $\implies$
\[
\Gal^*(\K^\sep/\K) \hsx\supset\hsx \Gal(\K^\sep/\LL).
\]
On the other hand, $\Gal(\K^\sep/\LL)$ is normal ($\LL/\K$ being Galois) and 
\[
\Gal(\K^\sep/\K) /  \Gal(\K^\sep/\LL) \hsx \approx \hsx \Gal(\LL/\K),
\]
which is abelian by hypothesis, thus
\[
\Gal(\K^\sep/\LL) \supset \Gal^*(\K^\sep/\K).
\]
Therefore
\[
\Gal(\K^\sep/\LL) \ = \  \Gal^*(\K^\sep/\K).
\]
And then
\begin{align*}
\LL \ 
&\vsy=\ \Inv(\Gal(\K^\sep/\LL))\\
&\vsy=\ \Inv(\Gal^*(\K^\sep/\K))\\
&\vsy=\ \K^\ab.]
\end{align*}
\end{x}

\vspace{0.1cm}

\begin{x}{\small\bf FACT} \ %21
$\K^\ab$ is generated by the set of finite abelian Galois extensions of $\K$ in $\K^\sep$.

\vspace{0.1cm}

[Every finite Galois extension of $\K$ inside $\K^\ab$ is necessarily abelian.]
\end{x}

\vspace{0.1cm}

\begin{x}{\small\bf DEFINITION} \ %22
Take $\K = \Q$ $-$then the splitting field $\Q(n)$ of the polynomial $X^n - 1$ is called the 
\un{cyclotomic field}
\index{cyclotomic field} 
of the $n^\nth$ roots of unity.
\end{x}

\vspace{0.1cm}

\begin{x}{\small\bf FACT} \ %23
$\Q(n)$ is a Galois extension of $\Q$ and $\Gal(\Q(n)/\Q)$ is isomorphic to 
$(\Z/n\Z)^\times$, hence $\Gal(\Q(n)/\Q)$ is abelian.
\end{x}
\vspace{0.1cm}
%%----------------------------------------------------------------------------------------------06

Accordingly, every intermediate field $\E$ between $\Q$ and $\Q(n)$ is abelian Galois (per $\Q$).

\vspace{0.1cm}

[$\Gal(\Q(n) / \Q)$ is abelian, hence every subgroup of $\Gal(\Q(n) / \Q)$ is normal, hence in particular 
$\Gal(\Q(n) / \E)$ is normal, hence $\E/\Q$ is Galois.  And 
\[
\Gal(\E) / \Q) \hsx \approx \hsx \Gal(\Q(n) / \Q) / \Gal(\Q(n) / \E).]
\]

\vspace{0.2cm}

The Kronecker-Weber theorem states that every finite abelian Galois extension of $\Q$ is contained in some $\Q(n)$, 
thus $\Q^\ab$ is the infinite cyclotomic extension $\Q(1, 2, \ldots)$.

\vspace{0.2cm}

\begin{x}{\small\bf SCHOLIUM} \ %24
$\Q^\ab$ is generated by the torsion points of the action of $\Z$ on $\C^\times$.

\vspace{0.1cm}

[Note: \ Given $n \in \Z$, $x \in \C^\times$, $(n,x) \ra n \cdot x = x^n$.]
\end{x}


\vspace{0.3cm}

\[
\text{ADDENDUM}
\]
\setcounter{theoremn}{0}

\vspace{0.1cm}

If \mG is a group, then the subgroup $G^*$ generated by the commutators $x y x^{-1}y^{-1}$ is the 
\un{commutator subgroup}
\index{commutator subgroup} 
of \mG.

\vspace{0.2cm}

\qquad\qquad \textbullet \quad $G^*$ is a normal subgroup of \mG.

\vspace{0.1cm}

\qquad\qquad \textbullet \quad $G / G^*$ is abelian.

\vspace{0.1cm}

And if $H \subset G$ is normal and if $G/H$ is abelian, then $H \supset G^*$.

\vspace{0.1cm}

%%\begin{x}{\small\bf FACT} \ %01

\vspace{0.3cm}

{\small\bf FACT}
If  $\LL/\K$ is an infinite Galois extension and if $N \subset \Gal(\LL/\K)$ is a normal subgroup, then 
$\ov{N} \subset \Gal(\LL/\K)$ is a closed normal subgroup. 
%%\end{x}









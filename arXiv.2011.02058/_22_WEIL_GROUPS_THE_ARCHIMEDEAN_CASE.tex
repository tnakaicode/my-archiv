\chapter{
$\boldsymbol{\S}$\textbf{22}.\quad  WEIL GROUPS: THE ARCHIMEDEAN CASE}
\setlength\parindent{2em}
\setcounter{theoremn}{0}
%%----------------------------------------------------------------------------------------------01

%\ \indent 

\begin{x}{\small\bf DEFINITION} \ %01
Put $W_\C = \C^\times$, call it the 
\un{Weil group}
\index{Weil group} 
of $\C$, and leave it at that.
\end{x}

\vspace{0.1cm}

\begin{x}{\small\bf DEFINITION} \ %01
Put 
\[
W_\R \ = \ \C^\times \ \cup \ \tJ \C^\times \quad \text{(disjoint union) \quad (J a formal symbol)},
\]
where $\tJ^2 = -1$ and $\tJ z \tJ^{-1} = \ov{z}$ (obvious topology on $W_\R$).  Accordingly, there is a nonsplit short exact sequence
\[
1 \lra \C^\times \lra W_\R \lra \Gal(\C / \R) \lra 1, 
\]
the image of J in $\Gal(\C / \R)$ being complex conjugation.

\vspace{0.1cm}

[Note: \ $H^2(\Gal(\C / \R),\C^\times)$ is cyclic of order 2, thus up to equivalence of extensions of $\Gal(\C / \R)$ by 
$\C^\times$ per the canonical action of $\Gal(\C / \R)$ on $\C^\times$, there are two possibilities:

\vspace{0.2cm}

1. \quad A split extension
\[
1 \lra \C^\times \lra E \lra \Gal(\C / \R) \lra 1.
\]

2. \quad A nonsplit extension
\[
1 \lra \C^\times \lra E \lra \Gal(\C / \R) \lra 1.
\]
The Weil group $W$ is a representative of the second situation which is why we took $\tJ^2 = -1$ 
(rather than $\tJ^2 = +1$).]
\end{x}
\vspace{0.1cm}

\begin{x}{\small\bf LEMMA} \ %03
The commutator subgroup $W_\R^*$ of $W_\R$ consists of all elements of the form 
$J z \tJ^{-1} z^{-1} = \ds\frac{\ov{z}}{z}$, i.e., $W_\R^* = S$, thus is closed.
\end{x}
\vspace{0.1cm}

%%----------------------------------------------------------------------------------------------02
Let 
\[
\pr: W_\R \lra \R^\times
\]
be the map sending J to $-1$ and $z$ to $\abs{z}^2$.
\vspace{0.2cm}

\begin{x}{\small\bf LEMMA} \ %04
$S$ is the kernel of pr and pr is surjective.
\end{x}
\vspace{0.1cm}

\begin{x}{\small\bf LEMMA} \ %05
The arrow 
\[
\pr^\ab: W_\R^\ab \lra \R^\times
\]
induced by pr is an isomorphism.
\end{x}
\vspace{0.1cm}

\begin{x}{\small\bf REMARK} \ %06
The inverse $\R^\times \ra W_\R^\ab$ of $\pr^\ab$ is characterized by the conditions 
\[
\begin{cases}
 \ -1 \ra \tJ W_\R^*\\
 \ x \ra \sqrt{x} \ W_\R^* \qquad (x > 0)
\end{cases}
.
\]
\end{x}
\vspace{0.1cm}

\begin{x}{\small\bf NOTATION} \ %07
Define
\[
\ncdot : W_\R \lra \R_{>0}^\times
\]
by the prescription 
\[
\norm{z} \ = \ z \ov{z} \quad (z \in \C), \quad \norm{\tJ} = 1.
\]
\end{x}
\vspace{0.1cm}

\begin{x}{\small\bf \un{N.B.}} \ %08
$\ncdot$ drops to a continuous homomorphism $W_\R^\ab \ra \R_{>0}^\times$.
\end{x}
\vspace{0.1cm}

\begin{x}{\small\bf DEFINITION} \ %09
A 
\un{representation}
\index{representation} 
of $W_\R$ is a continuous homomorphism 
$\rho:W_\R \ra \GL(V)$, where V is a finite dimensional complex vector space.
\end{x}
\vspace{0.1cm}

\begin{x}{\small\bf EXAMPLE} \ %10
If $s \in \C$, then the assignment $w \ra \norm{w}^s$ is a 1-dimensional
%%----------------------------------------------------------------------------------------------03
representation of $W_\R$, i.e., is a character.
\end{x}
\vspace{0.1cm}

\begin{x}{\small\bf \un{N.B.}} \ %11
If $\chi$ is a character of $\R^\times$, then $\chi \circ \pr$ is a character of $W_\R$ and all such have this form.

[For any $\rho \in \widetilde{W}_\R$, 
\[
\rho(\ov{z}) \ = \rho(\tJ z \tJ^{-1}) \ = \  \rho(\tJ) \rho(z) \rho(\tJ)^{-1} \ = \  \rho(z).
\]
Therefore
\[
1 \ = \ \rho(-1) \qquad (\text{cf.} \ \S7, \ \#12).
\]
But 
\[
\rho(-1) \ = \ \rho(\tJ^2) \ = \ \rho(\tJ)^2,
\]
so $\rho(\tJ) = \pm1$.  This said, the characters of $\R^\times$ are described in \S7, \#11, thus the 1-dimensional 
representations of $W_\R$ are parameterized by a sign and a complex number $s$:

\qquad \textbullet \quad $(+,s): \rho(z) = \abs{z}^s$, $\rho(\tJ) = + 1$\\

\qquad \textbullet \quad $(-,s): \rho(z) = \abs{z}^s$, $\rho(\tJ) = - 1$.]\\

\vspace{0.1cm}

Let $V$ be a finite dimensional complex vector space.

\vspace{0.1cm}


\end{x}
\vspace{0.1cm}

\begin{x}{\small\bf DEFINITION} \ %12
A linear transformation $T:V \ra V$ is 
\un{semisimple} 
\index{semisimple} 
if every $T$-invariant subspace has a complementary $T$-invariant subspace.
\end{x}
\vspace{0.1cm}

\begin{x}{\small\bf FACT} \ %13
$T$ is semisimple iff $T$ is diagonalizable, i.e., in some basis $T$ is represented by a diagonal matrix.

\vspace{0.05cm}

[Bear in mind that $\C$ is algebraically closed \ldots \ .]
\end{x}

\vspace{0.1cm}
%%----------------------------------------------------------------------------------------------04

\begin{x}{\small\bf DEFINITION} \ %14
A representation $\rho:W_\R \ra \GL(V)$ is 
\un{semisimple} 
\index{semisimple} 
if $\forall \ w \in W_\R$, 
$\rho(w):V \ra V$ is semisimple.
\end{x}
\vspace{0.1cm}

\begin{x}{\small\bf DEFINITION} \ %15
A representation $\rho:W_\R \ra \GL(V)$ is 
\un{irreducible}
\index{irreducible} 
if $V \neq 0$, 
and the only $\rho$-invariant subspaces are 0 and $V$.
\end{x}

\vspace{0.1cm}

The irreducible 1-dimensional representations of $W_\R$ are its characters (which, of course, are automatically semisimple).

\vspace{0.1cm}

\begin{x}{\small\bf LEMMA} \ %16
If $\rho:W_\R \ra \GL(V)$ is a semisimple irreducible representation of $W_\R$ of dimension $> 1$, then $\dim V = 2$.

\vspace{0.1cm}

PROOF \ 
There is a nonzero vector $v \in V$ and a charcter $\chi:\C^\times \ra \C^\times$ such that $\forall \ z \in \C^\times$, 
\[
\rho(z) v \ = \ \chi(z) v.
\]
Since the span $S$ of $v$, $\rho(\tJ)v$ is a $\rho$-invariant subspace, the assumption of irreducibility implies that $\dim V = 2$.

[To check the $\rho$-invariance of $S$, note that 
\[
\begin{cases}
 \ \rho(z) \rho(\tJ)v \ = \ \rho(z\tJ)v  \ = \ \rho(\tJ\ov{z})v \ = \ \rho(\tJ) \rho(\ov{z}) v \ = \ \rho(\tJ)\chi(\ov{z}) v\\
 \ \rho(\tJ) \rho(\tJ)v \ = \ \rho(\tJ^2) v \ = \ \rho(-1)v \ = \ \chi(-1) v.]
\end{cases}
.
\]

\vspace{0.1cm}

Given an integer $k$ and a complex number $s$, define a character $\chi_{k,s}:\C^\times \ra \C^\times$ by the prescription
\[
\chi_{k,s}(z) \ = \ \Bigl(\frac{z}{\abs{z}}\Bigr)^k \bigl(\abs{z}^2 \bigr)^s
\]
and let $\rho_{k,s} = \ind \chi_{k,s}$ be the representation of $W_\R$ which it induces.
\end{x}

\vspace{0.1cm}
%%----------------------------------------------------------------------------------------------05

\begin{x}{\small\bf LEMMA} \ %17
$\rho_{k,s}$ is 2-dimensional.
\end{x}

\vspace{0.1cm}

\begin{x}{\small\bf LEMMA} \ %18
$\rho_{k,s}$ is semisimple.
\end{x}

\vspace{0.1cm}

\begin{x}{\small\bf LEMMA} \ %19
$\rho_{k,s}$ is irreducible iff $k \neq 0$.
\end{x}

\vspace{0.1cm}

\begin{x}{\small\bf DEFINITION} \ %20
Let
\[
\begin{cases}
 \ \rho_1:W_\R \ra \GL(V_1)\\
 \ \rho_2:W_\R \ra \GL(V_2)\\
\end{cases}
\]
be representations of $W_\R$ $-$then $(\rho_1,V_1)$ is 
\un{equivalent} 
\index{equivalent representations} 
to $(\rho_2,V_2)$ if there exists an 
isomorphism $f:V_1 \ra V_2$ such that $\forall \ w \in W_\R$, 
\[
f \circ \rho_1(w) \ = \ \rho_2(w) \circ f.
\]
\end{x}

\vspace{0.1cm}

\begin{x}{\small\bf LEMMA} \ %21
$\rho_{k_1,s_1}$ is equivalent to $\rho_{k_2,s_2}$ iff $k_1 = k_2$, $s_1 = s_2$ or $k_1 = -k_2$, $s_1 = s_2$.
\end{x}

\vspace{0.1cm}

\begin{x}{\small\bf LEMMA} \ %22
Every 2-dimensional semisimple irreducible representation of $W_\R$ is equivalent to a unique $\rho_{k,s}$ $(k > 0)$.
\end{x}

\vspace{0.1cm}

\begin{x}{\small\bf \un{N.B.}} \ %23
Therefore the equivalence classes of 2-dimensional semisimple irreducible representations of $W_\R$ are parameterized by 
the points of $\N \times \C$.
\end{x}

\vspace{0.1cm}

\begin{x}{\small\bf DEFINITION} \ %24
A representation $\rho:W_\R \ra \GL(V)$ is 
\un{completely reducible}
\index{completely reducible} 
if $V$ is the direct sum of a collection of irreducible 
$\rho$-invariant subspaces.
\end{x}

\vspace{0.1cm}
%%----------------------------------------------------------------------------------------------06

\begin{x}{\small\bf LEMMA} \ %25
Let $\rho:W_\R \ra \GL(V)$ be a semisimple representation $-$then $\rho$ is completely reducible.

\vspace{0.1cm}

PROOF \ 
The characters of $\C^\times$ are of the form $z \ra z^\mu \ov{z}^\nu$ with $\mu$, $\nu \in \C$, $\mu-\nu \in \Z$ and $V$ 
is the direct sum of subspaces $V_{\mu,\nu}$, where 
$\restr{\rho(z)}{V_{\mu,\nu}} =  z^\mu \ov{z}^\nu \ \id_{V_{\mu,\nu}}$.  
Claim: 
\[
\rho(\tJ) V_{\mu,\nu} \ = \ V_{\nu,\mu}.
\]
Proof:
$\forall \ v \in V_{\mu,\nu}$,
\begin{align*}
\rho(z) \rho(\tJ) v \ 
&=\vsy\ \rho(\tJ \ov{z} \tJ^{-1}) \rho(\tJ) v\\
&=\vsy\ \rho(\tJ) \rho(\ov{z}) \rho(\tJ^{-1}) \rho(\tJ) v\\
&=\vsy\ \rho(\tJ) \rho(\ov{z}) v\\
&=\vsy\ \rho(\tJ) \ov{z}^\mu z^\nu v\\
&=\vsy\ \rho(\tJ) z^\nu \ov{z}^\mu v\\
&=\vsy\ z^\nu \ov{z}^\mu \rho(\tJ) v.
\end{align*}

Proceeding:

\vspace{0.1cm}

\qquad \textbullet \quad \un{$\mu = \nu$} \ Choose a basis of eigenvectors for $\rho(\tJ)$ on $V_{\mu,\nu}$ $-$then 
the span of each eigenvector is a 1-dimensional $\rho$-invariant subspace.

\vspace{0.1cm}

\qquad \textbullet \quad \un{$\mu \neq \nu$} Choose a basis $v_1, \ldots v_r$ for $V_{\mu,\nu}$ and put 
$v_i^\prime = \rho(\tJ) v_i$ $(1 \leq i \leq r)$ $-$then $\C v_i \oplus \C v_i^\prime$
is a 2-dimensional $\rho$-invariant subspace and the direct sum 
\[
\bigoplus\limits_{i = 1}^r \ (\C v_i \oplus \C v_i^\prime)
\]
equals 
\[
V_{\mu,\nu} \oplus V_{\nu,\mu}.
\]
\end{x}

\vspace{0.1cm}
%%----------------------------------------------------------------------------------------------07
\begin{x}{\small\bf REMARK} \ %26
Suppose that $\rho:W_\R \ra \GL(V)$ is a representation $-$then 
\begin{align*}
\tJ^2 = -1 
&\implies  (-1) \tJ \cdot \tJ = 1\\
&\implies (-1) \tJ  = \tJ^{-1}
\end{align*}
\qquad\qquad\qquad\qquad $\implies$ 
\begin{align*}
\rho(\tJ)^{-1}\ 
&= \ \rho(\tJ^{-1})\\
&= \ \rho((-1)\tJ)\\
&= \ \rho(-1)\rho(\tJ).
\end{align*}
On the other hand, if $\tJ^2 = 1$ (the split extension situation (cf. \#2)), then 
\begin{align*}
\id_V \ 
&= \  \rho(1) \\
&= \ \rho(\tJ^2)\\
&=\ \rho(\tJ)\rho(\tJ).
\end{align*}
\qquad\qquad\qquad\qquad $\implies$ 
\[
\rho(\tJ)^{-1} \ = \ \rho(\tJ).
\]
\end{x}
\vspace{0.1cm}
%%%%%%%%%%%%%%%%%%%%%%%%%%%%%%%%%%%%%%
%%%%%%%%%%%%%%%%%%%%%%%%%%%%%%%%%%%%%%
%%%%%%%%%%%%%%%%%%%%%%%%%%%%%%%%%%%%%%






















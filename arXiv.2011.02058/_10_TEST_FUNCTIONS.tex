\chapter{
$\boldsymbol{\S}$\textbf{10}.\quad  TEST FUNCTIONS}
\setlength\parindent{2em}
\setcounter{theoremn}{0}
%%----------------------------------------------------------------------------------------------01
 
\ 
\indent The 
\un{Schwartz space}
\index{Schwartz space} 
$\sS(\R^n)$ consists of those complex valued $\sC^\infty$ functions which, together with all their derivatives, vanish at infinity faster than any power of $\norm{\cdot}$.

\vspace{0.2cm}

\begin{x}{\small\bf DEFINITION} \ %1
The elements $f$ of $\sS(\R^n)$ are the 
\un{test functions}
\index{test functions} 
on $\R^n$.
\end{x}

\vspace{0.1cm}

\begin{x}{\small\bf EXAMPLE} \ %2
Take $n = 1$ $-$then
\[
f(x) = C x^A \exp(-\pi x^2),			%\quad \text{where A = 0 or 1, is a test function said to be \un{standard}}
\]
where $A = 0$ or 1, is a test function, said to be 
\un{standard}. 
\index{standard (test function)} 
Here
\[
\int_\R x^A \exp(-\pi x^2) e^{2 \pi \sqrt{-1} \ tx} dx = (\sqrt{-1})^A t^A \exp(- \pi t^2),
\]
thus $\sF_\R$ of a standard function is again standard (c.f. $\S 7$, $51$).
\end{x}

\vspace{0.1cm}

[Note: \  Henceforth, by definition, the Fourier transform of an $f \in  L^1(\R)$ will be the function

\[
\widehat{f}:\R \lra \C
\]
defined by the rule
\begin{align*}
\widehat{f}(t) \ \
&= \ \sF_\R f(t) \\
&= \  \int_{\R} f(x) e^{2 \pi \sqrt{-1}\  tx} dx.]
\end{align*}

\vspace{0.1cm}


\begin{x}{\small\bf EXAMPLE} \ %3
Take $n = 2$ and identify $\R^2$ with $\C$ $-$then
\[
f(z) = Cz^A \ov{z}^{B} \exp( -2 \pi \abs{z}^2),
\]
where $A,B \in \Z_{\ge 0}$ $\&$ $AB = 0$, is a test function, said to be 
\un{standard}.   
\index{standard (test function)}
Here
%%----------------------------------------------------------------------------------------------02
\[
\int_{\C} z^A \ov{z}^B \exp( -2 \pi \abs{z}^2) e^{2 \pi \sqrt{-1} \ (wz + \ov{w} \ov{z}}) \abs{dz \wedge d \ov{z}} 
\ = \ \sqrt{-1}^{A+B} \  w^B \ov{w}^A \exp( -2 \pi \abs{w}^2),
\]
thus $\sF_\C$ of a standard function is again standard ( c.f. $\S 7$, $\# 53$ ).
\end{x}

\vspace{0.1cm}

[Note: Henceforth, by definition, the Fourier transform of an $f \in L^1(\C)$ will be the function 
\[
\widehat{f} : \C \lra \C
\]
defined by the rule
\begin{align*}
\widehat{f}(w) \ 
&= \ \sF_\C f(w) \\
&= \ \int_\C f(z) e^{2\pi \sqrt{-1} \  (wz + \ov{w}\ov{z})} \abs{dz \wedge d\ov{z}} .]
\end{align*}

\vspace{0.2cm}

\begin{x}{\small\bf DEFINITION} \ %4
Let $G$ be a totally disconnected locally compact group $-$then a function $f:G \ra \C$ is said to be 
\un{locally constant} if for any $x \in G$, 
there is an open subset $U_x$ of $G$ containing $x$ such that $f$ is constant on $U_x$.
\end{x}

\vspace{0.1cm}

\begin{x}{\small\bf LEMMA} \ %5
A locally constant function $f$ is continuous.

\vspace{0.1cm}

PROOF \ 
Fix $x \in G$ and suppose that $\{x_i\}$ is a net converging to $x$ $-$then $x_i$ is eventually in $U_x$, hence there $f(x_i) = f(x)$.
\end{x}

\vspace{0.1cm}

\begin{x}{\small\bf DEFINITION} \ %6
The 
\un{Bruhat space}
\index{Bruhat space} 
$\sB(G)$
\index{$\sB(G)$} 
consists of those complex valued locally constant functions whose support is compact.
\end{x}

\vspace{0.1cm}

[Note: \  $\sB(G)$ carries a "canonical topology" but I shall pass in silence as regards to its precise formulation$]$.\\
%%----------------------------------------------------------------------------------------------03


\begin{x}{\small\bf DEFINITION} \ %7
The elements $f$ of $\sB(G)$ are the 
\un{test functions}
\index{test functions (on \mG)} 
on $G$.
\end{x}

\vspace{0.1cm}

\begin{x}{\small\bf LEMMA} \ %8
Given a test function $f$, there exists an open-compact subgroup $K$ of $G$, and integer 
$n \ge 0$, elements $x_1, \ldots, x_n$ in $G$ and elements $c_1, \ldots, c_n$ in $\C$ such that the union
$\bigcup\limits_{k=1}^n K x_k K	$ is disjoint and 
\[
f = \sum_{k=1}^n c_k \chi_{K x_k K},	
\]
$\chi_{K x_k K}$ the characteristic function of $K x_k K$.
\vspace{0.1cm}

\vspace{0.1cm}

PROOF \ 
Since $f$ is locally constant, for every $z \in \C$ the pre image $f^{-1}(z)$ is an open subset of $G$.  
Therefore $X = \{x:f(x) \ne 0\}$ is the support of $f$.  
This said, given $x \in X$, define a map
\[
\begin{aligned}
\phi_x: G \times\ G 		&\ra \C\\	
(x_1, x_2) 				&\mapsto f(x_1 x x_2)\\	
\end{aligned}
,
\]
thus $\phi_x(e,e) = f(x) $ and $\phi_x$ is continuous if $\C$ has the discrete topology. 
Consequently, one can find an open-compact subgroup $K_x$ of $G$ such that $\phi_x$ is constant on $K_x \times K_x$.  
Put $U_x = K_x \times K_x$ $-$then $U_x$ is open-compact and $f$ is constant on $U_x$.  
But $X$ is covered by the $U_x$, hence, being compact, is covered by finitely many of them.  
Bearing in mind that distinct double cosets are disjoint, consider now the intersection $K$ of the finitely many $K_x$ that occur.

\end{x}

\vspace{0.1cm}

Specialize and let $G = \Q_p$.
\\
\begin{x}{\small\bf EXAMPLE} \ %9
If $K \subset \Q_p$ is open-compact, then its characteristic function $\chi_K$ is a test function on $\Q_p$ .
\end{x}

\vspace{0.1cm}
%%----------------------------------------------------------------------------------------------04

\begin{x}{\small\bf LEMMA} \ %10
Every $f \in \sB(\Q_p)$ is a finite linear combination of functions of the form
\[
\chi_{x+p^n \Z_p}		\qquad (x \in \Q_p, \ n \in \Z).
\]

[This is an instance of $\# 8$ or argue directly (c.f. $\S 4$, $\#33$).]
\end{x}

\vspace{0.1cm}

\begin{x}{\small\bf DEFINITION} \ %11
Given $f \in L^1(\Q_p)$, its 
\un{Fourier transform}
\index{Fourier transform} 
is the function 
\[
\widehat{f}:\Q_p \lra \C
\]
defined by the rule
\begin{align*}
\widehat{f}(t) 
&= \   \int_{\Q_p} f(x) \chi_{p,t}(x) dx\\
&= \ \int_{\Q_p} f(x) \chi_p(tx) dx.	
\end{align*}
\end{x}

\vspace{0.1cm}

\begin{x}{\small\bf LEMMA} \ %12
$\forall$ $f \in L^1(\Q_p)$,
\[
\widehat{\ov{f\hspace{0.07cm}}}(t) = \overline{\widehat{f}(-t)}.
\]

\vspace{0.1cm}

PROOF \ 
\begin{align*}
\widehat{\ov{f\hspace{0.07cm}}}(t)   \ 	
&=\  \int_{\Q_p} \overline{f(x)} \chi_p(tx) dx \\
&=\  \int_{\Q_p} \overline{f(x) \chi_p(-tx)} dx \\
&=\  \int_{\Q_p} \overline{f(x) \chi_p((-t)x)} dx \\
&=\  \overline{\int_{\Q_p)} f(x) \chi_p((-t)x) dx} \\
&=\  \overline{\widehat{f}(-t)}. 
\end{align*}
\end{x}

\vspace{0.1cm}
%%----------------------------------------------------------------------------------------------05


\begin{x}{\small\bf SUBLEMMA} \ %13

\[
\int_{p^n\Z_p} \chi_p(x) dx \ = \ 
\begin{cases}
p^{-n} \qquad (n \geq 0)\\
0 \qquad\quad \ (n < 0)
\end{cases}
.
\]

\vspace{0.1cm}

[Recall that 
\[
\mu_{\Q_p}(p^n \Z_p) \ = \  p^{-n}
\]
and apply $\S 7$, $\# 46$ and $\S 8$, $\# 12$.]
\end{x}

\vspace{0.1cm}


\begin{x}{\small\bf LEMMA} \ %14
Take $f = \chi_{p^n\Z_p}$  $-$then
\[
\widehat{\chi}_{p^n\Z_p} = p^{-n} \chi_{p^{-n}\Z_p}.
\]

\vspace{0.1cm}

PROOF \ 
\begin{align*}
\widehat{\chi}_{p^n\Z_p}(t) \ 	
&=\  \int_{\Q_p} \chi_{p^n\Z_p}(x) \chi_{p,t}(x) dx\\
&=\  \int_{\Q_p} \chi_{p^n\Z_p}(x) \chi_p(tx) dx\\
&=\  \abs{t}_p^{-1}\int_{\Q_p} \chi_{p^n\Z_p}(t^{-1}x) \chi_p(x) dx\\
&=\  \abs{t}_p^{-1}\int_{p^{n+v(t)}\Z_p} \chi_p(x) dx.
\end{align*}
The last integral equals 
\[
p^{-n - v(t)}
\]
if $n + v(t) \geq 0$ and equals 0 if $n + v(t) < 0$ (cf. $\# 13$).  
But
%%----------------------------------------------------------------------------------------------06
\[
t \in p^{-n}\Z_p \Leftrightarrow v(t) \ge -n \Leftrightarrow n+v(t) \ge 0.
\]
Since
\[
\abs{t}^{-1}_pp^{v(t)} \ =\  1,
\]
it therefore follows that
\[
\widehat{\chi}_{p^n\Z_p} \ =\  p^{-n} \chi_{p^{-n}\Z_p}.
\]
In particular, 
\[
\widehat{\chi}_{\Z_p} \ =\  \chi_{\Z_p}.
\]
\end{x}

\vspace{0.1cm}

\begin{x}{\small\bf THEOREM} \ %15
Take $f = \chi_{x+p^n\Z_p}$ $-$then
\[
\widehat{\chi}_{x+p^n\Z_p}(t)  = \ 
\begin{cases}
\chi_p(tx)p^{-n} \qquad (\abs{t}_p \le p^n)\\
0 \qquad \qquad\qquad (\abs{t}_p > p^n)
\end{cases}
.
\]

\vspace{0.1cm}

PROOF \ 
%%----------------------------------------------------------------------------------------------07 (ish)
\begin{align*}
\widehat{\chi}_{x+p^n\Z_p}(t) \ 
&=\  \int_{\Q_p} \chi_{x+p^n\Z_p}(y) \chi_{p,t}(y) dy\\
&=\  \int_{\Q_p} \chi_{x+p^n\Z_p}(y) \chi_p(ty) dy\\
&=\  \int_{x+p^n\Z_p} \chi_p(ty) dy\\
&=\  \int_{p^n\Z_p} \chi_p(t(x+y)) dy\\
&=\  \int_{p^n\Z_p} \chi_p(tx+ty) dy\\
&=\  \int_{p^n\Z_p} \chi_p(tx) \chi_p(ty) dy\\
&=\  \chi_p(tx) \int_{p^n\Z_p} \chi_p(ty) dy\\
&=\  \chi_p(tx) \int_{\Q_p} \chi_{p^n\Z_p}(y) \chi_p(ty) dy\\										
&=\  \chi_p(tx) \int_{\Q_p} \chi_{p^n\Z_p}(y) \chi_{p,t}(y) dy\\
&=\  \chi_p(tx) \widehat{\chi}_{{p^n}\Z_p} (t)\\
&=\  \chi_p(tx) p^{-n}\chi_{p^{-n}\Z_p}(t).
\end{align*}
\end{x}

\vspace{0.1cm}

\begin{x}{\small\bf APPLICATION} \ %16
Taking into account $\# 10$, 
\[
f \in \sB(\Q_p) \Rightarrow \widehat{f} \in \sB(\Q_p). 
\]
\end{x}

\vspace{0.1cm}

\begin{x}{\small\bf THEOREM} \ %17
$\forall$ $f \in \mathbf{INV}(\Q_p)$, 
\[
\widehat{\widehat{f}\hspace{0.1cm}} = f(-x)	\qquad (x \in \Q_p).
\]

\vspace{0.1cm} 

PROOF \ 
It suffices to check this for a single function, so take $f = \chi_{Z_p}$ $-$then as noted above,
\[
\widehat{\chi}_{\Z_p} = \chi_{\Z_p},
\]
%%----------------------------------------------------------------------------------------------08
thus $\forall$ x,
\[
\widehat{\widehat{\chi}\hspace{0.05cm}}_{\Z_p}(x) = \chi_{\Z_p} (x) =  \chi_{\Z_p} (-x).
\]
\end{x}
\vspace{0.1cm}

\begin{x}{\small\bf \un{N.B.}} \ %18
It is clear that
\[
\sB(\Q_p) \subset \mathbf{INV}(\Q_p).
\]
\end{x}

\vspace{0.1cm}


\begin{x}{\small\bf SCHOLIUM} \ %19
The arrow $f \ra \widehat{f}$ is a linear bijection of $\sB(\Q_p)$ onto itself.

[Injectivity is manifest.  
As for surjectivity, the arrow $f \ra \check{f}$, where
\[
\widecheck{f\hspace{0.05cm}} = f(-x),
\]
maps $\sB(\Q_p)$ into itself.  And
\[
f \ 
=\ \widecheck{\widecheck{f\hspace{0.05cm}}} 
=\  (\widecheck{f\hspace{0.05cm}}) \raisebox{0.22cm}{$\text{ }\widecheck{}$}
=\  (\widecheck{f\hspace{0.05cm}})    \raisebox{0.11cm}{$\text{ }\widehat{} \hspace{0.15cm}\text{ }\widehat{}$}
=\  ((\widecheck{f\hspace{0.05cm}}) 
\raisebox{0.08cm}{$\text{ }\widehat{}$} \hspace{0.1cm} )
\raisebox{0.08cm}{$\text{ }\widehat{}$}.]
\]
\end{x}


\begin{x}{\small\bf REMARK} \ %20
As is well-known, the same conclusion obtains if $\Q_p$ is replaced by $\R$ or $\C$.
\end{x}

\vspace{0.1cm}

Pass now from  $\Q_p$ to  $\Q_p^\times$.

\vspace{0.2cm}

\begin{x}{\small\bf LEMMA} \ %21
Let $ f \in \sB(\Q_p^\times)$ $-$then $\exists$ $n \in \N:$
\[
\begin{cases}
\ \abs{x}_p < p^{-n} 	\implies f(x) = 0\\
\ \abs{x}_p > p^{n} \ \ 	\implies f(x) = 0
\end{cases}
.
\]

Therefore an element $f$ of $\sB(\Q_p^\times)$ can be viewed as an element of $\sB(\Q_p)$ with the property that $f(0) = 0$.
\end{x}

\vspace{0.1cm}

%%----------------------------------------------------------------------------------------------09
\begin{x}{\small\bf DEFINITION} \ %22
Given $f \in L^1(\Q_p^\times, d^\times x)$, its 
\un{Mellin transform}
\index{Mellin transform} 
$\widetilde{f}$ 
is the Fourier transform of $f$ per $\Q_p^\times:$
\[
\widetilde{f}(\chi) = \int_{\Q_p^\times} f(x) \chi(x) d^\times x.
\]

[Note: \  
By definition, 
\[
d^\times x = \frac{p}{p-1} \frac{dx}{\abs{x}_p}		\qquad \text{(c.f. } \S 6, \ \# 26),
\]
so
\[
\vol_{d^\times x}(\Z_p^\times) \ =\  \vol_{dx}(\Z_p) \ =\  1.]
\]
\end{x}

\vspace{0.1cm}


\begin{x}{\small\bf EXAMPLE} \ %23
Take $f = \chi_{\Z_p^\times}$ $-$then
\begin{align*}
\widetilde{\chi}_{\Z_p^\times} (\chi) \ 
&= \  \int_{\Q_p^\times} \chi_{\Z_p^\times} (x) \chi(x) d^\times x \\
&= \ \int_{\Z_p^\times} \chi(x) d^\times x.
\end{align*}
Decompose $\chi$ as in $\S 9$, $\# 10$, hence
%\[
%\int_{\Z_p^\times} \chi(x) d^\times x = \int_{\Z_p^\times} \abs{x}_p^{\sqrt{-1}} \chi(p^{-v(x)}x) d^\times x
%\]
\begin{align*}
\int_{\Z_p^\times} \chi(x) d^\times x \ 
&=\  \int_{\Z_p^\times} \abs{x}_p^{\sqrt{-1}\ t} \un{\chi}(p^{-v(x)}x) d^\times x\\
&=\  \int_{\Z_p^\times}  \un{\chi}(x) d^\times x\\
&=\ 
\begin{cases}
 \ 0  \qquad (\un{\chi} \not\equiv 1)\\
\ 1  \qquad (\un{\chi} \equiv 1)
\end{cases}
.
\end{align*}



%%----------------------------------------------------------------------------------------------10
According to $\S 9$, $\# 2$, a unitary character $\chi \in \widehat{(\Q_p^\times)}$ is unramified if its restriction 
$\un{\chi}$ to $\Z_p^\times$ is trivial.  
Therefore the upshot is that the Mellin transform of $\chi_{\Z_p^\times}$ 
is the characteristic function of the set of unramified elements of $\widehat{(\Q_p^\times)}$.
\end{x}

\vspace{0.1cm}
%%%%%



%%%%%%%%%%%%%%%%%%%
%%%%%

\[
\textbf{APPENDIX}
\]
\setcounter{theoremn}{0}
%\vspace{0.1cm}





Let $\K$ be a finite extension of $\Q_p$ $-$then
\[
\K^\times \thickapprox \Z \times R^\times
\]
and the generalities developed in \S9 go through with but minor changes when $\Q_p$ is replace by $\K$.

In particular$:$ $\forall$ $\chi \in \widehat{K}^\times$, there is a splitting
\[
\chi(a) = \abs{a}_\K^{\sqrt{-1}\  t} \un{\chi}(\pi^{-v(a)}a),
\]
where $t$ is real and 
\[
-(\pi/\log q) < t \le \pi / \log q.
\]

[Note: \  $\chi$ is \un{unramified} if it is trivial on $R^\times$.$]$


\begin{x}{\small\bf \un{N.B.}} \ %1
The $"\pi"$ in the first instance is a prime element (c.f. $\S 5$, $\# 10$) and $\abs{\pi}_\K = \ds\frac{1}{q}$.  
On the other hand, the $"\pi"$ in the second instance is $3.14\ldots$ .

\vspace{0.4cm}

The extension of the theory  from $\sB(\Q_p)$ to $\sB(\K)$ is straightforward, the point of departure being the observation that
\[
\int_{\pi^nR} \chi_{\K,p}(a) da = \mu_\K(R) \ 
\begin{cases}
\ q^{-n} \qquad (n = -d, -d+1, \ldots)\\
\  0 \qquad \quad (n = -d-1, -d-2, \ldots)
\end{cases}
.
\]
\end{x}

\vspace{0.1cm}

\begin{x}{\small\bf CONVENTION} \ %2
Normalize the Haar measure on $\K$ by stipulating that $\ds\int_R da = q^{-d/2}$.
\end{x}

\vspace{0.1cm}

\begin{x}{\small\bf DEFINITION} \ %3
Given $f \in L^1(\K)$, its 
\un{Fourier transform}
\index{Fourier transform} 
is the function 
\[
\widehat{f}:\K \lra \C
\]
defined by the rule
\[
\begin{aligned}
\widehat{f}(b)	
=&  \int_{\K} f(a) \chi_{\K,p,b} (a) da\\	
=&  \int_{\K} f(a) \chi_{\K,p} (ab) da.
\end{aligned}
\]
\end{x}

\vspace{0.2cm}

\begin{x}{\small\bf THEOREM} \ %4
$\forall$ $f \in \mathbf{INV}(\K)$,
\[
\widehat{\widehat{f}\hspace{0.1cm}}(a) = f(-a)	\qquad (a \in \K).
\]

\vspace{0.1cm}

PROOF \  
It suffices to check this for a single function, so take $f = \chi_R$, 
in which case the work has already been done in the Appendix to $\S 8$.  
To review:

\vspace{0.4cm}

\indent\indent$\text{\textbullet} \quad \widehat{\chi}_R(b)$
%\begin{align*} 
\indent\indent\indent\indent\quad \  \ $=\  \ds\int_{\K} \chi_R(a) \chi_{\K,p} (ab)da$\\
\indent\indent\indent\indent\indent\indent\indent\indent\indent $=\  \ds\int_{R} \chi_{\K,p} (ab)da$\\
\indent\indent\indent\indent\indent\indent\indent\indent\indent $=\  q^{-d/2} \chi_{\Delta_\K}(b)$.\\
%\end{align*}

\vspace{0.1cm}

\indent\indent$\text{\textbullet} \quad \int_{\K} q^{-d/2} \chi_{\Delta_\K}(b) \chi_{\K,p} (ab)db \quad $
%\begin{align*} 
$ =\  q^{-d/2} \ds\int_{\Delta_\K} \chi_{\K,p} (ab)db$\\
\indent\indent\indent\indent\indent\indent\indent\indent\indent $=\  \chi_R(a) \qquad \  \text{(loc. cit., $\# 13$)}$\\
\indent\indent\indent\indent\indent\indent\indent\indent\indent $=\  \chi_R(-a)$.
%\end{align*}

\end{x}

\vspace{0.1cm}

\begin{x}{\small\bf \un{N.B.}} \ %5
It is clear that 
\[
\sB(k) \subset \mathbf{INV}(\K).
\]
\end{x}

\vspace{0.1cm}

\begin{x}{\small\bf SCHOLIUM} \ %6
The arrow $f \ra \widehat{f}$ is a linear bijection of $\sB(k)$ onto itself.
\end{x}

\vspace{0.1cm}

\begin{x}{\small\bf CONVENTION} \ %7
Put
\[
d^\times a = \frac{q}{q-1} \frac{da}{\abs{a}_\K}.
\]
Then $d^\times a$ is a Haar measure on $\K^\times$ and 
\[
\vol_{d^\times a} (R^\times) = \vol_{da}(R) = q^{-d/2}.
\]
\end{x}

\vspace{0.1cm}

\begin{x}{\small\bf DEFINITION} \ %8
Given $f \in L^1(\K^\times, d^\times a)$, its 
\un{Mellin transform}
\index{Mellin transform} 
$\widetilde{f}$ is the Fourier transform of $f$ per $\K^\times:$
\[
\widetilde{f}(\chi) = \int_{\K^\times} f(a) \chi(a) d^\times a.
\]
\end{x}

\vspace{0.1cm}

\begin{x}{\small\bf EXAMPLE} \ %9
Take $f = \chi_{R^\times}$ $-$then

\[
\widetilde{\chi}_{R^\times}(\chi) = \ 
\begin{cases}
0 \quad\qquad \ \  (\un{\chi} \ne 1)\\
q^{-d/2} \qquad (\chi \equiv 1)
\end{cases}
.\]

\end{x}
%%%%%%%%%%%%%%%%%%%%%%%
%\quad \text{$(r \ge 1)$} \text{ }
%%%%%%%%%%%%%%%%%%%%%%%%%%%%%%%%%%%%%%
%%%%%%%%%%%%%%%%%%%%%%%%%%%%%%%%%%%%%%
%%%%%%%%%%%%%%%%%%%%%%%%%%%%%%%%%%%%%%






















\chapter{
%$\boldsymbol{\S}$\textbf{24}.\quad  THE WEIL-DELIGNE GROUP}
APPENDIX B: TOPICS IN ALGEBRA}
\setlength\parindent{2em}
\setcounter{theoremn}{0}
\[
\text{PRINCIPAL IDEAL DOMAINS}
\]

\[
\text{FIELD EXTENSIONS}
\]

\[
\text{ALGEBRAIC CLOSURE}
\]

\[
\text{TRACES AND NORMS}
\]
\newpage
%%----------------------------------------------------------------------------------------------01

\ \indent 

\[
\text{PRINCIPAL IDEAL DOMAINS}
\]

Let \mA be a commutative ring with unit.

\vspace{0.2cm}

\begin{x}{\small\bf DEFINITION} \ %01
An 
\un{ideal}
\index{ideal} 
\mI in \mA is an additive subgroup of \mA such that the relations $a \in A$, $x \in I$ imply that 
$a x$ ($= x a$) belongs to \mI.
\end{x}

\vspace{0.1cm}


\begin{x}{\small\bf DEFINITION} \ %02
An ideal \mI in \mA is a 
\un{prime ideal} 
\index{prime ideal} 
if $I \neq A$ and if $a b \in I$ implies that either $a \in I$ or $b \in I$.
\end{x}

\vspace{0.1cm}

\begin{x}{\small\bf DEFINITION} \ %03
An ideal \mI in \mA is a 
\un{maximal ideal} 
\index{maximal ideal} 
if $I \neq A$ and there is no larger proper ideal of \mA that contains \mI.
\end{x}

\vspace{0.1cm}

\begin{x}{\small\bf DEFINITION} \ %04
\mA is an
\un{integral domain} 
\index{integral domain} 
if $a b = 0$ implies that $a = 0$ or $b = 0$.
\end{x}

\vspace{0.1cm}

\begin{x}{\small\bf \un{N.B.}}  \ %05
Every field is an integral domain.
\end{x}

\vspace{0.1cm}

\begin{x}{\small\bf EXAMPLE} \ %06
$\Z$ is an integral domain but $Z/n\Z$ is an integral domain iff $n$ is prime.
\end{x}

\vspace{0.1cm}

\begin{x}{\small\bf FACT}  \ %07
An ideal $I \neq A$ in \mA is a prime ideal iff $A/I$ is an integral domain.
\end{x}

\vspace{0.1cm}

\begin{x}{\small\bf FACT}  \ %08
An ideal $I \neq A$ in \mA is a maximal ideal iff $A/I$ is a field.
\end{x}

\vspace{0.1cm}

\begin{x}{\small\bf EXAMPLE} \ %09
Take $A = \Z[X]$ $-$then $\langle X \rangle$ is a prime ideal 
(since $A/\langle X \rangle \approx \Z$ is an integral domain) but $\langle X \rangle$ is not a maximal ideal 
(since $A/\langle X \rangle \approx \Z$ is not a field).
\end{x}


\begin{x}{\small\bf DEFINITION} \ %10
An ideal \mI in \mA is a 
\un{principal ideal}
\index{principal ideal} 
if $I = A a_0$ ($\equiv \hsx \langle a_0\rangle$) for some $a_0 \in A$.
\end{x}

\vspace{0.1cm}

%%----------------------------------------------------------------------------------------------02

\begin{x}{\small\bf DEFINITION} \ %11
\mA is a 
\un{principal ideal domain}
\index{principal ideal domain} 
if \mA is an integral domain and if every ideal in \mA is principal.
\end{x}

\vspace{0.1cm}

\begin{x}{\small\bf FACT} \ %12
For any field $\K$, the polynomial ring $\K[X]$ is a principal ideal domain.

\vspace{0.1cm}

[If \mI is a nonzero ideal in $\K[X]$, then \mI consists of all the multiples of the monic polynomial in \mI of least degree.]
\end{x}

\vspace{0.1cm}

\begin{x}{\small\bf EXAMPLE} \ %13
The polynomial ring $\Z[X]$ is not a principal ideal domain.

\vspace{0.1cm}

[The ideal \mI consisiting of all polynomials with even constant term is not a principal ideal (but it is a maximal ideal).]
\end{x}

\vspace{0.1cm}



\begin{x}{\small\bf FACT} \ %14
If \mA is a principal ideal domain, then every nonzero prime ideal is maximal.
\end{x}

\vspace{0.1cm}


\begin{x}{\small\bf FACT} \ %15
For any field $\K$, the maximal ideals in $\K[X]$ are the nonzero prime ideals.
\end{x}

\vspace{0.1cm}


\begin{x}{\small\bf DEFINITION} \ %16
A 
\un{unit} 
\index{unit} 
in \mA is an element $u \in A$ with a multiplicative inverse, i.e., there is a $v \in A$ such that $u v = 1$.
\end{x}

\vspace{0.1cm}


\begin{x}{\small\bf EXAMPLE} \ %17
The units in $\K[X]$ are the nonzero constants.
\end{x}

\vspace{0.1cm}


\begin{x}{\small\bf EXAMPLE} \ %18
The units in $\Z$ are 1 and $-1$.
\end{x}

\vspace{0.1cm}


\begin{x}{\small\bf EXAMPLE} \ %19
The units in $\Z/n\Z$ are the congruence classes $[a]$ of a mod $n$ such that $(a,n) = 1)$.
\end{x}

\vspace{0.1cm}

\begin{x}{\small\bf DEFINITION} \ %20
The elements $a, b \in A$ are said to be 
\un{associates}
\index{associates} 
if there is a unit $u \in A$ such that $a = u b$.
\end{x}

\vspace{0.1cm}

%%----------------------------------------------------------------------------------------------03

\begin{x}{\small\bf DEFINITION} \ %21
A nonzero element $p \in A$ is said to be 
\un{irreducible}
\index{irreducible} 
if $p$ is not a unit and in every factorization $p = a b$, either $a$ or $b$ is a unit.
\end{x}



\vspace{0.1cm}

\begin{x}{\small\bf EXAMPLE} \ %22
Take $A = \Z[X]$ $-$then $2X + 2 = 2(X + 1)$ is not irreducible, yet it does not factor into a product of polynomials of lower degree.
\end{x}

\vspace{0.1cm}

\begin{x}{\small\bf SCHOLIUM} \ %23
For any field $\K$, a nonzero polynomial $p(X) \in \K[X]$ of degree $\geq 1$ is irreducible iff there is no factorizartion 
$p(X) = f(X) g(X)$ in $\K[X]$ with $\deg f < \deg p$ and $\deg g < \deg p$.
\end{x}

\vspace{0.1cm}

\begin{x}{\small\bf FACT} \ %24
If \mA is a principal ideal domain, then the nonzero prime ideals are the ideals $\langle p \rangle$, where $p$ is irreducible.
\end{x}

\vspace{0.1cm}


\begin{x}{\small\bf FACT} \ %25
If \mA is a principal ideal domain and if $p \in A$ is irreducible, then $A / \langle p \rangle$ is a field.

\vspace{0.1cm}

[For $\langle p \rangle$ is prime, hence maximal.]
\end{x}

\vspace{0.1cm}


\begin{x}{\small\bf DEFINITION} \ %26
\mA is a 
\un{unique factorization domain}
\index{unique factorization domain} 
if \mA is an integral domain subject to:

\vspace{0.1cm}

\un{E}\quad Every nonzero $a \in A$ that is not a unit is a product of irreducible elements.

\un{U}\quad If 
\[
p_1 \cdots p_m \hsx = \hsx q_1 \cdots q_n, 
\]
where the $p$ and $q$ are irreducible, then $m = n$ and there is a one-to-one correspondence between the 
factors such that the corresponding factors are associates.

\end{x}

\vspace{0.1cm}


\begin{x}{\small\bf FACT} \ %27
Every principal ideal domain is a unique factorization domain.
\end{x}

\vspace{0.1cm}

\begin{x}{\small\bf APPLICATION} \ %28
For any field $\K$, the polynomial ring $\K[X]$ is a unique factorization domain
\end{x}

\vspace{0.1cm}

%%----------------------------------------------------------------------------------------------04

\begin{x}{\small\bf DEFINITION} \ %29
Suppose that \mA is a unique factorization domain $-$then a 
\un{system of representatives of irreducible elements in \mA}
\index{system of representatives of irreducible elements in \mA} 
is a set of irreducible elements having exactly one element in common with the set of all associates of each irreducible element.
\end{x}

\vspace{0.1cm}

\begin{x}{\small\bf SCHOLIUM} \ %30
For any field $\K$, the monic irreducible polynomials constitute a system of representatives of irreducible elements in $\K[X]$. 

\vspace{0.1cm}

[Note: \ Let $f$ be a nonconstant polynomial in $\K[X]$ and let $f_1, \ldots, f_n$ be the distinct monic irreducible factors of 
$f$ in $\K[X]$ $-$then 
\[
f \hsx = \hsx C \prod\limits_{k = 1}^n \hsx f_k^{e_k},
\]
where $C$ is the leading coefficient of $f$ and $e_1, \ldots, e_n$ are positive integers.  Moreover, 
this representation of $f$ is unique up to a permutation of $\{1, \ldots, n\}$.]
\end{x}

\vspace{0.1cm}


\begin{x}{\small\bf FACT} \ %31
For any field $\K$ and for any irreducible polynomial $p(X)$, the quotient 
$\LL^\prime = \K[X]/\langle p(X) \rangle$ is a field containing an isomorphic copy $\K^\prime$ of $\K$ as a subfield 
and a zero of $p^\prime(X)$.

\vspace{0.1cm}

[Setting $I = \langle p(X) \rangle$, the map $a \ra a + I$ $(a \in \K)$ identifies $\K$ with a subfield $\K^\prime$ 
of $\LL^\prime$.  Write
\[
p(X) \hsx = \hsx a_0 + a_1 X + \cdots + a_n X^n.
\]
Then in $\K^\prime[X]$, 
\[
p^\prime(X) \hsx = \hsx (a_0 + I) + (a_1 + I)X + \cdots + (a_n + I) X^n.
\]
Now put $\theta = X + I$:
\begin{align*}
p^\prime(\theta) \ 
&=\vsx\ (a_0 + I) + (a_1 X + I) + \cdots + (a_n X^n + I)\\
&=\vsx\ a_0 + a_1 X + \cdots + a_n X^n + I\\
&=\vsx\ p(X) + I \\
&=\vsx\ I, 
\end{align*}
the zero element of $\LL^\prime$.]
\end{x}

\vspace{0.1cm}


%%%%%%%%%%%%%%%%%%%%%%%%%%%%%%%%%%%%%%
%%%%%%%%%%%%%%%%%%%%%%%%%%%%%%%%%%%%%%
%%%%%%%%%%%%%%%%%%%%%%%%%%%%%%%%%%%%%%


\setcounter{theoremn}{0}

\newpage
%%----------------------------------------------------------------------------------------------01


\[
\text{FIELD EXTENSIONS}
\]

\ \indent 

Let $\K$ be a field.

\vspace{0.3cm}

\begin{x}{\small\bf DEFINITION} \ %01
A 
\un{field extension} 
\index{field extension} 
of $\K$ is a field $\LL$ having $\K$ as a subfield.  
\end{x}

\vspace{0.1cm}

Given $\LL/\K$ and elements $x_1, \ldots, x_n \in \LL$, write $\K(x_1, \ldots, x_n)$ for the subfield of $\LL$ 
generated by $\K$ and the $x_i$ $(i = 1, \ldots, n)$.  In particular: $\K(x)$ is the subfield generated by $\K$ and $x$.

\begin{x}{\small\bf EXAMPLE} \ %02
Take $\K = \Q$, $\LL = \R$, $x = \sqrt{2}$ $-$then $\Q(\sqrt{2})$ consists of all real numbers of the form 
$r + s \sqrt{2}$ $(r, s \in \Q)$.

\vspace{0.1cm}

[Let \mF be the set of all real numbers of the indicated form, thus
\[
\Q \cup \{\sqrt{2}\} \hsx \subset \hsx \F \hsx \subset \hsx \Q(\sqrt{2}),
\]
and, by definition, $\Q(\sqrt{2})$ is the subfield of $\R$  generated by $\Q \cup \{\sqrt{2}\}$.  
Let now 
$x = r + s \sqrt{2}$ $(r, s, \in \Q)$: $r^2 - 2 s^2 \neq 0$ ($\sqrt{2}$ irrational)

\qquad\qquad\qquad\qquad $\implies$
\begin{align*}
\frac{1}{x} \hsx 
&= \hsx \frac{r}{r^2 - 2 s^2} + \frac{-s}{r^2 - 2 s^2} \sqrt{2}\\
&\in \F,
\end{align*}
so $\F$ is a field, so $\F = \Q(\sqrt{2})$.]
\end{x}

\vspace{0.1cm}

\begin{x}{\small\bf EXAMPLE} \ %03
Take $\K = \Q$, $\LL = \R$, $x = \sqrt{2}$, $y = \sqrt{3}$ $-$then
\[
\Q(\sqrt{2},\sqrt{3}) \hsx = \hsx \Q(\sqrt{2} + \sqrt{3}).
\]

[Obviously, $\sqrt{2} + \sqrt{3} \in \Q(\sqrt{2},\sqrt{3})$ hence 
$\Q(\sqrt{2} + \sqrt{3}) \subset \Q(\sqrt{2},\sqrt{3})$.  
In the other direction
%%----------------------------------------------------------------------------------------------02
\[
(\sqrt{2} + \sqrt{3}) (\sqrt{2} - \sqrt{3}) = -1
\]
\qquad\qquad\qquad\qquad $\implies$
\[
\sqrt{3} - \sqrt{2} \hsx = \hsx \ \frac{1}{\sqrt{2} + \sqrt{3}} \in \Q(\sqrt{2} + \sqrt{3})
\]
\qquad\qquad\qquad\qquad $\implies$
\[
\begin{cases}
\ \sqrt{3} \hsx = \hsx ((\sqrt{3} + \sqrt{2}) + (\sqrt{3} - \sqrt{2})) / 2\\
\ \sqrt{2} \hsx = \hsx ((\sqrt{3} + \sqrt{2}) - (\sqrt{3} - \sqrt{2})) / 2
\end{cases}
\in \Q(\sqrt{2} + \sqrt{3}).
\]
Therefore $\Q(\sqrt{2},\sqrt{3}) \subset \Q(\sqrt{2} + \sqrt{3})$.]
\end{x}

\vspace{0.1cm}

Given $\LL \supset \K$, view $\LL$ as a vector space over $\K$ and write $[\LL:\K]$ for its dimension, the 
\un{degree} 
\index{degree} 
of $\LL$ over $\K$.

\vspace{0.1cm}

[Note: \ In this context, the term ``dimension'' refers to the cardinal number of a basis for $\LL$ over $\K$.]

\vspace{0.2cm}

\begin{x}{\small\bf FACT} \ %04
Let $\F \subset \K \subset \LL$ be fields $-$then
\[
[\LL:\F] \hsx = \hsx [\LL:\K]  \cdot [\K:\F].  
\]
\end{x}

\vspace{0.1cm}

\begin{x}{\small\bf EXAMPLE} \ %05
Take $\F = \Q$, $\K = \Q(\sqrt{2})$, $\LL = \Q(\sqrt{2},\sqrt{3})$ $-$then
\begin{align*}
[\Q(\sqrt{2},\sqrt{3}):\Q] \ 
=\ [\Q(\sqrt{2},\sqrt{3}):\Q(\sqrt{2})] \cdot [\Q(\sqrt{2}):\Q]\\
=\ 2 \cdot 2 \\
=\  4.
\end{align*}
\end{x}

\vspace{0.1cm}

\begin{x}{\small\bf DEFINITION} \ %06
$\LL$ is a 
\un{finite extension}
\index{finite extension} 
of $\K$ if $[\LL:\K]$ is finite and $\LL$ is an 
\un{infinite extension}
\index{infinite extension} 
of $\K$ if $[\LL:\K]$ is infinite.
\end{x}

\vspace{0.1cm}

%%----------------------------------------------------------------------------------------------03

\begin{x}{\small\bf EXAMPLE} \ %07
$[\C:\R] = 2$ but $[\C:\Q] = 2^{\aleph_0}$.
\end{x}

\vspace{0.1cm}

Given $\LL/\K$ and $x \in \LL$, the 
\un{ideal $I_x$ of algebraic relations of $x$}
\index{ideal $I_x$ of algebraic relations of $x$} 
is the ideal in $\K[X]$ consisting of all polynomials admitting $x$ as a zero.

\vspace{0.2cm}

\begin{x}{\small\bf DEFINITION} \ %08
$x$ is 
\un{algebraic}
\index{algebraic}
over $\K$ 
(\un{transcendental} over $\K$)
\index{transcendental (element over a field $\K$)} 
according to whether $I_x$ is nonzero (zero).  
I.e.: $x$ is algebraic over $\K$ (transcendental over $\K$) according to whether it is (or is not) 
a zero of a nonzero polynomial in $\K[X]$.
\end{x}

\vspace{0.1cm}

\begin{x}{\small\bf EXAMPLE} \ %09
Take $\K = \Q$, $\LL = \C$ $-$then $\sqrt{-1}$ is algebraic over $\Q$ but $e$ and $\pi$ are transcendental over $\Q$.
\end{x}

\vspace{0.1cm}

\begin{x}{\small\bf FACT} \ %10
Let $x \in \LL$ $-$then $x$ is algebraic over $\K$ iff $I_x$ is a nonzero prime ideal in $\K[X]$ or still, 
is a maximal ideal in $\K[X]$.
\end{x}

\vspace{0.1cm}

\begin{x}{\small\bf FACT} \ %11
If $x \in \LL$ is algebraic over $\K$, then $I_x$ has a unique monic polynomial $p_x$ in $\K[X]$ as a generator: 
$I_x = \langle p_x \rangle$, the 
\un{minimal polynomial of $x$ over $\K$}.
\index{minimal polynomial of $x$ over $\K$} 

\vspace{0.1cm}

[Note: \  One can characterize $p_x$ as the monic polynomial in $\K[X]$ that admits $x$ as a zero and divides in $\K[X]$ 
every polynomial admitting $x$ as a zero.]

\end{x}

\vspace{0.1cm}


\begin{x}{\small\bf REMARK} \ %12
The minimal polynomial of an element depends on the base field.  
E.g.: If $\K = \Q$ and $\LL = \C$, then $p_{\sqrt{-1}}(X) = X^2 + 1$ but if $\K = \LL = \C$, then 
$p_{\sqrt{-1}}(X) = X - \sqrt{-1}$.
\end{x}

\vspace{0.1cm}

\begin{x}{\small\bf FACT} \ %13
If $x \in \LL$ is algebraic over $\K$, then its minimal polynomial $p_x$ is irreducible.
\end{x}

\vspace{0.1cm}

%%----------------------------------------------------------------------------------------------04

\begin{x}{\small\bf FACT} \ %14
If $x \in \LL$ is algebraic over $\K$ and if $n = \deg p_x$, then $p_x$ is the only monic polynomial in $\K[X]$ 
of degree $n$ admitting $x$ as a zero.
\end{x}

\vspace{0.1cm}


\begin{x}{\small\bf FACT} \ %15
If $x \in \LL$ is algebraic over $\K$, then the set $\{x^j: 0 \leq j \leq n-1\}$ is a linear basis of $\K(x)$ over $\K$, 
hence $[\K(x):\K] = n$.
\end{x}

\vspace{0.1cm}

\begin{x}{\small\bf EXAMPLE} \ %16
Take $\K = \Q$, $\LL = \R$, $x = (2)^{1/3}$ $-$then $\Q((2)^{1/3})$ is a subfield of $\R$ and $(2)^{1/3}$ 
is algebraic over $\Q$, its minimal polynomial being $X^2 - 2$, so $[\Q((2)^{1/3}):\Q] = 3$.
\end{x}

\vspace{0.1cm}

\begin{x}{\small\bf DEFINITION} \ %17
$\LL$ is an 
\un{algebraic extension}
\index{algebraic extension} 
of $\K$ if every element of $\LL$ is algebraic over $\K$.
\end{x}

\vspace{0.1cm}

\begin{x}{\small\bf FACT} \ %18
If $[\LL:\K] < \infty$, then $\LL$ is an algebraic extension of $\K$.

\vspace{0.1cm}

[If $n = [\LL:\K]$ and if $x \in \LL$, then the sequence $x^j$ $(0 \leq j \leq n)$ is linearly dependent over $\K$, 
so there exists a sequence $a_j$ $(0 \leq j \leq n)$ of elements of $\K$ (not all zero) such that 
$\ds\sum\limits_{j = 0}^n a_j x^j = 0$.]
\end{x}

\vspace{0.1cm}

\begin{x}{\small\bf FACT} \ %19
Suppose that $\K$ is infinite and $\LL$ is an algebraic extension of $\K$ $-$then 
\[
\card \K \hsx = \hsx \card \LL.
\]
\end{x}

\vspace{0.1cm}


\begin{x}{\small\bf EXAMPLE} \ %20
$\R$ is not an algebraic extension of $\Q$.
\end{x}

\vspace{0.1cm}

\begin{x}{\small\bf DEFINITION} \ %21
Let $\K$ be a field and let $\LL_1, \LL_2$ be field extensions of $\K$ $-$then 
%%----------------------------------------------------------------------------------------------05
a 
\un{$\K$-homomorphism}
\index{$\K$-homomorphism} 
$\phi:\LL_1 \ra \LL_2$ is a ring homomorphism such that 
$\restr{\phi}{\K} = \id_\K$, $\phi$ being called a 
\un{$\K$-isomorphism}
\index{$\K$-isomorphism} 
if it is in addition bijective (injectivity is automatic).

\vspace{0.1cm}

[Note: \ When $\LL_1 = \LL_2$, the term is 
\un{$\K$-automorphism}.]
\index{$\K$-automorphism} 

\end{x}

\vspace{0.1cm}

\begin{x}{\small\bf REMARK} \ %22
If $\LL_1 = \LL_2$, call it $\LL$, and if $\LL$ is an algebraic extension of $\K$, then every $\K$-homomorphims 
$\phi:\LL \ra \LL$ is a $\K$-isomorphism.
\end{x}

\vspace{0.1cm}

\begin{x}{\small\bf FACT} \ %23
Let $\K$ be a field and let $\LL_1$, $\LL_2$ be field extensions of $\K$.  Suppose that $f$ is an irreducible polynomial in 
$\K[X]$ and suppose that $x_1, x_2$ are, respectively, zeros of $f$ in  $\LL_1$, $\LL_2$  $-$then 
there is a unique $\K$-isomorphism 
$\K(x_1) \ra \K(x)$ such that $x_1 \ra x_2$.
\end{x}

\vspace{0.1cm}

[Note: \ The assumption that $f$ is irreducible cannot be dropped.]

\vspace{0.3cm}

\setcounter{theoremn}{0}
\[
\text{ADDENDUM}
\]

Let $\K$ be a field, $\LL/\K$ a field extension $-$then a sublset $S$ of $\LL$ is a 
\un{transcendence basis}
\index{transcendence basis} 
for $\LL/\K$ if \mS is algebraically independent over $\K$ and if $\LL$ is algebraic over $\K(S)$ 
(the subfield of $\LL$ generated by $\K \cup S$).

\begin{x}{\small\bf FACT} \ %01
A transcendence basis for $\LL/\K$ always exists and any two have the same cardinality.
\end{x}

\vspace{0.1cm}

\begin{x}{\small\bf DEFINITION} \ %02
The 
\un{transcendence degree}
\index{transcendence degree} 
$\trdeg(\LL/\K)$
\index{$\trdeg(\LL/\K)$} 
is the cardinality of any transcendence basis of $\LL/\K$.
\end{x}

\vspace{0.1cm}

%%----------------------------------------------------------------------------------------------06

\begin{x}{\small\bf EXAMPLE} \ %03
Take $\K = \Q$, $\LL = \C$ $-$then $\trdeg(\C/\Q)$ is infinite (in fact uncountable).
\end{x}

\vspace{0.1cm}

\begin{x}{\small\bf EXAMPLE} \ %04
Take $\K = \Q$, $\LL = \Q_p$ $-$then $\trdeg(\Q_p/\Q)$ is infinite (in fact uncountable).
\end{x}

\vspace{0.1cm}



%%%%%%%%%%%%%%%%%%%%%%%%%%%%%%%%%%%%%%
%%%%%%%%%%%%%%%%%%%%%%%%%%%%%%%%%%%%%%
%%%%%%%%%%%%%%%%%%%%%%%%%%%%%%%%%%%%%%


\setcounter{theoremn}{0}

\newpage
%%----------------------------------------------------------------------------------------------01

\ \indent 

\[
\text{ALGEBRAIC CLOSURE}
\]

Let $\K$ be a field, $\LL/\K$ a field extension.

\vspace{0.2cm}

\begin{x}{\small\bf NOTATION} \ %01
$A(\LL/\K)$ is the set of all elements of $\LL$ that are algebraic over $\K$.
\end{x}

\vspace{0.1cm}


\begin{x}{\small\bf DEFINITION} \ %02
$A(\LL/\K)$ is the 
\un{algebraic closure}
\index{algebraic closure} 
of $\K$ in $\LL$.
\end{x}

\vspace{0.1cm}

\begin{x}{\small\bf EXAMPLE} \ %03
Take $\K = \R$, $\LL = \C$ $-$then $A(\LL/\K) = \C$.

\vspace{0.1cm}

[Given $a + \sqrt{-1} \hsx b$, consider the polynomial
\[
(X - (a + \sqrt{-1} \hsx b)) (X - (a - \sqrt{-1} \hsx b)) \hsx = \hsx 
X^2 - 2 a X + a^2 + b^2.]
\]
\end{x}

\vspace{0.1cm}

\begin{x}{\small\bf FACT} \ %04
$\LL$ is an algebraic extension of $\K$ iff $A(\LL/\K) = \LL$.
\end{x}

\vspace{0.1cm}

\begin{x}{\small\bf DEFINITION}  \ %05
$\K$ is 
\un{algebraically closed}
\index{algebraically closed} 
in $\LL$ if every element of $\LL$ that is algebraic over $\K$ belongs to $\K$:
\[
A(\LL/\K) \hsx = \hsx \K.
\]
\end{x}

\vspace{0.1cm}

\begin{x}{\small\bf FACT} \ %06

\[
\K \hsx \subset\hsx A(\LL/\K) \hsx\subset\hsx \LL.
\]
\end{x}

\vspace{0.1cm}

\begin{x}{\small\bf FACT}  \ %07
$A(\LL/\K)$ is a field.
\end{x}

\vspace{0.1cm}

\begin{x}{\small\bf FACT}  \ %08
$A(\LL/\K)$ is algebracally closed in $\LL$.

\vspace{0.1cm}

[Spelled out, if $x \in \LL$ is algebraic over $A(\LL/\K)$, then $x \in A(\LL/\K)$.]
\end{x}

\vspace{0.1cm}

\begin{x}{\small\bf SCHOLIUM} \ %09
If $\K \subset \E \subset \LL$ and if $\E$ is an algebraic extension of $\K$, then
%%----------------------------------------------------------------------------------------------02
\[
\E \subset A(\LL/\K).
\]
\end{x}


\vspace{0.1cm}

\begin{x}{\small\bf DEFINITION} \ %10
Take $\K = \Q$, $\LL = \C$ $-$then an 
\un{algebraic number}
\index{algebraic number} 
is a complex number which is algebraic over $\Q$, i.e., is an element of $A(\C/\Q)$.
\end{x}

\vspace{0.1cm}


\begin{x}{\small\bf FACT} \ %11
$\card A(\C/\Q) \hsx = \hsx \aleph_0$.
\end{x}

\vspace{0.1cm}

\begin{x}{\small\bf FACT} \ %12
$[A(\C/\Q):\Q] \hsx = \hsx \aleph_0$.

\vspace{0.1cm}

[Let $n$ be a postive integer $-$then the polynomial $X^n - 2$ is irreducible in $\Q[X]$, thus is the minimal polynomial of 
$(2)^{1/2}$ over $\Q$, so $[Q((2)^{1/2}):\Q] = n$, from which 
\[
[A(\C/\Q):\Q] \hsx \geq \hsx n.
\]
And this implies that 
\[
[A(\C/\Q):\Q] \hsx \geq \hsx \aleph_0.
\]
On the other hand, 
\[
[A(\C/\Q):\Q] \hsx \leq \hsx \card A(\C/\Q) \hsx = \hsx \aleph_0.]
\]
\end{x}

\vspace{0.1cm}

\begin{x}{\small\bf DEFINITION} \ %13
A field $\F$ is 
\un{algebraically closed}
\index{algebraically closed} 
if every nonconstant polynomial in $\F[X]$ has a zero in $\F$.

\vspace{0.1cm}

[Note: \ This notion is absolute.]
\end{x}

\vspace{0.1cm}



\begin{x}{\small\bf EXAMPLE} \ %14
Neither $\Q$ nor $\R$ is algebraically closed but $\C$ is algebraically closed.
\end{x}

\vspace{0.1cm}


\begin{x}{\small\bf FACT} \ %15
$\F$ is algebraically closed iff every irreducible polynomial has degree 1.
\end{x}

\vspace{0.1cm}

%%----------------------------------------------------------------------------------------------03

\begin{x}{\small\bf FACT} \ %16
$\F$ is algebraically closed iff every nonconstant polynomial $f$ in $\F[X]$ splits in $\F[X]$.

\vspace{0.1cm}

[Note: \ I.e.: Given $f$, there exists a postive integer $n$ and elements 
$a, a_1, \ldots, a_n$ (not necessarily distinct) of $\F$ such that 
\[
f(X) \hsx = \hsx a \hsx \prod\limits_{k = 1}^n (X - a_k).]
\]
\end{x}

\vspace{0.1cm}


\begin{x}{\small\bf FACT} \ %17
If $\F$ is algebraically closed, then it is its only algebraic extension.
\end{x}

\vspace{0.1cm}


\begin{x}{\small\bf FACT} \ %18
If there is an algebraically closed field extension $\F^\prime$ of $\F$ in which $\F$ is algebraically closed, then 
$\F$ is algebraically closed.

\vspace{0.1cm}

[Let $f \in \F[X]$ be a nonconstant polynomial $-$then $f$ has a zero $a^\prime$ in $\F^\prime$, hence 
$a^\prime$ is algebraic over $\F$, hence $a^\prime \in \F$ (since $\F$ is algebraically closed in $\F^\prime$).]
\end{x}

\vspace{0.1cm}


\begin{x}{\small\bf APPLICATION} \ %19
Suppose that $\LL/\K$ is an algebraically closed field extension.  
Let $\F = A(\LL/\K)$, $\F^\prime = \LL$ to conclude that $A(\LL/\K)$ is algebraically closed.
\end{x}

\vspace{0.1cm}

\begin{x}{\small\bf EXAMPLE} \ %20
Take $\K = \Q$, $\LL = \C$ $-$then $\C$ is algebraically closed, hence $A(\C/\Q)$ is algebraically closed.
\end{x}

\vspace{0.1cm}



\begin{x}{\small\bf FACT} \ %21
Let $\K$ be a field, let $\LL$ be an algebraic closure of $\K$, and let $\M$ be an algebraically closed extension of $\K$ $-$then 
there exists a $\K$-monomorphism $\phi:\LL \ra \M$.
\end{x}



\vspace{0.1cm}

\begin{x}{\small\bf EXAMPLE} \ %22
Take $\K = \R$, $\LL = \C$, $\M = \C$ $-$then the inclusion $\R \ra \C$ admits two distinct extensions to $\C$, 
viz. the identity and the complex conjugation
%%----------------------------------------------------------------------------------------------04
(and these are the only $\R$-automorphisms of $\C$).

\vspace{0.1cm}

[Note: \ Therefore uniqueness of the extending $\K$-monomorphism cannot be asserted.]
\end{x}

\vspace{0.1cm}



\begin{x}{\small\bf EXAMPLE} \ %23
If $\E \neq \R$ is an algebraic extension of $\R$, then $\E$ is isomorphic to $\C$.

\vspace{0.1cm}

[Take $\K = \R$, $\LL = \E$, $\M = \C$ $-$then there exists an $\R$-monomorphism 
$\phi:\E \ra \C$, hence 
\[
2 \hsx = \hsx [\C:\R] \hsx = \hsx [\C:\phi(\E)] \cdot [\phi(\E):\R],
\]
from which $\C = \phi(\E) \approx \E$.]
\end{x}


\vspace{0.1cm}

\begin{x}{\small\bf DEFINITION} \ %24
Given a field $\F$, an 
\un{algebraic closure}
\index{algebraic closure} 
of $\F$ is an algebraicallly closed algebraic extension of $\F$.
\end{x}

\vspace{0.1cm}


\begin{x}{\small\bf EXAMPLE} \ %25
$\C$ is an algebraic closure of $\R$ but $\C$ is not an algebraic closure of $\Q$ (since it is not algebraic over $\Q$).
\end{x}

\vspace{0.1cm}


\begin{x}{\small\bf EXAMPLE} \ %26
$A(\C/\Q)$ is an algebraic closure of $\Q$.
\end{x}

\vspace{0.1cm}


\begin{x}{\small\bf STEINITZ THEOREM} \ %27
Every field $\F$ admits an algebraic closure $\F^\cl$ and any two algebraic closures of $\F$ are 
$\F$-isomorphic.
\end{x}

\vspace{0.1cm}


\begin{x}{\small\bf FACT} \ %28
Every automorphism of $\F$ can be extended to an automorphism of $\F^\cl$.

\vspace{0.1cm}

[Note: \ In general, if $\F_1$ and $\F_2$ are fields, then every isomorphism from $\F_1$ to $\F_2$ can be extended to an 
isomorphism from $\F_1^\cl$ to $\F_2^\cl$.]

\end{x}

\vspace{0.1cm}


\begin{x}{\small\bf FACT} \ %29
If $\LL / \K$ is an algebraic extension of $\K$, then $\LL$ is $\K$-isomorphic to a subfield of $\K^\cl$.
\end{x}

%%%%%%%%%%%%%%%%%%%%%%%%%%%%%%%%%%%%%%
%%%%%%%%%%%%%%%%%%%%%%%%%%%%%%%%%%%%%%
%%%%%%%%%%%%%%%%%%%%%%%%%%%%%%%%%%%%%%


\setcounter{theoremn}{0}

\newpage
%%----------------------------------------------------------------------------------------------01

\ \indent 

\[
\text{TRACES AND NORMS}
\]


Let $\K$ be a field, $\LL/\K$ a field extension of $\K$ $-$then each $x \in \LL$ gives rise to a linear transformation 
\[
M_x: \LL \ra \LL
\]
defined by 
\[
M_x(y) \hsx = \hsx x y.
\]

\vspace{0.2cm}

\begin{x}{\small\bf DEFINITION} \ %01
The 
\un{trace}
\index{trace}\index{$\tT_{\LL/\K}$} 
of $\LL$ over $\K$ is the function 
\[
\begin{cases}
\ \tT_{\LL/\K}:\LL \ra \K\\
\ \tT_{\LL/\K}(x) \ \hsx = \hsx \tr(M_x).
\end{cases}
\]
\end{x}

\vspace{0.1cm}


\begin{x}{\small\bf DEFINITION} \ %02
The 
\un{norm}
\index{norm}\index{$\tN_{\LL/\K}$}
of $\LL$ over $\K$ is the function 
\[
\begin{cases}
\ \tN_{\LL/\K}:\LL \ra \K\\
\ \tN_{\LL/\K}(x) \ \hsx = \hsx \det(M_x).
\end{cases}
\]
\end{x}

\vspace{0.1cm}

\begin{x}{\small\bf PROPERTIES} \ %03
$\forall \ x, y \in \LL$, $\forall \ a \in \K$:

\vspace{0.1cm}

1. \ $\tT_{\LL/\K}(x + y) \hsx = \hsx \tT_{\LL/\K}(x) + \tT_{\LL/\K}(y)$.

\vspace{0.1cm}

2. \ $\tT_{\LL/\K}(a) \hsx = \hsx [\LL:\K] a$.

\vspace{0.1cm}

3. \ $\tN_{\LL/\K}(x y) \hsx = \hsx \tN_{\LL/\K}(x) \tN_{\LL/\K}(y)$.

\vspace{0.1cm}

4. \ $\tN_{\LL/\K}(a) \hsx = \hsx a^{[\LL:\K]}$.
\end{x}

\vspace{0.1cm}

%%----------------------------------------------------------------------------------------------02

\begin{x}{\small\bf FACT} \ %04
If $\E$ is a subfield of $\LL$ containing $\K$, then 
\[
\begin{cases}
\ \tT_{\LL/\K}(x) \hsx = \hsx \tT_{\E/\K} ( \tT_{\LL/\E} (x))\\
\ \tN_{\LL/\K}(x) \hsx = \hsx \tN_{\E/\K} ( \tN_{\LL/\E} (x))
\end{cases}
.
\]
\end{x}

\vspace{0.1cm}

\begin{x}{\small\bf EXAMPLE} \ %05
Let $\theta \in \K^\times - (\K^\times)^2$ and put $\LL = \K(\sqrt{\theta})$ $-$then $\forall \ a, b \in \K$, 
\[
\begin{cases}
\ \tT_{\LL/\K}(a + b \sqrt{\theta}) \hsx = \hsx 2a\\
\ \tN_{\LL/\K}(x) (a + b \sqrt{\theta}) \hsx = \hsx a^2 - b^2 \theta
\end{cases}
.
\]
\end{x}









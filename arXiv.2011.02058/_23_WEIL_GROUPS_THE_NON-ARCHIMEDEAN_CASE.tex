\chapter{
$\boldsymbol{\S}$\textbf{23}.\quad  WEIL GROUPS: THE NON-ARCHIMEDEAN CASE}
\setlength\parindent{2em}
\setcounter{theoremn}{0}
%%----------------------------------------------------------------------------------------------01
\ \indent Let $\K$ be a non-archimedean local field. 

\vspace{0.25cm}

\begin{x}{\small\bf NOTATION} \ %01
Put 
\[
\begin{cases}
 \ G_\K \ = \ \Gal(\K^\sep / \K)\\
 \ G_\K^\ab \ = \ \Gal(\K^\ab / \K)
\end{cases}
.
\]
\end{x}

\vspace{0.1cm}

\begin{x}{\small\bf \un{N.B.}} \ %02
Every character of $G_\K$ factors through $\ov{G_\K^*}$, hence gives rise to a charcter of $G_\K^\ab$.
\end{x}
\vspace{0.1cm}

To study the characters of $G_\K^\ab$, precompose with the reciprocity map $\rec_\K:\K^\times \ra G_\K^\ab$, thus
\[
\chi_\K \ : \ 
\begin{cases}
 \ (G_\K^\ab)^{\widetilde{\ }} \ra (\K^\times)^{\widetilde{\ }}\\
 \ \chi \ra \chi \circ \rec_\K
\end{cases}
.
\]

\vspace{0.1cm}

\begin{x}{\small\bf LEMMA} \ %03
$\chi_\K$ is a homomorphism.
\end{x}
\vspace{0.1cm}

\begin{x}{\small\bf LEMMA} \ %04
$\chi_\K$ is injective.

\vspace{0.1cm}

PROOF \ 
Suppose that 
\[
\chi_\K(\chi) \ = \ \chi \circ \rec_\K
\]
is trivial $-$then $\restr{\chi}{\Img \rec_\K} = 1$. But $\Img  \rec_\K$ is dense in $G_\K^\ab$ (cf. \S21, \#13), 
so by continuity, $\chi \equiv 1$.
\end{x}

\vspace{0.1cm}

%%----------------------------------------------------------------------------------------------02

\begin{x}{\small\bf LEMMA} \ %05
$\chi_\K$ is not surjective.

\vspace{0.1cm}

PROOF \ 
$G_\K^\ab$ is compact abelian and totally disconnected.  
Therefore 
$(G_\K^\ab)^{\widetilde{\ }} = (G_\K^\ab)^{\widehat{\ }}$ 
and every $\chi$ is unitary and of finite order 
(cf. \S7, \#7 and \S8, \#2), thus the $\chi_\K(\chi)$ are unitary and of finite order.  
But there are characters of $\K^\times$ for which this is not the case.
\end{x}

\vspace{0.1cm}

\begin{x}{\small\bf \un{N.B.}} \ %06
The failure of $\chi_\K$ to be surjective will be remedied below (cf. \#19). 
\end{x}

\vspace{0.1cm}

The kernel of the arrow 
\[
\Gal(\K^\sep / \K) \lra \Gal(\K^\ur / \K)
\]
of restriction is $\Gal(\K^\sep / \K^\ur)$ and there is an exact sequence
\[
1 \lra \Gal(\K^\sep / \K^\ur) \lra \Gal(\K^\sep / \K)  \lra \Gal(\K^\ur / \K) \lra 1.
\]
Identify
\[
\Gal(\K^\ur / \K)
\]
with 
\[
\Gal(\F_q^\ab / \F_q) 
\]
and put 
\[
W(\F_q^\ab / \F_q) \ = \ \langle \sigma_q \rangle \qquad (\text{discrete topology}).
\]

\vspace{0.2cm}

\begin{x}{\small\bf DEFINITION} \ %07
The \un{Weil group} $W(\K^\sep / \K)$ is the inverse image of $W(\F_q^\ab / \F_q)$ in $\Gal(\K^\sep / \K)$, 
i.e., the elements of $\Gal(\K^\sep / \K)$ which induce an intgral power of $\sigma_q$.
\end{x}

\vspace{0.1cm}
%%----------------------------------------------------------------------------------------------03

\begin{x}{\small\bf NOTATION} \ %08
Abbreviate $W(\K^\sep / \K)$  to $W_\K$, hence $W_\K \subset G_\K$.
\end{x}

\vspace{0.1cm}

Setting 
\[
I_\K \ = \ \Gal(\K^\sep / \K^\ur) \qquad (\text{the \un{inertia group}}),
\]
there is an exact sequence
\[
\begin{tikzcd}[sep=large]
{1} \ar{r} 
&{I_\K} \ar{r}
&{W_\K} \ar{r}
&{W(\F_q^\ab / \F_q)} \ar{d} \ar{r}
&{1}\\
&&&{\Z} \ar{u}[swap]{\approx}
\end{tikzcd}
.
\]

\vspace{0.1cm}

[Note: \  Fix an element $\widetilde{\sigma}_q \in W_\K$ which maps to $\sigma_q$ $-$then structurally, 
$W_\K$ is the disjoint union
\[
\bigcup\limits_{n \in \Z} (\widetilde{\sigma}_q)^n I_\K.]
\]

Topologize $W_\K$ by taking for a neighborhood basis at the identity the 
\[
\Gal(\K^\sep / \LL) \cap I_\K,
\]
where $\LL$ is a finite Galois extension of $\K$.

\vspace{0.2cm}

\begin{x}{\small\bf REMARK} \ %09
$I_\K$ has the relative topology per the inclusion $I_\K \ra G_\K$ and any splitting 
$\Z \ra W_\K$ induces an isomorphism $W_\K \approx I_\K \times \Z$ of topological groups, 
where $\Z$ has the discrete topology.
\end{x}

\vspace{0.1cm}

\begin{x}{\small\bf LEMMA} \ %10
$W_\K$ is a totally disconnected locally compact group.

[Note: \ $W_\K$ is not compact \ldots \ .]
\end{x}

\vspace{0.1cm}

\begin{x}{\small\bf LEMMA} \ %11
The inclusion  $W_\K \ra G_\K$ is continuous and has a dense image.
\end{x}

\vspace{0.1cm}
%%----------------------------------------------------------------------------------------------04

\begin{x}{\small\bf LEMMA} \ %12
$I_\K$ is open in  $W_\K$.
\end{x}

\vspace{0.1cm}

\begin{x}{\small\bf LEMMA} \ %13
$I_\K$ is a maximal compact subgroup of $W_\K$.
\end{x}

\vspace{0.1cm}

Suppose that $\LL / \K$ is a finite extension of $\K$ $-$then $G_\LL \subset G_\K$ is the subgroup of $G_\K$ fixing $\LL$, hence 
\[
W_\LL \subset G_\LL \subset G_\K.
\]

\begin{x}{\small\bf LEMMA} \ %14
\[
W_\LL \ = \  G_\LL \cap W_\K \subset W_\K
\]
is open and of finite index in $W_\K$, it being normal in $W_\K$ iff $\LL / \K$ is Galois.
\end{x}

\vspace{0.1cm}

\begin{x}{\small\bf THEOREM} \ %15
The arrow 
\[
\LL \ra  W_\LL
\]
is a bijection between the finite extensions of $\K$ and the open subgroups of $W_\K$.

\vspace{0.1cm}

[By contrast, the arrow 
\[
\LL \ra \Gal(\K^\sep / \LL)
\]
is a bijection between the finite extensions of $\K$ and the open subgroups of $G_\K$.]
\end{x}

\vspace{0.1cm}

\begin{x}{\small\bf LEMMA} \ %16
\[
\ov{W_\K^*} \ = \ \ov{G_\K^*} .
\]
\end{x}

\vspace{0.1cm}

\begin{x}{\small\bf APPLICATION} \ %17
The homomorphism $W_\K^\ab \ra G_\K^\ab$ is 1-to-1.
\end{x}
\vspace{0.1cm}
%%----------------------------------------------------------------------------------------------05

\begin{x}{\small\bf THEOREM} \ %18
The image of 
$\rec_\K:\K^\times \ra G_\K^\ab$ is $W_\K^\ab$ and the induced map $\K^\times \ra W_\K^\ab$ is an isomorphism of topological groups (cf. \S21, \#13).
\end{x}
\vspace{0.1cm}

The characters of $W_\K$ ``are'' the characters of $W_\K^\ab$, so we have:

\vspace{0.2cm}

\begin{x}{\small\bf SCHOLIUM} \ %19
There is a bijective correspondence between the characters of $W_\K$ and the characters of $\K^\times$ or still, there is a 
bijective correspondence between the 1-dimensional representations of $W_\K$ and the 
1-dimensional representations of $\GL_1(\K)$.
\end{x}

\vspace{0.1cm}


Suppose that $\LL / \K$ is a finite Galois extension of $\K$ $-$then $G_\LL \subset G_\K$ and 
\[
G_\K / G_\LL \ \approx \ \Gal(\LL / \K)
\]
is finite of cardinality 
$[\LL:\K]$.  Since $W_\K$ is dense in $G_\K$, it follows that the image of the arrow
\[
\begin{cases}
 \ W_\K \lra G_\K / G_\LL\\
 \ w \lra w G_\LL
\end{cases}
\]
is all of $G_\K / G_\LL$, its kernel being those $w \in W_\K$ such that $w \in G_\LL$, i.e., its kernel is 
$G_\LL \cap W_\K$ or still, is $W_\LL$.

\vspace{0.2cm}

\begin{x}{\small\bf LEMMA} \ %20
\[
W_\K / W_\LL \ \approx \ G_\K / G_\LL \ \approx \ \Gal(\LL / \K).
\]
\end{x}
\vspace{0.1cm}

\begin{x}{\small\bf LEMMA} \ %21
$\ov{W_\LL^*}$ is a normal subgroup of $W_\K$.
\end{x}

\vspace{0.1cm}


%%----------------------------------------------------------------------------------------------06
[Bearing in mind that $W_\LL$ is a normal subgroup of $W_\K$, if $\alpha, \ \beta \in W_\LL^*$ and if $\gamma \in W_\K$, then 
\[
\gamma \alpha \beta \alpha^{-1} \beta^{-1} \gamma^{-1} \ = \ 
(\gamma \alpha \gamma^{-1}) (\gamma \beta \gamma^{-1}) 
(\gamma \alpha^{-1} \gamma^{-1}) (\gamma \beta^{-1} \gamma^{-1}).]
\]

\vspace{0.1cm}

There is an exact sequence 
\[
1 \lra 
W_\LL /\ov{W_\LL^*} \lra 
W_\K /\ov{W_\LL^*} \lra 
(W_\K /\ov{W_\LL^*})  / (W_\LL /\ov{W_\LL^*}) \lra 
1
\]
or still, there is an exact sequence
\[
1 \lra 
W_\LL /\ov{W_\LL^*} \lra 
W_\K /\ov{W_\LL^*} \lra 
W_\K   / W_\LL \lra 
1.
\]

\vspace{0.2cm}

\begin{x}{\small\bf NOTATION} \ %22
Put 
\[
W(\LL,\K) \ = \ W_\K /\ov{W_\LL^*}.
\]
\end{x}

\vspace{0.1cm}

\begin{x}{\small\bf SCHOLIUM} \ %23
There is an exact sequence
\[
1 \lra 
W_\LL^\ab \lra
W(\LL,\K) \lra 
W_\K /W_\LL \lra
1
\]
and a diagram
\[
\begin{tikzcd}[sep=large]
&{W_\LL^\ab}   \ar{r}
&{W(\LL,\K)} \ar{r}
&{W_\K / W_\LL} \ar{d}{\approx}\\
{1} \ar{r} &{\LL^\times} \ar{u}{\rec_\LL} &&{\Gal(\LL / \K)} \ar{r} &{1}
\end{tikzcd}
.
\]
\end{x}

\vspace{0.1cm}

\begin{x}{\small\bf NOTATION} \ %24
Given $w \in W_\K$, let $\norm{w}$ denote the effect on $w$ of passing
%%----------------------------------------------------------------------------------------------07
from $W_\K$ to $\R_{>0}^\times$ via the arrows 
\[
\begin{tikzcd}[sep=large]
{W_\K} \ar{r} 
&{W_\K^\ab} \ar{rr}{\rec_\K^{-1}}
&&{\K^\times} \ar{rr}{\mods_\K}
&&{\R_{>0}^\times}
\end{tikzcd}
.
\]
\end{x}

\vspace{0.1cm}

\begin{x}{\small\bf LEMMA} \ %25
$\norm{\cdot}:W_\K \ra \R_{>0}^\times$ is a continuous homomorphism and its kernel is $I_\K$.

[Under the arrow 
\[
W_\K \ra W_\K^\ab,
\]
$I_\K$ drops to 
\[
\Gal(\K^\ab / \K^\ur) \ \subset \ W_\K^\ab.
\]
Consider now the arrow
\[
\rec_\K:\K^\times \lra W_\K^\ab.
\]
Then $R^\times$ is sent to $\Gal(\K^\ab / \K^\ur)$ and a prime element $\pi \in R$ is sent to an element 
$\widetilde{\sigma}_q$ in $W_\K^\ab$ whose image in $W(\F_q^\ab / \F_q)$ is $\sigma_q$.  And 
\[
W_\K^\ab \ = \ \bigcup\limits_{n \in \Z} (\widetilde{\sigma}_q)^n \Gal(\K^\ab / \K^\ur).]
\]
\end{x}

\vspace{0.1cm}


\begin{x}{\small\bf DEFINITION} \ %26
A \un{representation} of $W_\K$ is a continuous homomorphism 
$\rho:W_\K \ra \GL(V)$, where $V$ is a finite dimensional complex vector space.
\end{x}

\vspace{0.1cm}


\begin{x}{\small\bf LEMMA} \ %27
A homomorphism 
$\rho:W_\K \ra \GL(V)$ 
is continuous per the usual topology on $\GL(V)$ iff it is continuous per the discrete topology on $\GL(V)$.

\vspace{0.1cm}

[$\GL(V)$ has no small subgroups.]
\end{x}

\vspace{0.1cm}
%%----------------------------------------------------------------------------------------------08

\begin{x}{\small\bf SCHOLIUM} \ %28
The kernel of every representation of $W_\K$ is trivial on an open subgroup $J$ of $I_\K$.  
Conversely, if $\rho:W_\K \ra \GL(V)$ is a homomorphism which is trivial on an open subgroup $J$ of $I_\K$, then 
the inverse image of any subset of $\GL(V)$ is a union of cosets of $J$, hence is open, hence $\rho$ is continuous, so by definition is a representation of $W_\K$.
\end{x}

\vspace{0.1cm}

\begin{x}{\small\bf EXAMPLE} \ %29
Suppose that $\LL / \K$ is a finite Galois extension of $\K$ $-$then 
\begin{align*}
W_\LL \cap I_\K \ 
&=\  G_\LL \cap W_\K \cap I_\K \\
&=\ G_\LL \cap I_\K
\end{align*}
is an open subgroup of $I_\K$.  But
\[
W_\K / W_\LL \ \approx \  \Gal(\LL / \K) \qquad (\text{cf.} \ \#20).
\]
Therefore every homomorphism $\Gal(\LL/\K) \ra \GL(V)$ lifts to a homomorphism $W_\K \ra \GL(V)$ which is trivial on an open subgroup of $I_\K$, hence is a representation of $W_\K$.
\end{x}

\vspace{0.1cm}

\begin{x}{\small\bf \un{N.B.}} \ %30
Representations of $W_\K$ arising in this manner are said to be of \un{Galois type}.
\end{x}

\vspace{0.1cm}

\begin{x}{\small\bf LEMMA} \ %31
A representation of $W_\K$ is of Galois type iff it has finite image.
\end{x}

\vspace{0.1cm}

\begin{x}{\small\bf EXAMPLE} \ %32
$\norm{\cdot}$ is a character of $W_\K$ but as a representation, is not of Galois type.
\end{x}

\vspace{0.1cm}

\begin{x}{\small\bf LEMMA} \ %33
Let $\rho:W_\K \ra \GL(V)$ be a representation $-$then the image $\rho(I_\K)$ is finite.
%%----------------------------------------------------------------------------------------------09

\vspace{0.1cm}

PROOF \ 
Suppose that $J$ is an open subgroup of $I_\K$ on which $\rho$ is trivial.  
Since $I_\K$ is compact and $J$ is open, the quotient $I_\K / J$ is finite, thus 
$\rho(I_\K) = \rho(I_\K / J)$ is finite.
\end{x}

\vspace{0.1cm}

\begin{x}{\small\bf DEFINITION} \ %34
A representation $\rho:W_\K \ra \GL(V)$ is \un{irreducible} if $V \neq 0$ 
and the only $\rho$-invariant subspaces are 0 and $V$.
\end{x}

\vspace{0.1cm}

\begin{x}{\small\bf THEOREM} \ %35
Given an irreducible representation $\rho$ of $W_\K$, there exists an irreducible representation $\widetilde{\rho}$ of $W_\K$ 
and a complex parameter $s$ such that $\rho \ \approx \ \widetilde{\rho} \ \un{\otimes} \ \norm{\cdot}^s$.
\end{x}

\vspace{0.1cm}

\begin{x}{\small\bf LEMMA} \ %36
Let $\rho:W_\K \ra \GL(V)$ be a representation $-$then $V$ is the sum of its irreducible $\rho$-invariant subspaces iff 
every $\rho$-invariant subspace has a $\rho$-invariant complement.
\end{x}

\vspace{0.1cm}

\begin{x}{\small\bf DEFINITION} \ %37
Let $\rho:W_\K \ra \GL(V)$ be a representation $-$then $\rho$ is \un{semisimple} if it satisfies either condition of the preceding lemma.
\end{x}

\vspace{0.1cm}

\begin{x}{\small\bf \un{N.B.}} \ %38
Irreducible representations are semisimple.
\end{x}

\vspace{0.1cm}

\begin{x}{\small\bf THEOREM} \ %39
Let $\rho:W_\K \ra \GL(V)$ be a representation $-$then the following conditions are equivalent

\qquad 1. \quad $\rho$ is semisimple.

\qquad 2. \quad $\rho(\widetilde{\sigma}_q)$ is semisimple.

\qquad 3. \quad $\rho(w)$ is semisimple $\forall \ w \in W_\K$.
\end{x}

\vspace{0.1cm}

%%%%%%%%%%%%%%%%%%%%%%%%%%%%%%%%%%%%%%
%%%%%%%%%%%%%%%%%%%%%%%%%%%%%%%%%%%%%%
%%%%%%%%%%%%%%%%%%%%%%%%%%%%%%%%%%%%%%






















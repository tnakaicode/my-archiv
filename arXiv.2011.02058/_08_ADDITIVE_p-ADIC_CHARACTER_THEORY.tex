\chapter{
$\boldsymbol{\S}$\textbf{8}.\quad  ADDITIVE p-ADIC CHARACTER THEORY}
\setlength\parindent{2em}
\setcounter{theoremn}{0}
%%----------------------------------------------------------------------------------------------01

%\ \indent 

\begin{x}{\small\bf FACT} \ %1
Every proper closed subgroup of $\T$ is finite.
\end{x}

\vspace{0.1cm}

Suppose that $G$ is compact abelian and totally disconnected.

\vspace{0.1cm}

\begin{x}{\small\bf LEMMA} \ %2
If $\chi \in \widehat{G}$, then the image $\chi(G)$ is a finite subgroup of $\T$.

\vspace{0.1cm}

PROOF \  $\ker \chi$ is closed and 
\[
\chi(G) \thickapprox G / \ker \chi.
\]
But the quotient $G / \ker \chi$ is 0-dimensional, hence totally disconnected.  
Therefore $\chi(G)$ is totally disconnected.  
Since $\T$ is connected, it follows that $\T \ne \chi(G)$, thus $\chi(G)$ is finite.
\end{x}

\vspace{0.1cm}

\begin{x}{\small\bf \un{N.B.}} \ %3
The torsion of $\R/\Z$ is $\Q/\Z$, so $\chi$ factors through the inclusion
\[
\Q/\Z \hookrightarrow \R/\Z, \quad \text{i.e., } \chi(G) \subset \Q/\Z.
\]
\end{x}

\vspace{0.1cm}

The foregoing applies in particular to $G = \Z_p$.

\vspace{0.1cm}

\begin{x}{\small\bf LEMMA} \ %4
Every character of $\Q_p$ is unitary.

\vspace{0.1cm}

PROOF \   This is because
\[
\Q_p \ = \  \bigcup_{n \in \Z} p^n \Z_p,
\]
where the $p^n \Z_p$ are compact, thus $\S7$, $\# 7$ is applicable.
\end{x}

\vspace{0.1cm}

\begin{x}{\small\bf LEMMA} \ %5
If $\chi \in \widehat{\Q}_p$ is nontrivial, then there exists an $n \in \Z$ such that $\chi \equiv 1$ on $p^n\Z_p$ but $\chi \not\equiv 1$ on $p^{n-1}\Z_p$.

\vspace{0.1cm}

PROOF \  Consider a ball $B$ of radius $\frac{1}{2}$ about 1 in 
$\C^\times$ $-$then the only subgroup of  $\C^\times$ contained in $B$ is the trivial subgroup and, by continuity, $\chi(p^n \Z_p)$ must be inside $B$ for all sufficiently large $n$, thus must be 
%%----------------------------------------------------------------------------------------------02
identically 1 there.
%\\
%$\hfill \blacksquare$
\end{x}
\vspace{0.1cm}

\begin{x}{\small\bf DEFINITION} \ %6
The 
\un{conductor}
\index{conductor} 
$\con \chi$ of a nontrivial $\chi \in \widehat{\Q}_p$ is the largest subgroup $p^n \Z_p$ 
on which $\chi$ is trivial (and $n$ is the minimal integer with this property).

\vspace{0.2cm}

A typical $x \ne 0$ of $\Q_p$ has the form
\begin{align*}
x \ 
&=\  \sum_{n = v(x)}^\infty a_n p^n \qquad ( a_n \in \sA, v(x) \in \Z) \\
&=\  f(x) + [x].
\end{align*}
Here the 
\un{fractional part}
\index{fractional part} 
f(x) of $x$ is defined by the prescription
\[
f(x) \ = \ 
\begin{cases}
\sum\limits_{n = v(x)}^{-1} a_np^n \quad \ \   \text{if} \  v(x) < 0\\
0  \qquad\qquad\qquad \text{if} \ v(x) \geq 0
\end{cases}
\]
and the 
\un{integral part} 
\index{integral part} 
$[x]$ of $x$ is defined by the prescription
\[
[x] = \sum_{n = 0 }^\infty a_np^n,
\]
with $f(0) = 0$, $[0] = 0$ by convention.
\end{x}
\vspace{0.1cm}

\begin{x}{\small\bf \un{N.B.}} \ %7
\[
f(x) \in \Z\bigl[\frac{1}{p}\bigr] \subset \Q,
\]
where
\[
\Z\bigl[\frac{1}{p}\bigr] = \{\frac{n}{p^k}: n \in \Z, k \in \Z\},
\]
%%----------------------------------------------------------------------------------------------03
while $[x] \in \Z_p$.
\end{x}
\vspace{0.1cm}


\begin{x}{\small\bf OBSERVATION} \ %8
\begin{align*}
0 \ 
&\leq \ f(x) \\
&= \ \sum\limits_{1 \le j \le -v(x)} \frac{a_{-j}}{p^j} \\
&<\  (p-1) \sum_{j = 1}^\infty \frac{1}{p^j}\\ 
&=\  1 
\end{align*}
\qquad\qquad\qquad\qquad$\implies$
\[
f(x) \in [0,1[ \ \cap \  \Z\bigl[\frac{1}{p}\bigr].
\]

Let $\mu_{p^\infty}$ stand for the group of roots of unity in $\C^\times$ having order a power of $p$, 
thus $\mu_{p^\infty}$ is a $p$-group and there is an increasing sequence of cyclic groups
\[
\begin{cases}
\mu_p \subset \mu_{p^2} \subset \cdots \subset \mu_{p^k} \subset  \cdots \\
\mu_{p^\infty} = \bigcup\limits_{k \ge 0}  \mu_{p^k} 
\end{cases}
,
\]
where
\[
 \mu_{p^k}  = \{z \in \C^\times : z^{p^k} = 1\}.
\]
\end{x}

\vspace{0.1cm}

\begin{x}{\small\bf REMARK} \ %9
Denote by $\mu$ the group of all roots of unity in $\C^\times$, hence
\[
\mu = \bigcup\limits_{m \ge 1}  \mu_m, \quad \mu_m = \{z \in \C^\times : z^m = 1\}.
\]
Then $\mu$ is an abelian torsion group and $\mu_{p^\infty}$ is the $p$-Sylow subgroup of $\mu$, 
i.e., the maximal $p$-subgroup of $\mu$.
\vspace{0.2cm}

%%----------------------------------------------------------------------------------------------04
Put
\[
\chi_p(x) = \exp(2\pi \sqrt{-1} \ f(x)) \qquad ( x \in \Q_p).
\]
Then
\[
\chi_p: \Q_p \ra \T
\]
and $\Z_p \subset \ker \chi_p$.
\end{x}

\vspace{0.1cm}

\begin{x}{\small\bf EXAMPLE} \ %10
Suppose that $v(x) = -1$, so $x = \ds\frac{k}{p} + y$ with $0 < k \le p-1$ and $y \in \Z_p:$
\[
\chi_p(x) = \exp(2\pi \sqrt{-1} \ \frac{k}{p}) = \zeta^k,
\]
where $\zeta = \exp(2\pi \sqrt{-1}/p)$ is a primitive $p^{th}$ root of unity.
\end{x}

\vspace{0.1cm}

\begin{x}{\small\bf LEMMA} \ %11
$\chi_p$ is a unitary character

\vspace{0.1cm}

PROOF \  Given $x, y \in \Q_p$, write
\begin{align*}
f(x+y) - f(x) - f(y) \ 
&=\  x + y - [x+y] - (x - [x]) - (y - [y]) \\
&=\  [x] + [y] - [x+y] \in \Z_p.
\end{align*}
But at the same time
\[
f(x+y) - f(x) - f(y) \in \Z\bigl[\frac{1}{p}\bigr].
\]
Thus
\[
f(x+y) - f(x) - f(y) \in \Z\bigl[\frac{1}{p}\bigr] \cap \Z_p = \Z
\]
and so
\[
\exp(2\pi \sqrt{-1} \ (f(x + y) - f(x) - f(y)) = 1
\]
%%----------------------------------------------------------------------------------------------05
or still, 
\[
\chi_p(x + y) = \chi_p(x) \chi_p(y).
\]
Therefore $\chi_p : \Q_p \ra \T$ is a homomorphism.  
As for continuity, it suffice to check this at 0, matters then being clear $($since $\chi_p$ is trivial in a neighborhood of 0$)$ 
$(\Z_p$ is open and $0 \in  \Z_p)$.
\end{x}

\vspace{0.1cm}

\begin{x}{\small\bf LEMMA} \ %12
The kernel of $\chi_p$ is $\Z_p$.

\vspace{0.1cm}

[A priori, the kernel of $\chi_p$ consists of those $x \in \Q_p$ such that $f(x) \in \Z$.  
Therefore
\[
\con  \chi_p = \Z_p.]
\]
\end{x}

\vspace{0.1cm}

\begin{x}{\small\bf LEMMA} \ %13 
The image of $\chi_p$ is $\mu_{p^\infty}$.

\vspace{0.1cm}

[A priori, the image of $\chi_p$ consists of the complex numbers of the form
\[
\exp( 2\pi \sqrt{-1} \ \frac{k}{p^m}) \ = \ \exp(2\pi \sqrt{-1}/p^m)^k.
\]
Since $\exp(2\pi \sqrt{-1}/p^m)$ is a root of unity of order $p^m$, these roots generate $\mu_{p^\infty}$ 
as $m$ ranges over the positive integers.]
\end{x}

\vspace{0.1cm}

\begin{x}{\small\bf SCHOLIUM} \ %14
$\chi_p$ implements an isomorphism
\[
\Q_p/\Z_p \thickapprox \mu_{p^\infty}.
\]
\end{x}

\vspace{0.1cm}

\begin{x}{\small\bf REMARK} \ %15
%%----------------------------------------------------------------------------------------------06
\begin{align*}
x \in p^{-k}\Z_p	&\Leftrightarrow p^{k}x \in \Z_p\\	
								&\Leftrightarrow \chi_p(p^{k}x) = 1\\
								&\Leftrightarrow \chi_p(x)^{p^k} = 1\\
								&\Leftrightarrow \chi_p(x) \in \mu_{p^k}.
\end{align*}
\end{x}

\vspace{0.1cm}

\begin{x}{\small\bf RAPPEL} \ %16
Let $p$ be a prime $-$then a group is 
\un{$p$-primary}
\index{$p$-primary group} 
if every element has order a power of $p$.
\end{x}
\vspace{0.1cm}

\begin{x}{\small\bf RAPPEL} \ %17
Every abelian torsion group $G$ is a direct sum of its $p$-primary subgroups $G_p$.
\end{x}

\vspace{0.1cm}

[Note: \ The $p$-primary component of $G_p$ is the $p$-Sylow subgroup of $G$.]

\vspace{0.1cm}

\begin{x}{\small\bf NOTATION} \ %18
$\Z(p^\infty)$ is the $p$-primary component of  $\Q /  \Z$.  

Therefore
\[
\Q / \Z \ \thickapprox \ \bigoplus_p \ \Z(p^\infty).
\]
\end{x}
\vspace{0.1cm}

\begin{x}{\small\bf LEMMA} \ %19
$\Z(p^\infty)$ is isomorphic to $\mu_{p^\infty}$.

\vspace{0.1cm}

[$\Z(p^\infty)$ is generated by the $1/p^n$ in $\Q /  \Z$.]

\vspace{0.2cm}

Therefore
\[
\Q /  \Z \ \thickapprox \ \bigoplus_p \  \mu_{p^\infty} \ \thickapprox \ \bigoplus_p \  \Q_p /  \Z_p.
\]

[Note: Consequently, 
\begin{align*}
\End(\Q / \Z)	\ 
&\thickapprox \ \End \bigl(\bigoplus\limits_p \ \Q_p /  \Z_p\bigr)\\	
&\thickapprox \ \prod\limits_p \End(\Q_p /  \Z_p)\\	
&\thickapprox \ \prod \limits_p \Z_p.]
\end{align*}
\end{x}

\vspace{0.1cm}
%%----------------------------------------------------------------------------------------------07 ish


\begin{x}{\small\bf REMARK} \ %20
$\widehat{\Z}_p$ is isomorphic to $\mu_{p^\infty}$ (c.f. $\# 26$ infra).
\end{x}

\vspace{0.1cm}


Given $t \in \Q_p$, let $L_t$ be left multiplication by $t$ and put $\chi_{p,t} = \chi_p \circ L_t$ 
$-$then $\chi_{p,t}$ is continuous and $\forall \ x \in \Q_p$,
\[
\chi_{p,t}(x) = \chi_p(tx).
\]

[Note: Trivially, $\chi_{p,0} \equiv 1$.  And $\forall$ $t \ne 0$, 
\[
\con \chi_{p,t} = p^{-v(t)}\Z_p.
\]

Proof: \ 
\begin{align*}
x \in \text{ con }\chi_{p,t}\ 	
&\Leftrightarrow\  tx \in  \Z_p\\			
&\Leftrightarrow\  \abs{tx}_p \le 1\\
&\Leftrightarrow\  \abs{x}_p \le \frac{1}{\abs{t}_p} = p^{v(t)}\\
&\Leftrightarrow\  x \in p^{-v(t)} \Z_p.]	
\end{align*}
Next
\begin{align*}
\chi_{p,t}(x+y)	\  
&=\  \chi_p(t(x+y))\\	
&=\  \chi_p(tx+ty)\\	
&=\  \chi_p(tx)\chi_p(ty)\\		
&=\   \chi_{p,t}(x)\chi_{p,t}(y).
\end{align*}
%%----------------------------------------------------------------------------------------------08
Therefore $\chi_{p,t} \in \widehat{\Q}_p$.

Next
\begin{align*}
\chi_{p,t+s}(x)	\ 
&= \ \chi_p((t+s)x)\\	
&= \ \chi_p(tx+sx)\\	
&= \ \chi_p(tx)\chi_p(sx)\\		
&= \ \chi_{p,t}(x)\chi_{p,s}(x).
\end{align*}
Therefore the arrow
\begin{align*}
\Xi_p:\Q_p	&\ra \widehat{\Q}_p\\	
t &\mapsto \chi_{p,t}
\end{align*}
is a homomorphism.

\vspace{0.2cm}
											
\begin{x}{\small\bf LEMMA} \ %21
If $t\ne s$, then $\chi_{p,t} \ne \chi_{p,s}$.

\vspace{0.1cm}

PROOF \  If to the contrary, $\chi_{p,t} = \chi_{p,s}$, then $\forall$ $x \in \Q_p$, $\chi_p(tx) = \chi_p(sx)$ or still, 
$\forall$ $x \in \Q_p$, $\chi_p((t-s)x) = 1$. 
But $L_{t-s}: \Q_p \ra \Q_p$ is an automorphism, hence $\chi_p$ is trivial, which it isn't.
\end{x}

\vspace{0.1cm}

\begin{x}{\small\bf LEMMA} \ %22
The set
\[
\Xi_p(\Q_p) \ = \  \{\chi_{p,t}: t \in \Q_p\}
\]
is dense in $\widehat{\Q}_p$.

\vspace{0.1cm}

PROOF \  Let $H$ be the closure in $\widehat{\Q}_p$ of the $\chi_{p,t}$.  
Consider the quotient $\widehat{\Q}_p/H$.  
To get a contradiction, assume that $H \ne\widehat{\Q}_p$, 
thus that there is a nontrivial $\xi \in \widehat{\widehat{\Q}}_p$ which is trivial on $H$.   
By definition, $H^\perp$ is computed in $\widehat{\widehat{\Q}}_p$, which by Pontryagin duality, is 
%%----------------------------------------------------------------------------------------------09
identified with $\Q_p$, so spelled out
\[
H^\perp = \{x \in \Q_p: \ev_{\Q_p}\restr{(x)}{H} = 1\}.
\]
Accordingly, for some $x$, $\xi=  \ev_{\Q_p}(x)$, hence $\forall \ t$, 
\begin{align*}
\xi(\chi_{p,t}) \ 
&=\  \ev_{\Q_p}(x)(\chi_{p,t})  \\
&=\  \chi_{p,t}(x) \\
&=\  \chi_p(tx) \\
&= 1,
\end{align*}
which is possible only if $x = 0$ and this implies that $\xi$ is trivial.
\end{x}

\vspace{0.1cm}

\begin{x}{\small\bf LEMMA} \ %23
The arrows
\[
\begin{cases}
\Q_p	\ra \Xi_p(\Q_p)\\	
\Xi_p(\Q_p)\ra  \Q_p\
\end{cases}
\]
are continuous.
\end{x}


Therefore $\Xi(\Q_p)$ is a locally compact subgroup of $\widehat{\Q}_p$.  
But a locally compact subgroup of a locally compact group is closed.  
Therefore $\Xi_p(\Q_p) = \widehat{\Q}_p$. 

In summary$:$
\begin{x}{\small\bf THEOREM} \ %24
$\widehat{\Q}_p$ is topologically isomorphic to $\Q_p$ via the arrow
\[
\Xi_p:\Q_p		\ra  \widehat{\Q}_p.
\]
\end{x}

\vspace{0.1cm}

\begin{x}{\small\bf LEMMA} \ %25
Fix $t$ $-$then $\restr{\chi_{p,t}}{\Z_p} = 1$ iff $t \in \Z_p$.

\vspace{0.1cm}

PROOF \  Recall that the kernel of $\chi_p$ is $\Z_p$.
%%----------------------------------------------------------------------------------------------10 ish
\[
\begin{aligned}
&\text{\textbullet} \quad t \in \Z_p, \ x \in \Z_p \implies tx \in \Z_p \implies \chi_p(tx) = 1 \implies \restr{\chi_{p,t}}{\Z_p} = 1.\\
&\text{\textbullet} \quad \restr{\chi_{p,t}}{\Z_p} = 1 \implies \chi_{p,t}(1) = 1 \implies \chi_p(t) = 1 \implies t \in \Z_p.
\end{aligned}
\]
\end{x}

\vspace{0.1cm}

\begin{x}{\small\bf APPLICATION} \ %26
$\widehat{\Z}_p$ is isomorphic to $\mu_{p^\infty}$.

\vspace{0.1cm}

$[\widehat{\Z}_p$ can be computed as $\widehat{\Q}_p/\Z_p^\perp$.  
But $\Z_p^\perp$, when viewed as a subset of $\Q_p$, consists of those $t$ such that $\restr{\chi_{p, t}}{\Z_p} = 1.$ 
Therefore 
\[
\widehat{\Z}_p \ 
\thickapprox \ \widehat{\Q}_p/\Z_p \ 
\thickapprox \ \Q_p /\Z_p \ 
\thickapprox \ \mu_{p^\infty}.]
\]
\end{x}

\vspace{0.1cm}


\begin{x}{\small\bf NOTATION} \ %27
Let
\[
x_\infty(x) = \exp(-2\pi \sqrt{-1} \  x) \qquad (x \in \R).
\]
\end{x}

\vspace{0.1cm}

\index{Product principle}
\begin{x}{\small\bf PRODUCT PRINCIPLE} \ %28
$\forall$ $x \in \Q$,
\[
\prod_{p \le \infty} \chi_p(x) \ = \ 1.
\]

\vspace{0.1cm}

PROOF \  Take $x$ positive $-$then there exist primes $p_1, \cdots, p_n$ such that $x$ admits a representation
\[
x \ = \ 
\frac{N_1}{p_1^{\alpha_1}} + 
\frac{N_2}{p_2^{\alpha_2}} + \cdots + 
\frac{N_n}{p_n^{\alpha_n}} + M,
\]
where the $\alpha_k$ are positive integers, the $N_k$ are positive integers 
$(1 \le N_k < p_k^{\alpha_k} - 1)$, and $M \in \Z$.  
Appending a subscript to $f$, we have
\[
f_{p_k}(x) \ = \ 
\frac{N_k}{p_k^{\alpha_k}}, \quad f_p(x) = 0 \quad (p \ne p_k, \ k = 1, 2, \ldots, n).
\]
Therefore
%%----------------------------------------------------------------------------------------------11 ish
\begin{align*}
\prod_{p < \infty} \chi_p(x) \ 	
&= \  \prod_{1 \le k \le n} \chi_{p_k}(x)\\	
&= \  \prod_{1 \le k \le n} \exp(2\pi\sqrt{-1}\ 	 f_{p_k}(x))\\
&= \  \exp(2\pi\sqrt{-1}\ 	 \sum_{k = 1}^n f_{p_k}(x))\\
&= \  \exp(2\pi\sqrt{-1}\ 	 (x - M))\\
&= \  \exp(2\pi\sqrt{-1}\ 	 x)
\end{align*}
\qquad\qquad$\implies$
\begin{align*}
\prod_{p \le \infty} \chi_p(x) \ 	
&=\ \prod_{p < \infty} \chi_p(x) \chi_\infty (x) \\	
&=\ \exp(2\pi\sqrt{-1}\ 	 x) \exp(-2\pi\sqrt{-1}\ 	 x)\\							
&=\ 1.
\end{align*}
\end{x}

\vspace{0.1cm}
%%%%%%%%%%%%%%%%%%%
%%%%%

\[
\textbf{APPENDIX}
\]
\setcounter{theoremn}{0}

Let $\K$ be a finite extension of $\Q_p$.
\vspace{0.25cm}

\begin{x}{\small\bf THEOREM} \ %1
The topological groups $\K$ and $\widehat{\K}$ are topologically isomorphic.

\vspace{0.1cm}

[Put
\begin{align*}
\chi_{\K,p}(a) \ 
&=\  \exp(2\pi \sqrt{-1} \ f(\tr_{\K/\Q_p}(a)))\\
&=\ \chi_p(\tr_{\K/\Q_p}(a))
\end{align*}
and given $b \in \K$, put
\[
\chi_{\K,p,b}(a) = \chi_{\K,p}(ab).
\]
Proceed from here as above.]
\end{x}

\vspace{0.1cm}
%%----------------------------------------------------------------------------------------------12

\begin{x}{\small\bf REMARK} \ %2
Every character of $\K$ is unitary.
\end{x}

\vspace{0.1cm}

\begin{x}{\small\bf LEMMA} \ %3
\[
\begin{cases}
\ a \in R 	&\implies \tr_{\K/ \Q_p}(a) \in \Z_p\\	
\ a \in P &\implies \tr_{\K/ \Q_p}(a) \in p\Z_p	
\end{cases}
.
\]
\end{x}

\vspace{0.1cm}

\begin{x}{\small\bf DEFINITION} \ %4
The 
\un{differential of $\K$}
\index{differential of $\K$} 
is the set
\[
\Delta_\K = \{b \in \K:\tr_{\K/ \Q_p}(R b) \subset \Z_p\}.
\]
\end{x}

\vspace{0.1cm}

\begin{x}{\small\bf LEMMA} \ %5
$\Delta_\K$ is a proper $R$-submodule of $\K$ containing $R$.
\end{x}

\vspace{0.1cm}

\begin{x}{\small\bf LEMMA} \ %6
There exists a unique nonnegative integer $d$ 
$-$\un{the differential exponent} \un{of $\K$}
\index{the differential exponent of $\K$} 
$-$characterized by the condition that
\[
\pi^{-d}R = \Delta_\K.
\]

[This follows from the theory of "fractional ideals" $($details omitted$)$.]
\end{x}

\vspace{0.1cm}

[Note: $\chi_{\K,p}$ is trivial on $\pi^{-d}R$ but is nontrivial on $\pi^{-d-1}R.]$

\vspace{0.2cm}

\begin{x}{\small\bf LEMMA} \ %7
Let $e$ be the ramification index of $\K$ over $\Q_p$ (cf. $\S5$, $\#17)$ $-$ then
\[
a \in P^{-e+1} \implies \tr_{\K/ \Q_p}(a) \in \Z_p.
\]

\vspace{0.1cm}

PROOF \  Let
\[
a \in P^{-e+1} = \pi^{-e+1}R = \pi^{-e}(\pi R) =  \pi^{-e}P,
\]
%%----------------------------------------------------------------------------------------------13
so $a = \pi^{-e}b \ (b \in P)$.  
Write $p = \pi^eu$ and consider $pa$:
\[
pa = \pi^eu\pi^{-e}b = ub.
\]
But
\begin{align*}
\abs{u} = 1, \  \abs{b} < 1 	\ 
&\implies\  \abs{ub} < 1\\	
&\implies\  ub \in P\\
&\implies\  \tr_{\K/ \Q_p}(ub) \in p\Z_p\\
&\implies\  \tr_{\K/ \Q_p}(pa) \in p\Z_p\\
&\implies\  p\tr_{\K/ \Q_p}(a) \in p\Z_p\\
&\implies\  \tr_{\K/ \Q_p} \in \Z_p.
\end{align*}
\end{x}

\vspace{0.1cm}

\begin{x}{\small\bf APPLICATION} \ %8
\[
d \ge e-1.
\]

[It suffices to show that
\[
P^{-e+1} \subset \Delta_\K \quad  (\equiv \pi^{-d}R).
\]
Thus let $a \in  P^{-e+1}$, say $a = \pi^eb$  $(b \in P)$, and let $r \in R$ $-$then the claim is that
\[
\tr_{\K/ \Q_p}(ar) \in \Z_p.
\]
But
\[
ar = \pi^{-e}br \in \pi^e P \quad (\abs{br} < 1)
\]
or still,
\[
ar \in P^{-e+1} \implies \tr_{\K/ \Q_p}(ar) \in \Z_p.]
\]
\end{x}

\vspace{0.1cm}
%%----------------------------------------------------------------------------------------------14

\begin{x}{\small\bf REMARK} \ %9
Therefore $d = 0 \implies e = 1$, hence in this situation, $\K$ is unramified.

[Note: There is also a converse, viz. if $\K$ is unramified, then $d = 0.$]
\end{x}

\vspace{0.1cm}

\begin{x}{\small\bf \un{N.B.}} \ %10
It can be shown that
\[
\tr_{\K/ \Q_p}(R) = \Z_p \ \text{ iff } \ d = e-1.
\]
\end{x}

\vspace{0.1cm}

\begin{x}{\small\bf CRITERION} \ %11
Fix $b \in \K$ $-$then
\[
b \in \Delta_\K \Leftrightarrow \forall \  a \in R, \ \chi_{\K,p}(ab) = 1.
\]

\vspace{0.1cm}

PROOF \ 

\indent\indent \textbullet \quad 
$a \in R, b \in \Delta_\K  \ \implies\  ab \in \Delta_\K$

$\indent\indent\indent\indent\indent\implies\  \tr_{\K/ \Q_p}(ab) \in \Z_p$

$\indent\implies$
\[
\chi_{\K,p}(ab) = \chi_p(\tr_{\K/ \Q_p}(ab)) = 1.
\]
\indent\indent\textbullet \quad 
$\forall \ a \in R, \ \chi_{\K,p}(ab) = 1$ 

$\indent\implies\ \forall \ a \in R, \  \tr_{\K/ \Q_p}(ab) \in \Z_p$

$\indent\implies\  b \in \Delta_\K$.

Normalize Haar measure on $\K$ by the condition
\[
\mu_\K(R) = \int_{R} da = q^{-d/2}.
\]
Let $\chi_R$ be the characteristic function of $R$ $-$then
%%----------------------------------------------------------------------------------------------15
\[
\int_\K \chi_R(a) \chi_{\K,p}(ab) da = \int_R \chi_{\K,p}(ab)da.
\]

\indent\indent\textbullet \quad 
$b \in \Delta_\K \implies \chi_{\K,p}(ab) = 1 \quad (\forall$ $a \in R)$

$\hspace{2.65cm} \implies \int_R\chi_{\K,p}(ab)da = \mu_\K(R) = q^{-d/2}$.

\indent\indent\textbullet \quad 
$b \notin \Delta_\K \implies \chi_{\K,p}(ab) \ne 1$ $(\exists$ $a \in R)$  

$\hspace{2.65cm}\implies \int_R\chi_{\K,p}(ab)da = 0$.


\text{}\\
Consequently, as a function of $b$,
\[
\int_R\chi_{\K,p}(ab)da =q^{-d/2} \chi_{\Delta_\K} (b),
\]
$\chi_{\Delta_\K}$ the characteristic function of $\Delta_\K$.
\end{x}

\vspace{0.1cm}

\begin{x}{\small\bf LEMMA} \ %12
\[
[\pi^{-d}R:R] = q^d.
\]
Therefore
\begin{align*}
\mu_\K(\Delta_\K) \ 
&=\  \mu_\K(\pi^{-d}R)\\	
&=\  q^d\mu_\K(R)\\
&= q^d q^{-d/2}\\								
&=\  q^{d/2}.
\end{align*}
\end{x}

\vspace{0.1cm}

\begin{x}{\small\bf LEMMA} \ %13
$\forall$ $a \in \K$,
\[
\int_\K q^{-d/2} \chi_{\Delta_\K} (b) \chi_{\K,p} (ab) db = \chi_R(a).
\]

\vspace{0.1cm}

%%----------------------------------------------------------------------------------------------16
PROOF \  The left hand side reduces to
\[
q^{-d/2} \int_{\Delta_\K} \chi_{\K,p} (ab) db
\]
and there are two possibilities

\indent\indent\textbullet \quad 
$a \in R  \implies ab \in \Delta_\K \quad (\forall \ b \in \Delta_\K)$

$\hspace{2.4cm}\implies \tr_{\K/ \Q_p}(ab) \in \Z_p$

$\hspace{2.4cm}\implies \chi_{\K,p}(ab) = 1$

$\indent\implies$
\begin{align*}
q^{-d/2} \int_{\Delta_\K} \chi_{\K,p}(ab)db \ 
&=\  q^{-d/2} \mu_\K(\Delta_\K) \\
&=\  q^{-d/2} q^{d/2}  \\
&=\  1.
\end{align*}


\indent\indent\textbullet \quad 
$a \notin R: \chi_{K,p}(ab) \ne 1 \quad (\exists \  b \in \Delta_\K)$

$\qquad\qquad\implies$
\[
q^{-d/2} \int_{\Delta_\K} \chi_{\K,p}(ab)db = 0.
\]

%$\hspace{2.65cm} q^{-d/2} \int_{\Delta_\K} \chi_{\K,p}(ab)db = 0$.
%\vspace{0.1cm}

To detail the second point of this proof, work with the normalized absolute value 
(cf. $\S 6, \ \# 18$) and recall that 
$\abs{\pi}_K = \ds\frac{1}{q}$ 
(cf. $\S 5, \ \# 21$).  
Accordingly, 
\[
x \in \pi^n R \Leftrightarrow \abs{x}_\K \le q^{-n}.
\]
Fix $a \notin R$ $-$then the claim is that $b \ra \chi_{\K,p}(ab)$ $(b \in \Delta_\K)$ is nontrivial.  
For
%%----------------------------------------------------------------------------------------------17 ish
\begin{align*}
\chi_{\K,p}(ab) = 1 	\ 
&\Leftrightarrow\  ab \in \pi^{-d}R\\
&\Leftrightarrow\  \abs{ab}_\K \le q^d\\
&\Leftrightarrow\  \abs{a}_\K \abs{b}_\K \le q^d\\
&\Leftrightarrow\   \abs{b}_\K \le \frac{q^d}{\abs{a}_\K} = q^{d+v(a)}.
\end{align*}
But
\begin{align*}
a \notin R 	
&\implies\  v(a) < 0\\
&\implies\  -v(a) > 0\\
&\implies\  -d-v(a) > -d\\
&\implies\ \pi^{-d-v(a)}R \subsetneqq \pi^{-d}R,
\end{align*}
a proper containment.
\end{x}

%%%%%%%%%%%%%%%%%%%%%%%%%%%%%%%%%%%%%%
%%%%%%%%%%%%%%%%%%%%%%%%%%%%%%%%%%%%%%
%%%%%%%%%%%%%%%%%%%%%%%%%%%%%%%%%%%%%%






















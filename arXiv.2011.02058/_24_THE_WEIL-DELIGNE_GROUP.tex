\chapter{
$\boldsymbol{\S}$\textbf{24}.\quad  THE WEIL-DELIGNE GROUP}
\setlength\parindent{2em}
\setcounter{theoremn}{0}
%%----------------------------------------------------------------------------------------------01

%\ \indent 

\begin{x}{\small\bf DEFINITION} \ %01
\  The 
\un{Weil-Deligne}
\index{Weil-Deligne group} 
group \ $WD_\K$ is the semidirect product \  $\C \rtimes W_\K$, the multiplication rule being 
\[
(z_1,w_1) \ (z_2,w_2) \ = \ (z_1  + \norm{w_1} z_2, w_1 w_2).
\]

[Note: \  The identity in $WD_\K$ is $(0,e)$ and the inverse of $(z,w)$ is $(-\norm{w}^{-1}z,w^{-1})$: 
\begin{align*}
(z,w) (-\norm{w}^{-1}z, w^{-1})  \ 
&=\  (z + \norm{w} (-\norm{w}^{-1}z),w w^{-1})\\
&=\ (z - z,e) \\
&=\  (0,e).]
\end{align*}
\end{x}

\vspace{0.1cm}

\begin{x}{\small\bf \un{N.B.}} \ %02
The topology on $WD_\K$ is the product topology.
\end{x}

\vspace{0.1cm}

\begin{x}{\small\bf DEFINITION} \ %03
A 
\un{Deligne representation}
\index{Deligne representation} 
of $W_\K$ is a triple $(\rho,V,N)$, where $\rho:W_\K \ra \GL(V)$ is a representation of $W_\K$ 
and $N:V \ra V$ is a nilpotent endomorphism of $V$ subject to the relation 
\[
\rho(w) N \rho(w)^{-1} \ = \ \norm{w} N \qquad (w \in W_\K).
\]

[Note: \ $N = 0$ is admissible so every representation of $W_\K$ is a Deligne representation.]
\end{x}

\vspace{0.1cm}

\begin{x}{\small\bf EXAMPLE} \ %04
Take $V = \C^n$, hence $\GL(V) = \GL_n(\C)$.  
Let $e_0, e_1, \ldots, e_{n-1}$ be the usual basis of $V$.  Define $\rho$ by the rule
\[
\rho(w) e_i \ = \ \norm{w}^i e_i \qquad (w \in W_\K, \ 0 \leq i \leq n-1)
\]
and define $N$ by the rule 
\[
N e_i \ = \ e_{i+1} \quad (0 \leq i \leq n -2), \quad N e_{n-1} = 0.
\]
%%----------------------------------------------------------------------------------------------02
Then the triple $(\rho, V, N)$ is a Deligne representation of $W_\K$, the 
\un{$n$-dimensional special} \un{representation}, 
\index{$n$-dimensional special representation}
denoted $\spp(n)$.
\end{x}

\vspace{0.1cm}

\begin{x}{\small\bf DEFINITION} \ %05
A 
\un{representation} 
\index{representation} 
of $WD_\K$ is a continuous homomorphism 
$\rho^\prime:WD_\K \ra \GL(V)$ whose restriction to $\C$ is complex analytic, where $V$ is a finite dimensional complex vector space.
\end{x}

\vspace{0.1cm}

\begin{x}{\small\bf LEMMA} \ %06
Every Deligne representation $(\rho,V,N)$ of $W_\K$ gives rise to a representation 
$\rho^\prime:WD_\K \ra \GL(V)$ of $WD_\K$.

\vspace{0.1cm}

PROOF \ 
Put
\[
\rho^\prime(z,w) \ = \ \exp(z N) \rho(w).
\]
Then
\begin{align*}
\rho^\prime(z_1,w_1) \rho^\prime(z_2,w_2)   \ 
&=\vsy\  \exp(z_1 N) \rho(w_1) \exp(z_2 N) \rho(w_2) \\
&=\vsy\  \exp(z_1 N) \rho(w_1) \exp(z_2 N) \rho(w_1^{-1}) \rho(w_1)  \rho(w_2) \\
&=\vsy\  \exp(z_1 N) \exp(z_2 \norm{w_1} N) \rho(w_1 w_2)\\
&=\vsy\  \exp(z_1 N + z_2 \norm{w_1} N) \rho(w_1 w_2)\\
&=\vsy\  \exp((z_1 + \norm{w_1} z_2) N) \rho(w_1 w_2)\\
%%----------------------------------------------------------------------------------------------03
&=\vsy\  \rho^\prime(z_1 + \norm{w_1} z_2, w_1 w_2) \\
&=\vsy\  \rho^\prime((z_1, w_1)(z_2, w_2)).
\end{align*}

[Note: \ 
The continuity of $\rho^\prime$ is manifest as is the complex analyticity of its restriction to $\C$.]
\end{x}
\vspace{0.1cm}


One can also go the other way but this is more involved.

\vspace{0.2cm}


\begin{x}{\small\bf RAPPEL} \ %07
If $T:V \ra V$ is unipotent, then 
\[
\log T \ = \ \sum\limits_{n \geq 1} \frac{(-1)^{n+1}}{n} \ (T - I)^n
\]
is nilpotent.
\end{x}
\vspace{0.1cm}

\begin{x}{\small\bf SUBLEMMA} \ %08
Let $\rho^\prime:WD_\K \ra \GL(V)$ be a representation of $WD_\K$ $-$then $\forall \ z \neq 0$, $\rho^\prime(z,e)$ is unipotent.
\end{x}

\vspace{0.1cm}

\begin{x}{\small\bf SUBLEMMA} \ %09
Let $\rho^\prime:WD_\K \ra \GL(V)$ be a representation of $WD_\K$ $-$then $\forall \ z \neq 0$, 
\[
\log \rho^\prime (z,e)
\]
is nilpotent and 
\[
(\log \rho^\prime (z,e)) / z \qquad (z \neq 0)
\]
is independent of $z$.
\end{x}
\vspace{0.1cm}

\begin{x}{\small\bf LEMMA} \ %10
Every representation $\rho^\prime:WD_\K \ra \GL(V)$ of $WD_\K$ gives rise to a Deligne representation 
$(\rho,V,N)$ of $W_\K$.

\vspace{0.1cm}

PROOF \ 
Put 
\[
\rho \ = \ \restr{\rho^\prime}{\{0\}} \times W_{\K}, \  N \ = \ \log \rho^\prime (1,e).
\]
%%----------------------------------------------------------------------------------------------04
Then $\forall \ w \in W_\K$, 

\allowdisplaybreaks
\begin{align*}
\rho(w) N \rho(w)^{-1}   \ 
&=\vsy\  \rho(w) \log \rho^\prime (1,e) \rho(w)^{-1}\\
&=\vsy\  \rho(w) \bigl( \sum\limits_{n \geq 1} \frac{(-1)^{n+1}}{n} \ (\rho^\prime(1,e) - I)^n\bigr) \rho(w)^{-1}\\
&=\vsy\ \sum\limits_{n \geq 1} \frac{(-1)^{n+1}}{n} (\rho(w) \rho^\prime(1,e) \rho(w)^{-1} - I)^n.
\end{align*}
And 
\allowdisplaybreaks
\begin{align*}
\rho(w) \rho^\prime(1,e) \rho(w)^{-1} \ 
&=\vsy\ \rho^\prime(0,w) \rho^\prime(1,e) \rho^\prime(0,w^{-1})\\
&=\vsy\ \rho^\prime ((0,w)(1,e)(0,w^{-1}))\\
&=\vsy\ \rho^\prime ((\norm{w},w)(0,w^{-1}))\\
&=\vsy\ \rho^\prime(\norm{w},e).
\end{align*}
Therefore
\allowdisplaybreaks
\begin{align*}
\rho(w) N \rho(w)^{-1} \ 
&=\vsy\ \sum\limits_{n \geq 1} \frac{(-1)^{n+1}}{n} (\rho^\prime(\norm{w},e)  - I)^n\\
&=\vsy\ \log \rho^\prime(\norm{w},e)\\
&=\vsy\ \norm{w} \log \rho^\prime (\norm{w}, e)) /  \norm{w}\\
&=\vsy\ \norm{w} \log \rho^\prime (1,e)\\
&=\vsy\ \norm{w} N.
\end{align*}

\end{x}

\vspace{0.1cm}
%%----------------------------------------------------------------------------------------------05
\begin{x}{\small\bf OPERATIONS} \ %11

\vspace{0.3cm}

\index{Deligne representation, direct sum}
\qquad \textbullet \quad \un{Direct Sum}: \ Let $(\rho_1,V_1,N_1)$, $(\rho_2,V_2,N_2)$ be Deligne representations 
$-$then their direct sum is the triple 
\[
(\rho_1 \oplus \rho_2,V_1 \oplus V_2 ,N_1 \oplus N_2).
\]

\index{Deligne representation, tensor product}
\qquad \textbullet \quad \un{Tensor Product}: \ Let $(\rho_1,V_1,N_1)$, $(\rho_2,V_2,N_2)$ be Deligne representations 
$-$then their tensor product is the triple
\[
(\rho_1 \  \otimes \  \rho_2,V_1 \  \otimes \ V_2 ,N_1 \ \otimes \  I_2 + I_1 \  \otimes\  \ N_2).
\]

\index{Deligne representations, contragredient}
\qquad \textbullet \quad \un{Contragredient}: \ Let $(\rho,V,N)$ be a Deligne representation $-$then its contragredient is the triple 
\[
(\rho^\vee, V^\vee, -N^\vee).
\]

[Note: \ 
$V^\vee$ is the dual of $V$ and $N^\vee$ is the transpose of $N$ (thus $\forall \ f \in V^\vee$, 
$N^\vee(f) = f \circ N$).]
\end{x}

\vspace{0.1cm}

\begin{x}{\small\bf REMARK} \ %12
The definitions of $\oplus$, $\otimes$, $\vee$ when transcribed to the ``prime picture'' are the usual 
representation-theoretic formalities applied to the group $WD_\K$.  
\end{x}

\vspace{0.1cm}

\begin{x}{\small\bf \un{N.B.}} \ %13
Let 
\[
\begin{cases}
\ (\rho_1,V_1,N_1)\\
\ (\rho_2,V_2,N_2)
\end{cases}
\]
be Deligne representations of $W_\K$ $-$then a morphism 
\[
(\rho_1,V_1,N_1) \ra (\rho_2,V_2,N_2)
\]
%%----------------------------------------------------------------------------------------------06
is a linear map $T:V_1 \ra V_2$ such that 
\[
T \rho_1 (w) \ = \  \rho_2(w) T \qquad (w \in W_\K)
\]
and $T N_1 = N_2 T$.

[Note: \ 
If $T$ is a linear isomorphism, then the Deligne representations 
\[
\begin{cases}
\ (\rho_1,V_1,N_1)\\
\ (\rho_2,V_2,N_2)
\end{cases}
\]
are said to be \un{isomorphic}.]
\index{Deligne representation, isomorphic}
\end{x}

\vspace{0.1cm}


\begin{x}{\small\bf DEFINITION} \ %14
Suppose that $(\rho,V,N)$ is a Deligne representation of $W_\K$ $-$then a subspace $V_0 \subset V$ is an 
\un{invariant subspace} 
\index{Deligne representation, invariant subspace}
if it is invariant under $\rho$ and $N$.
\end{x}
\vspace{0.1cm}

\begin{x}{\small\bf LEMMA} \ %15
The kernel of $N$ is an invariant subspace. 

\vspace{0.1cm}

PROOF \ 
If $N v = 0$, then $\forall \ w \in W_\K$, 
\[
N \rho (w) v \ = \ \norm{w^{-1}} \rho(w) N v = 0.
\]
\end{x}

\vspace{0.1cm}

\begin{x}{\small\bf DEFINITION} \ %16
A Deligne representation $(\rho,V,N)$ of $W_\K$ is 
\un{indecomposable} 
\index{Deligne representation, indecomposable}
if $V$ cannot be written as a direct sum of proper 
invariant subspaces.
\end{x}

\vspace{0.1cm}

\begin{x}{\small\bf EXAMPLE} \ %17
Consider $\spp(n)$ $-$then it is indecomposable.

\vspace{0.1cm}

[If $\C^n = S \oplus T$ was a nontrivial decomposition into  proper invariant subspaces, then both
$
\begin{cases}
\ S \cap \ker N\\
\ T \cap \ker N
\end{cases}
$
would be nontrivial.]
\end{x}
\vspace{0.1cm}
%%----------------------------------------------------------------------------------------------07

\begin{x}{\small\bf DEFINITION} \ %18
A Deligne representation $(\rho,V,N)$ of $W_\K$ is 
\un{semisimple} 
\index{Deligne representation, semisimple}
if $\rho$ is semisimple (cf. \S23, \#37).
\end{x}

\vspace{0.1cm}

\begin{x}{\small\bf EXAMPLE} \ %19
Consider $\spp(n)$ $-$then it is semisimple.
\end{x}

\vspace{0.1cm}

\begin{x}{\small\bf LEMMA} \ %20
Let $\pi$ be an irreducible representation of $W_\K$  $-$then $\spp(n) \ \otimes \  \pi$ is semisimple and indecomposable.  

[Note: \ 
Recall that $\pi$ is identified with $(\pi,0)$.]
\end{x}

\vspace{0.1cm}

\begin{x}{\small\bf THEOREM} \ %21
Every semisimple indecomposable Deligne representation of $W_\K$  is equivalent to a Deligne representation of the form 
$\spp(n) \ \otimes \  \pi$, where $\pi$ is an irreducible representation of $W_\K$ and $n$ is a positive integer. 
\end{x}

\vspace{0.1cm}

\begin{x}{\small\bf THEOREM} \ %22
Let $(\rho,V,N)$ be a semisimple Deligne representation of $W_\K$ $-$then there is a decomposition 
\[
(\rho,V,N) \ = \ \bigoplus\limits_{i= 1}^{s} \spp(n_i) \ \otimes \ \pi_i,
\]
where $\pi_i$ is an irreducible representation of $W_\K$ and $n_i$ is a positive integer.  
Furthermore, if  
\[
(\rho,V,N) \ = \ \bigoplus\limits_{j = 1}^t \spp(n_j^\prime) \  \otimes \  \pi_j^\prime
\]
is another such decomposition, then $s = t$ and after a renumbering of the summands, 
$\pi_i \approx \pi_i^\prime$ and $n_i = n_i^\prime$.
\end{x}

\vspace{0.1cm}


\[
\textbf{APPENDIX}
\]
\setcounter{theoremn}{0}

Instead of working with 
\[
WD_\K \ = \ \C \rtimes W_\K,
\]
%%----------------------------------------------------------------------------------------------08
some authorities work with 
\[
SL(2,\C) \times W_\K,
\]
the rationale for this being that the semisimple representations of the two groups are the ``same''.

Given $w \in W_\K$, let 
\[
h_w \ = \ 
\begin{pmatrix}
\norm{w}^{1/2} &0\\
0 &\norm{w}^{-1/2}\\
\end{pmatrix}
\]
and identify $z \in \C$ with 
\[
h_w \ = \ 
\begin{pmatrix}
1 &z\\
0 &1\\
\end{pmatrix}
.
\]

Then
\[
h_w 
\begin{pmatrix}
1 &z\\
0 &1\\
\end{pmatrix}
h_w^{-1}
\ = \ 
\begin{pmatrix}
1 &\norm{w}z\\
0 &1\\
\end{pmatrix}
.
\]
But  conjugation by $h_w$ is an automorphism of $\SL(2,\C)$, thus one can form the semisimple direct product 
$\SL(2,\C) \rtimes W_\K$, the multiplication rule being 
\[
(X_1, w_1) (X_2, w_2) \ = \ (X_1 h_{w_1} X_2 h_{w_1}^{-1}, w_1 w_2).
\]

\vspace{0.1cm}

%%----------------------------------------------------------------------------------------------09

\begin{x}{\small\bf LEMMA} \ %01
The arrow 
\[
(X,w) \lra (Xh_w,w)
\]
from 
\[
\SL(2,\C) \rtimes W_\K \quad \text{to} \quad \SL(2,\C) \times W_\K
\]
is an isomorphism of groups.
\end{x}

\vspace{0.1cm}

\begin{x}{\small\bf DEFINITION} \ %02
A 
\un{representation} 
\index{representation of $\SL(2,\C)$} 
of $\SL(2,\C) \times W_\K$ is a continuous homomorphism 
$\rho:\SL(2,\C) \times W_\K \ra \GL(V)$ ($V$ a finite dimensional complex vector space) such that the restriction of $\rho$ to 
$\SL(2,\C)$ is complex analytic.
\end{x}

\vspace{0.1cm}



\begin{x}{\small\bf \un{N.B.}} \ %03
$\rho$ is semisimple iff its restriction to $W_\K$ is semisimple.  

\vspace{0.1cm}

[The restriction of $\rho$ to $\SL(2,\C)$ is necessarily semisimple.]
\end{x}

\vspace{0.1cm}

The finite dimensional irreducible representations of $\SL(2,\C)$ are parameterized by the positive integers:
\[
n \longleftrightarrow \sym(n), \qquad \dim \sym(n) = n.
\]

\vspace{0.1cm}

\begin{x}{\small\bf THEOREM} \ %04
The isomorphism classes of semisimple Deligne representations of $W_\K$ are in a 1-to-1 correspondence with the isomorphism classes of semisimple representations of $\SL(2,\C) \times W_\K$.
\end{x}

\vspace{0.1cm}

To explicate matters, start with a semisimple indecomposable Deligne representation of $W_\K$, say 
$\spp(n) \ \otimes \  \pi$, and assign to it the external tensor product 
\fboxsep=0cm
$\sym(n)  \boxtimesdmc \hsx \pi$, hence in general 
\[
\bigoplus\limits_{i = 1}^s \spp(n_i)  \ \otimes \  \pi_i \lra 
%\bigoplus\limits_{i = 1}^s \sym(n_i) \boxed{\times} \hsx \pi_i.
\bigoplus\limits_{i = 1}^s \sym(n_i) \boxtimesdmc \hsx \pi_i.
\] 
%%%%%%%%%%%%%%%%%%%%%%%%%%%%%%%%%%%%%%
%%%%%%%%%%%%%%%%%%%%%%%%%%%%%%%%%%%%%%
%%%%%%%%%%%%%%%%%%%%%%%%%%%%%%%%%%%%%%






















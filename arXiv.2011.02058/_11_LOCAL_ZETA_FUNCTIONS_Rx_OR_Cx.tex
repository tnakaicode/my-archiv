\chapter{
$\boldsymbol{\S}$\textbf{11}.\quad  LOCAL ZETA FUNCTIONS: $\R^\times$ or $\C^\times$}
\setlength\parindent{2em}
\setcounter{theoremn}{0}
%%----------------------------------------------------------------------------------------------01
\ \indent 
We shall first consider $\R^\times$, hence $\widetilde{\R}^\times \thickapprox \Z / 2 \Z \times \C$ and every character has the form
\[
\chi(x) \equiv \chi_{\sigma, s}(x) = (\sgn x)^\sigma \abs{x}^s \quad 
(\sigma \in \{0, 1\}, \ s \in \C) \quad 
(\text{cf. } \S7, \  \#11).
\]

\vspace{0.1cm}

\begin{x}{\small\bf DEFINITION} \ %1
Given $f \in \sS(\R^n)$ and a character $\chi: \R^\times \ra \C^\times$, 
the 
\un{local zeta} \un{function}
\index{local zeta function} 
attached to the pair $(f, \chi)$ is
\[
Z (f,\chi) = \int_{\R^\times} f(x) \chi(x) d^\times x, \qquad \text{where $d^\times x = \frac{dx}{\abs{x}}$.}
\]

[Note: The parameters $\sigma$ and $s$ are implicit:
\[
Z(f,x) \equiv Z(f,\chi_{\sigma,s}).]
\]
\end{x}

\vspace{0.1cm}



\begin{x}{\small\bf LEMMA} \ %2
The integral defining $Z(f,\chi)$ is absolutely convergent for $\Re(s) > 0$.

\vspace{0.1cm}

PROOF \   
Since $f$ is Schwartz, there are no issues at infinity.  
As for what happens at the origin, let $I = ]-1,1[ \ -\  \{0\}$ and fix $C > 0$ such that $\abs{f(x)} \le C$ $(x \in I)$. $-$then
\begin{align*}
\abs{Z(f,\chi)} \ 	
&\le\  \int_{\R- \{0\}} \abs{f(x)} \abs{x}^{\Re(s) - 1} dx\\	
&\le\   \bigl(\int_{\R - I} + \int_I\bigr) \abs{f(x)} \abs{x}^{\Re(s) - 1} dx\\	
&\le\   M + C \int_I \abs{x}^{\Re(s) - 1} dx,
\end{align*}
a finite quantity.
\end{x}

\vspace{0.1cm}
%%----------------------------------------------------------------------------------------------02

\begin{x}{\small\bf LEMMA} \ %3
$Z(f,\chi)$ is a holomorphic function of $s$ in the strip $\Re(s) > 0$.

[Formally, 
\[
\frac{d}{ds}Z(f,\chi) \ =\  \int_{\R^\times} f(x) (sgn x)^\sigma (\log \abs{x}) \abs{x}^s d^\times x,
\]

and while correct, "differentiation under the integral sign" does require a formal proof $\ldots$ .]
\end{x}

\vspace{0.1cm}

\begin{x}{\small\bf NOTATION} \ %4
Put
\[
\widecheck{\chi} = \chi^{-1} \| \cdot \|.
\]

The integral defining $Z(f,\widecheck{\chi} )$ is absolutely convergent if $\Re(1-s) > 0$, i.e., if $1-\Re(s) > 0$ 
or still, if $\Re(s) < 1$.
\end{x}

\vspace{0.1cm}

\begin{x}{\small\bf LEMMA} \ %5
Let $f, g \in \sS(\R)$ and suppose that $0 < \Re(s) < 1$ $-$then
\[
Z(f,\chi) Z(\widehat{g},\widecheck{\chi}) =  Z(\widehat{f},\widecheck{\chi}) Z(g,\chi) 
\]

\vspace{0.1cm}

PROOF \ 
Write
\[
Z(f,\chi) Z(\widehat{g},\widecheck{\chi})  
= \int \int_{\R^\times \times \R^\times} f(x) \widehat{g}(y) \chi(xy^{-1}) \abs{y} d^\times x \hspace{0.05cm} d^\times y
\]
and make the substitution $t = yx^{-1}$ to get
\[
Z(f,\chi) Z(\widehat{g},\widecheck{\chi})  
= \int_{\R^\times} \bigl(\int_{\R^\times} f(x) \widehat{g}(tx) \abs{x} d^\times x\bigr) 
\chi(t^{-1}) \abs{t} d^\times t.
\]
The claim now is that the inner integral is symmetric in $f$ and $g$ (which then implies that
%%----------------------------------------------------------------------------------------------03
\[
Z(f,\chi) Z(\widehat{g},\widecheck{\chi})  = Z(g,\chi)   Z(\widehat{f},\widecheck{\chi}), 
\]
the desired equality$)$.   To see this is so, observe first that
\[
\abs{x} du \cdot d^\times x = \abs{u} dx \cdot d^\times u.
\]
Since $\R^\times$ and $\R$ differ by a single element, it therefore follows that
\begin{align*}
\int_{\R^\times} f(x) \widehat{g}(tx) \abs{x} d^\times x \ 
&=\  \int_{\R^\times} f(x) \abs{x} \bigl(\int_\R g(u) e^{2\pi\sqrt{-1}\  txu} du\bigr) d^\times x\\	
&=\   \int \int_{\R \times \R^\times} f(x) g(u) \abs{x} e^{2 \pi \sqrt{-1}\  txu} du d^\times x\\
&=\  \int_{\R^\times} g(u) \abs{u} \bigl(\int_\R f(x) e^{2 \pi \sqrt{-1}\  txu} dx \bigr) d^\times u\\
&=\  \int_{\R^\times} g(u) \widehat{f}(tu) \abs{u} d^\times u.
\end{align*}

Fix $\phi \in \sS(\R)$ and put
\[
\rho(\chi) = \frac{Z(\phi,\chi)}{Z(\widehat{\phi},\widecheck{\chi})}
\]
Then $\rho(\chi)$ is independent of the choice of $\phi$ and $\forall$ $f \in \sS(\R)$, the 
\un{functional equation}
\index{functional equation}
\[
Z(f,\chi) = \rho(\chi) Z(\widehat{f}, \widecheck{\chi})
\]
obtains.
\end{x}

\vspace{0.1cm}



\begin{x}{\small\bf LEMMA} \ %6
$\rho(\chi)$ is a meromorphic function of $s$ (cf. infra).
\end{x}

\vspace{0.1cm}
%%----------------------------------------------------------------------------------------------04


\begin{x}{\small\bf APPLICATION} \ %7
$\forall$ $f \in \sS(\R), Z(f,\chi)$ admits a meromorphic continuation to the whole $s$-plane.
\end{x}

\vspace{0.1cm}



\begin{x}{\small\bf NOTATION} \ %8
Set
\[
\Gamma_\R(s) = \pi^{-s/2} \Gamma(s/2).
\]
\end{x}

\vspace{0.1cm}



\begin{x}{\small\bf DEFINITION} \ %9
Write
\[L(\chi) =\ 
\begin{cases}
\Gamma_\R(s) &\quad (\sigma = 0)\\
\Gamma_\R(s+1) &\quad (\sigma = 1)
\end{cases}
.\]
\end{x}

\vspace{0.1cm}



Proceeding to the computation of $\rho(\chi)$, distinguish two cases.
\vspace{0.2cm}

\qquad \textbullet \quad \un{$\sigma = 0$} \quad Take $\phi_0(x)$ to be $e^{-\pi x^2}$ $-$then
\begin{align*}
Z(\phi_0, \chi) \ 
&=\  \int_{\R^\times} e^{-\pi x^2} \abs{x}^s d^\times x\\	
&=\  2\int_0^\infty e^{-\pi x^2} x^{s-1} dx\\	
&=\  \pi^{-s/2} \Gamma(s/2) \\
&=\  \Gamma_\R(s)\\
&=\  L(\chi).
\end{align*}
Next $\widehat{\phi}_0 = \phi_0$ ( cf. \S10, \#2) so by the above argument,
\[
Z(\widehat{\phi}_0,\widecheck{\chi}) = L(\widecheck{\chi}),
\]
from which
%%----------------------------------------------------------------------------------------------05 (ish)
\begin{align*}
\rho(\chi) \ 	
&=\  \frac{L(\chi)}{L(\widecheck{\chi})}\\	
&=\  \frac{\pi^{-s/2} \Gamma\bigl(\ds\frac{s}{2}\bigr)}{\pi^{-(1-s)/2} \Gamma\bigl(\ds\frac{1-s}{2}\bigr)}\\	
&=\  2^{1-s} \pi^{-s} \cos\bigl( \ds\frac{\pi s}{2}\bigr) \Gamma(s).
\end{align*}

\qquad \textbullet \quad \un{$\sigma = 1$} \quad Take $\phi_1(x)$ to be $xe^{-\pi x^2}$ $-$then
\allowdisplaybreaks
\begin{align*}
Z(\phi_1, \chi) \ 
&=\  \int_{\R^\times} xe^{-\pi x^2} \frac{x}{\abs{x}} \abs{x}^s d^\times x\\	
&=\  \int_{\R^\times} e^{-\pi x^2}  \abs{x}^{s+1} d^\times x\\
&=\  2\int_0^\infty e^{-\pi x^2} x^s dx\\	
&=\  \pi^{-(s+1)/2} \Gamma\bigl(\frac{s+1}{2}\bigr)\\
&=\  \Gamma_\R(s+1)\\
&=\  L(\chi).
\end{align*}
Next
\[
\widehat{\phi}_1(t) = \sqrt{-1}\ t \exp(- \pi t^2) \quad \text{(cf. \S10, \#2)}.
\]
Therefore
\begin{align*}
Z(\widehat{\phi_1}, \widecheck{\chi}) \ 	
&=\  \sqrt{-1}\   \int_{\R^\times} xe^{-\pi x^2} \frac{x}{\abs{x}} \abs{x}^{1-s} d^\times x\\	
&=\  \sqrt{-1}\   \int_{\R^\times} e^{-\pi x^2}  \abs{x}^{2-s} d^\times x\\
&=\  \sqrt{-1}\   2\int_0^\infty e^{-\pi x^2} x^{1-s} dx\\	
&=\  \sqrt{-1}\  \pi^{-(2-s)/2} \Gamma\bigl(\frac{2-s}{2}\bigr)\\
&=\  \sqrt{-1}\  \Gamma_\R(2-s)\\
&=\  \sqrt{-1}\  L(\widecheck{\chi}).
\end{align*}
%%----------------------------------------------------------------------------------------------06 (ish)
Accordingly
\begin{align*}
\rho(\chi) \ 
&=\  - \sqrt{-1}\  \ds\frac{L(\chi)}{L(\widecheck{\chi})}\\	
&=\  - \sqrt{-1}\   \frac{\pi^{-(s+1)/2} \Gamma(\ds\frac{s+1}{2})}{\pi^{(s-2)/2} \Gamma\bigl(\ds\frac{2-s}{2}\bigr)}\\	
&=\  - \sqrt{-1}\    2^{1-s} \pi^{-s} \sin(\ds\frac{\pi s}{2}) \Gamma(s).
\end{align*}

\vspace{0.1cm}

\begin{x}{\small\bf FACT} \ %10
\[
\begin{cases}
\ds\frac{\zeta(1-s)}{\zeta(s)} 	&= 2^{1-s} \pi^{-s} \cos\bigl(\ds\frac{\pi s}{2}\bigr) \Gamma(s)\\
\\
\ds\frac{\zeta(s)}{\zeta(1-s)} 	&= 2^{s} \pi^{s-1} \sin\bigl(\ds\frac{\pi s}{2}\bigr) \Gamma(1-s)
\end{cases}
.
\]

\vspace{0.1cm}

To recapitulate: $\rho(\chi)$ is a meromorphic function of $s$ and 
\[
\rho(\chi) 	= \epsilon(\chi) \frac{L(\chi)}{L(\widecheck{\chi})},
\]
where
\[\epsilon(\chi) =\ 
\begin{cases}
\  1 				&\quad \text{$(\sigma = 0)$}\\
\  - \sqrt{-1} 	&\quad \text{$(\sigma = 1)$}
\end{cases}
.\]
\end{x}

\vspace{0.1cm}

%%----------------------------------------------------------------------------------------------7
Having dealt with $\R^\times$, let us now turn to $\C^\times$, hence 
$\widetilde{\C}^\times$ $\thickapprox \Z \times \C$ and every character has the form
\[
\chi(x) \equiv \chi_{n,s}(x) = \exp(\sqrt{-1} \  n \  \arg x) \abs{x}^s 
\quad (n \in \Z, \ s \in \C) \quad (\text{cf. } \S 7, \ \#12).
\]
Here, however, it will be best to make a couple of adjustments.

1. Replace $x$ by $z$.

2. Replace $\acdot$ by $\acdot_\C$, the normalized absolute value, so
\[
\abs{z}_\C = \abs{z  \bar{z}} = \abs{z}^2	\qquad ( \text{cf. } \S6, \  \#15).
\]

\begin{x}{\small\bf DEFINITION} \ %11
Given $f \in \sS(\C)$ $(= \sS(\R^2))$ and a character $\chi : \C^\times \ra \C^\times$, the \un{local zeta function} attached to the pair $(f, \chi)$ is
\[
Z(f,\chi) = \int_{\C^\times} f(z) \chi(z) d^\times z,
\]
where $d^\times z = \ds\frac{\abs{dz \wedge d\ov{z}}}{\abs{z}_\C }$.
\vspace{0.2cm}

[Note: \  The parameters $n$ and $s$ are implicit:
\[
Z(f,\chi) \equiv Z(f, \chi_{n,s}).]
\]
\end{x}

\vspace{0.1cm}


\begin{x}{\small\bf NOTATION} \ %12
Put
\[
\widecheck{\chi} = \chi^{-1} \acdot_\C.
\]

The analogs of \#2 and \#3 are immediate, as is the analog of $\#5$ 
(just replace $\R^\times$ by $\C^\times$ and $\acdot$ by $\acdot_\C$), the crux then being the analog of $\#6$.
\end{x}

\vspace{0.1cm}



\begin{x}{\small\bf NOTATION} \ %13
Set
\[
\Gamma_\C(s) = (2\pi)^{1-s} \Gamma(s).
\]
\end{x}

\vspace{0.1cm}
%%----------------------------------------------------------------------------------------------08


\begin{x}{\small\bf DEFINITION} \ %14
Write
\[
L(\chi) = \Gamma_\C(s + \frac{\abs{n}}{2}).
\]

To determine $\rho(\chi)$ via a judicious choice of $\phi$ per the relation
\[
\rho(\chi) = \frac{Z(\phi,\chi)}{Z(\widehat{\phi},\widecheck{\chi})},
\]
let
\[
\phi_n(z) =\ 
\begin{cases}
\ \ov{z}^{n} e^{-2\pi\abs{z}^2}		&\quad (n \ge 0)\\
\ z^{-n} e^{-2\pi\abs{z}^2}	 		&\quad (n < 0)\\
\end{cases}
.\]
Then
\[
\widehat{\phi}_n = \sqrt{-1}^{\abs{n}} \phi_{-n} 	\qquad (\text{cf. } \S 10, \  \#3).
\]
\end{x}

\vspace{0.1cm}


\begin{x}{\small\bf \un{N.B.}} \ %15
In terms of polar coordinates $z = r e^{\sqrt{-1} \text{ } \theta}$,\\

\qquad\textbullet \quad$\phi_n(z) = r^{\abs{n}} \exp (-2\pi r^2 - \sqrt{-1} \ n \theta)$\\

\qquad\textbullet \quad$d^\times z =\ds\frac{2 r dr d \theta}{r^2} = \ds\frac{2}{r} dr d \theta$\\

\qquad\textbullet \quad$\chi(z) = e^{\sqrt{-1} \  n \theta} \abs{z}_\C^s = e^{\sqrt{-1} \ n \theta} r^{2s}$.\\

Therefore
%%----------------------------------------------------------------------------------------------9 (ish)


\begin{align*}
Z(\phi_n,\chi) \ 	
&=\  \int_0^{2\pi} \int_0^\infty r^{\abs{n}} \exp(-2\pi r^2 - \sqrt{-1} \  n \theta) e^{\sqrt{-1} \ n \theta} r^{2s} \frac{2}{r} dr d\theta\\	
&=\  \int_0^{2\pi} \int_0^\infty r^{2(s-1) + \abs{n}} \exp(-2\pi r^2) 2r dr d\theta\\		
&=\  2\pi \int_0^\infty t^{(s-1) + \abs{n}/2} \exp(-2\pi t) dt\\
&=\  (2\pi)^{1 - s - \abs{n}/2} \Gamma\bigl(s + \frac{\abs{n}}{2}\bigr)\\ 
&=\  \Gamma_\C\bigl(s + \frac{\abs{n}}{2}\bigr)\\
&=\  L(\chi)\\  
\end{align*}
and
\begin{align*}
Z(\widehat{\phi}_n,\widecheck{\chi}) \ 	
&=\  Z((\sqrt{-1})^{\abs{n}} \phi_{-n}, \widecheck{\chi})\\	
&=\  (\sqrt{-1})^{\abs{n}} (2\pi)^{1 - (1 - s)- \abs{n}/2} \Gamma\bigl(1 - s + \frac{\abs{n}}{2}\bigr)\\		
&=\  (\sqrt{-1})^{\abs{n}} (2\pi)^{s - \abs{n}/2} \Gamma\bigl(1 - s + \frac{\abs{n}}{2}\bigr)\\	
&=\  (\sqrt{-1})^{\abs{n}}  \Gamma_\C\bigl(1 - s + \frac{\abs{n}}{2}\bigr)\\
&=\  (\sqrt{-1})^{\abs{n}}  L(\widecheck{\chi}).
\end{align*}
Consequently, 
\begin{align*}
\rho(\chi)	\ 	
&=\  \frac{Z(\phi_n,\chi)}{Z(\widehat{\phi}_n,\widecheck{\chi}) }\\	
&=\  (\sqrt{-1})^{-\abs{n}} \frac{L(\chi)}{L(\widecheck{\chi})}\\		
&=\  \epsilon(\chi)\frac{L(\chi)}{L(\widecheck{\chi})},		
\end{align*}
%%----------------------------------------------------------------------------------------------10
where
\[
\epsilon(\chi) = (\sqrt{-1})^{-\abs{n}}.
\]
And
\[
\frac{L(\chi)}{L(\widecheck{\chi})} = (2\pi)^{1 - 2s} 
\frac{\Gamma\bigl(s + \ds\frac{\abs{n}}{2}\bigr)}{\Gamma\bigl(1-s + \ds\frac{\abs{n}}{2}\bigr)}.
\]
\end{x}
%%%%%%%%%%%%%%





%%%%%%%%%%%%%%%%%%%%%%%%%%%%%%%%%%%%%%
%%%%%%%%%%%%%%%%%%%%%%%%%%%%%%%%%%%%%%
%%%%%%%%%%%%%%%%%%%%%%%%%%%%%%%%%%%%%%






















\chapter{
$\boldsymbol{\S}$\textbf{17}.\quad  GLOBAL ZETA FUNCTIONS}
\setlength\parindent{2em}
\setcounter{theoremn}{0}
%%----------------------------------------------------------------------------------------------01

\ \indent 
Structurally, there is a short exact sequence
\[
1 \ra \I^1/ \Q^\times \ra \I/ \Q^\times \ra \R_{>0}^\times \ra 1 	\qquad 
(\text{cf.} \  \S14, \ \#27)
\]
and $\I^1/ \Q^\times$ is compact (cf. \S14, \ \#24).

\vspace{0.3cm}

\begin{x}{\small\bf DEFINITION} \ %01
Given $f \in \mathcal{B}_\infty(\A)$ and a unitary character $\omega:\I/ \Q^\times \ra \T$, the 
\underline{global zeta function}
\index{global zeta function} 
attached to the pair $(f,\omega)$ is 
\[
Z (f,\omega,s) \ =\  \int_\I f(x)\omega(x)\abs{x}_\A^s d^\times x		\qquad (\Re(s) > 1).
\]
\end{x}

\vspace{0.1cm}

\begin{x}{\small\bf EXAMPLE} \ %02
In the notation of \S16, take
\[
f(x) \ = \ \Phi(x) \ = \ e^{-\pi x_\infty ^2} \ \prod_p \chi_{\Z_p} (x_p)		\qquad (x \in \A)
\]
and let $\omega = 1$ $-$then as shown there
\[
Z(f,1,s) = \pi ^{-s/2} \Gamma(s/2) \zeta(s).
\]
\end{x}

\vspace{0.1cm}

\begin{x}{\small\bf LEMMA} \ %03
$Z(f,\omega,s)$ is a holomorphic function of $s$ in the strip $\Re(s) > 1.$
\end{x}

\vspace{0.1cm}

\begin{x}{\small\bf THEOREM} \ %04
$Z(f,\omega,s)$ can be meromorphically continued into the whole $s$-plane and satisfies the functional equation
\[
Z(f,\omega,s) = Z(\widehat{f},\ov{\omega},1-s).
\]

[Note:
\[
f \in \mathcal{B}_\infty(\A) \implies \widehat{f} \in \mathcal{B}_\infty(\A)	\qquad (\text{cf.} \  \S15, \  \#9).]
\]
The proof is a computation, albeit a lengthy one.
%%----------------------------------------------------------------------------------------------02

To begin with,
\[
\I \thickapprox \R_{>0}^\times \times \I^1	\qquad (\text{cf.} \   \S14, \  \#27).
\]
Therefore
\begin{align*}
Z (f,\omega,s) \ 	
&=\vsx \int_\I f(x)\omega(x)\abs{x}_\A^s d^\times x\\	
&=\vsx \int_{\R_{>0}^\times \times \I^1} f(tx)\omega(tx)\abs{tx}_\A^s \frac{dt}{t} d^\times x\\	
&=\vsx \int_0^\infty \bigl(\int_{\I^1} f(tx)\omega(tx)\abs{tx}_\A^s  d^\times x\bigr) \frac{dt}{t}.	
\end{align*}
\end{x}

\vspace{0.1cm}

\begin{x}{\small\bf NOTATION} \ %05
Put
\[
Z_t(f,\omega,s) = \int_{\I^1} f(tx)\omega(tx)\abs{tx}_\A^s  d^\times x.
\]
\end{x}

\vspace{0.1cm}

\begin{x}{\small\bf LEMMA} \ %06
\ \qquad\\
\vspace{0.1cm}

$
\text{\qquad\qquad} Z_t(f,\omega,s) + f(0) \ds\int_{\I^1/\Q^\times} \omega(tx)\abs{tx}_\A^s  d^\times x\\
\text{\qquad\qquad\qquad\qquad} = Z_{t^{-1}}(\widehat{f},\ov{\omega},1-s) + 
\widehat{f}(0) \int_{\I^1/\Q^\times} \ov{\omega}(t^{-1}x)\abs{t^{-1}x}_\A^{1-s}  d^\times x.\\
$

\vspace{0.1cm}

PROOF \ 
Write
\begin{align*}
\int_{\I^1} f(tx)\omega(tx)\abs{tx}_\A^s  d^\times x \ 
&=\vsx \int_{\I^1/\Q^\times} \bigl(\sum_{r \in \Q^\times} f(rtx) \omega(rtx) \abs{rtx}_\A^s \bigr) d^\times x\\	
&=\vsx \int_{\I^1/\Q^\times} \bigl(\sum_{r \in \Q^\times} f(rtx) \omega(tx) \abs{tx}_\A^s \bigr) d^\times x.	
\end{align*}
%%----------------------------------------------------------------------------------------------03
Then

\allowdisplaybreaks
$
Z_t(f,\omega,s) + f(0) \ds\int_{\I^1/\Q^\times} \omega(tx)\abs{tx}_\A^s  d^\times x
$
\begin{align*}
\quad \quad     
&=\vsx  \int_{\I^1/\Q^\times} (\sum_{q \in \Q} f(rtx) \omega(tx) \abs{tx}_\A^s d^\times x\\	
&=\vsx \int_{\I^1/\Q^\times} \bigl(\frac{1}{\abs{tx}_\A} \ \sum_{q \in \Q} \widehat{f}(qt^{-1}x^{-1})\bigr) \omega(tx) \abs{tx}_\A^s d^\times x \qquad (\text{cf.} \  \S15, \ \#13)\\
&=\vsx \int_{\I^1/\Q^\times} \bigl(\sum_{q \in \Q} \widehat{f}(qt^{-1}x)\bigr) \abs{t^{-1}x}_\A\omega(tx^{-1}) \abs{tx^{-1}}_\A^s d^\times x \qquad \text{$(x \ra x^{-1})$}\\
&=\vsx \int_{\I^1/\Q^\times} \bigl(\sum_{q \in \Q} \widehat{f}(qt^{-1}x)\bigr) \omega^{-1}(t^{-1}x) \abs{t^{-1}x}_\A^{1-s} d^\times x \\
&=\vsx \int_{\I^1/\Q^\times} \bigl(\sum_{q \in \Q} \widehat{f}(qt^{-1}x)\bigr) \ov{\omega}(t^{-1}x) \abs{t^{-1}x}_\A^{1-s} d^\times x \\
&=\vsx \int_{\I^1/\Q^\times} \bigl(\sum_{q \in \Q^\times} \widehat{f}(qt^{-1}x) \ov{\omega}(qt^{-1}x) \abs{qt^{-1}x}_\A^{1-s} \bigr)d^\times x \\
&\hspace{3.5cm} + \widehat{f}(0) \int_{\I^1/\Q^\times} \ov{\omega}(t^{-1}x)\abs{t^{-1}x}_\A^{1-s}  d^\times x \\
&=\vsx \int_{\I^1} \widehat{f}(t^{-1}x)) \ov{\omega}(t^{-1}x) \abs{t^{-1}x)}_\A^{1-s} d^\times x \\
&\hspace{3.5cm} + \widehat{f}(0) \int_{\I^1/\Q^\times} \ov{\omega}(t^{-1}x)\abs{t^{-1}x}_\A^{1-s}  d^\times x\\
&=\vsx Z_{t^{-1}}(\widehat{f},\ov{\omega},1-s) + \widehat{f}(0) \int_{\I^1/\Q^\times} \ov{\omega}(t^{-1}x)\abs{t^{-1}x}_\A^{1-s}  d^\times x.
\end{align*}


Return to $Z(f,\omega,s)$ and break it up as follows:
%%----------------------------------------------------------------------------------------------04
\[
Z(f,\omega,s) = \int_0^1 Z_t(f,\omega,s) \frac{dt}{t} + \int_1^\infty Z_t(f,\omega,s) \frac{dt}{t}.
\]

\vspace{0.1cm}


\begin{x}{\small\bf LEMMA} \ %07
The integral
\[
 \int_1^\infty Z_t(f,\omega,s) \frac{dt}{t}
\]
is a holomorphic function of $s$.

[It can be expressed as
\[
\int_{\I :\abs{x}_\A \ge 1} f(x)\omega (x) \abs{x}_\A^s d^\times x.]
\]
\end{x}

\vspace{0.1cm}


This leaves
\[
\int_0^1 Z_t(f,\omega,s) \frac{dt}{t},
\]
which can thus be represented as
\[
\int_0^1 (Z_{t^{-1}}(\widehat{f},\ov{\omega},1-s)  \  - \   f(0) \int_{\I^1/\Q^\times} \omega(tx)  \abs{tx}_\A^s d^\times x \  + \  \widehat{f}(0) \int_{\I^1/\Q^\times} \ov{\omega}(t^{-1}x)  \abs{t^{-1}x}_\A^{1-s} d^\times x )\frac{dt}{t}.
\]

To carry out the analysis, subject
\[
\int_0^1 Z_{t^{-1}}(\widehat{f},\ov{\omega},1-s) \frac{dt}{t}
\]
to the change of variable $t \ra t^{-1}$, thereby leading to
\[
\int_1^\infty Z_{t}(\widehat{f},\ov{\omega},1-s) \frac{dt}{t},
\]
a holomorphic function of $s$ (cf. \#7 supra).
%%----------------------------------------------------------------------------------------------05

\vspace{0.2cm}

It remains to discuss
\begin{align*}
R(f,\omega,s)     	
&=\vsx \int_0^1 (-f(0) \int_{\I^1/\Q^\times} \omega(tx)  \abs{tx}_\A^s d^\times x \  + \  \widehat{f}(0) \int_{\I^1/\Q^\times} \ov{\omega}(t^{-1}x)  \abs{t^{-1}x}_\A^{1-s} d^\times x )\frac{dt}{t}\\
&=\vsx \int_0^1 \bigl(-f(0)\omega(t) \abs{t}^s \int_{\I^1/\Q^\times} \omega(x) d^\times x \  + \  \widehat{f}(0) \ov{\omega}(t^{-1})\abs{t^{-1}}^{1-s} \int_{\I^1/\Q^\times} \ov{\omega}(x) d^\times x \bigr)\frac{dt}{t},
\end{align*}
there being two cases.

\vspace{0.1cm}

1. \quad $\omega$ is nontrivial on $\I^1$.  
Since $\I^1/\Q^\times$ is compact (cf. \S14, \#24), the integrals
\[
\int_{\I^1/\Q^\times} \omega(x) d^\times x, \qquad   \int_{\I^1/\Q^\times} \ov{\omega}(x) d^\times x 
\]
must vanish (cf. \S7, \#46).  
Therefore $R(f,\omega,s) = 0$, hence
\[
Z(f,\omega,s) = \int_1^\infty Z_t(f,\omega,s) \frac{dt}{t} + \int_1^\infty Z_t(\widehat{f},\ov{\omega},1-s) \frac{dt}{t},
\]
is a holomorphic function of $s$.

\vspace{0.1cm}

2. \quad $\omega$ is trivial on $\I^1$.  Let $\phi: \R_{>0}^\times \ra \I/\I^1$ be the isomorphism per \S14, \#27
$-$then $\omega \circ \phi: \R_{>0}^\times \ra \T$ is a unitary character of $\R_{>0}^\times$, 
thus for some $w \in \R$, $\omega \circ \phi = \acdot^{-\sqrt{-1}\ w}$, so
\[
\omega = \acdot^{-\sqrt{-1}\ w} \circ \phi^{-1} \implies \omega(x) = \abs{x}_\A^{-\sqrt{-1}\ w}.
\]
Therefore
%%----------------------------------------------------------------------------------------------06 (ish)

\begin{align*}
R(f,\omega,s)  \ 	
&=\  -f(0) \vol(\I^1/\Q^\times) \int_0^1 t^{-\sqrt{-1} \ w + s - 1} dt + \widehat{f}(0) \vol(\I^1/\Q^\times) \int_0^1 t^{-\sqrt{-1}\ w + s - 2} dt \\
&=\  -f(0) \frac{\vol(I^1/\Q^\times)}{-\sqrt{-1}\ w + s} + \widehat{f}(0) \frac{\vol(\I^1/\Q^\times)}{-\sqrt{-1}\ w + s - 1},
\end{align*}
a meromorphic function that has a simple pole at
\\
\[
\begin{cases}
\ s = \sqrt{-1}\ w \quad \quad \quad \text{ with residue } \quad -f(0) \ \vol(\I^1/\Q^\times) \quad \text{if} \ f(0) \ne 0\\
\ s = \sqrt{-1}\ w+1 \quad \ \text{ with residue } \qquad  \widehat{f}(0) \ \vol(\I^1/\Q^\times) \quad \text{if} \ \widehat{f}(0) \ne 0\\
\end{cases}
.\]

\vspace{0.1cm}



\begin{x}{\small\bf \un{N.B.}} \ %08
To explicate $\vol(\I^1/\Q^\times)$ use the machinery of \S16:  In the notation of \#2 above, 
\[
Z(f,1,s) = -\frac{1}{s} + \frac{1}{s-1} + \dotsb 
\]
\[
\implies \vol(\I^1/\Q^\times) = 1.
\]

[Note: \ Here, $w = 0$ and $f(0) = 1$, $\widehat{f}(0) = 1.]$
\end{x}

That $Z(f,\omega,s)  $ can be meromorphically continued into the whole $s$-plane is now manifest.  
As for the functional equation, we have
\\
\begin{align*}
Z(f,\omega,s)      
&= \vsx \int_1^\infty Z_t(f,\omega,s) \frac{dt}{t} 
+ \int_1^\infty Z_t(\widehat{f},\ov{\omega},1-s)  \frac{dt}{t} 
+ R(f,\omega,s)  \\
&= \vsx\int_1^\infty \bigl(\int_{\I^1} f(tx) \omega(tx) \abs{tx}_\A^s d^\times x\bigr)  \frac{dt}{t} 
+ \int_1^\infty \bigl(\int_{\I^1} \widehat{f}(tx) \ov{\omega}(tx) \abs{tx}_\A^{1-s} d^\times x\bigr)  \frac{dt}{t} 
+ R(f,\omega,s).
\end{align*}
%%----------------------------------------------------------------------------------------------07
And we also have
\\
\begin{align*}
Z(\widehat{f},\ov{\omega},1-s)      	
&=\vsx \int_1^\infty Z_t(\widehat{f},\ov{\omega},1-s) \frac{dt}{t} + \int_1^\infty Z_t(\widehat{\widehat{f}\hspace{0.125cm}},\ov{\ov{\omega}},1-(1-s))  \frac{dt}{t} + R(\widehat{f},\ov{\omega},1-s)  \\
&=\vsx \int_1^\infty Z_t(\widehat{f},\ov{\omega},1-s) \frac{dt}{t} 
+ \int_1^\infty Z_t(\widehat{\widehat{f}\hspace{0.125cm}},\omega,s)\bigr)  \frac{dt}{t} 
+ R(\widehat{f},\ov{\omega},1-s)  \\						
&=\vsx \int_1^\infty \bigl(\int_{\I^1} \widehat{f}(tx) \ov{\omega}(tx) \abs{tx}_\A^{1-s} d^\times x\bigr)  \frac{dt}{t} 
+ \int_1^\infty \bigl(\int_{\I^1} \widehat{\widehat{f}\hspace{0.125cm}}(tx) \omega(tx) \abs{tx}_\A^s d^\times x\bigr)  \frac{dt}{t} \\
&\hspace{7.5cm} +  R(\widehat{f},\ov{\omega},1-s).
\end{align*}
The first of these terms can be left as is (since it already figures in the formula for $Z(f,\omega,s)).$  
Recalling that
\[
\widehat{\widehat{f}\hspace{0.125cm}}(x) = f(-x) \quad (x \in \A) \qquad (\text{cf.} \  \S15, \ \#10),
\]
The second term becomes
\[
\int_1^\infty \bigl(\int_{\I^1} f(-tx) \omega(tx) \abs{tx}_\A^s d^\times x\bigr) \frac{dt}{t}
\]
%%----------------------------------------------------------------------------------------------08
or still,
\[
\int_1^\infty \bigl(\int_{\I^1} f(tx) \omega(-tx) \abs{-tx}_\A^s d^\times x\bigr) \frac{dt}{t} 
= \int_1^\infty \bigl(\int_{\I^1} f(tx) \omega(-tx) \abs{tx}_\A^s d^\times x\bigr) \frac{dt}{t}.
\]
But by hypothesis, $\omega$ is trivial on $\Q^\times$, hence
\[
\omega(-tx) = \omega((-1)tx) = \omega(-1) \omega(tx) = \omega(tx),
\]
and we end up with 
\[
 \int_1^\infty \bigl(\int_{\I^1} f(tx) \omega(tx) \abs{tx}_\A^s d^\times x\bigr) \frac{dt}{t}
\]
which likewise figures in the formula for $Z(f,\omega,s)$.  
Finally, if $\omega$ is trivial on $\I^1$, then
\\
\begin{align*}
R(\widehat{f},\ov{\omega},1-s) \   	
&= -\frac{\widehat{f}(0)}{\sqrt{-1} \ w + 1 - s} +  \frac{\widehat{\widehat{f}\hspace{0.125cm}}(0)}{\sqrt{-1} \ w + (1 - s) - 1}\\
&= \frac{f(0)}{\sqrt{-1}\  w - s} -  \frac{\widehat{f}(0)}{\sqrt{-1} \ w + 1 - s}\\
&= -\frac{f(0)}{-\sqrt{-1}\  w + s} +  \frac{\widehat{f}(0)}{-\sqrt{-1}\  w + s - 1}\\
&= R(f,\omega,s).
\end{align*}
On the other hand, if $\omega$ is nontrivial on $\I^1$, then $\ov{\omega}$ is nontrivial on $\I^1$ and 
\[
R(f,\omega,s) = 0, \quad R(\widehat{f},\ov{\omega},1-s) = 0.
\]
\end{x}
%%%%%%%%%%%%%%%%%%%%%%%%%%%%%%
%%%%%%%%%%%%%%%%%%%%%%%%%%%%%%%%%%%%%%
%%%%%%%%%%%%%%%%%%%%%%%%%%%%%%%%%%%%%%
%%%%%%%%%%%%%%%%%%%%%%%%%%%%%%%%%%%%%%






















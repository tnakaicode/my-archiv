\chapter{
$\boldsymbol{\S}$\textbf{13}.\quad  RESTRICTED PRODUCTS}
\setlength\parindent{2em}
\setcounter{theoremn}{0}
%%----------------------------------------------------------------------------------------------01

\ \indent 
Recall:

\begin{x}{\small\bf FACT} \ %1
Suppose that $X_i$ $(i \in I)$ is a nonempty Hausdorff space $-$then the product 
$\prod\limits_{i \in I} X_i$ is locally compact iff each 
$X_i$ is locally compact and all but a finite number of the $X_i$ are compact.
\end{x}

\vspace{0.1cm}


Let $X_i \  (i \in I)$ be a family of nonempty locally compact Hausdorff spaces and for each 
$i \in I$, let $K_i \subset X_i$ be an open-compact subspace.
\begin{x}{\small\bf DEFINITION} \ %2
The 
\underline{restricted product}
\index{restricted product}
\[
\prod\limits_{i \in I} (X_i : K_i)
\]
consists of those $x = \{x_i\}$ in $\prod\limits_{i \in I} X_i$ such that $x_i \in K_i$ for all but a finite number of $i \in I$. 
\end{x}

\vspace{0.1cm}

\begin{x}{\small\bf \un{N.B.}} \ %3
\[
\prod_{i \in I} (X_i : K_i) = \bigcup_{S \subset I} \ \prod_{i \in S} X_i \times \prod_{i \notin S} K_i,
\]
where $S \subset I$ is finite.
\end{x}

\vspace{0.1cm}

\begin{x}{\small\bf DEFINITION} \ %4
A 
\underline{restricted open rectangle}
\index{restricted open rectangle} 
is a subset of $\prod\limits_{i \in I} (X_i : K_i)$ of the form
\[
\prod_{i \in S} U_i \times \prod_{i \notin S} K_i,
\]
where $S \subset I$ is finite and $U_i \subset X_i$ is open.
\end{x}

\vspace{0.1cm}
%%----------------------------------------------------------------------------------------------02


\begin{x}{\small\bf LEMMA} \ %5
The intersection of two restricted open rectangles is a restricted open rectangle.
\end{x}

\vspace{0.1cm}


Therefore the collection of restricted open rectangles is a basis for a topology on $\prod\limits_{i \in I} (X_i : K_i)$, the \underline{restricted product topology}.  
\index{restricted product topology}

\vspace{0.2cm}

\begin{x}{\small\bf LEMMA} \ %6
If $I$ is finite, then
\[
\prod_{i \in I} X_i = \prod_{i \in I} (X_i : K_i)
\]
and the restricted product topology coincides with the product topology.
\end{x}

\vspace{0.1cm}


\begin{x}{\small\bf LEMMA} \ %7
If $I = I_1 \cup I_2$, with $I_1 \cap I_2 = \emptyset$, then
\[
 \prod_{i \in I} (X_i : K_i) \ \approx \  \bigl( \prod_{i \in I_1} (X_i : K_i) \bigr) \times \bigl(\prod_{i \in I_2} (X_i : K_i)\bigr),
\]
the restricted product topology on the left being the product topology on the right.
\end{x}

\vspace{0.1cm}


\begin{x}{\small\bf LEMMA} \ %8
The inclusion $\prod\limits_{i \in I} (X_i : K_i)  \hookrightarrow \prod\limits_{i \in I} X_i$ is continuous but the restricted product topology coincides with the relative topology only if $X_i = K_i$ for all but a finite number of $i \in I$.
\end{x}
\vspace{0.1cm}

\begin{x}{\small\bf LEMMA} \ %9
$\ds\prod\limits_{i \in I} (X_i : K_i)$ is a Hausdorff space.

\vspace{0.1cm}

PROOF \   
Taking into account \#8, this is because

1. A subspace of a Hausdorff space is Hausdorff;

2. Any finer topology on a Hausdorff space is Hausdorff.
\end{x}

\vspace{0.1cm}

\begin{x}{\small\bf LEMMA} \ %10
$\prod_{i \in I} (X_i : K_i)$ is a locally compact Hausdorff space.
%%----------------------------------------------------------------------------------------------03

\vspace{0.1cm}

PROOF \  
Let $x \in \prod\limits_{i \in I} (X_i:K_i)$ $-$then there exists a finite set $S \subset I$ such that $x_i \in K_i$ if $i \notin S$.  
Next, for each $i \in S$, choose a compact neighborhood $U_i$ of $x_i$.  
This done, consider
\[
\prod_{i \in S} U_i \times \prod_{i \notin S} K_i,
\]
a compact neighborhood of $x$.
\end{x}

\vspace{0.1cm}


From this point forward, it will be assumed that $X_i \equiv G_i$ is a locally compact abelian group and $K_i \subset G_i$ is an open-compact subgroup.
\vspace{0.2cm}

\begin{x}{\small\bf NOTATION} \ %11
\[
G = \prod_{i \in I} \ (G_i:K_i).
\]
\end{x}

\vspace{0.1cm}


\begin{x}{\small\bf LEMMA} \ %12
$G$ is a locally compact abelian group.  
\end{x}

\vspace{0.1cm}

Given $i \in I$, there is a canonical arrow
\begin{align*}
\ins_i : G_i	&\ra G\\	
x			&\mapsto (\cdots, 1, 1, x, 1, 1, \cdots).\\	
\end{align*}


\begin{x}{\small\bf LEMMA} \ %13
$\ins_i$ is a closed embedding.

\vspace{0.1cm}

PROOF \ 
Take $S = \{i\}$ and pass to
\[
G_i \times \prod_{j \ne i} K_j,
\]
an open, hence closed subgroup of G.  The image $\ins_i(G_i)$ is a closed subgroup of
%%----------------------------------------------------------------------------------------------04
\[
G_i \times \prod_{j \ne i} K_j
\]
in the product topology, hence in the restricted product topology.\\

Therefore $G_i$ can be regarded as a closed subgroup of $G$.
\end{x}

\vspace{0.1cm}

\begin{x}{\small\bf LEMMA} \ %14

1. Let $\chi \in \widetilde{G}$ 
$-$then $\chi_i = \chi \circ \ins_i = \restr{\chi}{G_i} \in \widetilde{G}_i$ and 
$\restr{\chi}{K_i} \equiv 1$ for all but a finite number of $i \in I$, so for each $x \in G$, 
\[
\chi(x) \ =\  \chi(\{x_i\}) \ =\  \prod_{i \in I} \chi_i(x_i).
\]

2.  Given $i \in I$, let $\chi_i \in \widetilde{G}_i$ and assume that $\restr{\chi}{K_i} \equiv 1$ 
for all but a finite number of $i \in I$ $-$then the prescription 
\[
\chi(x) \ =\  \chi(\{x_i\}) \ =\  \prod_{i \in I} \chi_i(x_i)
\]
defines a $\chi \in \widetilde{G}$.
\end{x}

\vspace{0.1cm}


These observations also apply if $\widetilde{G}$ is replaced by $\widehat{G}$, in which case more can be said.
\vspace{0.2cm}


\begin{x}{\small\bf THEOREM} \ %15
As topological groups,
\[
\widehat{G} \thickapprox \prod_{i \in I} \  (\widehat{G}_i:K_i^\perp).
\]

[Note: \   Recall that
\[
K_i^\perp = \{\chi_i \in \widehat{G}_i:\chi{|K_i} \equiv 1\} \qquad (\text{cf.} \  \S7, \  \#32)
\]
and a tacit claim is that $K_i^\perp$ is an open-compact subgroup of $\widehat{G}$.  
To see this, 
%%----------------------------------------------------------------------------------------------05
quote \S7, \#34 to get
\[
\widehat{K}_i \thickapprox \widehat{G} / K_i^\perp, \quad K_i^\perp \thickapprox \widehat{G / K_i}.
\]
Then

%\begin{itemize}
\qquad \textbullet \quad
$K_i$ compact $\implies \widehat{K}_i$ discrete $\implies \widehat{G}/K_i^\perp$ discrete $\implies K_i^\perp$ open.

\qquad \textbullet \quad
$K_i$ open $\implies G/K_i$ discrete $\implies \widehat{G/K_i}$ compact $\implies K_i^\perp$ compact.$]$
%\end{itemize}
\end{x}

\vspace{0.1cm}


Let $\mu_i$ be the Haar measure on $G_i$ normalized by the condition
\[
\mu_i(K_i) = 1.
\]
\begin{x}{\small\bf LEMMA} \ %16
There is a unique Haar measure $\mu_G$ on $G$ such that for every finite subset $S \subset I$, the restriction of $\mu_G$ to
\[
G_S \ \equiv \ \prod_{i \in S} \ G_i \ \times  \ \prod_{i \notin S}\  K_i
\]
is the product measure.
\end{x}

\vspace{0.1cm}


Suppose that $f_i$ is a continuous, integrable function on $G_i$ such that $\restr{f_i}{K_i} = 1$ 
for all $i$ outside some finite set and let $f$ be the function on $G$ defined by
\[
f(x) = f(\{x_i\}) = \prod_i f_i(x_i).
\]
Then $f$ is continuous.
Proof: The $G_S$ are open and cover $G$ and on each of them $f$ is continuous.

\vspace{0.2cm}


\begin{x}{\small\bf LEMMA} \ %17
Let $S \subset I$ be a finite subset of $I$ $-$then
%%----------------------------------------------------------------------------------------------06
\[
\int_{G_S} f(x) d\mu_{G_S} (x) \ =\  \prod_{i \in S} \  \int_{G_i} f_i(x_i)d\mu_{G_i}(x_i).
\]
\end{x}

\vspace{0.1cm}


\begin{x}{\small\bf APPLICATION} \ %18
If
\[
\sup_S \ \prod_{i \in S} \  \int_{G_i} \  \abs{f_i(x_i)} d\mu_{G_i}(x_i) \ < \  \infty,
\]
then $f$ is integrable on $G$ and
\[
\int_G f(x) d\mu_{G} (x) \ = \ \prod_{i \in I} \ \int_{G_i} f_i(x_i)d\mu_{G_i}(x_i).
\]
\end{x}

\begin{x}{\small\bf EXAMPLE} \ %19
Take  $f_i = \chi_{K_i}$ (which is continuous, $K_i$ being open-compact) $-$then $\widehat{f}_i = \chi_{K_i^\perp}.$  
Setting
\[
f \ = \ \prod_{i \in I} f_i,
\]
it thus follow that $\forall$ $\chi \in \widehat{G}$,
\[
\widehat{f}(\chi) \ = \  \prod_{i \in I} \widehat{f}_i(\chi_i).
\]
\end{x}

\vspace{0.1cm}


Working within the framework of \S7, \#45, let $\mu_{\widehat{G}_i}$ be the Haar measure on $\widehat{G}_i$ per Fourier inversion.

\vspace{0.1cm}

\begin{x}{\small\bf LEMMA} \ %20
\[
\mu_{\widehat{G}_i} (K_i^\perp) = 1.
\]

\vspace{0.1cm}

PROOF \ 
%%----------------------------------------------------------------------------------------------07 (ish)
Since $\chi_{K_i} \in \mathbf{INV}(G_i)$, $\forall$ $x_i \in G_i$,
\begin{align*}
\chi_{K_i} (x_i) \ 
&= \  \int_{\widehat{G}_i } \widehat{\chi}_{K_i} (x_i) \overline{\chi_i (x_i)} d \mu_{\widehat{G}_i} (\chi_i) \\
&= \  \int_{K_i^\perp} \overline{\chi_i (x_i)} d \mu_{\widehat{G}_i} (\chi_i).
\end{align*}
Now set $x_i = 1$ to get
\begin{align*}
1 
&=\  \int_{K_i^\perp} d\mu_{\widehat{G}_i} (\chi_i) \\
&=\  \mu_{\widehat{G}_i}(K_i^\perp).
\end{align*}
\end{x}

\vspace{0.1cm}



Let $\mu_{\widehat{G}}$ be the Haar measure on $\widehat{G}$ constructed as in \#16  (i.e., replace $G$ by 
$\widehat{G}$, bearing in mind \#20).

\vspace{0.2cm}


\begin{x}{\small\bf LEMMA} \ %21
$\mu_{\widehat{G}}$ is the Haar measure on $\widehat{G}$ figuring in the Fourier inversion per $\mu_G$.

\vspace{0.1cm}

PROOF \  Take
\[
f \ =\  \prod_{i \in I} f_i,
\]
where $f_i = \chi_{K_i}$ (cf. \#19 ) $-$then
\begin{align*}
\int_{\widehat{G}} \widehat{f}(\chi) \overline{\chi(x)} d\mu_{\widehat{G}} (\chi) \ 
&=\  \prod_{i \in I} \ \int_{\widehat{G}_i} \widehat{f}_i(\chi_i) \overline{\chi_i(x_i)} d\mu_{\widehat{G}_i} (\chi_i) \\	
&=\  \prod_{i \in I}\  f_i(x_i) \\	
&=\  f(\{x_i\}) \\
&=\  f(x).
\end{align*}
\end{x}
%%%%%%%%%%%%%%%%%%%%%%%%%%%%%%%%%%%%%%
%%%%%%%%%%%%%%%%%%%%%%%%%%%%%%%%%%%%%%
%%%%%%%%%%%%%%%%%%%%%%%%%%%%%%%%%%%%%%






















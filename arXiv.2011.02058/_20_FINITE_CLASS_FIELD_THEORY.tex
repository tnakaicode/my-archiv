\chapter{
$\boldsymbol{\S}$\textbf{20}.\quad  FINITE CLASS FIELD THEORY}
\setlength\parindent{2em}
\setcounter{theoremn}{0}
%%----------------------------------------------------------------------------------------------01

\ \indent 
Given a finite field $\F_q$ of characteristic $p$ (thus $q$ is an integral power of $p$), then in $\F_p^{\cl}$,
\[
\F_q \ = \ \{x:x^q = x\}.
\]
\vspace{0.1cm}

\begin{x}{\small\bf LEMMA} \ %01
The multiplicative group 
\[
\F_q^\times \ = \ \{x:x^{q-1} = 1\}
\]
is cyclic of order $q-1$.
\end{x}
\vspace{0.1cm}

\begin{x}{\small\bf NOTATION} \ %02
\[
\F_{q^n} \ = \ \{x:x^{q^n} = x\} \qquad (n \geq 1).
\]
\end{x}
\vspace{0.1cm}

\begin{x}{\small\bf LEMMA} \ %03
$\F_{q^n}$ is a Galois extension of $\F_q$ of degree $n$.
\end{x}
\vspace{0.1cm}

\begin{x}{\small\bf LEMMA} \ %04
$\Gal(\F_{q^n}/F_q)$ is a cyclic group of order $n$ generated by the element $\sigma_{q,n}$, where
\[
\sigma_{q,n}(x) \ = \ x^q \qquad (x \in \F_{q^n}).
\]
\end{x}
\vspace{0.1cm}

\begin{x}{\small\bf LEMMA} \ %05
The $\F_{q^n}$ are finite abelian extensions of $\F_q$ and they comprise all the finite extensions of $\F_q$, hence the algebraic closure of $\bigcup\limits_n \F_{q^n}$ is $\F_q^\ab$.
\end{x}
\vspace{0.1cm}

\begin{x}{\small\bf THEOREM} \ %06
There is a 1-to-1 correspondence between the finite abelian 
%%----------------------------------------------------------------------------------------------02
extensions of $\F_q$ and the subgroups of $\Z$ of finite index which is given by 
\[
\F_{q^n} \longleftrightarrow n\Z \qquad (n \geq 1).
\]
\end{x}
\vspace{0.1cm}

Schematically:\\


\[
\begin{tikzcd}[sep=tiny]
{\F_q} &{\subset} &{\F_{q^2}} &{\subset} &{\F_{q^4}}\\
{\cap} &&{\cap}\\
{\F_{q^3}} &{\subset} &{\F_{q^6}}\\
{\cap}\\
{\F_{q^9}}
\end{tikzcd}
\begin{tikzcd}[sep=tiny]
\\
\\
{ } \ar{rrrrr} &&&&&{} \ar{lllll}
\\
\\
\end{tikzcd}
\begin{tikzcd}[sep=tiny]
{\Z} &{\supset} &{2\Z} &{\supset} &{4\Z}\\
{\cup} &&{\cup}\\
{3\Z} &{\supset} &{6\Z}\\
{\cup}\\
{9\Z}
\end{tikzcd}
.
\]

The ``class field'' aspect of all this is the existence of a canonical homomorphism
\[
\rec_q:\Z \lra \Gal(\F_q^\ab / \F_q).
\]

\vspace{0.1cm}

\begin{x}{\small\bf NOTATION} \ %07
Define
\[
\sigma_q \in \Gal(\F_q^\ab / \F_q)
\]
by 
\[
\sigma_q(x) \ = \ x^q.
\]
\end{x}

\vspace{0.2cm}
%%----------------------------------------------------------------------------------------------03

\begin{x}{\small\bf \un{N.B.}} \ %08
Under the arrow of restriction
\[
\Gal(\F_q^\ab / \F_q) \lra \Gal(\F_{q^n} / \F_q),
\]
$\sigma_q$ is sent to $\sigma_{q,n}$.
\end{x}

\vspace{0.1cm}

\begin{x}{\small\bf DEFINITION} \ %09
\[
\rec_q(k) \ = \ \sigma_q^k \qquad (k \in \Z).
\]
\index{$\rec_q$}
\end{x}

\vspace{0.1cm}

\begin{x}{\small\bf LEMMA} \ %10
The identification
\[
\Z/n\Z \ \approx \ \Gal(\F_{q^n} / \F_q). 
\]
is the arrow $k \ra \sigma_{q,n}^k$.
\end{x}

\vspace{0.1cm}

On general grounds, 
\[
\Gal(\F_q^\ab / \F_q) \ = \ \lim\limits_{\lla} \Gal(\F_{q^n} / \F_q).
\]

[Note: \  The open subgroups of $\Gal(\F_q^\ab / \F_q)$ are the $\Gal(\F_{q}^\ab / \F_{q^n})$ and 
\[
\Gal(\F_q^\ab / \F_q) /  \Gal(\F_q^\ab / \F_{q^n}) \ \approx \ \Gal(\F_{q^n} / \F_q).]
\]

Therefore 
\[
\Gal(\F_q^\ab / \F_q)  \ \approx  \ \lim\limits_{\lla} \Z / n\Z,
\]
another realization of the RHS being $\prod\limits_p \Z_p$ which if invoked leads to 
\[
\sigma_q \longleftrightarrow (1, 1, 1, \ldots).
\]
%%----------------------------------------------------------------------------------------------04

\begin{x}{\small\bf \un{N.B.}} \ %11
The composition
\[
\begin{tikzcd}[sep=small]
{\Z} \ar{rr}{\rec_q} &&{\Gal(\F_q^\ab / \F_q) \ \approx \ \lim\limits_{\lla} \Z / n\Z} 
\end{tikzcd}
\]
coincides with the canonical map 
\[
k \ra (k  \ \modx n)_n.
\]
\end{x}

\vspace{0.2cm}

\begin{x}{\small\bf REMARK} \ %12
Give $\Z$ the discrete topology $-$then 
\[
\rec_q: \Z \lra \Gal(\F_q^\ab / \F_q) 
\]
is continuous and injective but it is not a homeomorphism ($\Gal(\F_q^\ab / \F_q)$ is compact).

\vspace{0.1cm}

[Note: \ 
The image $\rec_q(\Z)$ is the cyclic subgroup $\langle \sigma_q \rangle$ generated by $\sigma_q$.  And:

\vspace{0.2cm}

\qquad \textbullet \quad $\langle \sigma_q \rangle \ \neq \  \Gal(\F_q^\ab / \F_q)$\\

\qquad \textbullet \quad $\ov{\langle \sigma_q \rangle} \ = \  \Gal(\F_q^\ab / \F_q)$.]
\end{x}
\vspace{0.1cm}

\begin{x}{\small\bf SCHOLIUM} \ %13
The finite abelian extensions of $\F_q$ correspond 1-to-1 with the open subgroups of $\Gal(\F_q^\ab / \F_q)$.

\vspace{0.1cm}

[Quote the appropriate facts from infinite Galois theory.]
\end{x}

\vspace{0.1cm}

\begin{x}{\small\bf SCHOLIUM} \ %14
The open subgroups of $\Gal(\F_q^\ab / \F_q)$ correspond 1-to-1 with the open subgroups of $\Z$ of finite index.

\vspace{0.1cm}

[Given an open subgroup $U \subset \Gal(\F_q^\ab / \F_q)$, send it to $\rec_q^{-1}(U) \subset \Z$ (discrete topology).  
Explicated: 
\[
\rec_q^{-1}(\Gal(\F_q^\ab / \F_{q^n})) \ = \ n\Z.]
\]
\end{x}
\vspace{0.1cm}

%%----------------------------------------------------------------------------------------------05

\[
\textbf{APPENDIX}
\]
\setcounter{theoremn}{0}

The norm map 
\[
\tN_{\F_{q^n} / \F_q}: \F_{q^n}^\times \lra \F_q^\times
\]
is surjective.

\vspace{0.1cm}

[Let $x \in \F_{q^n}^\times$:
\begin{align*}
\tN_{\F_{q^n} / \F_q}(x) \ 
&= \ \prod\limits_{i = 0}^{n-1} \ (\sigma_{q,n})^{i_x}\\
&= \ \prod\limits_{i = 0}^{n-1} \ x^{q^i}\\
&= \ x^{\sum\limits_{i = 0}^{n-1} q^i}\\
&= \ x^{(q^n - 1) / (q-1)}.
\end{align*}
Specialize now and take for $x$ a generator of $\F_{q^n}^\times$, hence $x$ is of order $q^n - 1$, hence 
$\tN_{\F_{q^n} / \F_q}(x)$ is of order $q - 1$, hence is a generator of $\F_q$.]














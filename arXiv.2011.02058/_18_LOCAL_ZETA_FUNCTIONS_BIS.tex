\chapter{
$\boldsymbol{\S}$\textbf{18}.\quad  LOCAL ZETA FUNCTIONS (BIS)}
\setlength\parindent{2em}
\setcounter{theoremn}{0}
%%----------------------------------------------------------------------------------------------01

\ \indent 
To be in conformity with the global framework laid down in \S17, we shall reformulate the local theory of \S11 and \S12.

\vspace{0.25cm}

\begin{x}{\small\bf DEFINITION} \ %01
Given $f \in \mathcal{S}(\R)$ and a unitary character $\omega : \R^\times \ra \T$, the 
\un{local zeta function}
\index{local zeta function} 
attached to the pair $(f,\omega)$ is
\[
Z(f,\omega,s) = \int_{\R^\times} f(x) \omega(x) \abs{x}^s d^\times x 		\qquad (\Re (s) > 0).
\]
\end{x}

\vspace{0.1cm}



\begin{x}{\small\bf THEOREM} \ %02
There exists a meromorphic function $\rho(\omega,s)$ such that $\forall \ f$,
\[
\rho(\omega,s) = \frac{Z(f,\omega,s) }{Z(\widehat{f},\ov{\omega},1-s) }.
\]

Decompose $\omega$ as a product:
\[
\omega(x) = (\sgn\  x)^\sigma \abs{x}^{-\sqrt{-1} \ w} \qquad (\sigma \in \{0,1\}, w \in \R).
\]
\end{x}
\vspace{0.1cm}


\begin{x}{\small\bf DEFINITION} \ %03
Write (cf. \S11, \#9)
\[
L(\omega,s) =\ 
\begin{cases}
\Gamma_\R (s - \sqrt{-1} \ w) \quad \quad \quad (\sigma = 0)
\vspace{0.25cm}\\
\Gamma_\R (s - \sqrt{-1} \ w + 1 )  \ \quad (\sigma = 1)
\end{cases}
.
\]
\end{x}
\vspace{0.1cm}



\begin{x}{\small\bf FACT} \ %14
\[
\rho(\omega,s) = \ 
\begin{cases}
\ds\frac{L(\omega,s) }{L(\omega,1-s)} \quad \quad \quad \quad (\sigma = 0)
\vspace{0.25cm}\\
-\sqrt{-1} \ \ds\frac{L(\omega,s) }{L(\ov{\omega},1-s)}  \  \quad (\sigma = 1)
\end{cases}
.\]
\end{x}
\vspace{0.1cm}
%%----------------------------------------------------------------------------------------------02



\begin{x}{\small\bf REMARK} \ %05
The complex case can be discussed analogously but it will not be needed in the sequel.
\\
\end{x}

\vspace{0.1cm}

\begin{x}{\small\bf DEFINITION} \ %06
Given $f \in \sB(\Q_p)$ and a unitary character $\omega:\Q_p^\times \ra \T$, 
the 
\un{local zeta function}
\index{local zeta function} 
attached to the pair $(f,\omega)$ is
\[
Z(f,\omega,s) = \int_{\Q_p^\times} f(x) \omega(x) \abs{x}_p^s d^\times x \qquad (\Re (s) > 0).
\]
\end{x}

\vspace{0.1cm}

\begin{x}{\small\bf THEOREM} \ %07
There exists a meromorphic function $\rho(\omega,s)$ such that $\forall$ $f$,
\[
\rho(\omega,s) = \frac{Z(f,\omega,s) }{Z(\widehat{f},\ov{\omega},1-s) }.
\]

Decompose $\omega$ as a product:
\[
\omega(x) = \un{\omega}(x) \abs{x}_p^{-\sqrt{-1} \ w} \qquad (\un{\omega} \in \widehat{\Z_p^\times}, \ w \in \R).
\]
\end{x}

\vspace{0.1cm}

\begin{x}{\small\bf DEFINITION} \ %08
Write (cf. \S12, \ \#8)
\[
L(\omega,s) =\ 
\begin{cases}
(1 - \omega(p)p^{-s})^{-1}   \quad (\un{\omega} = 1)
\vspace{0.25cm}\\
\ 1  \  \quad \quad \quad \quad \quad \quad \quad (\un{\omega} \ne 1)
\end{cases}
.\]

[Note: \  if $\un{\omega}$ = 1, then
\[
\omega(p) = \abs{p}_p^{-\sqrt{-1}\ w} = p^{\sqrt{-1}\ w}.]
\]
\end{x}

\vspace{0.1cm}

\vspace{0.15cm}
\begin{x}{\small\bf FACT} \ %09
$(\un{\omega} = 1)$
\[
\rho(\omega,s) \ = \  \frac{L(\omega,s) }{L(\ov{\omega},1-s)} \ = \  \frac{1 - \ov{\omega}(p)p^{-(1-s)}}{1 - \omega(p)p^{-s}}.
\]
\end{x}

\vspace{0.1cm}
%%----------------------------------------------------------------------------------------------03


\begin{x}{\small\bf FACT} \ %10
$(\un{\omega} \ne 1$)
\[
\rho(\omega,s) = \tau(\omega) \hspace{0.05cm} \un{\omega}(-1) \hspace{0.05cm} p^{n(s \ + \ \sqrt{-1} \  -1)},
\]
where
\[
\tau(\omega) = \sum_{i = 1}^r \un{\omega}(e_i) \chi_p(p^{-n}e_i)
\]
and $\deg \omega = n \ge 1.$
\end{x}

%%%%%%%%%%%%%%%%%%%%%%%%%%%%%%%%%%%%%%%%%%%%%%%%%%
%%%%%%%%%%%%%%%%%%%%%%%%%%%%%%%%%%%%%%%%%%%%%%%%%%
%%%%%%%%%%%%%%%%%%%%%%%%%%%%%%%%%%%%%%%%%%%%%%%%%%

\[
\textbf{APPENDIX}
\]
\setcounter{theoremn}{0}


It can happen that
\\
\[
Z(f,\omega,s)  \ \equiv \  0.
\]
To illustrate, suppose that $\omega(-1) = -1$ and $f(x) = f(-x)$.  Working with $\Q_p^\times$ $($the story for $\R^\times$ being the same$)$, we have
\\
\begin{align*}
Z(f,\omega,s)	\ 	
&=\vsx\  \int_{\Q_p^\times} f(x) \omega(x) \abs{x}_p^s d^\times x\\
&=\vsx\  \int_{\Q_p^\times} f(-x) \omega(-x) \abs{-x}_p^s d^\times x\\
&=\vsx\ \omega(-1)  \int_{\Q_p^\times} f(x) \omega(x) \abs{x}_p^s d^\times x\\	
&=\vsx\ \omega(-1) Z(f,\omega,s)\\
&=\vsx\ -Z(f,\omega,s).
\end{align*}

%%%%%%%%%%%%%%%%%%%%%%%%%%%%%%%%%%%%%%
%%%%%%%%%%%%%%%%%%%%%%%%%%%%%%%%%%%%%%
%%%%%%%%%%%%%%%%%%%%%%%%%%%%%%%%%%%%%%






















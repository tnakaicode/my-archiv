\begin{chapter}{Computation Resources}
\label{app:comp}

\hspace{5 mm} 

Some of the computational tools developed during the work on this thesis might be of interest to other students, and the purpose of this section is to share these tools.
The tool below was built with the \texttt{Mathematica} software and its code is freely available on Github, with a short link provided.

\section{Tensor Contraction}

This program performs contraction of indices on Euclidean
tensors written with the Einstein summation convention.
There already are specific packages for tensor manipulations,
especially in the context of General Relativity, but they provide
more general functionality at the expense of a more complex
interface.
In the context of the RFD model (Chapter \ref{chap:rfd}), only manipulations of Euclidean tensors are necessary, a case in which there is no difference between covariant and contravariant indices.

The function \texttt{contract} was built in order to carry out tensor contractions in arbitrarily long products of Euclidean tensors. 
For instance, consider the isotropic tensor $G_{ijkl}$ defined in Eq.~\eqref{eq:tensor-g}:
\begin{equation}
G_{ijkl} = 2 \delta_{ik} \delta_{jl} - \frac12 \delta_{il} \delta_{jk} - \frac12 \delta_{ij} \delta_{kl} \ .
\end{equation}
The diagrammatic contributions of Eqs.~\eqref{eq:g_correction} and \eqref{eq:v_correction} are obtained as specific contractions of products of this tensor. 
As a demonstration of this function, some results are shown in Fig.~\ref{fig:contract}, corresponding to the following operations: $G_{ikil}$, $G_{iijj}$, $G_{ijij}$ and $G_{ijkl}G_{klmn}$.
% These resources were also needed in Chap. \ref{chap:shotnoise}.

\begin{figure}[h]
	\centering
	\includegraphics[width=\textwidth]{contract}
	\caption
	[Examples of tensor contractions performed with the \texttt{contract} function]
	{A picture from the \texttt{Mathematica} software, with 
	examples of tensor contractions performed with the \texttt{contract} function on the tensor $G_{ijkl}$.}
	\label{fig:contract}
\end{figure}

As another example, the specific contraction which produces the result of Eq.~\eqref{eq:g_correction} is depicted in Fig.~\ref{fig:contract-noise}.

\begin{figure}[h]
	\centering
	\includegraphics[width=\textwidth]{contract-noise.png}
	\caption
	[Tensor contraction for the noise renormalization in the RFD model.]
	{Tensor contraction for the noise renormalization in the RFD model.}
	\label{fig:contract-noise}
\end{figure}

As a last example, contractions of the tensor $G_{ijkl}$ with arbitrary vectors, denoted by $u$ and $v$, are shown, in Fig.~\ref{fig:contract-vector}.

\begin{figure}[ht]
	\centering
	\includegraphics[width=\textwidth]{contract-vector.png}
	\caption
	[Tensor contractions including products of tensors and vectors.]
	{Tensor contractions including products of tensors and vectors.}
	\label{fig:contract-vector}
\end{figure}

A final remark is that this function can only perform contractions of repeated indices, while the value of individual elements of a tensor should be obtained manually, or with other tools.

The code for the \texttt{contract} function is available at https://git.io/fjQe8.
%\href{https://git.io/fjQe8}{https://git.io/fjQe8}.

\end{chapter}

\begin{chapter}{Conclusion}
\label{chap:conclusion}

In this dissertation, two main approaches to the theoretical description of large fluctuations in turbulence have been pursued: functional methods and the modelling through non-Markovian stochastic processes, with techniques coming from statistical field theory, random fields, statistical analysis and numerical methods. In this closing chapter, the main results of each previous chapter and directions for further research are laid out.

In Chap.~\ref{chap:rfd}, an analytical technique for the description of PDF cores of non-gaussian observables is described and applied to the RFD model of Lagrangian turbulence. This analysis verified the accuracy of the analytical instanton approximation and determined a hierarchical ordering of the perturbative diagrams according to their relevance. The same technique was also applied to the Burgers model in Chap.~\ref{chap:burgers}, and it was revealed that a provisional noise renormalization procedure, previously observed on the velocity gradient PDFs can be explained as the product of fluctuations around the asymptotic instanton description.

Further questions remain on these matters, nevertheless. Numerical (exact) instantons could generate results of higher precision and a broader regime of validity to the studies of \textcite{apolinario2019instantons,apolinario2019onset}. The exact instantons have been studied in \textcite{grafke2015instanton,grigorio2017instantons,ebener2019}, but not the contribution from their fluctuations.
This analytical approach permits the study of complex statistical observables such as conditional measures without the need to generate huge ensembles for the measurement of the frequency of intense fluctuations.

% A useful aspect of the analytical approach to the velocity gradient PDFs is the possibility of generating large statistical ensembles for conditioned statistics. In this approach, ensembles can be easily produced which are considerably larger than those corresponding to a straightforward numerical solution of the model equations. This is an area of intense current investigation, particularly on algorithms for the preferential sampling of large deviations \parencite{giardina2006,giardina2011,margazoglou2019}.

In the stochastic Burgers model, the negative velocity gradient asymptotic exponent of $3/2$ (Eq.~\ref{eq:left-asymp}), although obtained as a theoretical result in \textcite{balkovsky1997}, has so far not been observed in numerical simulations of this system, instead the value of $1.16$ is reported \parencite{grafke2015relevance}. The investigation of the influence of noise in this problem, though, requires non perturbative techniques to describe the preasymptotic tails of the negative velocity gradient probability distribution.

Further applications of the functional methods described are also possible. For instance, the investigation of circulation statistics in three or two-dimensional Navier-Stokes turbulence \parencite{moriconi2004, smith1997, falkovich2011}, or the transport of passive scalars \parencite{balkovsky1998instanton}.

%Nevertheless, a full characterization of the velocity gradient PDFs in Burgers turbulence remains an open problem. Its behavior in the regime of fully developed turbulence is a challenge which cannot be undertaken with the cumulant expansion, and instead, nonperturbative techniques would be required for the path-integration of fluctuations around the instantons. Improved analytical solutions for the instantons would be required as well.

The work of Chap.~\ref{chap:shotnoise} adds to the effort of building a causal structure to continuous and scale invariant energy cascades. This is done through the modeling of the Lagrangian pseudo-dissipation with a multifractal random field driven by periodic shot noise, which is found to be lognormal and long-range correlated. Nevertheless, futher investigation is required to verify from DNS the statistical properties of Lagrangian dissipation and pseudo-dissipation, in order to understand how well such models represent real energy cascades. In this investigation, understanding the behavior of coarse-grained cascades is crucial, as discussed. 

%It is compelling to note that the analytical advantages of the lognormal formulation make it suitable for applications in several other fields where intermittent fluctuations play a role, besides turbulence, such as in financial economics \parencite{mandelbrot1997,ghashghaie1996,liu1999}, cosmology \parencite{coles1991} and condensed matter systems \parencite{kravtsov1997,serbyn2017}

The understanding of Lagrangian fluctuations is key to the effective modeling of transport properties, either of particles or fields, and to the understanding of the motion of extended structures in turbulence, such as filaments, rods and surfaces. They are heavily influenced by the localized intense bursts of energy dissipation, but still poorly understood theoretically.


\end{chapter}